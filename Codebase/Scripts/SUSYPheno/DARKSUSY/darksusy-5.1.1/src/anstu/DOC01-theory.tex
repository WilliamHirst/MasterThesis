%%%%%%%%%%%%%%%%%%%%%%%%%%%%%%%%%%%%%%%%%%%%%%%%%%%%%%%%%%%%%%%%%%%%

\section{Annihilation amplitudes for fermion-fermion annihilation}

In this directory, all the helicity amplitudes needed for
neutralino-neutralino, neutralino-chargino and chargino-chargino
annihilation are calculated. The helicity amplitudes have been
calculated with general expressions for vertices, masses etc.\ in
\code{Reduce} and converted to Fortran files. The calculation of these
are described in more detail in \cite{edsjo97}.

Each routine here adds the contribution to the helicity amplitudes
from one particular diagram and the sum over contributed diagrams is
done in the routines \codeb{an/dsandwdcosnn}, \codeb{an/dsandwdcoscn}
and \codeb{an/dsandwdcoscc}. The naming convention for the routines
here is the following: The first part of the routine name is
\codeb{dsan} to indicate that they deal with annihilations in \ds. The
next character tells which kind of process it is $s$-, $t$- or
$u$-channel and the next two caracters tell which initial state
particles we have (\codeb{f} for fermion), the next character is the
kind of propagating particle (\codeb{f} for fermion, \codeb{s} for
scalar and \codeb{v} for vector boson), and finally, the last two
characters tell the kind of final state particles. So, to take an
example, the routine \codeb{dsansffsvv} calculates the helicity
amplitudes for annihilation of two fermions to two vector
bosons via $s$-channel exchange of a scalar. There are also a few
special cases (routines ending in \codeb{ex} or \codeb{in}) for
diagrams with clashing arrows.
