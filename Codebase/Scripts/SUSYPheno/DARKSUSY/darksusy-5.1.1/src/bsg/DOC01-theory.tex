%%%%%%%%%%%%%%%%%%%%%%%%%%%%%%%%%%%%%%%%%%%%%%%%%%%%%%%%%%%%%%%%%%%%

\section{$b \rightarrow s \gamma$ -- theory}

The rare decay $b \rightarrow s \gamma$ can have large contributions
from loops of supersymmetric particles and one therefore has to check
that a particular supersymmetric model does not violate the observed
branching ratio for $b$ decay to $s \gamma$. In \ds\ we have several
expressions for calculations of the $b \rightarrow s \gamma$ decay. In
\codeb{src/ac}, some older obsolte expressions are found
(kept only for historical reasons). In this directory, 
\codeb{bsg/}, we have our best (and next-to-best)
implementation of the $b \rightarrow s \gamma$ decay.

Our best estimate of this process includes the
complete next-to-leading 
order (NLO) correction for the SM contribution and the dominant
NLO corrections for the SUSY term.  Until 2006, the NLO QCD SM calculation
was performed following the analysis in Ref.~\cite{bsgsm},
modified according to~\cite{bsgmagic}, 
and gave a branching ratio 
$\mathrm{BR}[B\rightarrow X_s\,\gamma] =3.72\times10^{-4}$ 
for a photon energy greater than  $m_b/20$. 
However, there was a big shift in the
theoretical calculation of the SM branching ratio in 2006. We thus now instead
have implemented the new results \cite{bsg2006} which instead give a SM 
branching ratio of $\mathrm{BR}[B\rightarrow X_s\,\gamma] =(3.15 \pm 0.23) \times10^{-4}$ 

In the SUSY contribution, we 
include the NLO contributions in the two Higgs doublet model,
following~\cite{bsgh2}, and the corrections due to SUSY particles. The latter
are calculated under the assumption of minimal flavour violation, with the 
dominant LO contributions from Ref.~\cite{bsgtan}, and with the NLO QCD term
with expressions of \cite{bsgsusy} modified in the large $\tan\beta$ regime
according to~\cite{bsgtan}. In the mSUGRA framework
(see, e.g., ~\cite{bsgcompare}), the largest discrepancy 
between the LO and the NLO SUSY corrections are found for ${\rm sign}{\mu}>0$, 
large $\tan\beta$ and low values of $m_{1/2}$: in this case the SUSY 
contribution to the decay rate is negative, and the discrimination of models 
based on the NLO analysis is less restrictive than the one in the LO analysis.

We will assume as allowed range of branching ratios
$2.71\times10^{-4}\leq\mathrm{BR}[B\rightarrow X_s\,\gamma] \leq4.39\times10^{-4}$,
which is obtained adding a theoretical uncertainty of 
$\pm 0.23 \times10^{-4}$ (both for the SM and for the SUSY calculation) to the
experimental value quoted by \cite{Barbiero-bsg2007}.

