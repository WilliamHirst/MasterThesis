%%%%%%%%%%%%%%%%%%%%%%%%%%%%%%%%%%%%%%%%%%%%%%%%%%%%%%%%%%%%%%%%%%%%

\section{Gamma rays from the halo -- theory}
\label{section_hr}

Among the yields of pair annihilations of halo dark matter particles,
the role played by gamma-rays could be a major one. Unlike the cases
involving charged particles, for gamma-rays it is straightforward
to relate the distribution of sources and the expected flux
at the earth. Most flux estimated can be obtained just by summing
over the contributions along 
lines of sight (or better, geodesics): gamma-rays have a low 
enough cross section on gas and dust and therefore the Galaxy is 
essentially transparent to them (except perhaps in the innermost part, 
very close to the region where a massive black hole is inferred); 
absorption by starlight and infrared background becomes effecient
only for very far away sources (redshift larger than about 1).

It follows that in case the gamma-ray signal is detectable, 
this might be the only chance for mapping the fine structure of a dark 
halo, with a much better resolution for inomogenities (clumps) with 
respect what is accevable through dynamical measurements or lensing effects. 
Turning the latter argument around, if the fine structure of the Galactic 
halo is clumpy, or if a large density enhancement is present towards the 
Galactic center, as seen in N-body simulations of dark matter halos,
this dark matter detection tecnique is much more promising than indicated 
by the earliest estimates in which smooth non-singular halo 
scenarios were considered (recall that the fluxes per unit volume are 
proportional to the square of the dark matter density locally in space).

A further reason to examine in details this detection methods is that
we are approaching what will probably be the golden age for
gamma-ray observations, with a several new experiments that are going 
to map the gamma-ray sky. These experiments will have unprecendented 
sensitivities and cover an energy range, namely 10 GeV -- few hundred GeV, 
in which very scarce data are available at the time being and which may 
turn out to be the most interesting for dark matter detection.
The hypothesis of a gamma-ray signal from neutralino annihilations 
will be tested for both by the upcoming space experiments 
(GLAST, AMS, AGILE) and by the new generation of ground-based 
air cherenkov telescopes (ACTs) being built (Magic, Hess, Veritas).

The bulk of the gamma-ray yield from neutralino annihilations arise in the
decay of neutral pions produced in the fragmetation processes initiated
by tree level final states~\cite{oldcontga,lpj,gahalo}
(analogously to the other halo signals,
in \ds\ we include all tree level final states and make use of a
Monte Carlo simulation for fragmentation and decay processes, see 
Section~\ref{sec:mcsim}). Unfortunately the $\pi^0$ 
intermediate state is common to other astrophysical processes, and this
may turn out to be a limiting factor to disentangle dark matter sources.
At the same time, however, a relevant gamma-ray contribution may arise 
directly (at one-loop level) in two body final states; although such 
photons are much fewer than those from $\pi^0$ decays they have a much 
better signature: neutralinos annihilating in the galactic halos move 
with a velocity of the order $v/c \sim$ 10$^{-3}$, hence these outgoing 
photons (as any particle in any of the allowed two body final states)
will then be nearly monochromatic, with energy of the order of the
neutralino mass\cite{charm,oldlines,jkline,lp,ub,lpj}. 
There is no other known astrophysical source with such a signature:
the detection of a line signal out of a spectrally smooth gamma-ray 
background would be a spectacular confirmation of the existence of dark 
matter in form of exotic massive particles.

If dark matter is in form of neutralinos, there are two processes
givin rise to line signals, the annihilation into two photons and into
one photon and a Z boson. Both of them are included in the \ds\ 
package, as well as the contribution with a continuum energy spectrum.
We review them briefly here, focussing first on annihilation rates
and giving then expressions for gamma-ray fluxes.


\subsection{$\chi\chi\to \gamma\gamma$}

In \ds\ the full expression for the annihilation cross section of
the process
\beq
\tilde{\chi}^{0}_{1} + \tilde{\chi}^{0}_{1} \rightarrow \gamma
+\gamma
\eeq
is computed at full one loop level, in the limit of vanishing relative 
velocity of the neutralino pair, i.e. the case of interest for neutralinos
in galactic halos; the outgoing photons have an energy equal to 
the mass of ${\chi}^{0}_{1}$:
\beq
E_{\gamma} = M_{\chi}.
\eeq
The neutralino pair must be in an S wave state with pseudoscalar quantum 
numbers; projecting out of the amplitude the $^{1}$S$_{0}$ state
simplifies the calculation, and a further simplification is obtained by
computing the amplitude in the non linear gauge defined in~\cite{fujikawa}, 
which is a slight variant of the usual linear R-gauge (or 't Hooft gauge). 

The amplitude of the annihilation process can be factorized in the
form
\beq
\mathcal{A} =\frac{e^2}{2 \sqrt{2} \; \pi^2}
  \epsilon\left(\epsilon_{1},\epsilon_{2},k_{1},k_{2} \right)
  \;\tilde{\mathcal{A}}
\eeq
where $\epsilon_{1}$, $\epsilon_{2}$ and $k_{1}$, $k_{2}$ are
respectively the polarization tensors and the momenta of the two outgoing 
photons. The cross section is then given, as a function of
$\tilde{\mathcal{A}}$, by the formula
\beq
  v\sigma_{2\gamma} = \frac{\alpha^2 M^2_{\chi}}{16 \pi^3} \left|\;
\tilde{\mathcal{A}}\; \right|^{2}\;\;\;.  \label{eq:sigmav2g}
\eeq

The total amplitude is implemented in \ds\ as the sum of the contributions 
obtained from four different classes of diagrams:
\begin{eqnarray*}
\tilde{\mathcal{A}}=\tilde{\mathcal{A}}_{f\tilde{f}}+
  \tilde{\mathcal{A}}_{H^+}+\tilde{\mathcal{A}}_{W}+\tilde{\mathcal{A}}_{G},
\end{eqnarray*}
where the indices label the particles in the internal loops, i.e.,
respectively, fermions and sfermions, charged Higgs and charginos, 
W-bosons and charginos, and, in the gauge we chose, charginos and Goldstone 
bosons. For every $\mathcal{A}$ term, real
and imaginary parts are splitted; the full set of analytic formulas are 
given in \cite{lp}, following the notation of \cite{linejk}, where some of 
the contributions were first computed. They are rather lengthy expressions
with non trivial dependences on various combinations of parameters in 
the MSSM. We reproduce here, as an example, the formulas for the diagrams
with W bosons and charginos, which, in  most cases, give the dominant 
contribution to the cross section as discovered in \cite{lp}. The sum over
$\chi^+_i$ includes the two chargino eigenstates:
\begin{eqnarray}
Re\,\tilde{\mathcal{A}}_{W} & = &
\sum_{\chi^+_i} \frac{1}{M^2_{\chi}} \; \left[ 2\;
\frac{\left(a-b\right)\;
  S_{\chi W}}{1+a-b} \;I_{1} \left( a,b \right)+\frac{S_{\chi W}-2
\sqrt{a}\;
  D_{\chi W}}{1-a-b} \;I_{1} \left( a,1 \right) \right. \nonumber \\
   &&\left. + \left( 2\;\frac{S_{\chi W}-2 \sqrt{a}\;D_{\chi
W}}{1-a-b}-
    \frac{3\,S_{\chi W}-4 \sqrt{a}\;D_{\chi W}}{1-b}  \right)\;I_{2}
    \left( a,b \right) \right. \nonumber \\
   && \left.+ \left( \frac{\left(2+b\right)\,S_{\chi W}-4 \sqrt{a}\;
    D_{\chi W}}{1-b} -2\;\frac{\left( 1-a+b \right)\; S_{\chi
W}}{1+a-b}
    \right)\;I_{3}\left( a,b \right) \right]
\end{eqnarray}

\begin{eqnarray}
Im\,\tilde{\mathcal{A}}_{W} & = &
  -\pi\;\sum_{\chi^+_i} \frac{1}{M^2_{\chi}} \; \left( 2\;
   \frac{\left(a-b\right)\;S_{\chi W}}{1+a-b} \right) \cdot \nonumber
\\
  &&\cdot \, \log \left( \frac{1+\sqrt{1-b/a}}{1-\sqrt{1-b/a}} \right)
   \theta \left(1-m^2_{W}\,/\,M^2_{\chi} \right) \label{imw}
\end{eqnarray}
where we defined:
\begin{eqnarray*}
a=\frac{M^2_{\chi^0_1}}{M^2_{\chi^+_i}} &&
b=\frac{m^2_{W}}{M^2_{\chi^+_i}}
\end{eqnarray*}
\begin{eqnarray*}
S_{\chi W}=\frac{1}{2}\;\left(g^L_{W1i}\;
g^{L\,\ast}_{W1i}+g^R_{W1i}\;
  g^{R\,\ast}_{W1i} \right) && D_{\chi W}=\frac{1}{2}\;\left(
g^L_{W1i}\;
  g^{R\,\ast}_{W1i}+g^R_{W1i}\; g^{L\,\ast}_{W1i} \right)\;\; ,
\end{eqnarray*}
and the functions $I_{1}\left( a,b \right)$,
$I_{2}\left( a,b \right)$ and $I_{3}\left( a,b \right)$, which arise
from the loop integrations, are given by:
\begin{eqnarray}
I_{1}\left( a,b \right) = \int_{0}^{1} \frac{d\,x}{x} \; \log \left(
\left|
\frac{4\,a\,x^2-4\;a\,x + b}{b} \right| \right)
\end{eqnarray}
\begin{eqnarray}
I_{2}\left( a,b \right) = \int_{0}^{1} \frac{d\,x}{x} \; \log \left(
\left|
\frac{-a\,x^2+(a+b-1)\,x + 1}{a\,x^2+(-a+b-1)\,x + 1} \right| \right)
\end{eqnarray}
\begin{eqnarray}
I_{3}\left( a,b \right) = \int_{0}^{1} \frac{d\,x}{x} \; \log \left(
\left|
\frac{-a\,x^2+(a+1-b)\,x + b}{a\,x^2+(-a+1-b)\,x + b} \right| \right) .
\end{eqnarray}
$I_{1}\left( a,b \right)$ is the well known three point function that
appears in triangle diagrams; it is an analytic function of
a/b. $I_{2}\left( a,b \right)$ and $I_{3}\left( a,b \right)$ may be
expressed in terms of dilogarithms. In \ds, they are computed in the
integral form as, for any physically interesting value of the
parameters a and b, the integrands are smooth functions of $x$.

The branching ratio for neutralino annihilations into $2\gamma$ is 
typically not larger than 1\%, with the largest values of
$v\sigma_{2\gamma}$, for neutralinos with a cosmologically significant
relic aboundance, in the range $10^{-29}$--$10^{-28}$~cm$^3$s$^{-1}$.
Such values may be large enough for the discovery of this signal
in upcoming measurements; at the same time it should be kept in mind
that very low values for the cross section are feasible as well.


%%%%%%%%%%%%%%%%%%%%%%%%%%%%%%
\subsection{$\chi\chi\to Z\gamma$}

The process of neutralino annihilation into a photon and a Z$^0$ boson
\cite{ub}
\beq
\tilde{\chi}^{0}_{1} + \tilde{\chi}^{0}_{1} \rightarrow \gamma
+ Z^0
\eeq
also gives a nearly
monochromatic line (with a small smearing caused by the finite
  width of the $Z^0$ boson), with  energy
\beq
E_{\gamma} = M_{\chi} - \frac{m_Z^2}{4\,M_{\chi}}.
\eeq

The steps followed in \ds\
to compute the cross section are essentially the same
as those described for the  $2\gamma$ case.
Again the total amplitude is obtained by summing the contribution
from four classes of diagrams and by splitting for each of them 
real and imaginary parts. The analytic formulas were derived in 
\cite{ub}, and are much less compact than those 
obtained for the process of neutralino annihilation into two photons.

The maximum value of $v\sigma_{Z\gamma}$, for neutralinos with a 
cosmologically significant relic aboundance, is around 
$2\cdot 10^{-28}$~cm$^3$s$^{-1}$ and occurs for a nearly
pure Higgsinos. In the heavy mass range, the value of $v\sigma_{Z\gamma}$
reaches a plateau of around $0.6 \cdot 10^{-28}$~cm$^3$s$^{-1}$. This
interesting effect of a non-diminishing cross section with higgsino mass
(which is due to a contribution to the real part of the amplitude)
is also valid for the $2\gamma$ final state in the corresponding limit, 
with a value of $1\cdot 10^{-28}$~cm$^3$s$^{-1}$ \cite{lp}.
Since the gamma-ray background drops rapidly with increasing photon
energy, these processes may be interesting for detecting dark
matter neutralinos near the upper range of allowed neutralino masses.

Whenever the lightest neutralino contain a significant Wino or Higgsino
component the value of $v\sigma_{Z\gamma}$ maybe as large as, or even larger 
than, twice the value of $v\sigma_{2\gamma}$. It is therefore usually
not a good approximation to neglect the $Z\gamma$ state compared to
$2\gamma$.


%%%%%%%%%%%%%%%%%%%%%%%%%%%%%%
\subsection{Gamma rays with continuum energy spectrum}

The advantage with the gamma-ray lines discussed in the previous Sections
is the distinctive spectral signature, which has no plausible astrophysical
counterpart. 

Compared to the monochromatic flux, the gamma-ray flux produced in 
$\pi^0$ decays is much larger but has less distinctive features.
The photon spectrum in the process of a pion decaying into $2\gamma$ is, 
independent of the pion energy, peaked at half of the $\pi^0$ mass, 
about 70~MeV, and symmetric with respect to this peak if plotted in 
logaritmic variables. Of course, this is true both for pions produced in 
neutralino annihilations and, e.g., for those generated by cosmic ray 
protons interacting with the interstellar medium.

When considered together with to the cosmic ray induced Galactic gamma-ray
background, the neutralino induced signal looks like a component analogous
to the secondary flux due to nucleon nucleon interactions: it is
drowned into the Bremsstrahlung component at low energy, while it may 
be the dominant contribution at energies above 1 GeV or so. 
In fact, if the exotic component is indeed significant
an option to disentangle it would be to search for a break in the
energy spectrum at about the neutralino mass, where the line feature
might be present as well: while the maximal energy for a photon emitted 
in neatralino pair annihilations is equal to the neutralino mass,
the component from cosmic ray protons extends to much higher energies,
essentially with the same spectral index as for the proton spectrum
(the role played by the third main background component, 
inverse Compton emission, has still to settled at the time being and
may worsen the problem of discrimitation against background).

Besides this (weak) spectral feature, another way to disentangle
the dark matter signal may be to exploit a directional signature:
data with a wide angular coverage should be analyzed to seach for 
a gamma-ray flux component following the shape and density profile 
of the dark halo, including eventual contributions from clumps.


\subsection{Sources and fluxes}

Given a density distribution of dark matter neutralinos along some line
of sight $l$, the monochromatic gamma-ray flux per unit solid
angle in that direction is:
\beq
{d\Phi_{\gamma}(\psi)\over d\Omega} 
= \frac{N_{\gamma}\;v\sigma_{X^0\gamma}}{8\pi M_\chi^2}
\int_{line\;of\;sight}\rho_{\chi}^2(l)\; d\,l(\psi)\;,
\label{eq:gaflux}
\eeq
where $\psi$ is an angle to label the direction of observation and where 
$N_{\gamma} = 2$ for $\chi\,\chi\rightarrow \gamma\,\gamma$, 
$N_{\gamma} = 1$ for $\chi\,\chi\rightarrow Z\,\gamma$. Analogously,
the gamma-ray flux with continuum energy spectrum is obtained by replacing
$N_{\gamma}\;v\sigma_{X^0\gamma}$ with 
$\sum_f dN_{\gamma}^f/dE\;v\sigma_{f}$, where the sum is over all tree
level final states. Separating the dependence on the dark matter distribution
from the part which is related to values of the cross section and the 
neutralino mass, we rewrite Eq.~(\ref{eq:gaflux}) as:
\beq
{d\Phi_{\gamma}(\psi)\over d\Omega} 
\simeq  9.395 \cdot 10^{-12}\left( \frac{N_{\gamma}\;v\sigma_{X^0\gamma}}
{10^{-29}\ {\rm cm}^3 {\rm s}^{-1}}\right)\left( \frac{10\,\rm{GeV}}
{M_\chi}\right)^2 \cdot 
J\left(\psi\right)\;\rm{cm}^{-2}\;\rm{s}^{-1}\;\rm{sr}^{-1}\;,
\eeq
where we have defined the dimensionless function
\beq
J\left(\psi\right) = \frac{1} {8.5\, \rm{kpc}}
\cdot \left(\frac{1}{0.3\,{\rm GeV}/{\rm cm}^3}\right)^2
\int_{line\;of\;sight}\rho_{\chi}^2(l)\; d\,l(\psi)\;.
\label{eq:jpsi}
\eeq
The relevant quantity for a measurement is, rather than $J\left(\psi\right)$,
the integral of $J\left(\psi\right)$ over the solid angle given by the
angular acceptance $\Delta\Omega$ of a detector which is pointing in
the direction $\psi$. Defining:
\begin{equation}
\langle\,J\left(\psi\right)\rangle_{\Delta\Omega}
= \frac{1}{\Delta\Omega} \int_{\Delta\Omega} d\Omega^{\prime}
J\left(\psi^{\prime}\right)\;, 
\label{eq:jave}
\end{equation}
the flux measured in a detector is:
\begin{equation}
\Phi_{\gamma}(\psi, \Delta\Omega) = 9.395 \cdot 10^{-12}
\left(\frac{N_{\gamma}\;v\sigma_{X^0\gamma}}
{10^{-29}\ {\rm cm}^3 {\rm s}^{-1}}\right)
\left( \frac{10\,\rm{GeV}}{M_\chi}\right)^2 
\langle\,J\left(\psi\right)\rangle_{\Delta\Omega}\times\Delta\Omega 
\;\rm{cm}^{-2}\;\rm{s}^{-1}\; \rm{sr}^{-1}\; .
\label{signal}
\end{equation}
Finally, the formalism we introduced can be used also to estimate
the flux in the simple case of a single source which, for the given 
detector, can be approximated as point-like (see examples below). 
If such 
source is in the direction $\psi$ at a distance $d$, Eq.~(\ref{eq:jave})
becomes:
\begin{equation}
\langle\,J\left(\psi\right)\rangle_{\Delta\Omega}
= \frac{1} {8.5\, \rm{kpc}}
\cdot \left(\frac{1}{0.3\,{\rm GeV}/{\rm cm}^3}\right)^2
\cdot \frac{1}{d^2} \cdot \frac{1}{\Delta\Omega}
\int d^3r \;\rho_{\chi}^2(\vec{r}) 
\end{equation}
where the integral is over the extention of the source (much smaller
than $d$).

Several targets have been discussed as sources of gamma-rays from the
annihilation of dark matter particles. An obvious source is the dark
halo of our own galaxy~\cite{galga} and in particular the Galactic center,
as the dark matter density profile is expected, in most models, to be 
picked towards it, possibly with huge enhancements close to te central 
black hole. The Galactic center is an ideal target for both ground-
and space-based gamma-ray telescopes. As satellite experiments have 
much wider field of views and will provide a full sky coverage,
they will test the hypothesis of gamma-rays emitted in clumps of dark 
matter which may be present in the 
halo~\cite{clumpyga,gahalo,clumpy,clumpybeg}. 
Another possibility which has been considered is the case of 
gamma-ray fluxes from external nearby galaxies~\cite{extergal}. 
Furthermore, it has 
been proposed to search for an extragalactic flux originated by all 
cosmological annihilations of dark matter 
particles~\cite{extragal,extragalbeu}.

\ds\ is suitable to compute the gamma-ray flux from all these (and possibly 
other) sources. Two cases are fully included in the package:
assuming neutralinos are smoothly distributed in the Galactic halo
with $\rho_{\chi}$ equal to the dark matter density profile, in 
\ds\ Eq.~\ref{eq:jave} is computed for a specified halo profile and 
any given $\psi$ and $\Delta\Omega$~\cite{lpj}. 
The second option deals with the
case of a portion of dark matter being in the form of clumps, each of
which is treated as a non-resolvable source in the detector, distributed
in the Galaxy according to a probability distribution which
follows the dark matter density profile (see \cite{clumpy}
for details). It is straightforward to extend this to all other 
astrophysical sources; in case of cosmological sources one has just 
to pay attention to include redshift effects and absoption on starlight 
and infrared background, see~\cite{extragalbeu}.



%%%%%%%%%%%%%%%%%%%%%%%%%%%%%%
\section{Neutrinos from halo -- theory}

Usually, the flux of neutrinos from annihilation of neutralinos in
the Milky Way halo is too small to be detectable, but for some clumpy
or cuspy models, it might be detectable. The calculation of the
neutrino-flux follows closely the calculation of the continuous gamma
ray flux, with the main addition that neutrino interactions close to
the detector are also included. Hence, both the neutrino flux and the
neutrino-induced muon flux can be obtained. The neutrino to muon
conversion rate in the Earth can also be obtained.
