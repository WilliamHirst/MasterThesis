%%%%%%%%%%%%%%%%%%%%%%%%%%%%%%%%%%%%%%%%%%%%%%%%%%%%%%%%%%%%%%%%%%%%
\section{Positrons from the halo -- theory}

Neutralino annihilations in the halo will give rise to positrons either
directly or from decaying mesons in hadron jets.  We thus expect to get both
monochromatic positrons (at an energy of $m_{\chi}$) from direct annihilation
into $e^{+}e^{-}$ and continuum positrons from the other annihilation
channels. In general, the branching ratio for annihilation directly into $e^+
e^-$ is rather small due to the helicity-flip suppression $\propto m_e$ for
S-wave annihilation in the halo, but for some classes of models one can still
obtain a large enough branching ratio for the line to be observable.

The computation of the positron flux from neutralino annihilations used in \ds\
resembles the calculation of the antiproton flux, with some important changes
due to other mechanisms of energy loss with different energy dependence.  Due
to the fact that energy losses for positrons are more rapid than for
antiprotons, the computed signal is less sensitive to the global structure of
the dark matter halo.  (On the other hand, it is more sensitive to possible
local sources of background, such as supernova remnants etc.)

The calculation of the neutralino-induced positron flux performed in \ds\
follows the analysis in \cite{baltz}.  For continuum positrons, we have again
simulated the decay and/or hadronization with the Lund Monte Carlo {\sc Pythia}
as described in section \ref{sec:mcsim}.
%For any given MSSM model, the positron spectrum is
%\begin{equation}
%  \frac{d\phi}{dE} =
%  \left. \frac{d\phi}{dE} \right|_{\rm cont.} +
%  \left. \frac{d\phi}{dE} \right|_{\rm line}
%  = \sum_{F \neq e^{+}e^{-}} B_{F}
%  \left. \frac{d\phi}{dE} \right|_{F} + B_{e^{+}e^{-}}\,
%  \delta(E-m_\chi),
%\end{equation}
%in units of $e^{+}/{\rm annihilation}$ where $B_{F}$ is the branching ratio
%into a given final state $F$ and $\left.d\phi/dE\right|_{F}$ is the
%spectrum of positrons from annihilation channel $F$\@.
We have included all two-body final states in \ds\ (except the three lightest
quarks which are completely negligible) at tree level and the $Z\gamma$
\cite{bub} and $gg$ \cite{lp} final states which arise at one-loop level.


%%%%%%%%%%%%%%%%%%%%%%%%%%%%%%%%%%%%%%%%%%%%%%%%%%%%%%%%%%%%%%%%%%%%%%
\subsection{Propagation and the interstellar flux}

We consider a standard diffusion model, somewhat less
sophisticated than in the case of antiprotons, for the propagation of
positrons in the galaxy.  Charged particles move under the influence
of the galactic magnetic field.  For the relevant energies
  the magnetic gyroradii of the particles are quite small.
However, the magnetic field is tangled, and even with small gyroradii,
particles can jump to nearby field lines which will drastically alter
their courses.  This entire process can be modeled as a random walk,
which can be described by a diffusion equation.

Positron propagation is complicated by the fact that light particles lose
energy quickly due to inverse Compton and synchrotron processes.  Diffuse
starlight and the Cosmic Microwave Background (CMB) both contribute appreciably
to the energy loss rate of high energy electrons and positrons via inverse
Compton scattering.  Electrons and positrons also lose energy by synchrotron
radiation as they spiral around the galactic magnetic field lines.

Our detailed treatment of positron diffusion employed in \ds\ is as follows.
First define a dimensionless energy variable $\varepsilon=E/(1\;{\rm GeV})$,
and the dimensionless mass $\mct=m_\chi/(1\;{\rm GeV})$\@.  The
standard diffusion-loss equation for the space density of cosmic rays per unit
energy, $dn/d\varepsilon$, is then given by
\begin{equation}
   \frac{\partial}{\partial t}\frac{dn}{d\varepsilon}=\vec{\nabla}\cdot
   \left[K(\varepsilon,\vec{x})\vec{\nabla}\frac{dn}{d\varepsilon}\right]+
   \frac{\partial}{\partial \varepsilon}\left[b(\varepsilon,\vec{x})
   \frac{dn}{d\varepsilon}\right]+Q(\varepsilon,\vec{x}),
   \label{eq:diffloss}
\end{equation}
where $K$ is the diffusion constant, $b$ is the energy loss rate and $Q$ is the
source term.  We consider only steady state solutions, setting the left hand
side of Eq.\ (\ref{eq:diffloss}) to zero.

We assume that the diffusion constant $K$ is constant in space throughout a
``diffusion zone'', but it may vary with energy.  At energies above a few GeV,
we can represent the diffusion constant as a power law in energy \cite{wlg},
\begin{equation}
   K(\varepsilon)=K_0\varepsilon^\alpha\approx
   3\times 10^{27}\varepsilon^{0.6}{\rm cm}^2\;{\rm s}^{-1}.
   \label{eq:KA}
\end{equation}
However, at energies below about 3 GeV, there is a cutoff in the diffusion
constant that can be modeled as
\begin{equation}
   K(\varepsilon)=K_0\left[C+\varepsilon^\alpha\right]\approx
   3\times 10^{27}\left[3^{0.6}+\varepsilon^{0.6}\right]
   {\rm cm}^2\;{\rm s}^{-1}.
   \label{eq:KB}
\end{equation}
Both of these models for the diffusion constant can be used in \ds\ but the
second expression is the default.\footnote{In fact, a third option can be
   chosen in \ds\ as well, employing the propagation model of \cite{kamturner}.}
The function $b(\varepsilon)$ represents the (time) rate of energy loss.  We
allow energy loss via synchrotron emission and inverse Compton scattering.  The
rms magnetic field in the diffusion zone is about $3~\mu$G, an energy density
of about 0.2 eV cm$^{-3}$\@. We allow inverse Compton scattering on both the
cosmic microwave background and diffuse starlight, which have energy densities
of 0.3 and 0.6 eV cm$^{-3}$ respectively.  These two processes combined give an
energy loss rate \cite{energy-loss} \def\erf{\mathop{\rm erf}}
\begin{equation}
   b(\varepsilon)_{e^\pm}=\frac{1}{\taue}\varepsilon^2
   \approx 10^{-16}\varepsilon^2\;{\rm s}^{-1},
   \label{eq:bofe}
\end{equation}
where we have neglected the space dependence of the energy loss rate.  Lastly,
the function $Q$ is the source of positrons in units of cm$^{-3}$ s$^{-1}$\@.

We model the diffusion zone as a slab of thickness $2L$\@.  We fix $L$ to be 3
kpc, which fits observations of the cosmic ray flux \cite{wlg}.  We impose free
escape boundary conditions, namely that the cosmic ray density drops to zero on
the surfaces of the slab, which we let be the planes $z=\pm L$\@.  We neglect
the radial boundary usually considered in diffusion models.  This is justified
when the sources of cosmic rays are nearer than the boundary, as is usually the
case with galactic sources.  We will see that the positron flux at Earth,
especially at higher energies, mostly originates within a few kpc and hence
this approximation is well justified in our case. (This is different from the
case of antiprotons, where the flux from the Galactic center can be very
important at the Earth's location \cite{pbar}.)
%From \cite{baltz} we find the solution to the diffusion equation
%\begin{equation}
%  \frac{dn}{d\varepsilon}=\frac{\taue}{\varepsilon^2}\int
%  ^\infty_\varepsilon d\varepsilon'\,\int d^3\vec{x}\,'\,G_{2L}
%  \left(v(\varepsilon)-v(\varepsilon'),\vec{x}-\vec{x}\,'\right)
%  Q(\varepsilon',\vec{x}\,').
%  \label{eq:dnde}
%\end{equation}
%where the function $G_{2L}$ is the Green's function for a slab with thickness
%$2L$, satisfying the free escape boundary conditions\@.
%The free space Green's function is given by
%\begin{equation}
%G_{\rm free}(v-v',\vec{x}-\vec{x}\,')=\left(4\pi K_0\taue\deltav\right)^{-3/2}
%\exp\left(-\frac{\left(\vec{x}-\vec{x}\,'\right)^2}{4K_0\taue\deltav}\right)
%\theta(\deltav).
%\end{equation}
%With image charges $x_n=x$, $y_n=y$, $z_n=2Ln+(-1)^nz$, the full Green's
%function is
%\begin{equation}
%G_{2L}(v-v',\vec{x}-\vec{x}\,')=\sum_{n=-\infty}^\infty
%(-1)^nG_{\rm free}(v-v',\vec{x}-\vec{x}_n').
%\end{equation}
The spatial part of the Green's function is performed once, independently of
the supersymmetric model, yielding an energy dependent diffusion time
\begin{equation}
\taud(\varepsilon,\varepsilon')=\frac{1}{4K_0\deltav}
\sum^\infty_{n=-\infty}\sum_\pm\erf\left(\frac{(-1)^nL+2Ln\pm z}
{\sqrt{4K_0\taue\deltav}}\right)\times\hspace{1in}
\end{equation}
\begin{displaymath}
\int_0^\infty dr'\,r'f(r')
I_0\left(\frac{2rr'}{4K_0\taue\deltav}\right)
\exp\left(\frac{r^2+r'^2}{4K_0\taue\deltav}\right)
\theta(\deltav),
\end{displaymath}
where $f(r)$ is the effective halo profile squared, and the expression is
evaluated for $r$ and $z$ appropriate for the observer.  The function
$v(\varepsilon)$ depends on the diffusion model: the default model has
$v(\varepsilon)=C/\varepsilon+\varepsilon^{\alpha-1}/(1-\alpha)$.  The function
$\taud$ is the effective diffusion time for particles emitted at energy
$\varepsilon'$ and observed at energy $\varepsilon$.  Of course if the
observed energy is larger than the emitted energy, $\taud=0$.  The spatial
integrand is smooth, and is computed for a range of values, equally spaced in
$\log(\deltav)$\@ for use in \ds.  Likewise, the series of image charges used
in the Green's function converges rapidly, and with the range of
$\deltav$ values we are concerned with, need not be taken past $n=\pm 10$\@.
The total positron spectrum is now given by
\begin{equation}
   \frac{dn}{d\varepsilon}=n_0^2\sigv\frac{1}{\varepsilon^2}\Bigg\{B_{\rm
   line}\taud
   \left(\varepsilon,\mct\right)+\int_{\varepsilon}^{\mct}d\varepsilon'\,
   \left.\frac{d\phi}{d\varepsilon}\right|_{\rm
   cont.}\taud(\varepsilon,\varepsilon') \Bigg\},
\label{eq:dndefinal}
\end{equation}
where $B_{\rm line}$ is the branching ratio directly to $e^+e^-$, and
$d\phi/d\varepsilon|_{\rm cont.}$ is the spectrum of continuum positrons per
annihilation.  Remembering that this is an expression for the number density of
positrons, the flux is given by
\begin{equation}
   \frac{d\Phi}{d\varepsilon} = \frac{\beta c}{4 \pi} \frac{dn}{d\varepsilon}
   \simeq \frac{c}{4 \pi} \frac{dn}{d\varepsilon},
\end{equation}
where $\beta c$ is the velocity of a positron of energy
$\varepsilon$\@.  For the energies we are interested in, $\beta c
\simeq c$ is a very well justified approximation.

%The integral is performed in the variable $v$\@ (see \cite{baltz} 
%for details).
%Replacing the Jacobian factor $w(v)=-d\varepsilon/dv$, one finds
%\begin{equation}
%\frac{dn}{d\varepsilon}=n_0^2\sigv\frac{1}{\varepsilon^2}\Bigg\{B_{\rm line}
%\taud\left(\varepsilon,\mct\right)+\int_{v(\mct)}^{v(\varepsilon)}dv'\,
%w(v')\left.\frac{d\phi}{d\varepsilon}\right|_{\rm cont.}
%\left(\varepsilon(v')\right)
%\taud\left(\varepsilon,\varepsilon(v')\right)\Bigg\}.
%\end{equation}


%%%%%%%%%%%%%%%%%%%%%%%%%%%%%%
\subsection{Solar modulation}

Again there is a complication in that interactions with the solar wind and
magnetosphere, solar modulation, alter the spectrum.  This can be neglected at
high energies, but at energies below about 10 GeV, the effects of solar
modulation become important. However, its effects can be reduced by considering
the positron fraction, $e^+/(e^+ + e^-)$, instead of the absolute positron
fluxes. This is possible to obtain from \ds, since included in the package is
an estimate of the background $e^+$ and $e^-$ flux taken from
\cite{MoskStrong98}.



