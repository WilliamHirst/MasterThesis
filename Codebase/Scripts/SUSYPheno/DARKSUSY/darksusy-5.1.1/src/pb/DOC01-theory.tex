%%%%%%%%%%%%%%%%%%%%%%%%%%%%%%%%%%%%%%%%%%%%%%%%%%%%%%%%%%%%%%%%%%%%
%%%%%%%%%%%%%%%%%%%%%%%%%%%%%%
\section{Antiprotons -- theory}

Neutralinos can annihilate each other in the halo producing leptons,
quarks, gluons, gauge bosons and Higgs bosons. The quarks, gauge
bosons and Higgs bosons will decay and/or form jets that will give
rise to antiprotons (and antineutrons which decay shortly to
antiprotons). Since antiprotons are not very abundant in the Universe,
this could in principle be a good signature for supersymmetric dark
matter. However, the cosmic rays (mainly protons) may produce
secondary antiprotons in collisions with the interstellar medium,
giving an important background. It was hoped that the difference in
kinematics between such secondary antiprotons and the primary ones
generated in neutralino annihilations would give an unambiguous
signature at low antiproton energy. However, recent calculations
indicate that other effects spoil this picture to a large degree
\cite{pbar,gaisserpbar}. It still remains true, however, that present
measurements and upper limits to the antiproton flux may be used as a
constraint to rule out some MSSM configurations with large rates.

Unfortunately, there is a larger uncertainty in
limits thus obtained than, for example, for the signal from neutrinos
from the Earth and Sun. This is due to the severe astrophysical
uncertainties about the phase space structure of the dark matter
halo, in particular the density profile towards the Galactic center.
This uncertainty will plague all indirect detection signals from
the halo: antiprotons, positrons and gamma-rays. Therefore, the
limits that can be put generally involve a combination of
MSSM and halo model parameters, and are therefore of limited use
constraining the MSSM alone.

At tree level the relevant final states for $\bar{p}$ production
are $q \bar{q}$, $\ell \bar{\ell}$, $W^{+} W^{-}$, $Z^{0} Z^{0}$,
$W^{+} H^{-}$, $Z H_{1}^{0}$, $Z H_{2}^{0}$, $H_{1}^{0} H_{3}^{0}$ and
$H_{2}^{0} H_{3}^{0}$. We have included in \ds\ all the heavier quarks ($c$,
$b$ and $t$), gauge bosons and Higgs boson final states.
  In addition, we have included the $Z \gamma$ (\cite{ub}) and
the 2 gluon (\cite{bua}; \cite{lp}) final states which occur at one
loop-level.


The hadronization and/or decay of all final states (including) gluons
is simulated with {\sc Pythia} as described in section \ref{sec:mcsim}.
A word of caution should be raised, however, that
antiproton data is not very abundant, in particular not at the
lowest antiproton lab energies which tend to dominate the signal.
Therefore an uncertainty in normalization, probably of the order of a
factor 2, cannot be excluded at least in the low energy region.

%%%%%%%%%%%%%%%
\subsection{The Antiproton Source Function}

The source function $Q_{\bar{p}}^{\chi}$ gives the number of antiprotons
per unit time, energy and volume element produced in annihilation
of neutralinos locally in space. It is given by
\begin{equation}
   Q_{\bar{p}}^{\chi}(T,\vec{x}\,)=
   (\sigma_{\rm ann}v)
   \left(\frac{\rho_{\chi}(\vec{x}\,)}{m_{\chi}}\right)^{2}
   \sum_{f}^{}\frac{dN^{f}}{dT}B^{f}
   \label{sourcefkn}
\end{equation}
where $T$ is the $\bar{p}$ kinetic energy. For a given annihilation
channel $f$, $B^{f}$ and $dN^{f} / dT$ are, respectively, the branching ratio
and the fragmentation function, and $(\sigma_{\rm ann}v)$ is the
annihilation rate at $v=0$ (which is very good approximation since the
velocity of the neutralinos in the halo is so low).
As dark matter neutralinos annihilate in pairs, the source function is
proportional to the square of the neutralino number density
$n_{\chi} = \rho_{\chi} / m_{\chi}$.
Assuming that most of the dark matter in the Galaxy is made up of neutralinos
and that these are smoothly distributed in the halo, one can directly relate
the neutralino number density to the dark matter density profile in the
galactic halo $\rho$.
%
Although what is implemented is a smooth distribution of dark
matter particles in the halo, an extension to a clumpy distribution is
potentially interesting as well (\cite{clumpy}; \cite{pieroclumpy}).

%%%%%%%%%%%%%%%%%%%%%%%%%%%%%%
\subsection{Propagation model}
\label{sec:prop}

In the absence of a well established theory to describe the
interactions of charged particles with the magnetic field of the
Galaxy and the interstellar medium, the propagation of cosmic rays has
generally been treated by postulating a semiempirical model and
fitting the necessary set of unknown parameters to available data.  A
common approach is to use a diffusion approximation defined by a
transport equation and an appropriate choice of boundary
conditions~(see e.g.\ \cite{Berezinskii}; \cite{Gaisserbook} and
references therein).

We have chosen to compute the propagation of cosmic rays in the Galaxy by
means of a
  transport equation of the diffusion type
  (see \cite{Berezinskii}; \cite{Gaisserbook}).
In the case of a stationary solution, the number density $N$ of a
stable cosmic ray
species whose distribution of sources is defined by the function of
energy and space $Q(E,\vec{x})$, is given by:
\begin{equation}
\frac{\partial{N(E,\vec{x})}}{\partial{t}} = 0 = \nabla \cdot
\left(D(R,\vec{x})\,\nabla N(E,\vec{x})\right)
- \nabla \cdot \left( \vec{u}(\vec{x})\,N(E,\vec{x}) \right)
- p(E,\vec{x})\,N(E,\vec{x}) + Q(E,\vec{x}) \;\;.
\label{eq:diff}
\end{equation}

On the right hand side
of Eq.~(\ref{eq:diff}) the first term implements the diffusion approximation
for a given diffusion coefficient $D$, generally assumed to be a function of
rigidity $R$, while the second term describes a large-scale convective
motion of velocity $\vec{u}$. The third term is added to take into account
losses due to to collisions with the interstellar matter.
It is a very good approximation to include in this term only the interactions
with interstellar hydrogen, in this case
$p$ is given by:
\begin{equation}
p(E,\vec{x}) = n^H(\vec{x}) \, v(E) \, \sigma^{\rm in}_{cr\,p}(E)
\end{equation}
where $n^H$ is the hydrogen number density in the Galaxy, $v$ is the
velocity of the cosmic ray particle considered `$cr$', while
$\sigma^{\rm in}_{cr\,p}$ is the inelastic cross section for $cr$-proton
collisions.


The propagation region is assumed to have a cylindrical symmetry: the
Galaxy is split into two parts, a disk of radius $R_h$ and height
$2\cdot h_g$, where most of the interstellar gas is confined, and a
halo of height $2\cdot h_h$ and the same radius. We assume that the
diffusion coefficient is isotropic with possibly two different values
in the disk and in the halo, reflecting the fact that in the disk
there may be a larger random component of the magnetic fields.
The spatial dependence is then:
\begin{equation}
D(\vec{x}) = D(z) = D_g \, \theta(h_g - |z|) + D_h \, \theta(|z| - h_g)\;\;.
\end{equation}
Regarding the rigidity dependence,
we consider the same functional form as in
\cite{Chardonnet} and \cite{bottinolast}:
\begin{equation}
D_l(R) = D_l^0 \left(1+\frac{R}{R_0}\right)^{0.6}
\label{eq:diffco}
\end{equation}
where $l=g,h$.

The convective term has been introduced in Eq.~(\ref{eq:diff}) to describe the
effect of particle motion against the wind of cosmic rays leaving the disk,
assuming a galactic wind of velocity
\begin{equation}
\vec{u}(\vec{x}) = \left(0, 0, u(z)\right)
\end{equation}
where
\begin{equation}
u(z)= {\rm sign}(z) \, u_h \, \theta(|z| - h_g)\;\;.
\end{equation}
An analytic solution is possible also in the case of a linearly increasing
wind~(\cite{pieroclumpy}).
The distribution of gas in the Galaxy is for convenience assumed
to have the very simple $z$ dependence
\begin{equation}
n^H(\vec{x}) = n^H(z)
= n_g^H \, \theta(h_g - |z|) + n_h^H \, \theta(|z| - h_g)
\end{equation}
where $n_h \ll n_g$ (in practice, $n_h=0$ is taken) and an average
in the radial direction is performed.

As boundary condition, it is usually assumed that cosmic rays can escape
freely at the border of the propagation region, i.e.\
\begin{equation}
N(R_h,z) = N(r,h_h) = N(r,-h_h) = 0
\label{eq:bound}
\end{equation}
as the density of cosmic rays is assumed to be negligibly small in the
intergalactic space.

The cylindrical symmetry and the free escape at the
boundaries makes it possible
to solve in \ds\
the transport equation expanding the number density distribution $N$
in a Fourier-Bessel series:
\begin{equation}
N(r,z,\theta) = \sum_{k=0}^{\infty} \,\sum_{s=1}^{\infty} \;
J_k \left(\nu_s^k \frac{r}{R_h}\right) \cdot
\left[ M_s^k(z) \cos(k\theta) + \tilde{M}_s^k(z) \sin(k\theta)\right] \\
\end{equation}
which automatically satisfies the boundary condition at $r=R_h$,
$\nu_s^k$ being the $s$-th zero of $J_k$ (the Bessel function of the
first kind  and of order $k$).
In the same way the source function can be expanded as:
\begin{equation}
Q(r,z,\theta) = \sum_{k=0}^{\infty} \,\sum_{s=1}^{\infty} \;
J_k \left(\nu_s^k \frac{r}{R_h}\right) \cdot
\left[ Q_s^k(z) \cos(k\theta) + \tilde{Q}_s^k(z) \sin(k\theta)\right]
\end{equation}
where
\begin{equation}
Q_s^k(z) = \frac{2}{{R_h}^2\,{J_{k+1}}^2(\nu_s^k)}
\int\limits_0^{R_h} dr^{\prime} \;r^{\prime}
J_k \left(\nu_s^k \frac{r^{\prime}}{R_h}\right)
\frac{1}{\alpha_k\,\pi}
\int\limits_{-\pi}^{\;\pi} d\theta^{\prime} \; \cos(k\theta^{\prime})
\,Q(r^{\prime},z,\theta^{\prime})\;\;.
\end{equation}
The equation relevant for the propagation in the $z$ direction is \cite{pbar}:
\begin{equation}
\frac{\partial}{\partial{z}} D(z) \frac{\partial}{\partial{z}} M_s^k(z)
- D(z) \left(\frac{\nu_s^k}{R_h}\right)^2 M_s^k(z)
- \frac{\partial}{\partial{z}} \left(u(z)\,M_s^k(z) \right)
- p(z) M_s^k(z) + Q_s^k(z) =0 \;\;.
\end{equation}
For $-h_g \leq z \leq h_g$ the solution is given by:
\begin{equation}
M_s^k(z) = M_s^k(0) \cosh(\lambda_g^{ks} z) - \frac{1}{D_g\, \lambda_g^{ks}}
\int\limits_{0}^{z} \;dz^{\prime}
\sinh\left(\lambda_g^{ks} (z-z^{\prime})\right) Q_s^k(z^{\prime})
\label{eq:zdep}
\end{equation}
where
\begin{eqnarray}
M_s^k(0) \ & = & \ \frac{1}{\cosh(\lambda_g^{ks} h_g)}
\left\{\frac{I_H}{\sinh\left(\lambda_h^{ks} (h_h-h_g)\right)}+
\frac{D_h\,I_{GS}}{D_g\, \lambda_g^{ks}} \,\left[\gamma_h +
\lambda_h^{ks} \coth\left(\lambda_h^{ks} (h_h-h_g)\right)\right]
+I_{GC}\right\} \nonumber \\
& & \times \left[D_g\,\lambda_g^{ks}\,\tanh\left(\lambda_g^{ks} h_g\right)
+ D_h\,\gamma_h + D_h\,\lambda_h^{ks}\,
\coth\left(\lambda_h^{ks} (h_h-h_g)\right) \right]^{-1}
\label{eq:halodiff}
\end{eqnarray}
with
\begin{eqnarray}
\lambda_g^{ks} = \sqrt{\left({\nu_s^k\over R_h}\right)^2 +
\frac{n_g^H v \sigma^{\rm in}_{cr\,p}}{D_g}} \;,\;\;\;\;\
\lambda_h^{ks} = \sqrt{\left({\nu_s^k\over R_h}\right)^2 +
\frac{n_h^H v \sigma^{\rm in}_{cr\,p}}{D_h} + {\gamma_h}^2}\;,\;\;\;\; \
\gamma_h = \frac{u_h}{2\,D_h}
\label{eq:halodiff2}
\end{eqnarray}
and
\begin{eqnarray}
I_H & = & \int\limits_{h_g}^{h_h} \;dz^{\prime} \,
\sinh\left(\lambda_h^{ks} (h_h-z^{\prime})\right) \,
\exp\left(\gamma_h (h_g-z^{\prime})\right)
\cdot \frac{Q_s^k(z^{\prime}) + Q_s^k(-z^{\prime})}{2} \nonumber \\
I_{GS} & = & \int\limits_{0}^{h_g} \;dz^{\prime} \,
\sinh\left(\lambda_g^{ks} (h_g-z^{\prime})\right)
\cdot \frac{Q_s^k(z^{\prime}) + Q_s^k(-z^{\prime})}{2} \nonumber \\
I_{GC} & = & \int\limits_{0}^{h_g} \;dz^{\prime} \,
\cosh\left(\lambda_g^{ks} (h_g-z^{\prime})\right)
\cdot \frac{Q_s^k(z^{\prime}) + Q_s^k(-z^{\prime})}{2} \;.
\label{eq:int}
\end{eqnarray}

In \ds\ we have also as an option included the propagation models by
Chardonnay et al.\ \cite{chardonnay} and Bottino et al.\
\cite{bottinopbar}.

%%%%%%%%%%%%%%%%%%%%%%%%%%%%%%
\subsection{Solar Modulation}

A complication when comparing predictions of a theoretical
model with data on cosmic rays taken at Earth is given by the solar
modulation effect. During their propagation from the interstellar
medium through the solar system, charged particles are affected by the
solar wind and tend to lose energy. The net result of the modulation
is a shift in energy between the interstellar spectrum and the
spectrum at the Earth and a substantial depletion of particles with
non-relativistic energies.

The simplest way to describe the phenomenon is the analytical force-field
approximation by Gleeson \& Axford \cite{GleesonAxford} for a spherically
symmetric model. The prescription of
this effective treatment is that, given an interstellar flux at the
heliospheric boundary, $d\Phi_{\rm b}/dT_{\rm b}$, the flux at the Earth
is related to this by
\begin{equation}
   \frac{d\Phi_{\oplus}}{dT_\oplus}(T_\oplus) = \frac{p_\oplus^2}{p_{\rm
   b}^2} \frac{d\Phi_{\rm b}}{dT_{\rm b}} (T_{\rm b})
\end{equation}
where the energy at the heliospheric boundary is given by
\begin{equation}
   E_{\rm b} = E_\oplus + |Ze|\phi_F
\end{equation}
and $p_{\otimes}$ and $p_{\rm b}$ are the momenta at the Earth and
the heliospheric boundary respectively.
Here $e$ is the absolute value of the electron charge and $Z$ the
particle charge in units of $e$ (e.g.\ $Z=-1$ for antiprotons).

An alternative approach is to solve numerically the propagation equation
of the spherically symmetric model~(\cite{fisk}): the solar modulation
parameter one has to introduce with this method roughly corresponds to
$\phi_F$ as given above. When computing solar modulated antiproton fluxes,
the two treatments seem not to be completely equivalent in the low energy
regime. Keeping this in mind, we have
anyway
implemented the force field approximation in \ds\, avoiding the
CPU time-consuming problem of
having to solve a partial differential equation for each
supersymmetric model.


