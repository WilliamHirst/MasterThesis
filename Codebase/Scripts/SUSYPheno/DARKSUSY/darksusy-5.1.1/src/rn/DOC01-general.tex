%%%%%%%%%%%%%%%%%%%%%%%%%%%%%%%%%%%%%%%%%%%%%%%%%%%%%%%%%%%%%%%%%%%%

\section{Relic density of neutralinos}

The relic density routines in \codeb{src/rd} solve the Boltzmann
equation for any cold dark matter particle and it is up to us to tell
it what kind of particles that can participate in coannihilations and
what the effective annihilation rate is. This set-up for neutralino
dark matter is done in \codeb{dsrdomega}. This routine is therefor the
main routine the user should call, when the relic density of
neutralinos is wanted. 

What it does internally is the following:
\begin{itemize}
  \item It determines which particles that can coannihilate (based on
    their mass differences) and puts these particles into a common
    block for the annihlation rate routines (\codeb{dsanwx}) and an
    array for the relic density routines. The relic density routines
    need to know their masses and internal degrees of freedom.
  \item It checks where we have resonances and thresholds and adds
    these to an array, which is passed to the relic density
    routines. The relic density routines then use this knowledge to
    make sure the tabulation of the cross section and the integrations
    are performed correctly at these difficult points.
  \item It then calls the relic density routines to calculate the
    relic density.
\end{itemize}

The returned value is $\Omega_\chi h^2$.
