%%%%%%%%%%%%%%%%%%%%%%%%%%%%%%%%%%%%%%%%%%%%%%%%%%%%%%%%%%%%%%%%%%%%%%
\section{Kinetic decoupling and microhalos (mh) -- theory}
\label{sec:mh}

Even after \emph{chemical decoupling}, which sets the DM relic density (see Section 
\ref{sec:Boltzmann}), DM  frequently scatters with the very abundant 
standard model particles and thereby stays in local thermal equilibrium with the 
heat bath until the temperature has dropped by another factor of between 10 and
 a few 1000; after \emph{kinetic decoupling}, even these scattering events cease and DM 
no longer interacts with standard model particles. Inhomogeneities 
in the DM density can only develop after this has happenened, and the DM particles have
sufficiently cooled down so that free streaming becomes negligible.
The scale of kinetic decoupling can therefore directly be translated 
into a cutoff in the power spectrum of (dark) matter fluctuations and thus the size of the 
smallest (at least when not taking into account primordial black holes) gravitationally 
bound objects in the universe.



\subsection{Kinetic decoupling}

The process of kinetic decoupling is governed by the full Boltzmann equation for the WIMP 
distribution function $f(\mathbf{x},\mathbf{p})$, which in 
a flat Friedmann-Robertson Walker spacetime reads
\begin{equation}
  \label{eq:fullboltz}
  E(\partial_t-H\,\mathbf{p}\cdot\nabla_\mathbf{p})\,f=C[f]\,,
\end{equation}
where $C[f]$ is known as the collision term.
The Boltzmann equation quoted in Eq.~(\ref{eq:Boltzmann}) for the description of 
the (chemical) DM freeze-out process is actually just the first moment of 
this expression, i.e.~obtained by integrating it over $\int d^3p$ (after dividing it by $E$). 
As was realized int \cite{Bringmann:2006mu,Bringmann:2009vf}, kinetic decoupling can be decribed to a very high precision 
by considering, instead, the \emph{second} moment of Eq.~(\ref{eq:fullboltz}). For this purpose,
one introduces the WIMP "temperature" $T_\chi$,
\begin{equation}
  \int\frac{d^3p}{(2\pi)^3}\mathbf{p}^2f(\mathbf{p})\equiv3\,m_\chi T_\chi n_\chi\,,
\end{equation}
as a parameter that characterizes the deviation from thermal equilibrium (for which 
$T_\chi=T$ holds). Multiplying Eq.(\ref{eq:fullboltz}) by $\mathbf{p}^2/E$, integrating it 
over $\mathbf{p}$ and keeping only the leading order terms in $\mathbf{p}^2/m_\chi^2$ 
then results in \cite{Bringmann:2009vf}
\begin{equation}
  \label{eq:boltz2}
  \left(\partial_t+5H\right)T_\chi=2\,m_\chi\, c(T)\left(T-T_\chi\right)\,,
\end{equation}
where
\begin{equation}
  \label{eq:cTdef}
  c(T) =  \sum_i\frac{g_\mathrm{SM}}{6(2\pi)^3m_\chi^4T} \int dk\,k^5 \omega^{-1}\,
        g^\pm\left(1\mp g^\pm\right)\mathop{\hspace{-11ex}
        \overline{\left|\mathcal{M}\right|}^2_{t=0}}_{\hspace{4ex}s=m_\chi^2+
                   2m_\chi\omega+m_i^2}\,.
\end{equation}
In the above expression, the sum runs over all standard model scattering partners with 
4-momentum $(\omega,\mathbf{k})$ and (thermal) distribution $g(\omega)$ (note, however, that no 
assumption has been made about the form of the WIMP distribution function $f$).

For practical purposes, one may now further introduce
\begin{eqnarray}
   x &\equiv& m_\chi/T\,,\\
   y &\equiv& m_\chi g_{\rm eff}^{-1/2} T_\chi/T^2\,,
\end{eqnarray}
and bring Eq.~(\ref{eq:boltz2}) into a form that can more easily be integrated numerically:
\begin{equation}
  \label{eq:process}
  \frac{dy}{dx} = 2\frac{m_\chi\, c(T)}{H\tilde g^{-1/2}}\left(1-\frac{T_\chi}{T}\right)\,,
\end{equation}
with $\tilde g^{1/2}\equiv g_{\rm eff}^{1/2}/(1+\frac{1}{4}\frac{g_{\rm eff}'}{g_{\rm eff}}T)$. 
This form of the equation is also very convenient in that it allows to directly read off 
its asymptotic behaviour: 
At large $T$, the factor in front of the right-hand side is much larger than unity, thus enforcing
$T_\chi=T$; when $T$ becomes small, the WIMPs completely decouple from the thermal 
bath and $y$ stays constant, i.e. $T_\chi\propto T^2 g_{\rm eff}^{1/2}\propto a^{-2}$. 
Since this transition happens on a rather short time scale, a sensible definition of 
the \emph{kinetic decoupling} temperature $T_{\rm kd}$ is thus given by equating these 
two regimes (as if the decoupling process were to occur instantaneously -- 
see \cite{Bringmann:2009vf}  for a more detailed discussion):
\begin{equation}
  \label{eq:tkddef}
  x_{\rm kd}=\frac{m_\chi}{T_{\rm kd}}\equiv g_{\rm eff}(T_{\rm kd})\, \left.y\right|_{x\rightarrow\infty}\,.
\end{equation}





\subsection{The smallest protohalos}


Before kinetic decoupling, WIMPs are tightly coupled to the heat bath, so
any small-scale perturbations in the DM fluid would immediately be washed out.
For temperatures $T\lesssim T_{\rm kd}$, however, this is no longer the case and perturbations
 in the DM density start to devolop under the influence of gravity; however, the 
remaining viscous coupling between the two fluids and, subsequently, the free-streaming 
of the WIMPs generate an exponential cutoff in the power spectrum \cite{Green:2005fa}, 
with a characteristic comoving damping scale $k_{\rm fs}$. The WIMP mass contained in a 
sphere of the corresponding size is thus given by \cite{Bringmann:2009vf}
\begin{equation}
   M_{\rm fs}\approx\frac{4\pi}{3}\rho_\chi\left(\frac{\pi}{k_{\rm fs}}\right)^3
=4.0\times 10^{-6}\left(\frac{1+{\rm ln}\left(g_{\rm eff}^{1/4}T_{\rm kd}/30\;{\rm MeV}\right)/18.6}{\left(m_\chi/100\; {\rm GeV}\right)^{1/2} g_{\rm eff}^{1/4}\left(T_{\rm kd}/30\;{\rm MeV}\right)^{1/2}}\right)^3M_\odot\,.
\end{equation}
Acoustic oscillations also have to be taken into account as a damping mechanism 
and lead to a similar exponential cutoff in the power spectrum
\cite{Loeb:2005pm,Bertschinger:2006nq}. In this case, the characteristic damping mass 
 is given by the total amount of DM inside the horizon at the time of kinetic decoupling:
\begin{equation}
  M_{\rm ao}\approx\frac{4\pi}{3}\left.\frac{\rho_\chi}{H^3}\right|_{T=T_{\rm{kd}}}
  =3.4\times10^{-6}\left(\frac{T_{\rm kd}g_{\rm eff}^{1/4}}{50\,{\rm MeV}}\right)^{-3}M_\odot\,.
\end{equation}

In general, the actual cutoff in the power spectrum is  given by
$M_{\rm cut}=\max\left[M_{\rm fs},M_{\rm ao}\right]$; which of the two
physically independent damping mechanisms is more efficient depends on the 
particle nature of the WIMP. 
The expected mass for the smallest gravitationally bound objects in the 
universe is then also simply given by $M_{\rm cut}$. Numerically, the formation of 
protohalos with masses down to the cutoff scale has been confirmed
and their evolution could be 
followed until a redshift of $z\sim26$ \cite{Diemand:2005vz}; the further survival 
probabilities of these objects, however, as well as the resulting consequences for the indirect 
detection of DM, are subject to a presently still ongoing discussion.




\subsection{Implementation in \ds}

Before using any of the routines provided in {\tt src/mh} for the first time, one has to 
call \ftb{dsmhset} in order to choose the DM model and to make some necessary initilizations; 
currently, both the case of neutralino DM in the general SUSY case as well as Kaluza-Klein DM 
in minimal universal extra 
dimensions (mUED) are implemented. However, it is straightforward to add any additional, 
user-defined  DM model by extending the relevant routines and following the comments
given in the correponding files.  
For the average user, the subroutines of greatest interest will be \ftb{dsmhtkd} and 
\ftb{dsmhmcut}, and there should be no need to call any of the other routines directly.

\ftb{dsmhtkd} numerically solves the Boltzmann equation (\ref{eq:boltz2}) and returns
$T_{\rm kd}$ as given in the definition of Eq.~(\ref{eq:tkddef}). Here, special care is taken to 
accurately handle potential resonances in the scattering amplitude; to this end, 
\ftb{dsmhboltz\_init} identifies the location of all relevant resonances and passes this
information to \ftb{dsmhboltz} where the integral of Eq.~(\ref{eq:cTdef}) is perfomed.

Finally, \ftb{dsmhmcut} returns the mass cutoff in the power spectrum, with an input parameter 
determining whether it is $M_{\rm fs}$ or $M_{\rm ao}$; the default call results in 
$M_{\rm cut}=\max\left[M_{\rm fs},M_{\rm ao}\right]$, i.e.~the mass of 
the smallest protohalos.
