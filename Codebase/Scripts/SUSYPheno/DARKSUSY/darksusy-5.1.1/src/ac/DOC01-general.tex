%%%%%%%%%%%%%%%%%%%%%%%%%%%%%%%%%%%%%%%%%%%%%%%%%%%%%%%%%%%%%%%%%%%%

\section{Accelerator bounds}

\ds\ contains a set of routines to check if a given model is excluded by
accelerator constraints. These routines are called \ftb{dsacbnd[number]}. The
policy is that when we update \ds\ with new accelerator constraints, we keep
the old routine, and add a new routine with the last number incremented by one.
Which routine that is called is determined by calling \ftb{dsacset} with a
tag determining which routine to call. To check the accelerator constraints, 
then call \ftb{dsacbnd} which calls the right routine for you. Upon return,
\ftb{dsacbnd} returns an exclusion flag, \code{excl}. If zero, the model
is OK, if non-zero, the model is excluded. The cause for the exclusion is coded
in the bits of \code{excl} according to table \ref{tab:acexcl}


\begin{table}[!h]
\centering
\begin{tabular}{rrrcl} \hline
\multicolumn{3}{c}{\code{excl}} && \\ \cline{1-3}
Bit set & Octal value & Decimal value && Reason for exclusion \\ \hline
 0 &             1 &            1 && Chargino mass \\
 1 &             2 &            2 && Gluino mass \\
 2 &             4 &            4 && Squark mass \\
 3 &            10 &            8 && Slepton mass \\
 4 &            20 &           16 && Invisible $Z$ width \\
 5 &            40 &           32 && Higgs mass \\
 6 &           100 &           64 && Neutralino mass \\
 7 &           200 &          128 && $b \rightarrow s \gamma$ \\
 8 &           400 &          256 && $\rho$ parameter \\ \hline
\end{tabular}
\caption{The bits of \code{excl} are set to indicate by which process this
particular model is excluded. Check if a bit is set with 
\code{btest(excl,bit)}.}
\label{tab:acexcl}
\end{table}

