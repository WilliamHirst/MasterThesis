%%%%%%%%%%%%%%%%%%%%%%%%%%%%%%%%%%%%%%%%%%%%%%%%%%%%%%%%%%%%%%%%%%%%
\section{Neutrinos from Sun and Earth --  routines}

\comment{NOTE: This section is not up-to date with the current
  \ds\ release.}

This set of routines contain routines to calculate the
neutrino-induced muon flux from the Earth and the Sun in various
models. It also includes routines that calculate the neutrino-induced
muon flux from other sources, like the Sun's atmosphere, the Earth's
atmosphere \mcomment{(include these???)}.

There are three different methods of calculation available (determined
by \texttt{ntcalcmet} in \texttt{dsntcom.h}). Method 1 uses the
approximate formulae for the capture rates in the Earth/Sun from the
Jungman, Kamionkowski and Griest review \cite{jkg}. Method 2, uses the
same expression for the Sun, but the full expression from Gould
\cite{Gould321} for capture in the Earth (this is the default). Method
3, finally, is the same as 2, but it also includes capture in the
Earth from the Damour-Krauss population of WIMPs that have scattered
in the outscirts of the Sun. The easiest way to select method is by
calling \texttt{dsntset}, with the argument 'jkg' for method 1, 'gould'
or 'default' for method 2 and 'dk' for method 3. A call to 
\texttt{dsntset('default')} is made in \texttt{dsinit}, but can be
changed by the user by calling \texttt{dsntset} after \texttt{dsinit}.

To calculate the neutrino-induced muon flux from the Earth, you call

\begin{sub}{subroutine
\ftb{dsntrates}(emuth,thmax,rtype,rateea,ratesu,istat)}
  \itit{Purpose:} Calculate the rate of neutrinos or neutrino-induced
  muons in a neutrino telescope from neutralino annihilation in the
  Earth and the Sun.
  \itit{Input:}
  \itv{emuth}{r8} The neutrino or muon energy threshold in GeV.
  \itv{thmax}{r8} The half-aperture opening angle (in degrees) 
  towards the center
  of the Sun or the Earth (i.e.\ the flux will be summed in a cone
  towards the center of the Sun or the Earth, where the top-angle of
  the cone is 2*\ft{thmax}).
  \itv{rtype}{i}Type of flux to calculate:\\
  =1: muon neutrino-flux (neutrino and anti-neutrino summed) in units
  of km$^{-2}$ yr$^{-1}$.\\
  =2: neutrino-to-muon conversion rate (muons and anti-muons summed)
  in units of km$^{-3}$ yr$^{-1}$.\\
  =3: muon flux (muons and anti-muons summed)
  in units of km$^{-2}$ yr$^{-1}$.
  \itit{Output:}
  \itv{rateea}{r8} The rate from neutralino annihilation in the Earth
  in the above units.
  \itv{ratesu}{r8} The rate from neutralino annihilation in the Sun
  in the above units.
  \itv{istat}{i} =0: Everything went OK.\\
  $\neq0$: Some of the tables of neutrino or muon yields had to be
  used outside their tabulated regions. Extrapolations have been used.
\end{sub}

\begin{sub}{subroutine
\ftb{dsntdiffrates}(emu,theta,rtype,rateea,ratesu,istat)}
  \itit{Purpose:} Calculate the differential 
  rate of neutrinos or neutrino-induced
  muons in a neutrino telescope from neutralino annihilation in the
  Earth and the Sun.
  \itit{Input:}
  \itv{emu}{r8} The neutrino or muon energy in GeV.
  \itv{theta}{r8} The angle (in degrees) from the center of 
  the Sun or the Earth.
  \itv{rtype}{i}Type of flux to calculate:\\
  =1: muon neutrino-flux (neutrino and anti-neutrino summed) in units
  of km$^{-2}$ yr$^{-1}$ GeV$^{-1}$ degrees$^{-1}$.\\
  =2: neutrino-to-muon conversion rate (muons and anti-muons summed)
  in units of km$^{-3}$ yr$^{-1}$ GeV$^{-1}$ degrees$^{-1}$.\\
  =3: muon flux (muons and anti-muons summed)
  in units of km$^{-2}$ yr$^{-1}$ GeV$^{-1}$ degrees$^{-1}$.
  \itit{Output:}
  \itv{rateea}{r8} The rate from neutralino annihilation in the Earth
  in the above units.
  \itv{ratesu}{r8} The rate from neutralino annihilation in the Sun
  in the above units.
  \itv{istat}{i} =0: Everything went OK.\\
  $\neq0$: Some of the tables of neutrino or muon yields had to be
  used outside their tabulated regions. Extrapolations have been used.
\end{sub}
