%%%%%%%%%%%%%%%%%%%%%%%%%%%%%%
\section{Halo models -- theory}
\label{sec:halo}

All the dark matter detection rates depend in one way or another on
the properties of the Milky Way dark matter halo. We will here outline
the halo model that by default is included with \ds.

The mass distribution in the Milky Way and the relative importance of
\mbox{its} three components, the bulge, the disk and the halo, are
poorly constrained by available observational data.  Although the
dynamics of the satellites of the galaxy clearly indicates the
presence of a non-luminous matter component, a discrimination among
the different radial dark matter halo profiles proposed in the
literature is not possible at the time being ~\cite{binney}.  One
approach is to assume that dark matter profiles are of a universal
functional form and to infer the Milky Way dark matter distribution
from the results of N-body simulations of hierarchical clustering in
cold dark matter cosmologies.  The predicted profiles in these
scenarios have been tested to a sample of dark matter dominated dwarf
and low-surface brightness galaxies which provide the best
opportunities to test the spatial distribution of dark matter.
Actually this field of research is in rapid evolution and slightly
discrepant results have recently been presented
\cite{navarro,carlberg,kravtsov,moore}.

In \ds, we include a dark matter halo profile of the form
\beq
   \rho(r) \propto \frac{1}{(r/a)^{\gamma}\;[1+(r/a)^{\alpha}]^
   {(\beta-\gamma)/\alpha}}.
\eeq
With this family of profiles, we have a parameterization of most
spherically symmetric profiles
in the literature. In Table~\ref{tab:halo-profile} we list the
corresponding values of $\alpha$, $\beta$ and $\gamma$ for some
popular profiles.

\begin{table}
\begin{center}
\begin{tabular}{lll}
{\bfseries Model} & {\bfseries $(\alpha,\beta,\gamma)$}
& {\bfseries $a$ (kpc)} \\ \hline
Isothermal sphere \cite{kravtsov} & $(2,2,0)$ \\
Kravtsov et al.\ \cite{kravtsov} & $(2,3,0.2-0.4)$ \\
Navarro, Frenk and White\ \cite{navarro} & $(1,3,1)$ \\
Moore et al.\ \cite{moore} & $(?,?,?)$ \\ \hline
\end{tabular}
\caption{Different halo dark matter profiles and their corresponding
parameters.} 
\label{tab:halo-profile}
\end{center}
\end{table}

One should keep in mind that some of these profiles are very steep at
the center of the galaxy which might be in conflict with observations.
In fact, this is a topic under rapid evolution at present.  Some
researchers have taken the view that the steep profiles seen in
simulations are impossible to match with observations, and therefore
drastic modifications of the cold dark matter scenario have been
proposed.  Examples are self-interacting \cite{self} or strongly
self-annihilating \cite{annihil} dark matter models.  None of these
proposals can be made to work in MSSM models, so we do not consider
them in \ds.  It should also be noted that there could be other,
astrophysics-related solutions to these problems, which involve the
interplay between the baryons and the dark matter.

Our galactocentric distance $R_{0}$ is not entirely known. Estimates
for $R_{0}$ range from 7.1 kpc to 8.5 kpc \cite{oll,reid,???} and in
\ds\ we use $R_{0}=8.5$ kpc as a default. 
We also choose the modified isothermal distribution as
a default, but this can be changed by the user.

As a further uncertainty, it is unknown precisely how the black hole
at the galactic center would have interacted with the halo neutralino
distribution. In fact, there are indications that a profile more
singular than NFW would cause a very steep cusp (a ``spike'')
near the Galactic center, with a high enough density that even
the flux of neutrinos from that population could be detected \cite{gs}.
If this really exists, essentially all MSSM models may already be excluded
through the non-detection of radio emission from electrons and
positrons generated in the annihilations \cite{spike}. It should be
noted though that these estimates involve many uncertainties.

We only consider spherical profiles; introducing a flattening
parameter may enhance the value of the flux but the effect is not
expected to be dramatic for this neutralino detection method and we
prefer not to introduce another factor of uncertainty.

We also need to specify the normalization constant of the halo
profile, which we choose as the value of the halo density $\rho_0$ at
our galactocentric distance $R_0$, the core radius $a$ and $R_0$.  One
should keep in mind that there is correlation between the allowed
values of $a$ and $\rho_{0}$ and the chosen halo profile
\cite{bub,pierothesis} due to constraints on e.g.\ the total mass of
the galaxy within 100 pc and the dark matter contribution to the local
rotation curve.  In Table~\ref{tab:halo-profiles} we list typical
values of $a$ and $\rho_{0}$ that we have chosen  based on
these constraints for the different profiles. For more details about these
arguments, see \cite{bub,pierothesis}. The \ds\ default value for
$\rho_{0}$ is 0.3 GeV/cm$^3$.

Usually, the local galactic dark matter velocity distribution is taken
to be a truncated gaussian, which in the detector frame moving at
speed $v_O$ relative to the galactic halo reads
\begin{equation}
   f(v) = {1\over {\mathcal N}_{\rm cut}} { v^2 \over u v_O \sigma} \left\{
   \exp\left[{-{(u-v_O)^2\over2\sigma^2}}\right] -
   \exp\left[{-{\min(u+v_O,v_{\rm cut})^2\over2\sigma^2}}\right]
   \right\}
\end{equation}
for $ v_{\rm esc} < v < \sqrt{v_{\rm esc}^2 + (v_O + v_{\rm cut} ) ^2
} $ and zero otherwise, with $ u = \sqrt{v^2 + v_{\rm esc}^2} $ and
\begin{equation}
   {\mathcal N}_{\rm cut} =
   {v_{\rm cut}\over\sigma} \exp\left( {-{v_{\rm cut}^2\over2\sigma^2}} \right)
   -
   \sqrt{\pi\over2} {\rm erf} \left( {v_{\rm cut}\over\sqrt{2}\sigma} \right) .
\end{equation}
As default, we have taken the halo line-of-sight (one-dimensional) velocity
dispersion $\sigma = $120 km/s,\footnote{Other authors write
   $\exp(-3v^2/2\overline{v}^2)$, in which case $\overline{v} = \sqrt{3}
   \sigma$.}  the galactic escape speed $ v_{\rm cut} = $ 600 km/s, the relative
Earth-halo speed $ v_O = $ 264 km/s (a yearly average) and the Earth escape
speed $ v_{\rm esc} = $ 11.9 km/s. These parameters can be changed by
the user. In some instances, like neutralino capture in the
Earth, the user can specify an arbitrary velocity distribution by
providing a subroutine.

%%%%%%%%%%%%%%%%%%%%%%%%%%%%%%
\subsection{Rescaling of the neutralino density}
\label{sec:rescale}

It is natural to assume that the neutralinos make up most of the dark
matter in our galaxy.  One may therefore only consider MSSM models
which are cosmologically interesting, i.e.\ where the neutralinos can
make up a major fraction of the dark matter in the Universe without
overclosing it.  This range is usually chosen to be $0.025 <
\Omega_{\chi}h^{2} <1$.  However, the user may want to either enlarge
or narrow this range.  If, as is perhaps most natural, the neutralino
alone contributes the major fraction of non-baryonic dark matter in
the Universe, one may want to refer to the current values of
cosmological parameters and fix $\Omega_{\chi}h^{2}$ to be in the
interval between, say 0.1 and 0.3.  If there are other components of
the dark matter, one may want to tolerate smaller numbers.  If one
makes use of the poor knowledge of how galaxy halos were formed, all
the range down to 0.025 may be taken as acceptable.  However, if
$\Omega_{\chi}h^{2}$ drops below 0.025, it cannot account for all the
dark matter associated with galaxy halos.  A frequently used recipe is
then to rescale the estimated local dark matter density $\rho_0\sim
0.3$ GeV/cm$^3$ by $\Omega_{\chi}h^2/0.025$, giving a lower local
density in the form of neutralinos.  Although this may seem a harmless
procedure, one should keep in mind that it is very {\em ad hoc} and
that it may overestimate the preponderance of models with large direct
detection rates.  This is because of the general result that
$\Omega_{\chi}h^{2}\sim 1/\sigma_{ann}v$, and crossing symmetry
generally relates a large annihilation cross section to a large
scattering cross section.  (For indirect detection in the halo, the
effect is moderated by the fact that the rates are proportional to the
square of the density, which thus involves the square of the rescaling
factor.)  In \ds, the user can set the value of the local dark matter
density (the default is $0.3$ GeV/cm$^3$) and determine whether
rescaling is to be used or not, and in that case the lowest tolerable
$\Omega^{\rm min}_{\chi}h^{2}$ below which rescaling should take
place.  If rescaling is used, all output detection rates are computed
with the rescaled value when appropriate.

