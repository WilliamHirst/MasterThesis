%%%%%%%%%%%%%%%%%%%%%%%%%%%%%%%%%%%%%%%%%%%%%%%%%%%%%%%%%%%%%%%%%%%%

\section{Initialization routines}

Before \ds\ is used for some calculations, it needs to be
initialized. This is done with a call to \codeb{dsinit}. This routine
makes sure that all standard parameters are defined, such as standard
model parameters and particle codes. It also calls the different
\codeb{ds*set} routines with the argument \code{default}. E.g., the
halo model is set to the default choice with a call to
\codeb{dshmset}\code{('default')}. Analogously, all other routines
with a \codeb{ds*set} routine is also called to set them up to the
default model/parameters.

This means that the call to \codeb{dsinit} should be the first call in
any program using \ds. Any calls the user makes to other routines,
either to calculte things or select a different model (e.g.\ a
different halo model) should come after the call to \codeb{dsinit}.
