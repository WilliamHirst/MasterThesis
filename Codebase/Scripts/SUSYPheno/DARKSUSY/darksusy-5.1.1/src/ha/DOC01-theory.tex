%%%%%%%%%%%%%%%%%%%%%%%%%%%%%%%%%%%%%%%%%%%%%%%%%%%%%%%%%%%%%%%%%%%%%%
\section{Annihilation in the halo, yields -- theory}

Here we calculate yields from annihilation in the halo.

\subsection{Monte Carlo simulations}
\label{sec:ha-mcsim}

We need to
evaluate the yield of different particles per neutralino annihilation.
The hadronization and/or decay of the annihilation products are
simulated with {\sc Pythia} \cite{pythia} 6.154.
The simulations are done for a set of 18 neutralino
masses, $m_{\chi}$ = 10, 25, 50, 80.3, 91.2, 100, 150, 176, 200, 250,
350, 500, 750, 1000, 1500, 2000, 3000 and 5000 GeV\@. We tabulate the
yields and then interpolate these tables in \ds.

     The simulations are here
     simpler than those for annihilation in the Sun/Earth
    since we don't have a surrounding medium that can stop the
     annihilation products.  We here simulate for 8 `fundamental'
     annihilation channels $c\bar{c}$, $b\bar{b}$,
     $t\bar{t}$, $\tau^+\tau^-$, $W^+W^-$, $Z^0 Z^{0}$, $g g$ and
     $\mu^{+} \mu^{-}$. Compared to the simulations in the Earth and
     the Sun, we now let pions and kaons decay and we also let
     antineutrons decay to antiprotons. For each mass we simulate
     $2.5 \times 10^{6}$ annihilations and tabulate the yield of
     antiprotons, positrons, gamma rays (not the gamma lines),
     muon neutrinos and neutrino-to-muon conversion rates and the
     neutrino-induced muon yield, where in the last two cases the
     neutrino-nucleon interactions has been simulated with {\sc
     Pythia} as outlined in section~\ref{sec:nt-mcsim}

With these simulations, we can calculate the yield for any of these
particles for a given MSSM model.  For the Higgs bosons, which decay
in flight, an integration over the angle of the decay products with
respect to the direction of the Higgs boson is performed.  Given the
branching ratios for different annihilation channels it is then
straightforward to compute the muon flux above any given energy
threshold and within any angular region around the Sun or the center
of the Earth.
