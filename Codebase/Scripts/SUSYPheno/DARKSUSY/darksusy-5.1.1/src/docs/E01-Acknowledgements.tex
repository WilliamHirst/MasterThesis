%%%%%%%%%%%%%%%%%%%%%%%%%%%%%%%%%%%%%%%%%%%%%%%%%%%%%%%%%%%%%%%%%%%%%

\chapter*{Acknowledgements}
\addcontentsline{toc}{chapter}{Acknowledgements}

P. Gondolo created \ds\ in 1994, took care of its organization,
arranged it for release, and prepared the documentation. He contributed
\cite{bergstrom96} the routines on the supersymmetric spectrum and mixing, the
original calculation of the neutralino relic density without coannihilations,
the direct detection rates and the accelerator bounds. P. Gondolo and J.
Edsj\"o \cite{edsjo97} included coannihilations in the relic density routines.
J. Edsj\"o contributed the package for the neutrino--induced muons from the Sun
and the Earth \cite{edsjo95+}, and organized the routines for annihilations in
the galactic halo, incorporating the code for the gamma-ray continuum by
himself, for the antiprotons \cite{bergstrom99} and the gamma-ray lines
\cite{bergstrom97+} by P. Ullio, and for the positrons by E.  Baltz
\cite{baltz99}. Finally, \ds\ includes adapted versions of (1)
routines by Carena, Quir\'os and Wagner on the Higgs boson masses, (2) routines
from CMLIB (URL: http://www.netlib.org), specifically \ft{dqagse} and its
dependencies, (3) routines from CERNLIB, specifically \ft{gpindp} by X and
\ft{gadap} by T. Johansson.

