%%%%%%%%%%%%%%%%%%%%%%%%%%%%%%%%%%%%%%%%%%%%%%%%%%%%%%%%%%%%%%%%%%%%
\chapter{Introduction}

\ds\ is a set of Fortran routine to make calculations for
supersymmetric dark matter in the Minimal Supersymmetric Standard
Model, the MSSM. The physics involved is covered in the \ds\ paper
\cite{dspaper}. In this manual we will mainly cover the more techincal
aspects of \ds, i.e.~ how to call different subroutines and how to
change switches and options. We will only briefly review the necessary
physics involved when needed and refer the reader to \cite{dspaper}
and the original papers behind \ds\ \cite{dsoriginal} for more
details. If you use \ds\, please consider the original physics work
behind and give proper credit to \cite{dspaper} and the relevant
references in \cite{dsoriginal}. If you use non-standard options,
e.g.\ a different propagation model for antiprotons, please remember
to give proper credit to that model.

