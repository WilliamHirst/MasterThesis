%%%%%%%%%%%%%%%%%%%%%%%%%%%%%%%%%%%%%%%%%%%%%%%%%%%%%%%%%%%%%%%%%%%%
\chapter{General remarks on notation}


In an attempt to keep this manual reasonably easy to follow we will
need to specify our notation.  We will use the following convention
for fonts,
\begin{sub}{Convention for fonts}
  \itv{\rmfamily text}{} This font is used for normal text.
  \itv{variable}{} This font is used for variables or other things
  in the code that is mentioned.
  \itv{\ftb{routine}}{} This font is used for subroutine or function
  names or for header file names.
  \itv{\tt dump}{} This font will be used for screen dumps of outputs.
  \itv{\ttfamily \em input}{} This font will be used for user
  input, i.e.\ where you are supposed to write something.
\end{sub}

Subroutines and functions will be described with the following
structure
\begin{sub}{subroutine \ftb{example}(in1,in2,in3,in4,in5,in6,in7,out1)}
  \itit{Purpose:} Here the routine will be explained.
  \itit{Inputs:}
  \itv{in1}{i} This is an input argument, declared as \ft{integer}.
  \itv{in2}{r} This is an input argument, declared as \ft{real}.
  \itv{in3}{r8} This is an input argument, declared as \ft{real*8}.
  \itv{in4}{c} This is an input argument, declared as \ft{complex}.
  \itv{in5}{c16} This is an input argument, declared as \ft{complex*16}.
  \itv{in6}{ch2} This is an input argument, declared as \ft{character*2}.
  \itv{in7}{ch*} This is an input argument, declared as \ft{character*(*)}.
  \itit{Outputs}
  \itv{out1}{r8} This is an output argument, declared as \ft{real*8}
\end{sub}
where the shorthand notation for the type of the arguments is
indicated. For functions, the type is indicated on the first line,
\begin{sub}{function \ftb{fun}(arg) \hfill r8}
  \itit{Purpose:} Here the function will be explained.
  \itit{Inputs:}
  \itv{arg}{i} This is an input argument, declared as \ft{integer}.
\end{sub}
i.e., in this case the function is declared as \ft{real*8}.

The subroutines always reside in a file with the
same name as the subroutine/function. Routines that belong together
are put in separate subdirectories in the \ft{src} directory. The different
subdirectories are
\begin{sub}{Subdirectories in src/}
  \itv{ac}{}     accelerator constraints
  \itv{an}{}     driver routines for neutralino and chargino annihilation
  \itv{an1l}{}   1-loop neutralino annihilation amplitudes
  \itv{anstu}{}  tree-level neutralino and chargino annihilation amplitudes
  \itv{dd}{}     direct detection and neutralino scattering
  \itv{ep}{}     positron fluxes from the halo
  \itv{ge}{}     general routines
  \itv{ha}{}     yields of halo annihilation products (from Pythia
                 simulations in vacuum)
  \itv{hm}{}     halo models
  \itv{hr}{}     driver routines for rates from the halo
  \itv{ib}{}     internal bremsstrahlung
  \itv{ini}{}    initialization routines
  \itv{mh}{}     kinetic decoupling and microhalos
  \itv{mu}{}     neutrino and muon yields from the neutralino
                 annihilations in the Earth/Sun (from Pythia
                 simulations in medium)
  \itv{nt}{}     driver routines for rates and fluxes in neutrino telescopes
  \itv{pb}{}     antiprotons from annihilation in the halo
  \itv{rd}{}     relic density routines (general)
  \itv{rn}{}     driver routines for neutralino relic density
  \itv{su}{}     general MSSM routines, couplings, masses, etc.
  \itv{xcern}{}  routines from CERNLIB
  \itv{xcmlib}{} routines from CMLIB 
\end{sub}


Common blocks are all declared in header files in the \ft{inc}
directory. When discussing switches and parameters in common blocks we
will, instead of describing the common blocks in detail, 
mention which header file they reside in. If you want to access these
variables, you should then include the corresponding header
file. E.g., it can look like this
\begin{sub}{Example parameters in \ftb{headerfile.h}}
  \itit{Purpose:} Description of this set of variables.
  \itv{par1}{r8} Description of a \ft{real*8} parameter.
\end{sub}

