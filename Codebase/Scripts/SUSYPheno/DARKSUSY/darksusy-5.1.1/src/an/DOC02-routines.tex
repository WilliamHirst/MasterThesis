%%%%%%%%%%%%%%%%%%%%%%%%%%%%%%%%%%%%%%%%%%%%%%%%%%%%%%%%%%%%%%%%%%%%

\section{Annihilation routines - general remarks}

The annihilation cross section routines is divided into several parts,
mostly for historical reasons. The layout is roughly as follows:
\begin{description}
  \item{\code{src/an}} Here we keep the main routins for both neutralino-
  neutralino annihilation cross sections and the effective annihilation
  cross section in the relic density calculations. The steering routines
  for neutralino and chargino coannihilations are also kept here.
  
\item{\code{src/anstu}} Here keep the $t-$, $u-$ and $s-$ diagram expressions
for fermion-fermion coannihilations (i.e.\ neutralino and chargino
coannihilations).

\item{\code{src/as}} Here all the coannihilation cross sections including
sfermions are kept.

\end{description}

We will here describe the \code{src/an}-routines.


%%%%%%%%%%
\subsection{General routines}

The general routine to call for an effective annihilation cross section (to
be used for relic density calculations) is \codeb{dsanwx}, which returns the
invariant annihilation rate (integrated over $\cos \theta$). The actual
cross section, differential in $\cos \theta$ is calculated by 
\codeb{dsandwdcos} which includes all the coannihilations needed. This is set
up in \codeb{rn/dsrdomega} which determines which coannhilating particles
to include.

For other applications where the annihilation rate is needed, e.g.\ annihilation
in the galactic halo, one can call the specific annihilation rate routine
directly. The main one is \codeb{dsandwdcosnn} for neutralino-neutralino
annihilation. To simplify this task, we supply a routine \codeb{dssigmav} which
calls \codeb{dsandwdcosnn} for neutralino-neutralino annihilation at zero
relative velocity and returns the result, either as the total annihilation
cross section, or the cross section for a specific channel. See the header of
\codeb{dssigmav} for details.

%%%%%%%%%%
\subsection{Neutralino and chargino (co)annihilation cross sections}
                                   
The routines \codeb{dsandwdcosnn}, \codeb{dsandwdcoscn} and \codeb{dsandwdcoscc}
calculate the annihilation cross sections (returning the invariant 
annihilation rate) for neutralino-neutralino, neutralino-chargino and
chargino-chargino annihilations. Which particles the cross section is
calculated for is given by particle indices as defined in \codeb{inc/dsmssm.h}.

All the annihilation routines return the invariant rate instead of the
cross section. The invariant annihilation rate between particle $i$
and $j$ is defined as 
\begin{equation}
  W_{ij} = 4 p_{ij} \sqrt{s} \sigma_{ij} = 4 \sigma_{ij} \sqrt{(p_i
\cdot p_j)^2 - m_i^2 m_j^2} = 4 E_{i} E_{j} \sigma_{ij} v_{ij} .
\end{equation}
See chapter \ref{ch:src-rd} for more details.
