%%%%%%%%%%%%%%%%%%%%%%%%%%%%%%%%%%%%%%%%%%%%%%%%%%%%%%%%%%%%%%%%%%%%%%
\section{Internal Bremsstrahlung (IB) -- theory}

Whenever WIMPs annihilate into pairs of charged particles $X\bar X$, this process 
will inevitably be accompanied by internal 
bremsstrahlung (IB), i.e. the emission of an additional photon in the final state 
(note that in contrast to ordinary, or external, bremsstrahlung no external 
electromagnetic field is required for the emission of the photon). In many cases, 
IB photons completely dominate the annihilation spectrum at the highest accessible 
energies. The resulting, characteristic sharp step at an energy corresponding to 
the dark matter particle's mass, often accompanied by a bump-like feature at slightly smaller energies, is a spectral signature that is hard to mimic by 
astrophysical sources and in that respect similar to monochromatic photons. 
In fact, being an $\mathcal{O}(\alpha_\mathrm{em})$ correction to the tree-level 
annihilation rate, one would generically expect IB photons to be more copiously 
produced than the loop-suppressed [i.e.~$\mathcal{O}(\alpha_\mathrm{em}^2)$] 
monochromatic photons; this has been confirmed in \cite{ib_susy} for a large part 
of the supersymmetric parameter space, not the least due to the appearance of 
efficient enhancement mechanisms. 



\subsection{General considerations}


For didactic purposes, one may follow \cite{ib_susy} and distinguish between 
photons directly radiated off the external legs (\emph{final state radiation}, FSR) 
and photons radiated from virtual charged particles (\emph{virtual internal 
bremsstrahlung}, VIB). The IB photons are thus  the total contribution from both 
FSR and VIB photons.

For relativistic charged final states, FSR diagrams are always dominated by photons 
emitted \emph{collinearly} with $X$ or $\bar X$.  This is a purely kinematical 
effect and related to the fact that the propagator of the corresponding outgoing 
particle,
\begin{equation}
  D(p)\propto\left((k+p)^2-m_X^2\right)^{-1},
\end{equation}
diverges in this situation. Here, $k$ and $p$ denote the momenta of the photon and 
the outgoing particle, respectively. The resulting photon spectrum turns out to be 
of a universal form, almost independent of the underlying particle physics model 
(see, e.g., \cite{birkedal}):
\begin{equation}
  \label{fsr}
  \frac{dN^{X\bar X}_{\gamma,\mathrm{FSR}}}{dx}\approx\frac{\alpha 
Q_X^2}{\pi}\mathcal{F}_X(x)\log \left(\frac{s(1-x)}{m_X^2}\right)\,.
\end{equation}
Here, $Q_X$ and $m_X$ are the electric charge and mass of $X$; the splitting 
function $\mathcal{F}(x)$ depends only on the spin of the final state particles and 
takes the form
\begin{equation}
  \mathcal{F}_\mathrm{fermion}(x)=\frac{1+(1-x)^2}{x}
\end{equation}
for fermions and
\begin{equation}
  \mathcal{F}_\mathrm{boson}(x)=\frac{1-x}{x}
\end{equation}
for bosons. Due to the logarithmic enhancement that becomes apparent in 
Eq.~(\ref{fsr}), FSR photons are often the main source for IB.
A prominent example where FSR in this universal form not only dominates IB but in 
fact the total gamma-ray spectrum from WIMP annihilations, is the case of 
Kaluza-Klein dark matter \cite{Bergstrom:2004cy}.

In general, one can single out two situations where photons emitted from virtual 
charged particles may give an even more important contribution to the total IB 
spectrum than FSR: i) the three-body final state $X\bar X\gamma$ satisfies a 
symmetry of the initial state that cannot be satisfied by the two-body final state 
$X\bar X$ or ii) $X$ is a boson and the annihilation into $X\bar X$ is dominated by 
$t$-channel diagrams, with the $t$-channel particle almost degenerate in mass with 
the annihilating WIMP. In contrast to FSR, the contribution from VIB photons can 
not be given in a model-independent way but is very sensitive to the underlying 
short-distance physics. For more details, see \cite{ib_susy}. 




\subsection{IB from neutralino annihilations}

For supersymmetric dark matter annhilations,  the relevant final states for IB are 
$W^+W^-$, $W^\pm H^\mp$, $H^+H^-$ and $f\bar f$; both of the situations just 
mentioned above can arise, and \emph{VIB} contributions become important in 
considerable regions of the parameter space.

Let us first note that for neutralino annihilations, in contrast to the situation 
for, e.g.,  Kaluza-Klein dark matter, we cannot in general expect very large 
\emph{FSR} contributions. This is because the lightest charged final states, for 
which the logarithmic enhancement shown in Eq.~(\ref{fsr}) would be most effective, 
are fermionic and therefore strongly helicity suppressed. Fermion final states 
containing an additional photon, however, are not subject to such a suppression 
\cite{lbe89}. In the limit of vanishing fermion mass, and assuming that both 
corresponding sfermions have the same mass, the photon multiplicity is given by 
\cite{ib_susy}

\begin{eqnarray}
  &&\frac{\mathrm{d}N^{f^+f^-}} {\mathrm{d}x}= 
\alpha_\mathrm{em}Q^2_f\frac{\left|\tilde g_R\right|^4+\left|\tilde 
g_L\right|^4}{64\pi^2} \Big(m_\chi^2 \langle\sigma v\rangle_{\chi\chi\rightarrow 
f\bar f}\Big)^{-1}\\
 &&\times(1-x)\left\{\frac{4x}{(1+\mu)(1+\mu-2x)}-\frac{2x}{(1+\mu-x)^2}
  -\frac{(1+\mu)(1+\mu-2x)}{(1+\mu-x)^3}\log\frac{1+\mu}{1+\mu-2x}\right\}\,,\nonumber
\end{eqnarray}
where $\mu\equiv m_{{\tilde f}_R}^2/m_\chi^2=m_{{\tilde f}_L}^2/m_\chi^2$ and 
$\tilde g_RP_L$ ($\tilde g_LP_R$) denotes the coupling between neutralino, fermion 
and right-handed (left-handed) sfermion. In the above expression, a large factor 
$m_\chi^2/m_f^2$ due to the lifted helicity suppression (from ${\langle\sigma 
v\rangle}_{\chi\chi\rightarrow f\bar f}\propto m_f^2m\chi^{-4}$) appears, and 
another enhancement at high photon energies for sfermions degenerate with the 
neutralino.

For large neutralino masses $m_\chi\gg m_W$ and charginos almost degenerate with 
the neutralino, $W^+W^-$ and $W^\pm H^\mp$ final states are affected by the second 
of the mechanisms for VIB enhancement that were discussed in the previous 
subsection. Of these two channels, IB from $W^+W^-$ nearly always dominates; for 
pure Higgsinos (or Winos), the resulting photon multiplicity in this limit is well 
approximated by \cite{ib_susy,ib_higgsinos}:
\begin{equation}
 \frac{\mathrm{d}N^{W^+W^-}} {\mathrm{d}x} \approx
  \frac{\alpha_\mathrm{em}}{\pi}\frac{4(1-x+x^2)^2}{(1-x+\epsilon/2)x}\left[\log\left(2\frac{1-x+\epsilon/2}{\epsilon}\right)-1/2+x-x^3\right]\,,
\end{equation}
where $\epsilon \equiv m_W/m_\chi$. 

Charged Higgs pairs $H^+H^-$, finally, provide yet another interesting example 
where the two-body final state is not allowed due to symmetry restrictions; in this 
case, in the limit $v\rightarrow0$, enforced by  $CP$ conservation. The 
annihilation into $H^+H^-\gamma$, on the other hand, \emph{is} possible. However, 
since charged Higgs bosons in most models have considerably larger masses than 
gauge bosons, they are expected to give negligible IB contributions compared to the 
latter.

\subsection{The implementation in \ds}

Let us now briefly describe how  IB is implemented in \ds. The total gamma-ray 
spectrum  from WIMP annihilations is given by
\begin{equation}
  \frac{dN_{\gamma,\mathrm{tot}}}{dx}=\sum_f 
B_f\left(\frac{dN^f_{\gamma,\mathrm{sec}}}{dx}+\frac{dN^f_{\gamma,\mathrm{IB}}}{dx} 
+ \frac{dN^f_{\gamma,\mathrm{line}}}{dx}\right)\,,
\end{equation}
where $B_f$ denotes the branching ratio into the annihilation channel $f$. The 
first term encodes the contribution from secondary photons, mainly produced through 
the decay of neutral pions, as described in Section \ref{section_hr}.
 We recall here that these contributions are included by using the Monte Carlo code 
{\sf Pythia} \cite{pythia} to simulate the decay of a hypothetical particle with 
mass 
$2m\chi$ and user-specified branching ratios $B_f$.  In this way, also FSR 
associated to this decay  is automatically included in 
${dN^f_{\gamma,\mathrm{sec}}}/{dx}$ (the main contribution here comes from photons 
directly radiated off the external legs, but also photons radiated from other 
particles in the decay cascade are taken into account). 

Of course, IB from the decay of such a hypothetical particle cannot in general be 
expected  to show the same characteristics as IB from the actual annihilation of 
two WIMPs and for this reason an additional term ${dN^f_{\gamma,\mathrm{IB}}}/{dx}$ 
is included that accounts for the difference between the full IB contribution and 
the FSR part already taken into account for by {\sf Pythia}. For details regarding 
the 
implemented procedure of separating these two contributions in a consistent way, 
see \cite{ib_susy}.
The contributions ${dN^f_{\gamma,\mathrm{IB}}}/{dx}$, in contrast to 
${dN^f_{\gamma,\mathrm{sec}}}/{dx}$, are generically highly model-dependent. At the 
moment, they are fully implemented only for neutralino annihilations and included 
by default. However, one may easily switch to a user-defined contribution, choose 
to neglect IB completely or, for comparison, to only include the FSR part (for 
details, see 
\ftb{dshaib}). 

For the supersymmetric case, the 
full expressions for all relevant three-body final states are implemented, i.e.~not 
just the 
approximations given in the last subsection, which only apply to the limits 
described there.
For performance reasons, IB is only included when virtual $t$-channel particles 
are sufficiently degenerate in mass that large IB contributions to the total 
spectrum can be expected. With a call to  \ftb{dsIBset}, this 
default
behaviour can be 
customized.
Finally, radiative corrections to the annihilation into charged particles $X\bar X$ 
of course also change the number of $X\bar X$ pairs per annihilation and thus the 
corresponding yield of particles in the further decay of the annihilation products. 
At the moment, apart from the photon yield, only the IB positron yield is 
implemented in \ds. By default, only the annihilation into the dominating channel $e^+e^-\gamma$ is taken into account in this case; again, this behaviour can be modified by a call to \ftb{dsIBset}.


