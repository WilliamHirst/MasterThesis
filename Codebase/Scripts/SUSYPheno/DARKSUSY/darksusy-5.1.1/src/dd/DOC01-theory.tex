%%%%%%%%%%%%%%%%%%%%%%%%%%%%%%%%%%%%%%%%%%%%%%%%%%%%%%%%%%%%%%%%%%%%
\section{Direct detection -- theory}

{\bfseries NOTE: The description below is outdated. \ds\ now includes
much better form factors.}

If  neutralinos are indeed the CDM needed on galaxy scales and larger,
there should be a substantial flux of these particles in the Milky
Way halo. Since the interaction strength  is
essentially given by the same weak couplings as, e.g., for neutrinos
there is a non-negligible chance of detecting them in low-background
counting experiments \cite{goodmanwitten}.
Due to the large parameter space of MSSM, even
with the simplifying assumptions above, there is a rather wide span of
predictions for the event rate in detectors of various types. It is
interesting, however, that the models giving the largest rates are
already starting to be probed by present direct detection
experiments \cite{bg,bottino}.


The rate for direct detection of galactic neutralinos, integrated over
deposited energy assuming no energy threshold, is
\begin{equation}
   R = \sum_i N_i n_\chi \langle \sigma_{i\chi} v \rangle ,
\end{equation}
where $ N_i $ is the number of nuclei of species $i$ in the detector,
$n_\chi$ is the local galactic neutralino number density, $
\sigma_{i\chi} $ is the neutralino-nucleus elastic cross section, and
the angular brackets denote an average over $ v $, the neutralino
speed relative to the detector as described in Section~\ref{sec:halo}.

In \ds, the basic quantities computed are the neutralino-nucleon cross
sections, which are free of complications related to nuclear
structure, and various experimental details like energy threshold,
efficiencies etc.  However, as a crude estimate of the expected rates
in realistic detectors, the total neutralino-nucleus scattering rates
can be obtained for $^{76}$Ge, Al$_2$O$_3$ (sapphire) and NaI.
Usually, it is the spin-independent interaction that gives the most
important contribution in target materials such as Na, Cs, Ge, I, or
Xe, due to the enhancement caused by the coherence of all nucleons in
the target nucleus.

The neutralino-nucleus elastic cross section can be written as
\begin{equation}
   \sigma_{i\chi} = {1 \over 4 \pi v^2 } \int_{0}^{4 m^2_{i\chi} v^2}
   \mbox{\rm d} q^2 G_{i\chi}^2(q^2) ,
\end{equation}
where $ m_{i\chi} $ is the neutralino-nucleus reduced mass, $q$ is the
momentum transfer and $G_{i\chi}(q^2) $ is the effective
neutralino-nucleus vertex. We write
\begin{equation}
   G^2_{i\chi}(q^2) = A_i^2 F^2_S(q^2) G_S^2 +
   4 \Lambda_i^2 F^2_A(q^2) G_A^2 ,
   \label{detrate1}
\end{equation}
which shows the coherent enhancement factor $A_i^2$ for the
spin-independent cross section
(often $ \Lambda_i^2 $ is written as $ \lambda^2 J(J+1) $ ). We assume
gaussian nuclear form factors \cite{Gould87} 
\begin{equation}
   F_S(q^2) = F_A(q^2) =
   \exp(-q^2R_i^2/6\hbar^2) ,
\end{equation}
\begin{equation}
   R_i = ( 0.3 + 0.89 A_i^{1/3} ) {\rm fm} ,
\end{equation}
which should provide us with a good approximation of the integrated
detection rate \cite{EllisFlores}, in which we are only interested.
(To obtain the differential rate with reasonable accuracy, better
approximations are needed \cite{engel}.)

Using heavy-squark effective lagrangians
\cite{efflagrange}, we get
\begin{equation}
   G_S = \sum_{q={\rm u,d,s,c,b,t}} \langle \bar{q} q \rangle
   \left( \sum_{h=H_1,H_2} { g_{h\chi\chi} g_{hqq} \over m_h^2 } -
   {1\over 2} \sum_{k=1}^6 { g_{L\tilde{q}_k\chi q} g_{R\tilde{q}_k\chi q} \over
   m^2_{\tilde{q}_k} } \right)
\label{GS}
\end{equation}
and
\begin{equation}
   G_A = \sum_{q={\rm u,d,s}} \Delta q
    \left( { g_{Z\chi\chi} g_{Zqq} \over m_Z^2 } + {1\over 8} \sum_{k=1}^6
   { g_{L\tilde{q}_k\chi q}^2 + g_{R\tilde{q}_k\chi q}^2 \over m^2_{\tilde{q}_k}
   } \right) .
\label{GA}
\end{equation}
The $g$'s are elementary vertices involving the particles indicated by
the indices, and they read
\begin{eqnarray}
   g_{h\chi\chi} & = & \left\{  \begin{array}{ll}
   \left( g Z_{\chi2} - g_y Z_{\chi1} \right)
   \left( - Z_{\chi3} \cos\alpha + Z_{\chi4} \sin\alpha \right) ,
   & {\rm for\ } H_1 , \\
   \left( g Z_{\chi2} - g_y Z_{\chi1} \right)
   \left( Z_{\chi3} \sin\alpha + Z_{\chi4} \cos\alpha \right) ,
   & {\rm for\ } H_2 ,
   \end{array} \right. \\
   g_{hqq} & = & \left\{ \begin{array}{ll}
   - Y_q \cos\alpha / \sqrt{2} , & {\rm for\ } H_1 , \\
   + Y_q \sin\alpha / \sqrt{2} , & {\rm for\ } H_2 , \end{array} \right. \\
   g_{Z\chi\chi} & = & {g \over 2 \cos\theta_W}
   \left( Z_{\chi3}^2 - Z_{\chi4}^2 \right)\\
   g_{Zqq} & = & - {g \over 2 \cos\theta_W} T_{3q} , \\
   g_{L\tilde{q}_k\chi q} & = & g_{LL} \Gamma_{QL}^{kq} +
   g_{RL} \Gamma_{QR}^{kq} , \\
   g_{R\tilde{q}_k\chi q} & = & g_{LR} \Gamma_{QL}^{kq} +
   g_{RR} \Gamma_{QR}^{kq} ,
\end{eqnarray}
with
\begin{eqnarray}
   g_{LL} & = & - {1\over\sqrt{2}}
   \left( T_{3q} g Z_{\chi2} + {1\over3} g_y Z_{\chi1} \right) , \\
   g_{RR} & = & \sqrt{2} e_q g_y Z_{\chi1} , \\
   g_{LR} & = & g_{RL} \, = \, \left\{ \begin{array}{ll}
   - Y_q Z_{\chi3} , & {\rm for\ } q={\rm u,c,t} , \\
   - Y_q Z_{\chi4} , & {\rm for\ } q={\rm d,s,b} ,
   \end{array} \right.
\end{eqnarray}
and
\begin{eqnarray}
   Y_q & = & \left\{ \begin{array}{ll}
   m_q / v_2 , & {\rm for\ } q={\rm u,c,t}, \\
   m_q / v_1 , & {\rm for\ } q={\rm d,s,b}.
   \end{array} \right.
\end{eqnarray}
Defining ($N=n,p$)
\beq
f^N_{Tq}\equiv {\langle N |m_q\bar q q|N\rangle\over m_N},
\eeq
we take in \ds\ the numerical values \cite{Gasser}
\begin{eqnarray}
   & f^p_{Tu} = 0.023, \qquad f^p_{Td} = 0.034, \qquad
    &\nonumber\\
&f^p_{Tc} = 0.0595, \qquad f^p_{Ts} = 0.14, \qquad&\nonumber\\
&f^p_{Tt} = 0.0595, \qquad f^p_{Tb}=0.0595 \qquad
\end{eqnarray}
and
\begin{eqnarray}
   & f^n_{Tu} = 0.019, \qquad f^n_{Td} = 0.041, \qquad
    &\nonumber\\
&f^n_{Tc} = 0.0592 \qquad f^n_{Ts} = 0.14, \qquad&\nonumber\\
&f^n_{Tt} = 0.0592, \qquad f^n_{Tb} = 0.0592. \qquad
\end{eqnarray}

For the quark contributions to the nucleon spin we
take \cite{SMC}
\begin{equation}
   \Delta {\rm u} = 0.77, \qquad \Delta {\rm d} = -0.40, \qquad
   \Delta {\rm s} = -0.12 .
\end{equation}
However, the older set of data \cite{jaffe}
\begin{equation}
   \Delta {\rm u} = 0.77, \qquad \Delta {\rm d} = -0.49, \qquad
   \Delta {\rm s} = -0.15
\end{equation}
can optionally be used.

Moreover, we take for the $\Lambda$ factors
\begin{equation}
   \Lambda^2_{\rm Al} = 0.087, \qquad
   \Lambda^2_{\rm Na} = 0.041 \quad {\rm and} \quad
   \Lambda^2_{\rm I} = 0.007,
   \label{detrate2}
\end{equation}
according to the odd-group model \cite{EngelVogel}.


