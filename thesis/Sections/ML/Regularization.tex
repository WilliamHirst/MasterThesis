\section{Regularization}
In \ac{ML}, overfitting occurs when a model becomes overly tuned to the training data and as a consequence 
fails to extract trends which would allow it to predict previously unseen data. The architecture of a \ac{NN}, 
the maximum depth of a boosted-\ac{DT} or even the size of the dataset can all contribute to overfitting. In the 
case of deep learning especially, overfitting can be a large problem and is therefore of focus in this thesis. Apart 
from predicting on a new data set, there are no rigid methods to detect overfitting. Instead, there exists many 
attempts to minimize the risk of it, we call these methods' \emph{regularization}. In \ac{ML}, regularization 
is known as any attempt to reduce the error in a prediction by reducing overfitting. Generally one can categorize
regularization as being either implicit or explicit. Explicit regularization means adding terms to the optimization 
problem. This is a very direct way of insuring no part of the model becomes overly dominant. Examples of this 
could be adding a penalty in the cost function of a \ac{NN} to ensure no weights become to large. Implicit
regularization is a less direct attempt of hindering overfitting. This could be changing the depth of the \ac{ML}-model,
varying the cost function or altering the data itself.
\subsection{Early stopping}
A simple implicit regularization method is to introduce \emph{early stopping} in the training pipeline. Early stopping 
simply means to stop training before the parameters of the model are allowed to over tune. The usual approach 
is to introduce a goal for the training, which when reached, ends the training. Examples of these goals could be a 
predetermined loss value on the training set or training until lack of progress on a second unseen dataset. The latter
approach is the one I will use in this thesis. 
\subsection{Ensembles}
When comparing different \ac{ML} methods by performance, more often then not the top performing model includes 
ensembling in one way or another. Like the word suggests, ensembling in \ac{ML} means using a collection of 
\ac{ML} models to create one complex model. There are many ways to create ensembles of models, most methods 
fall in to one of three categories; \emph{bagging}, \emph{boosting} and \emph{stacking}. Creating an ensemble of models 
through bagging, means to use several models each trained on their own sample from the same dataset. The overarching 
new model is created by averaging the predictions from the ensemble of models. The method seeks to create a unique set of 
models through exposing each individual model to different training sets. Boosting is different to bagging in that it uses 
the same training data on all the models. The diversity in the models when boosting, stems from intentionally choosing the 
architecture of the models such that it reduces the error made by the previous ensemble of models. Finally, stacking uses 
a predetermined model to decide how to combine the predictions made by the ensemble. More on the specifics of bagging, 
boosting and stacking in later sections.  

  
