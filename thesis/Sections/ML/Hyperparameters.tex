\section{Hyperparameters}
The tuning of parameters is a vital part of building an optimal \ac{ML}
model, though not all parameters are set during training. Parameters 
that are chosen prior to and fixed during training are called \emph{hyperparameters}. We differentiate
between two types of hyperparameters; model hyperparameters and algorithm 
hyperparameters. Model hyperparameters refer to parameters used to define the 
architecture of the model. Examples of this could be the size and depth of 
a \ac{NN} or the maximum depth of a \ac{DT}. These parameters are not tuned
during training, but will nonetheless have a great impact on the performance 
of the model. Algorithm hyperparameters on the other hand, do not have a direct impact 
on the performance of the model. These are parameters that mainly affect the 
effectiveness and quality of the training process. Examples of this are the 
learning rate of a \ac{NN}, or the batch-size used during training. Regardless 
of whether we are discussing model- or algorithm hyperparameters, it is in our interest
to find the optimal choice of parameters. The choice of parameters is made 
prior to the final training, and will be discussed in the following section.