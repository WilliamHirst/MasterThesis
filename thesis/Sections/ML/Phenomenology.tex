\section{Phenomenology}\label{sec:MLPhen}
\ac{ML} models differs from other analysis tools in its ability to
learn. Where a purely analytical model is static in both method 
and performance, a \ac{ML} model aims to be dynamic and self 
improving. \ac{ML} utilizes data to leverage towards an optimal
model. The extent of utilization of data defines a \ac{ML} model as being
either \emph{supervised}, \emph{semi-supervised} or \emph{unsupervised}. In the case 
of supervised \ac{ML}, a set of \emph{targets} are provided along with the
data which allows a \ac{ML} model to learn how to map a set of inputs to a target. In this 
thesis I will use the notation of $X = [\bf{x}_0,\bf{x}_1,\bf{x}_2,...,\bf{x}_N]$ to refer to 
a dataset, $T = [t_0,t_1,t_2,...,t_N]$ to the corresponding target and $Y=[y_0,y_1,y_2,...,y_N]$ 
to the output of the model where N is a number of points. The datasets contain a set of vectors with length equal to the number 
of features. The target and output could similarly contain vectors, but will in this thesis be restricted 
to scalar values. Generally, the target values can be both continues values, in which case the \ac{ML} 
method aims to perform a \emph{regression}, or discrete values, in which case the \ac{ML} method aims 
to perform a \emph{classification}. 
\\
The goal of supervised learning is the model can apply any attained knowledge from 
the study of a set of inputs and targets to predict the target of a new dataset. 
The success of any prediction is dependent on the quality of the 
data used during training, or \emph{training-data}. The training-data is required to 
be both representative of any trends you hope to detect in the new data set (test-data)
and be of sufficient amount. The latter point stems from a phenomenon known as \emph{over-fitting},
which is a problem where the training data becomes overly specialized to only predict 
the target of the training-data, and nothing else\footnote{More on this in later sections.}.
In this thesis the main focus will be on the application of supervised learning.
\\
In the case of unsupervised \ac{ML}, no target is provided. The motivation for unsupervised
learning is to create a model which is independent of any target. Such models are often 
useful in the case where one is not certain what one is looking for. An independence of target
means that the model is not overly sensitive to any specific trends or patterns, we call
this being \emph{unbiased}. Instead of learning to detect or predict specific phenomenon,
unsupervised learning is often used to detect anomalies in the data, which means it relies
heavily on statistics. Because of this, some use the term \emph{anomaly detection} and 
unsupervised learning interchangeably.
\\
Semi-unsupervised learning is a loose term
which finds itself in the middle of the previous two. It often refers to methods where 
no concrete target is provided, but instead uses the data provided to create a target. The goal 
is to alleviate as much bias as possible, but at the same time converge towards the usually 
superior performance of supervised learning.  