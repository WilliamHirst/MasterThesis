\chapter{Introduction to Machine Learning and Data Analysis}\label{chap:Intro ML}
\acf{ML} is rapidly becoming an overwhelming presence in many scientific fields.
In areas ranging from cancer research to stock-trading, machine learning is being applied to problems
once thought as impossible to solve. Particle physics, like many other fields is no exception. Jet flavor classification \cite{Guest_2016}, 
separating jets from gluons \cite{PhysRevD.44.2025} or using \ac{ML} to create efficient \ac{SR} are just some examples
where \ac{ML} is a vital tool. The traditional approach for ML in high-energy physics is the use of supervised
\ac{DNN}. \ac{DNN}'s are famous for their versatility which is partly due to their diversity in architecture and 
application. In later years \ac{BDT} have become more and more popular, especially after the release of XGBoost 
in 2014. The performance of XGBoost matches that of the \ac{DNN} in many cases, as well as having the advantage 
of being both stable and easy to use. In this section I will discuss many of the key concepts of \ac{ML} as 
well as go into detail of specific models applied in this search.

