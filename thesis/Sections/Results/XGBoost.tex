\section{Benchmarking the Analysis with a BDT}\label{sec:XGBoost}
\ac{BDT}s have been an essential part of \acf{HEP} analysis for many years (see 
\cite{cms_collaboration_reconstruction_2016}). More recently, the \verb!XGBoost! 
framework has become the main \ac{BDT} implementation in use in \ac{HEP} due 
to its excellent performance and scalability. Another reason for its popularity is the simple \acf{API}, allowing 
to create and apply a model in two lines of code. Additionally, its boosting capabilities means 
that it is little affected by variations in its structure. This leads many to the conclusion that the default 
parameters (see section \ref{subsec:XGBoost}) are often the best. 
\\
I choose to perform a sensitivity analysis for the original signal data set using an \verb!XGBoost! model. The results
will act as a benchmark when performing further testing with \ac{NN} variants. In figure \ref{fig:XGBoost} 
I have presented a grid displaying the expected significance (see section \ref{subsec:Sensitivity}) for each 
mass combination in the original signal data set. The significance presented in the grid (and will be in the next results to come) 
was calculated with equation \ref{eq:Z1}. In the figure we can observe that the \verb!XGBoost! model performs
better for smaller masses. This results can be somewhat counterintuitive, due to signal with smaller masses 
having a larger resemblance to background than signal with large masses. The explanation is simply that there are
far more events with smaller masses than there are with larger masses. By studying figure \ref{fig:nrSignal} we know that 
there are a total of 134 events with $\{200,400\}_{GeV}$\footnote{For the remainder of this thesis I will use the notation $\{A,B\}_{GeV}$, 
where A and B are the masses of $\tilde{\chi}$ and $\tilde{\chi}_2$, respectively.} compared to 6 events 
with $\{400,800\}_{GeV}$. Not only does this mean that the model will have had more small mass signals to 
train on than large mass, but also that a potential signal region would have to keep far more of the large 
mass signal to achieve a high significance.\\
\begin{figure}
    \centering
    \includegraphics[width=0.7\textwidth]{Figures/MLResults/XGB/SUSY/Grid/XGBGridSig.pdf}
    \caption{A grid displaying the expected significance on the original signal set, using the signal region 
    created by the \emph{XGBoost} model.}
    \label{fig:XGBoost}
\end{figure}
To further investigate the performance of the \verb!XGBoost! model, I have drawn the distribution of the output for the 
entire background data set as well as for signal with 4 different mass combinations. The result is found in figure 
\ref{fig:XGBDistComp}. The figure shows the output of the \verb!XGBoost! model with the full output range \ref{fig:XGBDist}
and the output ranging from [0.975,1.00]\footnote{This interval was chosen to study the higher output region.} (\ref{fig:XGBDist_95}). 
From the figure we observe a clear separation between signal and background, where most of the signal is given a high output ($>0.9$) 
and most of the background is given small output ($<0.1$). To further support the effect of statistics in the signal, we can take note of the 
amount of signal in the higher range of the output ([0.975,1]). Although the model is able to achieve a much higher 
effectiveness (is able to preserve more of the signal) for higher mass signals, there are over 4 times as many 
events with masses $\{250,400\}_{GeV}$ (the lightest mass combination in the figure) then there are of any other. Note 
that the bins are filled using the event-spesific sample weights from the simulated data which explains why some bins
are of the magnitude $10^{-2}$.
\\
A final observation from figure \ref{fig:XGBDistComp}, is the study of which \ac{SM} processes contribute the most 
for higher model output. In figure \ref{fig:XGBDist_95}, we can see that in the output range of [$0.975,1$], the highest 
contributing processes are $Diboson(lll)$, $Top-Other$ and $t\bar{t}$. On the other hand $Z-jets$, which is originally the 
largest contributing processes, is almost removed in the higher output range. This aligns with my predictions made when studying the 
different processes in section \ref{sec:bkg}. 
\begin{figure}
    \makebox[\linewidth][c]{%
    \centering
    \hfill
    \begin{subfigure}{.6\textwidth}
        \includegraphics[width=\textwidth]{Figures/MLResults/XGB/SUSY/MLDist/xgbDist.pdf}
        \caption{}
        \label{fig:XGBDist}
    \end{subfigure}
    \begin{subfigure}{.6\textwidth}
        \includegraphics[width=\textwidth]{Figures/MLResults/XGB/SUSY/MLDist/xgbDist_C7.pdf}
        \caption{}
        \label{fig:XGBDist_95}
    \end{subfigure}
    \hfill
    }
    \caption[The output distribution from a trained XGBoost model for the background and signals with 4 different mass combination.]{
    The output distribution from a trained XGBoost model for the background and signals with 4 different mass combinations:
    $\{250,400\}_{GeV}$, $\{400,650\}_{GeV}$, $\{50,700\}_{GeV}$ and $\{400,800\}_{GeV}$. 
    The figures include the full output range (\ref{fig:XGBDist}) and the output ranging from 0.975-1.00 (\ref{fig:XGBDist_95}).
    The number in parentheses indicate the fraction of each background and the absolute number of events for each signal point. 
    The total number of background events is also shown.}
    \label{fig:XGBDistComp}
\end{figure}
