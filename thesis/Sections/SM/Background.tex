\section{The Background Channels}\label{sec:bkg}
We define background as anything that is not of interest, i.e. not signal. 
As will be further explained in  later sections, we will explicitly demand a three lepton final state in all collisions 
considered in the analysis. This will remove a lot of background, but not all of it. Due to the imperfect nature of the 
reconstruction of events, a demand for three lepton final state will not be without errors. This leaves room for more 
variation in the background than one might expect. In this section I will cover the channels\footnote{By channels,
I refer to the stages by which a particle decays to a specific final state. Often this refers to 
the different particles which mediates the process from initial- to final state.} which will 
be of importance during the analysis. I will also discuss which background channels are the hardest 
to reduce in a potential new physics signal region, also called the irreducible background. These are 
channels whose features are largely indistinguishable from the signal. Note that the sections bellow
are listed by the size of the contribution to the three lepton final state data (from greatest to least).  
\\
The processes are classified similarly to how they were classified in the original \ac{MC} simulated data produced 
by ATLAS and used in this study. For the sake of making several of the figures in the thesis more readable, I decided 
to merge several of the classifications. Diboson(llll), diboson(lll) and diboson(ll) are in later figures categorized as 
simply diboson. Similarly, the three smallest processes, triboson, W-jets and Higgs, are all merged to the category
\emph{Others}.
\subsection*{Z-Jets}
The Z-jets channel contribute the most amount of events in the data. The channel consists of all events
resulting in a Z-boson alongside jets. If the Z-boson decays into two leptons, the additional 
jet can act as a fake lepton and the channel looks like a three lepton final state. In figure \ref{fig:z_pjets} 
I have written the Feynman diagram of an example of such a channel. The figure shows a quark-antiquark pair leading 
to a Z-boson and gluon. The Z boson decays into two leptons (normally $e^-e^+$ or $\mu^- \mu^+$) and the gluon hadronizes 
as a jet of hadrons which may obtain a b-hadron. From this point, a b-hadron can decay semileptonically, creating what is called
a non-prompt lepton, or be misidentified as a lepton. Both scenarios constitute what is normally called fake non-prompt leptons.

\subsection*{Diboson (lll)}
Diboson channels are defined as channels resulting in two bosons. In the case of (lll), the dibosons
decay into a total of three leptons. In figure \ref{fig:wz} I have drawn the Feynman diagram of an 
example of such a channel. The figure shows a W- and Z-boson production through a quark-antiquark pair.
The W-boson decays into a lepton with missing transverse energy and the Z-boson decays into a pair of leptons.
The similarity to the signal, i.e. a three lepton final state with missing transverse energy, makes the diboson(lll)
process hard to separate from the signal.  

\subsection*{$t\bar{t}$}\label{subsec:ttbar}
The $t\bar{t}$ channel is defined as a proton-proton collision resulting in a top quark-antiquark production 
through the strong interaction. In figure \ref{fig:ttbar} I have drawn a Feynman diagram of an example of such 
a channel. The figure shows gluon-gluon fusion producing a pair of top quarks. The top quark-antiquark pair decay into 
a bottom-quark and a W boson. The channel constitutes a background when both W bosons decay into a charged and a 
neutral lepton and one of the b-quarks are misidentified as a lepton. 

\subsection*{Diboson (llll)}
In the case of diboson (llll), the channel refers to events resulting in two Z-bosons which decay 
into four leptons. In figure \ref{fig:zz} I have drawn a Feynman diagram of an example of 
such a diagram. The figure shows a quark-antiquark pair annihilating into two Z-bosons.
The two Z-bosons decay into two pairs of leptons. This process constitutes a background when one 
of the leptons is not reconstructed in the detector.

\subsection*{Top Others}
The top other channel is similar to the $t\bar{t}$ channel, in that it results in top quarks. The main difference between 
the two, is that the top other process does not produce a top quark-antiquark pair through strong interaction of quarks. 
In figure \ref{fig:topOthers} I have drawn an example of a top other process, where a top quark-antiquark pair is produced 
from a bottom quark-antiquark pair collision mediated through a W boson. From this point, the top quark-antiquark pair decays 
to a similar state as described in section regarding the $t\bar{t}$ processes. Both the top other and $t\bar{t}$ processes exhibit
large amounts of missing transverse energy in the final state, making them both difficult to separate from the signal. 

\subsection*{Single Top}
The single top channel, similarly to top other and the $t\bar{t}$ channel also produces a top quark, but in the case of single top 
only one top quark is produced. In figure \ref{fig:topOthers} I have drawn the Feynman diagram of such a channel. The Feynman diagram displays a
top quark produced through the strong interaction of a bottom quark and a gluon. The top quark is produced through the interaction with a W boson 
which decays into a charged-neutral lepton pair and the top quark decays to a W boson and bottom quark. 

\subsection*{Diboson (ll)}
The diboson(ll), similarly to diboson(llll) produces two bosons, but instead of ending in a four lepton final state, ends in a two lepton final state.
Figure \ref{fig:ww}, displays the Feynman diagram of a diboson(ll) process where a W-pair is produced through the annihilation of two fermions through 
a charged boson. To produce the two lepton final state, the two bosons must each decay into a charged-neutral lepton pair. 

\subsection*{Triboson}
The triboson channel is defined as a proton-proton collision producing three bosons.  In figure \ref{fig:zzz}, I have drawn a Feynman diagram 
displaying an example of a triboson process. The Feynman diagram shows a quark-antiquark pair annihilating to three bosons, two W and one Z. The Z-boson decays 
to a pair of leptons, and the two W bosons each decay into a charged-neutral lepton pair resulting in a 4-lepton final state with missing transverse energy.

\subsection*{W-jets}
The W-jets is defined as a proton-proton collision producing a W boson alongside jets. In figure \ref{fig:w_pjets} I have drawn an example 
of a Feynman diagram for a W-jets processes. The Feynman diagram displays a pair of quarks colliding to form a W boson alongside a gluon. The gluon decays 
into a bottom quark-antiquark pair and the W boson decays into a charged-neutral lepton pair. Given that this only produces the one lepton, it 
relies heavily on poor reconstruction, and is hence quite rare.

\subsection*{Higgs}
Finally, we have the Higgs process, which is defined as proton-proton collision producing a Higgs boson. The Higgs boson is relatively heavy (125~Gev) and couples to mass. Therefore, the 
Higgs process is the rarest process in the data set. In figure \ref{fig:h}, I have drawn a Feynman diagram of a collision producing a Higgs boson. The Feynman diagram displays a gluon pair annihilation
through exchanging a virtual top quark, and producing a virtual top quark-antiquark pair. The latter pair, annihilate, producing a Higgs boson, which quickly (due to its mass) decay into a 
pair of Z bosons where one is virtual. The Z boson pair then further decay into a total of 4 leptons. 

\begin{figure}
    \makebox[\linewidth][c]{%
        \begin{subfigure}{.5\textwidth}
            \includegraphics[width=0.45\textwidth, angle = -90]{Figures/FDiagrams/Z_pjets.png}
            \caption{}
            \label{fig:z_pjets}
        \end{subfigure}
        \hspace{1.5cm}
        \begin{subfigure}{.5\textwidth}
            \includegraphics[width=0.45\textwidth, angle = -90]{Figures/FDiagrams/wz.png}
            \caption{}
            \label{fig:wz}
        \end{subfigure}
    }
    \\
    \newline
    \makebox[\linewidth][c]{%
        \begin{subfigure}{.5\textwidth}
            \includegraphics[width=0.45\textwidth, angle = -90]{Figures/FDiagrams/ttbar.png}
            \caption{}
            \label{fig:ttbar}
        \end{subfigure}
        \hspace{1.5cm}
        \begin{subfigure}{.5\textwidth}
            \includegraphics[width=0.45\textwidth, angle = -90]{Figures/FDiagrams/zz.png}
            \caption{}
            \label{fig:zz}
        \end{subfigure}
    }
    \\
    \newline
    \makebox[\linewidth][c]{%
        \begin{subfigure}{.5\textwidth}
            \includegraphics[width=0.435\textwidth, angle = -90]{Figures/FDiagrams/topOther.png}
            \caption{}
            \label{fig:topOthers}
        \end{subfigure}
        \hspace{1.5cm}
        \begin{subfigure}{.5\textwidth}
            \includegraphics[width=0.45\textwidth, angle = -90]{Figures/FDiagrams/singleTop.png}
            \caption{}
            \label{fig:singleTop}
        \end{subfigure}
    }
    \\
    \newline
    \makebox[\linewidth][c]{%
        \begin{subfigure}{.5\textwidth}
            \includegraphics[width=0.45\textwidth, angle = -90]{Figures/FDiagrams/ww.png}
            \caption{}
            \label{fig:ww}
        \end{subfigure}
        \hspace{1.5cm}
        \begin{subfigure}{.5\textwidth}
            \includegraphics[width=0.46\textwidth, angle = -90]{Figures/FDiagrams/WZW.png}
            \caption{}
            \label{fig:zzz}
        \end{subfigure}
    }
    \\
    \newline
    \makebox[\linewidth][c]{%
        \begin{subfigure}{.5\textwidth}
            \includegraphics[width=0.45\textwidth, angle = -90]{Figures/FDiagrams/w_pjets.png}
            \caption{}
            \label{fig:w_pjets}
        \end{subfigure}
        \hspace{1.1cm}
        \begin{subfigure}{.5\textwidth}
            \includegraphics[width=0.43\textwidth, angle = -90]{Figures/FDiagrams/h.png}
            \caption{}
            \label{fig:h}
        \end{subfigure}
    }
    \caption[Feynman diagrams of background processes.]{A collection of examples of Feynman diagrams for the \ac{SM} background processes.
    The diagrams display an example of the processes $Z-jets$ (\ref{fig:z_pjets}), $Diboson(lll)$
    (\ref{fig:wz}), $t\bar{t}$ (\ref{fig:ttbar}), $Diboson(llll)$ (\ref{fig:zz}), $TopOthers$ (\ref{fig:topOthers}),
    $SingleTop$ (\ref{fig:singleTop}), $Diboson(ll)$ (\ref{fig:ww}), $Triboson$ \ref{fig:zzz},
    $W-jets$ (\ref{fig:w_pjets}) and $Higgs$ (\ref{fig:h}).}
    \label{fig:Feynman}
\end{figure}
\newpage
