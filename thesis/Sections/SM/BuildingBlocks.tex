\section{The Building Blocks}
As early as Ancient Greece, humans pondered the nature of the most elementary building blocks of
the universe. The Greeks imagined a rope of a given length and a pair of scissors with adjustable size.
Then one could ask, how many times can you cut the rope in half? If the answer is less than infinite,
what are you left with?
\\
In 1897, Joseph John Thomson (\emph{1856-1940}) discovered the first elementary particle using the Cathode Ray Tube \cite{JJ}. 
The particle that Thomson discovered was named the \emph{electron}. Prior to the time of discovery, we believed atoms to 
be the smallest building blocks. After the discovery of the electron, the discovery of the 
proton and neutron quickly followed. It was not until more than 50 years after the discovery of 
the proton (by Ernest Rutherford \emph{1871-1937} \cite{Rutherfoord}) that we discovered that also protons and neutrons 
could be further dissected to smaller particles. We call these particles quarks. The 'final-piece'\footnote{Given the
nature of this thesis, the existence of further pieces is implied.} of the puzzle came
in 1956 \cite{Reines} when we discovered the (at that time thought of as massless) neutrino. In 
later years, we have discovered the tau, muon and their 'neutrino partners'. Together, the 
electron, muon, tau and the neutrinos form the leptons. We refer to both leptons and quarks as fermions.
\\
Upon the evolution of quantum mechanics and physics as a whole, we started to divert
our focus from 'what' and over to 'how'. How can we explain all the complex interactions
that emerge between these fundamental particles? Through the creation of the \ac{SM} and countless 
experiments, we discovered that forces are interactions between particles and fields.
The \ac{SM} describes forces as fields which are mediated through particles called gauge bosons. 
\\
The four fundamental forces responsible in the universe are electromagnetism (\ac{QED}), the weak-force, 
the strong-force (\ac{QCD}) and gravity. The most familiar boson is the photon. The photon is responsible 
for the mediation of \ac{QED} and is responsible for all electromagnetic effects, such as the ones allowing
us to see objects using our eyes. Similarly, the $W^{\pm}$ and $Z$ bosons are responsible for the weak-force which
allows for radioactive decay. And gluons are responsible for \ac{QCD} which holds the quarks inside the protons and 
neutrons together. Gravity is the only force not included in the SM, but would (if one day included)
presumably have its own force carrying particle, the graviton. 
\\
The final building block in the universe introduced and described by the \ac{SM} is the Higgs boson. The Higgs boson was 
proposed by Robert Brout, Francois Englert and Peter Higgs in 1964 and discovered at CERN in 2012 \cite{Aad_2012,the_cms_collaboration_observation_2012}. 
The Higgs boson, sometimes called the God particle, is responsible for giving mass to the particles in a process called
spontaneous symmetry breaking of the electroweak theory \cite{SSB}. Together the fermions and the bosons make up all the 
particles in the \ac{SM}.
\subsection{The Leptons} 
In table \ref{table:Leps}, a summary of all leptons is found, along with the respective mass and electric charge.  
The leptons are all elementary particles with half-integer spin\footnote{Spin is a quantum number
which predicts the effect of an applied electromagnetic field.}, $\pm 1/2$. For reasons that are yet to be known, the 
leptons come in three generations. Each generation contains a pair of charged and neutral lepton. The first generation contains the
electron, $e^-$ and the electron-neutrino, $\nu_e$. The second contains the muon, $\mu$ and the
muon-neutrino, $\nu_\mu$. And the third generation contains the tau, $\tau^-$ and $\nu_\tau$. The generations
are numbered by the mass of the charged lepton, where the first generation is the lightest. As is often the case
in particle physics, the heavier a particle, the lower the probability of it being produced in particle interactions. 
This, and the fact that they sometimes behave similarly to jets in a detector, explains why particle physicists often neglect 
the $\tau$ when speaking about leptons.
\\
The charged leptons are all massive particles ranging from a fraction of 1eV to more than $10^9$eV.
The neutrinos were up until the turn of the millennia assumed to be massless. This was not only backed by experiments
but also by the \ac{SM} which had not previously been experimentally challenged. In 1998 \cite{NeutrinoMass}, it was discovered 
that neutrinos in fact do have mass, although a very small mass. Given the size of the masses we are yet to accurately measure 
the mass of the neutrinos, but we have found them all to be less than 20 MeV\footnote{The lightest neutrino $\nu_e$, is found to 
have an upper bound of $10^{-6}$eV.}.
\begin{table}
    \centering
    $
    \begin{array}{cccc}
        \hline \text{Generation} & Flavor  &\text{ Mass [MeV]} & \text{Charge [Elementary charge]} \\
        \hline 1st & \text{e}  &\text{0.511}  & -1 \\
        1st & \text{$ \nu_e$}   &\text{$<10^{-6}$}  & 0 \\
        \hline
        2nd & \text{$\mu$}  &\text{105.66}  & -1 \\
        2nd & \text{$ \nu_\mu$}   &\text{$<0.17$} & 0 \\
        \hline
        3rd & \text{$\tau$}  &\text{1776.8} & -1 \\
        3rd & \text{$ \nu_\tau$}   &\text{$<18.2$} & 0 \\
        \hline
    \end{array}
    $
    \caption{A list of all leptons along with their generation, flavor, mass and \acs{EM} charge.}
    \label{table:Leps}
\end{table}
\subsection{The Quarks}
\begin{center}
    \hyphenblockcquote{UKenglish}{joyce1999finnegans}{
        'Three quarks for Muster Mark! \\
        Sure he hasn't got much of a bark.\\
        And sure any he has it's all beside the mark.'
        }
\end{center}
The poem above was written by James Joyce (\emph{1882-1941}) in 1939, and was the motivation for Gell-Mann (\emph{1929-2019}) 
when naming the inner particles of hadrons, quarks. We can categorize quarks as being either a down- or up-type. All down-type quarks have a 
negative electrical charge equal to 1/3 that of the electron (e) and all positive quarks have a positive charge equal to 2/3 that of the electron (+e).
Similarly to leptons, all quarks have a spin equal to 1/2 and like the leptons, are divided into three generations. Each generation
of quarks are made of a pair of one up- and one down-type quark. The first generation contains the up, $u$ and the down, $d$ quark,
the second the charm, $c$ and the strange, $s$ quark and third the top, $t$ and the bottom, $b$ quark. Table \ref{table:Quarks} presents  
a summary of all quarks, along with the respective mass and electric charge. Also similarly to leptons, the higher the generation and mass 
the more energy is needed to create them. \\  
\begin{table}[H]
    \centering
    $
    \begin{array}{cccc}
        \hline \text{Generation} & Flavour  &\text{ Mass [MeV]} & \text{Charge [Elementary charge]} \\
        \hline 1st & \text{u}  &\text{2.2}  & +2/3 \\
        1st & \text{$d$}   &\text{4.7}  & -1/3 \\
        \hline
        2nd & \text{$c$}  &\text{1280}  & +2/3 \\
        2nd & \text{$s$}   &\text{96} & -1/3 \\
        \hline
        3rd & \text{$t$}  &\text{173100} & +2/3 \\
        3rd & \text{$b$}   &\text{4180} & -1/3 \\
        \hline
    \end{array}
    $
    \caption{A list of all quarks along with their generation, flavor, mass and \ac{EM} charge.}
    \label{table:Quarks}
\end{table}
Similarly to how difference in spin allows leptons to stay in an otherwise similar quantum state, the quarks have 'color'.
The colors of quarks are what connect them to the strong-force. \ac{QCD} is what allows quarks to change color. 
It predicts asymptotic freedom when quarks are free at short distances, also known as
color confinement. Briefly explained, color confinement results in quarks never existing
in isolation but always in a quark-antiquark pair (meson) or in three quark state (baryons) such as protons and neutrons. Given 
color confinement, quarks are never directly observed in experiments, instead we detect the signature of quarks forming hadrons in a 
process called hadronization. At high energies, quark hadronization leads to narrow, collimated jets of charged particles, which explains why we call 
these signatures 'jets of hadrons', or simply jets. 
