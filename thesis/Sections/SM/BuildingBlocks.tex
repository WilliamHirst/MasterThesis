\section{The bulding blocks}
As early as ancienct greece, humans pondered the nature of the most elementary building blocks of
the universe. They imagined a rope of a given length, with a pair of scissors of adjustable size.
Then one could ask, how many times can you cut the rope in half? If the answer is less than infinite,
what are you left with?
\\
In 1897, Joseph John Thomson (\emph{1856-1940}) discovered the first elementary particle using the Cathode Ray Tube. 
This particle we later named the electron. Prior to the time of discovery, we believed atoms to 
be the smallest building blocks. After the discovery of the electron, the discovery of the 
proton and neutron quickly followed. It was not until more than 50 years after the discovery of 
the proton (by Ernest Rutherford) that we discovered that also protons and neutrons could be further
dissected to smaller particles. We call these particles quarks. The "final-piece"\footnote{Given the
nature of this thesis, the existence of further pieces is implied.} of the puzzle came
in 1956 when we discovered the, at that time thought of as massless neutrino. Together, the 
electron and the neutrino form the leptons. We refer to both the leptons and the quarks as fermions.
\\
Upon the evolution of the quantum mechanics and a physics as a whole, we started to divert
our focus from the what and over to the how. How can we explain all the complex interactions
that emerge between these relatively simple particles? Through the creation of \ac{SM} and countless 
experiment, we discovered that forces are nothing but interactions between particles and fields.
The \ac{SM} describes all forces as a field which are mediated through a particle, we call bosons. 
\\
The four forces responsible for all the forces in the universe are electromagnetism (\ac{QED}), the weak-force, the strong-force(\ac{QCD})
and gravity. The boson most familiar to most is the photon. The photon is responsible for the mediation 
of \ac{QED} and is responsible for all electromagnetic effect, such as the ones allowing
us to see objects using our eyes. Similarly, the W and Z bosons are responsible for the weak-force which
allows for radioactive decay. And the gluon is responsible for \ac{QCD} which holds protons and 
neutrinos together. Gravity is the only force not described in the SM, but would (if one day included)
have its own force carrying particle, graviton. 
\\
The final building block in the universe introduced and described by \ac{SM} is the Higgs boson.
The Higgs boson was proposed by Peter Higgs in 1964 and discovered at CERN in 2012. The Higgs boson,
also called the God particle is responsible for giving particles mass in a process called
spontaneous symmetry breaking (more on this in later sections). Together the fermions and the bosons
make up all the particles in the \ac{SM} as it now stands.
\subsection{The leptons}
The leptons are all elementary particle with half-integer spin, $\pm 1/2$. A lepton can either be charged
or neutral. For reasons that are yet to be known, the leptons come in 3 generations.
Each generation containing a pair of charged and neutral lepton. The first generation contains the
electron, $e^-$ and the electron-neutrino, $\nu_e$. The second contains the muon, $\mu$ and the
muon-neutrino, $\nu_\mu$. And the third generation contain the tau, $\tau^-$ and $\nu_\tau$. The generations
are numbered by the mass of the charged lepton, where the first generation is the lightest. As is often the case
in particle physics, the heavier a particle, the rarer. This is due to the heavier particles (higher generations) quickly
decaying into lighter particles (lower generation), in a process we call particle decay. This explains why particle physicists
neglect the $\tau$ when speaking about leptons, given that this is by far the heaviest and also the rarest.
\\
The charged leptons are all massive particles ranging from a fraction of 1eV to more than a thousand times that.
The neutrinos were up until the turn of the millennia presumed to be massless. This was not only backed by experiments
but also by the SM which seldom seemed to be wrong. In 1998 it was discovered that neutrinos in fact do have mass
although being extremely light. Given the size of the masses we are yet to accurately measure the mass of the neutrinos,
but we have found them all to be less than 20 MeV. The fact that the neutrinos in fact do have mass is a problem 
which will be discussed further in later section. In table \ref{table:Leps}, a summary of all leptons are found,
along with the respective mass and charge.  
\begin{table}
    \centering
    $
    \begin{array}{cccc}
        \hline \text{Generation} & Flavour  &\text{ Mass [MeV]} & \text{Charge [Elementary charge]} \\
        \hline 1st & \text{e}  &\text{0.511}  & -1 \\
        1st & \text{$ \nu_e$}   &\text{$<0.001$}  & 0 \\
        \hline
        2nd & \text{$\mu$}  &\text{105.66}  & -1 \\
        2nd & \text{$ \nu_\mu$}   &\text{$<0.17$} & 0 \\
        \hline
        3rd & \text{$\tau$}  &\text{1776.8} & -1 \\
        3rd & \text{$ \nu_\tau$}   &\text{$<18.2$} & 0 \\
        \hline
    \end{array}
    $
    \caption{A list of all leptons along with their generation, flavor, mass and charge.}
    \label{table:Leps}
\end{table}
\subsection{The quarks}
\begin{center}
    \hyphenblockcquote{UKenglish}{joyce1999finnegans}{
        'Three quarks for Muster Mark! \\
        Sure he hasn't got much of a bark.\\
        And sure any he has it's all beside the mark.'
        }
\end{center}
The poem above was written by James Joyce in 1939, and was the motivation for Gell-Mann when naming the 
inner particles of hadrons, quarks. Quarks were introduced to explain some strong-force
properties of hadrons. We categorize quarks as either up- or down quarks. All up-quarks have a negative charge
equal to 2/3 that of the electron and all down quarks have a positive charge equal to 1/3 that of the electron.
Similarly to leptons, all quarks have a spin equal to 1/2 and are divided in 3 generations. Each generation
of quarks are made of a pair of one up- and one down-quark. The first generation contains the up, $u$ and the down, $d$ quark,
the second the charm, $c$ and the strange, $s$ quark and third the top, $t$ and the bottom, $b$ quark. Also similarly to leptons,
the higher the generation and mass the rarer the quarks.  
\begin{table}
    \centering
    $
    \begin{array}{cccc}
        \hline \text{Generation} & Flavour  &\text{ Mass [MeV]} & \text{Charge [Elementary charge]} \\
        \hline 1st & \text{u}  &\text{2.2}  & -2/3 \\
        1st & \text{$d$}   &\text{4.7}  & +1/3 \\
        \hline
        2nd & \text{$c$}  &\text{1280}  & -2/3 \\
        2nd & \text{$s$}   &\text{96} & +1/3 \\
        \hline
        3rd & \text{$t$}  &\text{173100} & -2/3 \\
        3rd & \text{$d$}   &\text{4180} & +1/3 \\
        \hline
    \end{array}
    $
    \caption{A list of all quarks along with their generation, flavor, mass and charge.}
    \label{table:Quarks}
\end{table}
Similarly to how difference in spin allows leptons to stay in an otherwise similar quantum state, the quarks have color.
The colors of quarks are what connects them to the strong-force. The strong force is what allows quarks to change color
and also explains a phenomenon known as color confinement. Briefly explained, color confinement results in quarks never existing
in isolation but always in pairs, mesons or in threes, baryons (like protons and neutrons). Given color confinement,
quarks are never directly observed in experiments, instead we detect the signature of quarks forming hadrons in a process
called hadronisation. We call these signature jets. 
