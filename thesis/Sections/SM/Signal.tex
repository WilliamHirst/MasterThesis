\section{The Signal}\label{sec:signal}
% \subsection{Heavy Neutral Lepton}
% \begin{figure}
%     \centering
%     \makebox[0.75\linewidth][c]{%
%     \includegraphics[width=0.225\textwidth, angle = -90]{Figures/FDiagrams/HNLSignal.png}
%     }
%     \caption{The Feynman diagram of the signal-channel.}
%     \label{fig:signal}
% \end{figure}
% \subsection{Neutralinos}
In this thesis, I will compare \ac{ML} models in their ability to learn the trends of the data which will contribute  
in our ability separate the signal from the background. Specifically for this thesis, I will be studying an expansion of the 
\ac{SM} which includes super symmetry (see section \ref{subsec:SS}). The signal I will aim to separate from the background, is one 
which produces a WZ pair, through two neutralinos. In figure \ref{fig:signal}, I have drawn a Feynman diagram for such a process.
The Feynman diagram shows a pair of neutralinos, the two lightest neutralinos introduced by the \ac{MSSM} ($\tilde{\chi}_1$ and $\tilde{\chi}_2$), 
and where each neutralino produces a boson (W and Z respectively) and a $\tilde{\chi}_1$.
\begin{figure}
    \centering
    \makebox[0.75\linewidth][c]{%
    \includegraphics[width=0.225\textwidth, angle = -90]{Figures/FDiagrams/WZSignal.png}
    }
    \caption{The Feynman diagram of the signal-channel.}
\end{figure}