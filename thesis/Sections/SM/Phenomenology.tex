\section{Phenomenology - What is it?}
\subsection{The bulding blocks}
As early as ancienct greece, humans pondered the nature of the most elementary building blocks of
the universe. They imagined a rope of a given length, with a pair of scissors of adjustible size.
Then one could ask, how many times can you cut the rope in half? If the answer is less than infinite,
what are you left with?
\\
In 1897, Joseph John Thomson discovered the first elementary particle using the Cathode Ray Tube. 
This particle we later named the electron. Prior to the time of discovery, we belived atoms to 
be the smalles building blocks. After the discovery of the electron, the discovery of the 
proton and neutron quickly followed. It was not until more than 50 years after the discovery of 
the proton (by Ernest Rutherford) that we discovered that also protons and neutrons could be further
disected to smaller particles. We call these particles quarks. The "final-piece"\footnote{Given the
nature of this thesis, the existence of further pieces is implied.} of the puzzle came
in 1956 when we discovered the, at that time thought of as massless neutrino. Together
with the electron, the neutrino is defined as a lepton. 
\\
Upon the evolution of the quantum mechanics and a physics as a whole, we started to divert
our focus from the what and over to the how. How can we explain all the complex interactions
between this relativly simple particles? Through the creation of \ac{SM} and countless 
experiment, we discovered that fources are nothing but interactions between particles
through what we call, force mediating particles. The \ac{SM} decribes all forces as a fields which 
are mediated through a particles, we call bosons. 
The four forces responsible
\\
for all the forces in the universe are electro-magnetism (\ac{QED}), the weak-force, the strong-force(\ac{QCD})
and gravity. The boson most familiar to most is the photon. The photon is responsible for the mediation 
of \ac{QED} and is responisble for all electro-magnetic effect, such as the ones allowing
us to sea objects using our eyes. Similarly the W and Z bosons are responsible for the weak-force whcih
allows for radioactive decay. And the gluon is responsible for \ac{QCD} whcih holds protons and 
neutrinos together. Gravity is the only force not described in the SM, but would (if one day included)
have its own force carrying particle, graviton.
\\
The final building block in the universe introduced and described by \ac{SM} is the Higgs boson.
The Higgs boson was proposed by Peter Higgs in 1964 and discovered at CERN in 2012. The Higgs boson,
also called the God particle is responsible for giving particles mass in a process called
spontaneous symmetry breaking (more on this in later sections). 
\\
\subsection{Quantum numbers}
Each particle plays its own important part in the tapestry of the univere. What makes the particle different
and allows them to contribute in the way they do are the quantum numbers. The quantum numbers are used
to explain the attributes of the elementary particles and range from spin to charge. A table 
of all quantum numbers along with allowed values are found in table 

\begin{table}
    \centering
    $
    \begin{array}{ccc}
        \hline \text { Quantum number } & \text { Values } \\
        \hline\text{Charge} & \text{$0, \pm 1/3, \pm 1/2, \pm 2/3, \pm 1$}  \\
        \text{Spinn} & \text{$0, \pm 1/2, \pm 1$} \\
        \hline
    \end{array}
    $
    \caption{A list of relevant quantum numbers along with the respective
    possible values.}
\label{table:SG}
\end{table}
