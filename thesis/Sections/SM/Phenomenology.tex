\section{Phenomenology - What is it?}
\subsection{The bulding blocks}
As early as ancienct greece, humans pondered the nature of the most elementary building blocks of
the universe. They imagined a rope of a given length, with a pair of scissors of adjustible size.
Then one could ask, how many times can you cut the rope in half? If the answer is less than infinite,
what are you left with?
\\
In 1897, Joseph John Thomson discovered the first elementary particle using the Cathode Ray Tube. 
This particle we later named the electron. Prior to the time of discovery, we belived atoms to 
be the smalles building blocks. After the discovery of the electron, the discovery of the 
proton and neutron quickly followed. It was not until more than 50 years after the discovery of 
the proton (by Ernest Rutherford) that we discovered that also protons and neutrons could be further
disected to smaller particles. We call these particles quarks. The "final-piece"\footnote{Given the
nature of this thesis, the existence of further pieces is implied.} of the puzzle came
in 1956 when we discovered the, at that time thought of as massless neutrino. Together
with the electron, the neutrino is defined as a lepton. 
\\
Upon the evolution of the quantum mechanics and a physics as a whole, we started to divert
our focus from the what and over to the how. How can we explain all the complex interactions
between this relativly simple particles? Through the creation of \ac{SM} and countless 
experiment, we discovered that fources are nothing but interactions between particles
through what we call, force mediating particles. The one most familiar to most is the photon.
The photon is responsible for the mediation of \ac{QED} 

