\section{Beyond the Standard Model}
\subsection{Why look beyond?}
\begin{center}
    \hyphenblockcquote{UKenglish}{Kelvin}{
        'There is nothing new to be discovered in physics now.\\
        All that remains is more and more precise measurement.'
        }
\end{center}
The quote above is rumored to have been spoken by William Thompson(\emph{1824–1907}), better
known as Lord Kelvin when addressing the British Association for the Advancement
of Science in 1900. The statement was followed by a long period of advancements in the
field of physics by the likes of Max Planck (\emph{1858–1947}) and 
Albert Einstein (\emph{1879–1955}). Less than half a decade after Lord Kelvin
uttered the famous words, began the development of Quantum Mechanics. 
Just as Kelvin was wrong back then, would he be wrong today. For although \ac{SM} explains 
a large range of phenomena, there are yet many mysteries to explain in the universe and even 
problems rooted in \ac{SM}. In this section I will present some of the problems we hope to 
tackle in the future. 
\begin{itemize}
    \item \ac{SM} in its current form cannot incorporate \emph{gravity}, as it does not have sufficient symmetry. 
    The hope has been to integrate gravity into \ac{SM} through the discovery of a gravity-carrying particle, 
    the gravtion \cite{Graviton}. So far, no-such particle is found.
    \item \emph{Dark matter} and \emph{dark energy} make up more than $90\%$ of the mass in the observable universe,
    but is found to lie beyond the \ac{SM} \cite{DarkME}.
    \item Inflation is today the leading explanation to what happened in the early-stages
    (the first fraction of a second) of the universe. It explains a universe in which all space
    undergoes a rapid increase in rate of expansion. None of the fields explained by \ac{SM} are 
    capable of causing any such expansion.
    \item Finally, and the one most relevant for this analysis; what is the origin of the neutrino 
    mass and is \ac{CP} violated in the neutrino sector? 
\end{itemize}
\subsection{Neutrino-Mass problem}
Neutrinos have a special place in physics. For one, they are the only particles that
only interact by the weak force. This means that neutrinos rarely interact at all. It is often used
as a (granted for many not incredibly exciting) conversation piece that a colossal amount ($ca.10^{14}$) of 
neutrinos pass through your body every second. This is harmless to us exactly because of the rarity
of neutrinos interaction with anything. For this reason we call neutrinos ghostly. 
\\
Another reason neutrinos are special is that they exhibit flavor mixing. In 2015 Arthur McDonald (\emph{1943-}) and 
Takaaki Kajita (\emph{1959-}) were awarded the Nobel Prize for their contributions in the discovery. In simple terms,
flavor mixing is a process where a particle oscillates between different flavors. For neutrinos this 
means oscillating between $\nu_e,\ \nu_\mu$ and $\nu_\tau$. Flavor mixing is in itself not special.
Quarks have been observed to exhibit the same behavior. The reason this is interesting in the case of 
neutrinos, is that it implies that neutrinos are massive. Before this discovery, we believed neutrinos to be
massless. But why are massive neutrinos a problem?
\\
All (previously) known massive particles gain mass through interactions with the \ac{BEH} field. For particles
to gain mass they need to have a right- and left-handed\footnote{For a massles particle NB FIX .The handedness of a particle is defined
as a relation between the direction of spin and momentum for a particle. Right means the two are directed
in the same direction and left corresponds to opposite direction.} particle. So far, no right-handed neutrinos
have been observed, due to them being sterile\footnote{Non-interacting.} in the \ac{SM}. Generally there are two schools of thought for why a right-handed neutrino has not been 
observed. Either, it is very heavy, and we are yet to generate energies large enough to recreate it, or 
it is indistinguishable to the left-handed neutrino. In the first scenario we would call the right-handed 
neutrino a \emph{Dirac} fermion. This means that it is no different from any other right-handed lepton as far as mass is concerned. 
In the second the right-handed neutrino is a \emph{Majorana} fermion. A Majorana fermion is simply a lepton where the particle
and the antiparticle are the same. In this thesis I have searched for a right-handed neutrino and considered
both Majorana and Dirac neutrinos.
\subsection{Super Symmetry and the Neutralino}
A supersymmetric theory has for many years been an interesting candidate for a new physics beyond the standard 
model. \ac{SUSY} aims to alter the standard model such that mass and force (or fermions and bosons) is treated equally. 
\ac{SUSY} suggests that each \ac{SM} particle has an additional \ac{SP} which we call a sparticle. 
The sparticles all differ with half a spin from the original \ac{SM} particle. Because \ac{SUSY} is a broken symmetry, 
sparticles are expected to be much higher than the corresponding \ac{SM} masses, often in the range of 100- 1000GeV. 
The difference in spin means that the \ac{SP} of a fermion is a scalar boson and the \ac{SP} of a boson is a fermion. \ac{SUSY} is a 
candidate to fix many problems in physics, some of which are; the hierarchy problem, fixing the mass of the higgs 
and possibly the mystery of dark matter. For a more thorough explanation of \ac{SUSY} and its application, the 
reader is referred to Refs.\cite{SUSY}

