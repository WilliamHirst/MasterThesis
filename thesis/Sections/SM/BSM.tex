\section{Beyond the Standard Model}
\subsection{Why look beyond?}
\begin{center}
    \hyphenblockcquote{UKenglish}{Kelvin}{
        'There is nothing new to be discovered in physics now.\\
        All that remains is more and more precise measurement.'
        }
\end{center}
The quote above is rumored to have been spoken by William Thompson(\emph{1824–1907}), better
known as Lord Kelvin when adressing the British Association for the Advancement
of Science in 1900. The statement followed a long period of advancements in the
field of physics by the likes of James Clerk Maxwell (\emph{1831–1879}) and 
Michael Faraday (\emph{1791–1867}). It would take less than half a decade
before he would understand the magnitude of his misscalculation, when Einstein and 
Planck began the development of Quantum Mechanics. Just as Kelvin was wrong back then, 
would he be wrong today. For allthough \ac{SM} explains a large range of phenomena,
there are yet many mistories to explain in the universe and even problems rooted in \ac{SM}.
In this section I will explain some of the problems we hope to tackle in the future. 
\\ \newline
As mentioned in previous section, \ac{SM} is yet to explain \emph{gravity}. The hope has been
to integrate gravity into \ac{SM} through the discovery of a gravity-carrying particle, 
the gravtion. So far, no-such particle is found. \emph{Dark matter} and \emph{dark energy} are 
also not described by \ac{SM} , even though the two make up more than $90\%$ of the mass in the 
observable universe. Inflation is today the leading explaination to what happend in the early-stages
(the first fraction of a second) of the universe. It explains a universe in which all space
undergoes a rapid increase in rate of expantaion. None of the fields explained by \ac{SM} are 
capable of causing any such expantion. Finally, and the one most relavant for this analysis is the
neutrino-mass and \ac{CP}-violation problems, but this will be discussed in the next section.
\subsection{CP-Violation}

\subsection{Neutrino-Mass}

\subsection{Dirac or Majorana}
 