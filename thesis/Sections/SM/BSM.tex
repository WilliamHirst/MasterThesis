\section{Beyond the Standard Model}
\subsection{Why look beyond?}
\begin{center}
    \hyphenblockcquote{UKenglish}{Kelvin}{
        'There is nothing new to be discovered in physics now.\\
        All that remains is more and more precise measurement.'
        }
\end{center}
The quote above is rumored to have been spoken by William Thompson(\emph{1824–1907}), better
known as Lord Kelvin when addressing the British Association for the Advancement
of Science in 1900. The statement was followed by a long period of advancements in the
field of physics by the likes of Max Planck (\emph{1858–1947}) and 
Albert Einstein (\emph{1879–1955}). Less than half a decade after Lord Kelvin
uttered the famous words, began the development of Quantum Mechanics. 
Just as Kelvin was wrong back then, would he be wrong today. For although \ac{SM} explains 
a large range of phenomena, there are yet many mysteries to explain in the universe and even 
problems rooted in \ac{SM}. In this section I will present some of the problems we hope to 
tackle in the future. 
\begin{itemize}
    \item \ac{SM} in its current form cannot incorporate \emph{gravity}, as it does not have sufficient symmetry. 
    The hope has been to integrate gravity into \ac{SM} through the discovery of a gravity-carrying particle, 
    the gravtion \cite{Graviton}. So far, no-such particle is found.
    \item \emph{Dark matter} and \emph{dark energy} make up more than $90\%$ of the mass in the observable universe,
    but is found to lie beyond the \ac{SM} \cite{DarkME}.
    \item Inflation is today the leading explanation to what happened in the early-stages
    (the first fraction of a second) of the universe. It explains a universe in which all space
    undergoes a rapid increase in rate of expansion. None of the fields explained by \ac{SM} are 
    capable of causing any such expansion.
    \item Finally,what is the origin of the neutrino 
    mass and is \ac{CP} violated in the neutrino sector? 
\end{itemize}
\subsection{Supersymmetry, the Chargino and the Neutralino}\label{subsec:SS}
A supersymmetric theory has for many years been an interesting candidate for a new physics beyond the standard 
model. \ac{SUSY} aims to alter the standard model such that mass and force (or fermions and bosons) is treated equally. 
\ac{SUSY} suggests that each \ac{SM} particle has an additional \ac{SP} which we call a sparticle. 
The sparticles all differ with half a spin from their original \ac{SM} particle. Because \ac{SUSY} is a broken symmetry, 
sparticles are expected to be much higher than the corresponding \ac{SM} masses, often in the range of 100-1000GeV. 
The difference in spin means that the \ac{SP} of a fermion is a scalar boson and the \ac{SP} of a boson is a fermion. \ac{SUSY} is a 
candidate to fix many problems in physics, some of which are; the hierarchy problem, fixing the mass of the Higgs 
and possibly the mystery of dark matter. There are many variants of \ac{SUSY}, all of which introducing a set of new 
particles. In this thesis I will study a signal which stems from the simplest variant of \ac{SUSY}, which is the most 
similar to \ac{SM}, namely the \ac{MSSM}. 
\\
In this thesis I will be studying \ac{ML} models as they analyze data including particles introduced by \ac{MSSM}, the \emph{neutralino} ($\tilde{\chi}_{1,2,3,4}$\footnote{I will in this thesis 
use the notation of $\tilde{\chi}_{1,2,3,4}$ to refer to the neutralino, instead of $\tilde{\chi}^0_{1,2,3,4}$}) and 
the \emph{chargino}($\tilde{\chi}^{\pm}_{1,2}$). The neutralino is the lightest sparticle introduced by \ac{MSSM}, and is therefore stable, as it can not decay into a lighter sparticle. 
It is the sparticle of a mixture of the neutral gauge bosons introduced by the weak force, and the Higgs gauge boson, making the neutralino a 
fermion. Similar to the neutrino, the neutralino has no \ac{EM} charge, and only interacts through the weak force, making it an ideal candidate 
for dark matter. The chargino is similar to the neutralino, with the exception that it is charged in the \ac{EM} force. 
For a more thorough explanation of \ac{SUSY} and its application, the reader is referred to \cite{SUSY}. 