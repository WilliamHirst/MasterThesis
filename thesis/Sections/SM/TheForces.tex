\section{The Forces}
Why do the nuclei of atoms stick together? Why do the electrons revolve
around said nuclei? And why can neutrons convert to protons and vice versa? These phenomena 
are all described through the interactions of leptons and quarks. But how do we describe the 
interactions? In the previous section I briefly mentioned that the interactions of the leptons 
and quarks (or fermions) are mediated by the gauge bosons. The explanation of force as 
a mediation of bosons is the cornerstone of the \ac{SM} and is explained through 
the introduction of \ac{QFT}. \ac{QFT} is a precise mathematical framework which relies on the 
properties of local symmetries (Gauge symmetries) and field theory. Given the scope of this thesis
an introduction to \ac{QFT} will not be given, yet certain attributes of the forces themselves
are of interest. 
\subsection{Electromagnetism}
The \acf{EM} force is one of the two macroscopic forces\footnote{The other being gravity.}.
This means that most people have directly experienced it and therefore have built an intuition for
it. If you place two oppositely charged objects close enough, they attract. The closer they are, 
the more they attract. Electric charge is the property that causes particles to 
experience electromagnetic forces. This is due to the boson responsible for mediating it, 
the photon ($\gamma$). The photon is massless, and only couples to particles with an electric charge,
which means only particles with a non-zero charge can interact through the electromagnetic force.
\subsection{The Strong Force}
The strong force is, alongside the weak force a microscopic force. The strong force 
holds nuclei together and is (as the name suggests) the strongest force. Similarly to how electric 
charge plays a role in electromagnetism, the strong force has \emph{color}. Not to be mistaken 
with the spectrum of frequencies of light, color in the \ac{SM} is known as the charges 
associated with the strong force. There are three conserved color charges, 'r', 'b' and 'g', or red, blue and green. 
Unlike photons which are neutral, the gluons carry color and anticolor and only couple to 
colorful\footnote{Meaning particles with a non-zero color charge.} particles. 
Due to the three color charges, there exists a collection of independent color states, 8 which 
corresponds to the number of gluons. The only particles with color are the quarks, antiquarks and gluons, which explain why 
only they experience the strong force. 
\subsection{The Weak Force}
The weak force is the weakest of all the forces of the \ac{SM}. This is due to the bosons 
coupling weakly to fermions, and the fact that the $W^{\pm}$ and $Z$ are massive, which makes them rare. 
The charged weak force is the only force which allows for flavor change\footnote{Flavor is a term used to differentiate the 
fermions, i.e. the six leptons (electron, muon, electron neutrino etc.) and six quarks (top, bottom, charm etc.).
A change in flavor therefore means to transition from one of these flavor to another through the weak 
charge bosons. For example $\mu_e \rightarrow e^{-}W^+$ }. They interact only with left-handed particles. The charge 
associated with the weak force is the weak isospin. All left-handed fermions and right handed-handed antifermions 
have non-zero isospin (1/2), meaning the W and Z bosons interact with both the leptons and the quarks.  