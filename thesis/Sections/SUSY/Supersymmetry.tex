\section{Supersymmetry}
The basis of Supersymmetry(SUSY) is that it is a symmetry relating fermionic and bosonic states. For the last decades it has been one of the most prominent extensions of the Standard model of particle physics (SM), and it has therefore been extensively studied. Theoretically, there is no doubt that SUSY plays an important role, as it has been used to prove index theorems [ref to Witten], deriving positive energy theorems, setting lower bounds on soliton masses and in construction of consistent fermionic string theories. The caveat is that despite the enormous effort in its study, there is no experimental evidence of SUSY, so the big question is if SUSY is a symmetry of nature?

\subsection{Hierarchy problem}

\subsection{Supersymmetry Algebra}
A base concept of all Quantum Field Theories is that they are invariant under spacetime transformations of the Poincare group. This is the group of all transformations on the form
\begin{align}
    x^{\mu}\rightarrow x'^{\mu}=\Lambda^{\mu}_{\nu}x^{\nu}+a^{\mu}
\end{align}

\medskip
The idea of Supersymmetry originated from attempts to look for extensions of the Poincare group. In 1967, Coleman and Mandula(ref) proved a theorem that says:
\begin{itemize}
    \item In a generic quantum field theory, under a number of reasonable and physical assumptions, like locality, causality, positivity of energy, finiteness of particle number, etc..., the only possible continous symmeties of the S-matrix are those generated by the Poincare group generators, $P_{\mu}$ and $M_{\mu\nu}$, plus some internal symmetry group G where the generators $B_{i}$ of G commutes with them
    \begin{align*}
        [P_{\mu},B_{i}]=[M_{\mu\nu},B_{i}]=0
    \end{align*}
\end{itemize}
From this one draws the conclusion that any extension of the Poincare group to include gauge symmetries is isomorphic to the symmetry group $G\times P^{\uparrow}_{+}$, or in other words the most general symmetries of the S-matrix is $G\times P^{\uparrow}_{+}$. Now since Supersymmetry is a symmetry that relates fermions and bosons, which has different space-time properties, it is indeed a space-time symmetry. Therefore the theorem by Coleman and Mandula seems to shut down the possibility of relating fermions and bosons. The next question is then: What happens if one tries to loosen some of the assumptions at the basis of this theorem?

As is well known the internal symmetry group of the SM, $G_{SM}$,  are Lie groups with Lie algebra valued generators. Therefore they obey a set of commutation relations, and are therefore bosonic in character. However, we know that there exists fermions, and there is no apparent reason to believe that nature only prefers bosonic generators, and not fermionic generators. One of the assumptions in Coleman and Mandulas theorem is that the symmetry algebra only includes commutators, but what happens if one weakens this assumption and allows for anti-commutation relations?

In 1975 Haag, Lopuszanski and Sohnius introduced the concept of graded Lie algebras to evade Coleman and Mandulas theorem, where they showed that by introducing fermionic generators in a certain way, the set of allowed symmetries could be enlarged. They also showed that Supersymmetry is the only possible such option. This makes the Poincare group becoming SuperPoincare, and then the most general symmetry of the S-matrix is $SP\times G$.

The extension restricts the possible supersymmetries acceptable in a Quantum Field Theory with interactions, as only theories with one spinorial charge $Q_{\alpha}$, known as $N=1$ supersymmetry, allows for chiral fermions which are important for phenomenology. For this reason we only consider the $N=1$ case, where a single set of four fermionic generators are introduced through a two component Weyl spinor $Q_{a}$ and its Hermitian conjugate $(Q_{a})^{\dagger}=\Bar{Q}_{\dot{a}}$. The supersymmetric extension to the Poincare algebra is then given by the following set of commutation and anti-commutation relations
\begin{align*}
    [P_{\mu},Q_{\alpha}]&=0
    \\
    [Q_{\alpha},M^{\mu\nu}]&=\frac{1}{2}(\sigma^{\mu\nu})_{\alpha}^{\beta}Q_{\beta}
    \\
    \{\}
    \\
    \{Q_{\alpha},\Bar{Q}_{\beta}\}&=2(\gamma^{\mu})_{\alpha\beta}P_{\mu}
\end{align*}
This shows that the superalgebra closes to yield the the generator of the Poincare group, which shows that it is indeed a space-time symmetry. By taking the trace of the last identity, we get that the Hamiltonian is given by
\begin{align*}
    H=P^{0}=\frac{1}{4}(\Bar{Q_{1}}Q_{1}+Q_{1}\Bar{Q_{1}}+Q_{2}\Bar{Q_{2}}+\Bar{Q_{2}}Q_{2})
\end{align*}
So by considering some state $\ket{\Psi}$ we get that the expectation value of the Hamiltonian is
\begin{align*}
    \ev{H}{\Psi}\geq 0
\end{align*}
Therefore the expectation value is zero only if the state $\ket{\Psi}$ is annihilated by all supersymmetry generators. Such a state would have to be the state with lowest possible energy, which of course is the vacuum state $\ket{0}$. From basic principles in physics, we know that if we have a symmetry it commutes with the Hamiltonian and the ground state has to be invariant. Therefore we have that
\begin{align}
    Q_{a}\ket{0}=\Bar{Q}_{\dot{a}}\ket{0}=0
\end{align}
Which means that $\ev{H}{0}=0$, i.e the ground state has zero energy if supersymmetry is preserved. The converse situation where $H\ket{0}\neq0$ means that supersymmetry is broken, which is a relation that will be useful when we are forced to break supersymmetry. The scenario with zero vacuum energy is the first indication of cancellation between fermions and bosons. In non-supersymmetric Quantum Field Theories the zero point energy of the bosonic oscillators is positive and add up to infinity, whereas fermionic oscillators add up to negative infinity. These are generally set to zero as energy is measured in terms of energy differences, but in supersymmetry they naturally cancel.
\medskip

\subsection{Irreducible representations of the Superalgebra}
As usual in physics we want the members of our group, i.e the generators, to act on physical states. Therefore we must ask the question of what kind of particles transform under the SuperPoincare group? Which is the same as asking: what are the irreducible representations of the SuperPoincare group?

From Schur's lemma we know that for any irreducible representation of a Lie group, the Casimir operator(s) are proportional to the identity. It can be shown that the Casimir operators of the SuperPoincare group is $P^{2}=P_{\mu}P^{\mu}$ and $C^{2}$(ref to KW for general expressions).

Without loss of generality one can go to a particles rest frame $P=(m,0)$, and a state conveniently labeled by the $m$, will as expected give eigenvalue $m^{2}$ when acted upon by $P^{2}$. It can further be shown that in this scenario the other Casimir operator takes the form(KW)
\begin{align}
    C^{2}=2m^{4}J_{k}J^{k}
\end{align}
where $J_k$ is an abstraction of the spin operator as it obey the same algebraic strucutre. Then a state $\ket{m,j,j_{3}}$ when acted upon by $C^{2}$ will give
\begin{align}
    C^{2}\ket{m,j,j_3}=-m^{4}j(j+1)\ket{m,j,j_3}
\end{align}
where $j=0,\frac{1}{2},1,...$ and $j_3=-j,-j+1,...,j$, since $J_k$ obeys the angular momentum commutation relation. Therefore one concludes that the irreducible representaitions of the superalgebra can be labeled by $(m,j)$, and any given set of $m$ and $j$ will give $2j+1$ states with different $j_3$.

We are now in a position to construct all the states for a given representation labeled bu $(m,j)$, and it is convenient to use the Weyl spinor representation of the $Q$ generators. It can be shown that there exists a state $\ket{\Omega}$, such that
\begin{align}
    Q_{A}\ket{\Omega}=0
\end{align}
which we call the Clifford vacuum. By using the explicit expression for $J_{k}$
\begin{align}
    J_{k}=S_{k}-\frac{1}{4m}\bar{Q}_{\dot{B}}\bar{\sigma}_{k}^{\dot{B}A}Q_{A}
\end{align}
giving
\begin{align}
    J_{3}\ket{\Omega}=S_{k}\ket{\Omega}=j_{3}\ket{\Omega}
\end{align}
meaning that $s=j$ and $s_{3}=j_{3}$ are the eigenvalues of $S^{2}$ and $S_{3}$ for the clifford vacuum. We can construct three more states from the Clifford vacuum
\begin{align}
    &\bar{Q}^{\dot{1}}\ket{\Omega}
    \\
    &\bar{Q}^{\dot{2}}\ket{\Omega}
    \\
    &\bar{Q}^{\dot{1}}\bar{Q}^{\dot{2}}\ket{\Omega}
\end{align}
Specifically, we can show that
\begin{align}
    S_{3}\bar{Q}^{\dot{A}}\ket{\Omega}&=\big(j_{3}\pm\frac{1}{2}\big)\bar{Q}^{\dot{A}}\ket{\Omega}
    \\
    S_{3}\bar{Q}^{\dot{1}}\bar{Q}^{\dot{2}}\ket{\Omega}&=j_{3}\bar{Q}^{\dot{1}}\bar{Q}^{\dot{2}}\ket{\Omega}
\end{align}
In total this means that each set of quantum numbers $m,j,j_3$ gives two states with $s_3=j_3$, and two states with $s_3=j_3\pm\frac{1}{2}$, giving two bosonic and two fermionic states. This highlights the feature of supersymmetric theories that there are an equal number of bosonic and fermionic states, with the same mass.

\subsection{The $j=0$ and $j=1/2$ irreducible representations}
For $j=0$ we must have that $j_3=0$ and as a result the Clifford vacuum $\ket{\Omega}$ must have $s=0$, and is a bosonic state. Further, there are two states $\bar{Q}^{\dot{A}}\ket{\Omega}$ with $s=\frac{1}{2}$, and $\bar{Q}^{\dot{1}}\bar{Q}^{\dot{2}}\ket{\Omega}$ with $s=0$. In total there are two bosonic states and two fermionic states, which we later will represent with what we call a \emph{scalar superfield}.
\medskip
For $j=\frac{1}{2}$ we have two Clifford vacua with $s=\frac{1}{2}$ and $s_{3}=\pm\frac{1}{2}$, which we label $\ket{\Omega;\frac{1}{2}}$ and $\ket{\Omega;-\frac{1}{2}}$. Further, we construct from each of these two new fermionic states $\bar{Q}^{\dot{1}}\bar{Q}^{\dot{2}}\ket{\Omega;\pm\frac{1}{2}}$ with $s_{3}=\mp\frac{1}{2}$. In addition we have two states with $s_3=0$, one state with $s_3=1$, and one state with $s_3=-1$. In total there are four fermionic states and four bosonic states. When talking about particles, these correspond to two fermions, one vector boson and one scalar particle. These states will later be referred to as the \emph{vector superfield}.


\subsection{Supermultiplets}
Since the single-particle fermionic and bosonic states transform into one another under $Q_{a}$ and $\bar{Q}_{\dot{a}}$, they are called \textit{superpartners}. These are combined in \textit{supermultiplets} that form irreucible representations of the superalgebra. From the relation $\{Q_{a},Q_{\dot{a}}\}\sim P$, with $P$ a one-to-one mapping, the number of bosonic and fermionic degrees of freedom in a supermultiplet must be equal, $n_B=n_F$.

As the most general symmetry group is given by the direct product $SP\times G$, the supersymmetry generators commutes with the gauge group generators, from which it follows that all particles in a supermultiplet will have the same gauge quantum numbers. However, in the Standard Model there are no fermion-boson pairs with the same mass and gauge quantum numbers, and for this reason supersymmetry must be a broken symmetry in nature's current vacuum state. The breaking of supersymmetry we alluded to when we studied the Hamiltonian in terms of the operators $Q_{a}$, and is a topic that we will return to later.
\medskip

In the simplest case of $N=1$ supersymmetry, there are only two relevant supermultiplets, which are called \textit{chiral multiplets} and \textit{vector mulitplets}. The chiral multiplets are the smallest possible irreducible representation, and contains a single Weyl fermion with $n_F=2$, and a complex scalar field with $n_B=2$\footnote{Or two real scalar fields $\phi_1$ and $\phi_2$ with $n_B=1$ each, that are assembled into a complex scalar field}. The fields in each multiplet must transform according to the same representation of any gauge symmetry. We know from the SM that the left handed fermions transform differently under the gauge group than the right handed fermions, and chiral multiplets are the only multiplets that has this property. Therefore SM fermions must be members of chiral supermultiplets, with the consequence that the superpartners of quarks and leptons are spin-0 bosons. These superpartners are of course scalars, which is called \textit{scalar fermions} or \textit{sfermions}. The SM fermions are Dirac particles\footnote{Maybe with the exception of neutrinos}, which has left and right-handed pieces that are two-component Weyl spinors. Therefore each of these have a complex scalar partner, e.g the electron $e$ has two scalar partners, $\Tilde{e}_{L}$ and $\Tilde{e}_R$. The selectrons are scalars so they do not have a handedness, thus the $L$ and $R$ are only names to indicate which component of the Dirac fermion they are superpartners of.

Vector supermultiplets contains a vector boson and a Weyl spinor, both with two degrees of freedom. The vector bosons of the SM transform in the adjoint representation, thus the members of a vector supermultiplet must belong to the adjoint representation of the gauge group. The superpartners of the SM gauge bosons are fermions and are called \textit{gauginos}, e.g the supersymmetric partner of the gauge boson \textit{gluino} are called \textit{gluino}.


\subsection{Constructing Supersymmetric Lagrangians}
The construction of supersymmetric field theories can be done in to different but equally valid set of formalisms. The first is the regular Quantum Field Theory approach, with the formulation in terms of regular spacetime dependent component fields (see Stephen Martin). This formalism is very explicit, but can tend to be extremely tedious as there are an enormous amount of possible terms that has to be written out explicitly. The second and more elegant approach is to use the language of \textit{superfields}, where there are one superfield per supermultiplet.

The superfields are constucted by defining them as functions on \textit{superspace}\footnote{Introduced by Salam and Strathdee}, which is an extension of Minkowski spacetime with a set of four anticommuting co-ordinates. From this formalism, the regular spacetime dependent Lagrangian are found by integrating out the fermionic co-ordinates, a process that has the advantage of keeping only the supersymmety-invariant terms in the final Lagrangian.

\subsection{Superspace}
In superspace the invariance of supersymmetry transformations are manifest, just as invariance of Lorentz transformations are manifest in Minkowski space. Superspace is an eight dimensional manifold that are constructed from the coset space of the super-Poincare group and the Lorentz group, $SP/L$. Co-ordinates are then given by $z^{\pi}=(x^{\mu},\theta^{a},\bar{\theta}_{\dot{a}})$, where $x^{\mu}$ are the usual Minkowski co-ordinates, and $\theta^{a}(\bar{\theta}_{\dot{a}})$ are four anti-commuting Grassmann numbers, being the parameters of the $Q$ operators in the superalgebra. These Grassmann numbers we have labeled by $a,\dot{a}$ as we want four of them and in addition we want to place them in Weyl spinors. As Grassmann numbers anti-commute, we have the following relations
\begin{align}
    \{\theta^{a},\theta^{b}\}=...=0
\end{align}
Giving the relations
\begin{align}
    \theta_{a}^{2}&=\theta_{a}\theta_{a}=-\theta_{a}\theta_{a}=0
    \\
    \theta^{2}&\equiv\theta^{a}\theta_{a}=-2\theta_{1}\theta_{2}
    \\
    \bar{\theta}^{2}&\equiv\bar{\theta}_{\dot{a}}\bar{\theta}^{\dot{a}}=-2\bar{\theta}^{\dot{1}}\bar{\theta}^{\dot{2}}
\end{align}
This has the important consequence that if we have a generic function $f(\theta)$, that can be expanded in a power series, it is given by
\begin{align*}
    f(\theta)=a+b\theta_{a}
\end{align*}
as all the higher order terms vanish. Further, Grassmann calculus becomes very simple. Differentiating $f(\theta)$ just gives
\begin{align}
    \pdv{f}{\theta_{a}}=a
\end{align}
and by defining the integrals
\begin{align}
    \int d\theta_{a}&\equiv0
    \\
    \int d\theta_{a}\,\theta_{a}&\equiv1
\end{align}
with the condition of linearity, then you can easily see that
\begin{align}
    \int d\theta_{a}\,f(\theta_{a})=\pdv{f}{\theta_{a}}=a
\end{align}
meaning that with Grassmann numbers integration and differentiation are the same operation.

Integration over multiple Grassmann numbers are
\begin{align}
    \int d^{2}\theta\,\theta\theta&=1
    \\
    \int d^{2}\bar{\theta}\,\bar{\theta}\bar{\theta}&=1
    \\
    \int d^{4}\theta\,(\theta\theta)(\bar{\theta}\bar{\theta})&=1
\end{align}
where the volume elements have been defined as
\begin{align}
    d^{2}\theta&=-\frac{1}{4}d\theta^{a}d\theta_{a}
    \\
    d^{2}\bar{\theta}&=-\frac{1}{4}d\bar{\theta}_{\dot{a}}d\bar{\theta}^{\dot{a}}
    \\
     d^{4}\theta&=d^{2}\theta d^{2}\bar{\theta}
\end{align}

Further we are interested in the differential representation of the supersymmetric generators. In order to find their expression we use the exponential map, from which we know a group element of $g\in SP$ can be written in the following way
\begin{align}
    g=\exp\big(-ix^{\mu}P_{\mu}+i\theta^{A}Q_{A}+i\bar{\theta}_{\dot{A}}\bar{Q}^{\dot{A}}-\frac{i}{2}w_{\rho\nu}M^{\rho\nu}\big)
\end{align}
where $x^{\mu}, \theta, \bar{\theta}$ and $w_{\rho\nu}$ are the parametrisation of the group, and $P_{\mu},Q_{A},\bar{Q}^{\dot{A}}$ and $M^{\rho\nu}$ are the generators. We can now parametrise the coset space $SP/L$ by simply setting $w_{\mu\nu}=0$, and the remaining parameters of $SP/L$ span superspace. In order to show the effect of supersymmetry transformations, we begin by noting that any $SP$ transformation can effectively be written as
\begin{align*}
    L(a,\alpha)=\exp\big(-ia^{\mu}P_{\mu}+i\theta^{A}Q_{A}+i\bar{\theta}_{\dot{A}}\bar{Q}^{\dot{A}}\big)
\end{align*}
beacuse one can show that
\begin{align*}
    \exp\big(-\frac{i}{2}w_{\mu\nu}M^{\mu\nu}\big)L(a,\alpha)=L(\Lambda a,S(\Lambda)\alpha)\exp\big(-\frac{i}{2}w_{\mu\nu}M^{\mu\nu}\big)
\end{align*}
which mean that all Lorentz boost does is to transform spacetime coordinates by $\Lambda(M)$ and Weyl spinors by $S(\Lambda(M))$, which is a spinor representation of $\Lambda(M)$. Thus, we can pick frames, do our thing with the transformation, and boost back to any frame we wanted. Now we can find the transformation of superspace coordintes under a supersymmetry transformation, just as we have all seen the transformation of Minkowski coordinates under Lorentz transformations. The effect of $L(a,\alpha)$ on a superspace coordinate $z^{\pi}$ is defined by the mapping $z^{\pi}\rightarrow z'^{\pi}$, given by
\begin{align*}
    \exp{iz'^{\pi}K_{\pi}}=L(a,\alpha)\exp{iz^{\pi}K_{\pi}}
\end{align*}
where $z^{\pi}=(x^{\mu},\theta_{A},\bar{\theta}^{\dot{A}})$ and $K_{\pi}=(P_{\mu},Q_{A},\bar{Q}^{\dot{A}})$. Writing this out we get
\begin{align*}
    \exp{iz'^{\pi}K_{\pi}}&=L(a,\alpha)\exp{iz^{\pi}K_{\pi}}
    \\
    &=\exp\big(-ia^{\mu}P_{\mu}+i\theta^{A}Q_{A}+i\bar{\theta}_{\dot{A}}\bar{Q}^{\dot{A}}\big)\exp{iz^{\pi}K_{\pi}}
\end{align*}
Using the Baker-Hausdorff formula, we can show that the commutator $$[(-ia^{\mu}P_{\mu}+i\theta^{A}Q_{A}+i\bar{\theta}_{\dot{A}}\bar{Q}^{\dot{A}}),iz'^{\pi}K_{\pi}]\propto P_{\mu}$$
which commutes with all operators, and therefore all the higher commutators in the expansion are zero. So effectively this means that we are left with
\begin{align*}
    \exp{iz'^{\pi}K_{\pi}}=\exp{-i(x^{\mu}+a^{\mu}-i\alpha^{A}\sigma^{\mu}_{A\dot{A}}\bar{\theta}^{\dot{A}}+i\theta^{A}\sigma^{\mu}_{A\dot{A}}\bar{\alpha}^{\dot{A}})P_{\mu}+i(\theta^{A}+\alpha^{A})Q_{A}+i(\bar{\theta}_{\dot{A}}+\bar{\alpha}_{\dot{A}})\bar{Q}^{\dot{A}}}
\end{align*}
From which we draw the conclusion that superspace coordinates transform under supersymmetry transformations as
\begin{align*}
    (x^{\mu},\theta^{A},\bar{\theta}_{\dot{A}})\rightarrow f(a^{\mu},\alpha^{A},\bar{\alpha}_{\dot{A}})=(x^{\mu}+a^{\mu}-i\alpha^{A}\sigma^{\mu}_{A\dot{A}}\bar{\theta}^{\dot{A}}+i\theta^{A}\sigma^{\mu}_{A\dot{A}}\bar{\alpha}^{\dot{A}},\theta^{A}+\alpha^{A},\bar{\theta}_{\dot{A}}+\bar{\alpha}_{\dot{A}})
\end{align*}
Now we can write down the differential representation of the supersymmetry generators by applying the standard expression for generators $X_i$ of a Lie algebra, given the functions $f_{\pi}$ for the transformation of the parameters:
\begin{align*}
    X_{i}=\pdv{f_{\pi}}{a_i}\pdv{}{z_{\pi}}
\end{align*}
which gives us
\begin{align*}
    P_{\mu}&=i\partial_{\mu}
    \\
    iQ_{A}&=\partial_{A}-i(\sigma^{\mu}\bar{\theta})_{A}\partial_{\mu}
    \\
    i\bar{Q}^{\dot{A}}&=\partial^{\dot{A}}-i(\sigma^{\mu}\theta)^{\dot{A}}\partial_{\mu}
\end{align*}


\subsection{Superfields}
In general, any superfield as a function of superspace $\Phi(x^{\mu},\theta,\bar{\theta})$, can be taylor expanded in the Grassmann valued variables, with spacetime dependent components. A general superfield can then be written as
\begin{align}
    \Phi &= a(x)+\theta\xi(x)+ \bar{\theta}\bar{\chi}(x)+\theta\theta b(x)+\bar{\theta}\bar{\theta} c(x)+\bar{\theta}(\bar{\sigma}^{\mu})\theta v_{\mu}(x)\nonumber
    \\
    &\hspace{1cm}+\bar{\theta}\bar{\theta}\theta\eta(x)+\theta\theta\bar{\theta}\bar{\psi}(x)+\theta\theta\bar{\theta}\bar{\theta}d(x)
\end{align}
where $a(x), b(x)$, $c(x)$ and $d(x)$ are complex scalar fields (or complex pseduo scalar). $\xi(x)$ and $\eta(x)$ field are left-handed Weyl spinors, $\bar{\chi}(x)$ and $\bar{\psi}(x)$ are right-handed Weyl spinors, and lastly $V_{\mu}$ is a bosonic vector field.

As we know from classic quantum field theory, we want the Lagrangian to be invariant under certain kinds of symmetry transformations, like Lorentz transformations and gauge transformations. In the case of local gauge transformations we need to introduce the covariant derivative in order to compare points living in different tangent spaces. The same idea applies for SUSY transformations, and the general supersymmetric covariant derivatives can be written as
\begin{align}
    D_{A}&\equiv\partial_{A}+i(\sigma^{\mu}\bar{\theta})_{A}\partial_{\mu}
    \\
    \bar{D}^{\dot{A}}&\equiv-\partial^{\dot{A}}-i(\sigma^{\mu}\theta)^{\dot{A}}\partial_{\mu}
\end{align}
which acts on the superfields $\Phi$.
\subsection{Chiral Supermultiplets}
A chiral supermultiplet is represented by a chiral superfield, and as by its defining name we distinguish between left-chiral and right-chiral superfields. By definition these must obey the following constraints
\begin{align}
    \bar{D}_{\dot{A}}\Phi&=0\hspace{2cm}(\text{left-chiral})
    \\
    {D}^{A}\Phi^{\dagger}&=0\hspace{2cm}(\text{right-chiral})
\end{align}
With the further requirement that $\Phi$ should be Lorentz scalars or pseudo-scalars, it can be shown that the left and right-handed chiral fields can be written in terms of their component fields as
\begin{align}
    \Phi(x,\theta,\bar{\theta})&=\phi(x)+i(\theta\sigma^{\mu}\bar{\theta})\partial_{\mu}\phi(x)-\frac{1}{4}\theta\theta\bar{\theta}\bar{\theta}\partial_{\mu}{\partial^{\mu}}\phi(x)+\sqrt{2}\theta\psi(x)
    \\
    &\hspace{1cm}-\frac{i}{\sqrt{2}}\theta\theta\partial_{\mu}\psi(x)\sigma^{\mu}\bar{\theta}+\theta\theta F(x)
\end{align}
where $\phi(x)$ and $F(x)$ are complex scalars, and $\psi_{A}$ are a left handed Weyl spinor. Taking the hermitian conjugate we will get the complex versions $A^{*}$ and $F^{*}$ together with a right handed Weyl spinor $\psi^{\dot{A}}$. Now we are in a position to compare the above with the $j=0$ irreducible representation, which had two scalar states and two fermionic states. After applying the e.o.m we see that the auxiliary field $F$ can be eliminated together with two fermionic degrees of freedom. That leaves us with two scalar states and two fermionic states, exactly the same as in the $j=0$ representation.

However, in the SM we have Dirac fermions and Weyl spinors can not alone describe them. Therefore the scalar superfields will not directly correspond to particle states for the SM, but if we construct particle representations by taking one left handed scalar superfield and one different right handed scalar superfield, we form a Dirac fermion and two scalars (and their antiparticles) after applying the e.o.m.

\subsection{Vector Supermultiplets}
The vector superfield $V(x,\theta,\bar{\theta})$ is used to represent the vector supermultiplet. A vector superfield is obtained by requiring it to be real
\begin{align}
    V^{\dagger}(x,\theta,\bar{\theta})=V(x,\theta,\bar{\theta})
\end{align}
With this constraint it can be shown by taking the the general superfields $\Phi^{\dagger}$ and $\phi$, that a general vector superfield can be written as
\begin{align*}
    V(x,\theta,\bar{\theta})&=f(x)+\theta\psi(x)+\bar{\theta}\bar{\psi}(x)+\theta\theta m(x)+\bar{\theta}\bar{\theta}m^{*}(x)
    \\
    &\hspace{0.5cm}+\theta(\sigma^{\mu})\Bar{\theta} \,V_{\mu}(x)+\theta\theta\bar{\theta}\bar{\lambda}(x)+\bar{\theta}\bar{\theta}\theta\lambda(x)+\theta\theta\bar{\theta}\bar{\theta}d(x)
\end{align*}
where $f(x)$, $d(x)$ are real scalar fields, $\psi(x)$ and $\lambda(x)$ are Weyl spinors, $m(x)$ is a complex scalar field and $V_{\mu}(x)$ is a vector field. Examples of a vector superfield are $V=\Phi^{\dagger}\Phi$, $\Phi^{\dagger}+\Phi$ and $i(\Phi^{\dagger}-\Phi)$. It is however unfortunate that this does not correspond to the wanted degrees of freedom in the $j=1/2$ representation of the superalgebra. There is of course a solution, and that is gauge freedom. Given a vector superfield $V(x,\theta,\bar{\theta})$, an abelian supergauge transformation is defined as
\begin{align}
    V'(x,\theta,\bar{\theta})=V(x,\theta,\bar{\theta})+i(\Lambda(x,\theta,\bar{\theta})-\Lambda^{\dagger}(x,\theta,\bar{\theta}))
\end{align}
Using the gauge freedom, we can choose component fields to eliminate degrees of freedom. One particular choice is the Wess-Zumino gauge, which defines
\begin{align}
    \chi(x)&=-\frac{1}{\sqrt{2}}\psi(x)
    \\
    F(x)&=-m(x)
    \\
    \alpha(x)&=\phi(x)+\phi^{*}(x)=-f(x)
\end{align}
Thus, a vector superfield in the WZ gauge can be written as
\begin{align}
    V_{WZ}(x,\theta,\bar{\theta})=\frac{1}{2}(\theta\sigma^{\mu}\theta)\big(V_{\mu}(x)+i\partial_{\mu}\alpha(x))\big)+\theta\theta\bar{\theta}\bar{\lambda}(x)+\bar{\theta}\bar{\theta}\lambda(x)+\theta\theta\bar{\theta}\bar{\theta}d(x)
\end{align}
which contains one real scalar d.o.f, three gauge field d.o.f and four fermion d.o.f, corresponding to the $j=1/2$ representation. The WZ gauge is particularly convenient for calculations as
\begin{align}
    V_{WZ}^{2}=\frac{1}{2}\theta\theta\bar{\theta}\bar{\theta}\big(V_{\mu}(x)+i\partial_{\mu}\alpha(x)\big)\big(V_{\mu}(x)+i\partial_{\mu}\alpha(x)\big)
\end{align}


\subsection{Supersymmetric Lagrangian}
A basic feature of all field theories is that the symmetry transformations of the Lagrangian $\mathcal{L}$, should leave the action $\mathcal{S}$ invariant
\begin{align}\label{eq:Action}
    \mathcal{S}=\int d^{4}x\,\mathcal{L}
\end{align}
This is fulfilled as long as the symmetry transformation leaves the Lagrangian invariant up to a total derivative
\begin{align*}
    \mathcal{L}\rightarrow \mathcal{L'}=\mathcal{L}+\partial_{\mu}K(x)
\end{align*}
where $K(x)$ is a spacetime function that vanishes on the surface of integration. For supersymmetric transformations it can be shown that the highest order component fields in $\theta$ and $\bar{\theta}$ of a superfield always transform in this way. From the Grassmann calculus we had the identity
\begin{align*}
    \int d^{4}\theta \, (\theta\theta)(\bar{\theta}\bar{\theta})=1
\end{align*}
Therefore, the highest order component can be projected out by modifying the action as
\begin{align}\label{eq:SAction}
    \mathcal{S}=\int d^{4}x\int d^{4}\theta \,\mathcal{L}
\end{align}
which ensures that the action is invariant under supersymmetry transformations. Here $\mathcal{L}$ is a function of superfields and it is not the same as in \cref{eq:Action}. In classical quantum field theories renormalizable Lagrangians has mass dimension $[\mathcal{L}]=M^{4}$, but counting dimension in \cref{eq:SAction} we see that $[\mathcal{L}]=M^{2}$. Including Grassmann variables renormalizability forbids terms with mass dimension higher than four, which means that $\mathcal{L}$ can have at most three powers of scalar superfields $\Phi$. The general supersymmetric Lagrangian can therefore be written as
\begin{align}
    \mathcal{L}=\Phi_{i}^{\dagger}\Phi_{i}+\bar{\theta}\bar{\theta}W[\Phi]+\theta\theta W[\Phi^{\dagger}]
\end{align}
where $\Phi_{i}^{\dagger}\Phi_{i}$ is the kinetic term, and $W[\Phi]$ is a holomorphic function of $\Phi$ called the $\emph{superpotential}$, given by
\begin{align*}
    W[\Phi]&=a_{i}\Phi_{i}+m_{ij}\Phi_{i}\Phi_{j}+\lambda_{ijk}\Phi_{i}\Phi_{j}\Phi_{k}
\end{align*}
where $m_{ij}^{2}$ and $\lambda_{ijk}$ are symmetric. In order to specify a supersymmetric Lagrangian with a given superfield content we only need to specify the superpotential.


\subsection{Supergauge}
As in the Standard Model we want our supersymmetric theory to be gauge invariant. This derivation follows in the same manner as in (ref to gauge chapter), but for concreteness we follow through the steps. We consider a general gauge group $G$ with Lie algebra
\begin{align}
    [t^{a},t^{b}]=if^{abc}t^{c}
\end{align}
where $t^{a}$ are the usual gauge group generators and $f^{abc}$ are the structure constants of the gauge group. As usual a group element $g$ can be represented by the unitary exponential map, which give a local gauge transformation
\begin{align}\label{eq:supergauge}
    U(x)=\exp(ig\Lambda^{a}t^{a})
\end{align}
The gauge transformation for a left handed superfield is then
\begin{align}
    \Phi\rightarrow\Phi'=U\Phi
\end{align}
where $q$ is the charge of the superfield $\Phi$ under $G$, and $\Lambda^{a}(x)$ is the parameters of the gauge transformation. As $\Phi$ is a left handed superfield, so for this transformation to make sense $\Lambda^{a}$ must be a left handed superfield. Likewise, $\Lambda^{\dagger\,a}$ must be a right handed superfield as $\Phi^{\dagger}$ is a right handed superfield. For the superpotential to be gauge invariant $W[\Phi']=W[\Phi]$, we find the following requirements
\begin{align}
    a_{i}&=0\hspace{3mm}\text{if}\hspace{3mm} a_{i}U_{ir}\neq a_{r}
    \\
    m_{ij}&=0\hspace{3mm}\text{if}\hspace{3mm} m_{ij}U_{ir}U_{js}\neq m_{rs}
    \\
    \lambda_{ijk}&=0\hspace{3mm}\text{if}\hspace{3mm} \lambda_{ijk}U_{ir}U_{js}U_{kt}\neq \lambda_{rst}
\end{align}
For the kinetic term to be invariant we must introduce a gauge compensating vector superfield $V^{a}(x)$ for each Lie algebra generator $t^{a}$. We represent the vector superfield as
\begin{align}
    V(x)=\exp(gV^{a}t^{a})
\end{align}
The kinetic term can then written as
\begin{align}
    \mathcal{L}_{kin}=\Phi^{\dagger}V\Phi
\end{align}
Under gauge transformation this will give
\begin{align}
    \mathcal{L}_{kin}\rightarrow\mathcal{L'}_{kin}=\Phi^{\dagger}e^{-ig\Lambda^{\dagger\,a}t^{a}}e^{gV'^{a}t^{a}}e^{ig\Lambda^{a}t^{a}}\Phi
\end{align}
which mean that the vector superfield must transform as
\begin{align}
    V\rightarrow V'=U^{\dagger}VU
\end{align}
On infinitesimal and component form this reduces to the standard non-abelian gauge transformations
\begin{align}
    V_{\mu}^{a}=V_{\mu}^{a}+\partial_{\mu}\alpha^{a}+gf^{abc}V_{\mu}^{b}\alpha^{c}
\end{align}

\subsection{Supersymmetric Field Strength}
As we have introduced new dynamical gauge fields we must define the field strength in order to fully construct the supersymmetric Lagrangian. We want the field strengths themselves to be superfields, so we define the supersymmetric field strengths as
\begin{align}
    W_{A}&\equiv -\frac{1}{4}\bar{D}\bar{D}e^{-V^{a}t^{a}}D_{A}e^{V^{a}t^{a}}
    \\
    \bar{W}_{\dot{A}}&\equiv-\frac{1}{4}DDe^{-V^{a}t^{a}}D_{\dot{A}}e^{V^{a}t^{a}}
\end{align}
We know that $D^{3}=\bar{D}^{3}=0$, so
\begin{align}
    \bar{D}_{\dot{A}}W_{A}&=0
    \\
    D_{A}\bar{W}_{\dot{A}}&=0
\end{align}
which mean that $W_{A}$ is a left handed superfield and $\bar{W}_{\dot{A}}$ is a right handed superfield. As in the Standard Model the gauge invariant field strength terms for Non-Abelian gauge fields is given by the trace of the field strengths, Tr$[W^{A}W_{A}]$. Since $W_{A}$ is spanned by the Lie group generators, we have that
\begin{align}
    \text{Tr}[W^{A}W_{A}]=W^{a\,A}W_{A}^{b}\text{Tr}[t^{a}t^{b}]=W^{a\,A}W_{A}^{b}\delta^{ab}T(R)=T(R)W^{a\,A}W_{A}^{a}
\end{align}
where $T(R)$ is the Dynkin index. Further, by expanding in terms of component fields it can be shown that it contains the ordinary non-abelian field strength
\begin{align}
    F_{\mu\nu}^{a}=\partial_{\mu}V_{\nu}^{a}-\partial_{\nu}V_{\mu}^{a}+gf^{abc}V_{\mu}^{b}V_{\nu}^{c}
\end{align}
Now we have all the ingredients to write down the full unbroken supersymmetric Lagrangian. Normalizing the field strength terms with the Dynkin index, the most general supersymmetric Lagrangian in terms of superfields is
\begin{align}\label{eq:superlagrangian}
    \mathcal{L}(\Phi)=\Phi^{\dagger}e^{gV^{a}t^{a}}\Phi+\bar{\theta}\bar{\theta}W[\Phi]+\theta\theta W[\Phi^{\dagger}]+\frac{1}{2}W^{a\,A}W_{A}^{a}
\end{align}
For clarity it is instructive to present this Lagrangian in terms of the spacetime dependent component fields. For details on how to expand in component fields, see \cite{Martin:1997ns}, which will give the Lagrangian
\begin{align}\label{eq:component superLagrangian}
    \mathcal{L}=-\frac{1}{4}&(F_{\mu\nu}^{a})^{2}-(D_{\mu}\phi_{i})^{*}(D^{\mu}\phi_{i})+i\bar{\psi}_{i}\bar{\sigma}^{\mu}D_{\mu}\psi_{i}+i\bar{\lambda}^{a}\bar{\sigma}^{\mu}D_{\mu}\lambda^{a}
    \\
    &-\frac{1}{2}(W_{ij}\psi_{i}\psi_{j}+h.c)-\sqrt{2}g(\phi_{i}^{*}t^{a}\psi_{i}\lambda^{a}+h.c)
    \\
    &-W_{i}W_{i}^{*}-\frac{1}{2}g(\phi_{i}^{*}t^{a}\phi_{i})^{2}
\end{align}
where $D_{\mu}$ are the usual covariant derivative defined in [ref to covariant derivative]. The functions $W_{i}$ and $W_{ij}$ are the derivatives of the superpotential in terms the component fields
\begin{align}
    W_{i}&=\pdv{W}{\phi_{i}}=m_{ij}\phi_{j}+y_{ijk}\phi_{j}\phi_{k}
    \\
    W_{ij}&=\frac{\partial^{2}W}{\partial\phi_{i}\partial\phi_{j}}=m_{ij}+y_{ijk}\phi_{k}
\end{align}
As shown in [ref to superpotential trnaformation] gauge invariance restricts the parameters in the superpotential $W$. The tadpole term is only allowed if $\phi_{i}$ is a gauge singlet, and mass terms $m_{ij}\phi_{i}\phi_{j}$ can only appear if the representations the fields transform under are conjugates of one another. The yukawa term $y_{ijk}\phi_{i}\phi_{j}\phi_{k}$ require that the fields transform under representations that combine to a gauge singlet. It is worth noting that the mass parameter $m_{ij}$ and yukawa parameter $y_{ijk}$ appear in mass terms of both scalars and fermions, highlighting that members of each supermultiplet are mass degenerate. Also, the same set of yukawa couplings enter linearly in $\phi\psi\psi$ interactions and quadratically in $\phi^{4}$ interactions, which explains the coupling relation $|\lambda_{f}|^{2}=\lambda_{s}$
needed to cancel the quadratic divergences in the Higgs mass correction.


\subsection{Supersymmetry Breaking}
As mentioned above, supersymmetry predicts that all superpartners within a supermultiplet has the same mass. This scenario has been excluded by experiments, as none of the sparticles have been observed, but their Standard Model partners have been observed in vast numbers. The only solution to this problem is that supersymmetry must badly broken. Fortunately, we have a similar problem in the Standard Model, where we use the Higgs mechanism to spontaneously break the electroweak gauge invariance such that the weak gauge bosons aquire a mass. So, if we believe that supersymmetry is an exact symmetry of a fundamental theory, it must be spontaneously broken, which happens when the vacuum state is not invariant under the symmetry. Let us go back to the discussion of the vacuum state we had in [ref to supersymmetric hamiltonian], where we found the Hamiltonian
\begin{align}
    H=\frac{1}{4}(\Bar{Q_{1}}Q_{1}+Q_{1}\Bar{Q_{1}}+Q_{2}\Bar{Q_{2}}+\Bar{Q_{2}}Q_{2})
\end{align}
We showed then that we must have $\ev{H}{\psi}\geq 0$, where $\psi$ is some state. Then we showed that if the vacuum state is supersymmetric, we must have $\ev{H}{0}=0$. So, if the converse is true and there exist a positive vacuum energy, supersymmetry must be broken. As in the Standard Model we want to find a scalar potential that does the job, and in supersymmetry we have the scalar potential
\begin{align}
    V(\phi_{i},\phi_{i}^{*})=\sum_{i=1}^{n}\big|\pdv{W[\phi_{1},\dots,\phi_{n}]}{\phi_{i}}\big|^{2}
\end{align}
By applying the equation of motion for the auxiliary field $F_{i}$, we find that
\begin{align}
    F_{i}=\pdv{W}{\phi_{i}}
\end{align}
So if the auxiliary field $F_{i}$ acquires a vacuum expectation value $\langle F_{i}\rangle>0$, supersymmetry is spontaneously broken. This mechansism is known as F-term breaking. Another way of breaking supersymmetry spontaneously is by adding a term $\mathcal{L}\sim 2kV$, where $V$ is a vector superfield. If the auxiliary field $D(x)$ acquires a vacuum expectation value $\langle D\rangle >0$, we get what we call a D-term breaking.

Unfortunately, these procedures makes phenomenologically uncacceptable predicitons where at least one of the superpartners are lighter than the Standard model partner. The Standard model is gauge anomaly free, which is a feature we want a supersymmetric theory to inherit. At tree level one show that the weighted sum over tree-level squared-mass eigenvalues, known as the $\emph{supertrace}$, vanish in theories with non-anomolous gauge symmetries. The supertrace is given by
\begin{align}
    \mathcal{S}\text{Tr}[m^{2}]=\sum_{j}(-1)^{2j}(2j+1)\text{Tr}(m_{j}^{2})
\end{align}
where $m^{2}$ is the total squared mass matrix of the Lagrangian, $j$ is the spin of the particles and $m_{j}^{2}$ is the squared mass matrix for the spin-$j$ particles. If we have a theory with only scalar superfields, yielding two fermionic and two bosonic degress of freedom each we have that
\begin{align}
    \text{Tr}[(m_{s}^{2}-2m_{f}^{2})]=0
\end{align}
with the consequence that not all scalar partners can be heavier than the known fermions. A further complication is that the renormalizable supersymmetric Lagrangian \cref{eq:component superLagrangian}, does not contain any Yukawa terms $\phi\lambda\lambda$ that turn into mass terms for the gauginos if the scalar field acquires a vacuum expectation value. Therefore, to spontaneously break supersymmetry using fields that are coupled at tree level to the supermultiplets of the known particles seems not to work. In order to circumvent these difficulties, one generally assume that there is a sector at high energy scale that are hidden and not, or minimally, charged under $G_{SM}$. Then, hidden sector fields can acquire vacuum expectation values that spontaneously breaks supersymmetry, and the effects of supersymmetry breaking can be mediated from the hidden sector to the visible sector through non-renormalizable interactions or loop-processes. Through this process phenomenologically possible superpartner masses can be generated. The most studied frameworks for spontaneous supersymmetry breaking are \emph{Planck-scale-mediated supersymmetry breaking} (PMSB) and \emph{Gauge-mediated supersymmetry breaking} (GMSB), but as we will take another approach we refer to \cite{Martin:1997ns} for more details.

\subsection{Soft Supersymmetry Breaking}
Another popular approach is to not care about any specific model dependent breaking mechanism, but add terms to the Lagrangian that explicitly break supersymmetry, and treat the coefficients of these terms as free model parameters. However, we can not just simply add any supersymmetry-breaking term to the Lagrangian, with the risk of spoiling all the attractive features of low-scale supersymmetry. Terms that we do not want to add are the terms that can lead to quadratic divergences to scalar masses, and of course no terms that breaks renormalizability. Therefore, we add so called \emph{soft}-breaking terms where the total mass dimension for the interacting terms are at most 3. Ignoring all \textquote{maybe-soft} terms (ref to Are), the allowed terms can be written in terms of their component fields as
\begin{align}
    \mathcal{L}_{soft}=&-\frac{1}{2}M\lambda^{a}\lambda^{a}-m_{ij}^{2}\phi_{i}^{*}\phi_{j}
    \\
    &-t_{i}\phi_{i}-\frac{1}{2}b_{ij}\phi_{i}\phi_{j}-\frac{1}{6}a_{ijk}\phi_{i}\phi_{j}\phi_{k}+h.c
\end{align}
Where $M$ is repeated for each gauge group, and will give mass to the gauginos, while $m_{ij}^{2}$ and $b_{ij}$ will give additional scalar mass terms. If one of the scalar field in the trilinear term acquires a vacuum expectation value, $a_{ijk}$ will contribute to a mass terms for the remaining fields. The tadpole term with parameter $t_{i}$ is only allowed if $\phi_{i}$ is a gauge singlet. In total, the soft breaking terms give masses to both the scalar and fermionic superpartners of the Standard model particles.

\subsection{Minimal Supersymmetric Standard Model}
The Minimal Supersymmetric Standard Model (MSSM) is a supersymmetric extension of the Standard Model containing a minimal field content consistent with the Standard Model. We want to construct the MSSM on the same premiss as the SM, which is a theory based on $SU(3)_{C}\times SU(2)_{L}\times U(1)_{Y}$, and therefore all the known particles must be placed in appropriate supermultiplets. Standard model fermion fields and the scalar Higgs field are contained in chiral supermultiplets, and the Standard model gauge fields are contained in vector supermultiplets. In order to construct Dirac fermions we need to use a left chiral superfield and a \emph{different} right chiral superfield, which forms two fermions-particle and antiparticle-and four scalar particles-a pair of left and right handed scalar particles, and their antiparticles.

\subsection{Leptons}
For the left-handed leptons, the supermultiplets are contained in $SU(2)_{L}$ doublets, and the right-handed leptons are contained in $SU(2)_{L}$ singlets,
\begin{align}
    L_{i}=\begin{pmatrix}
            \nu_i\\
            l_i
\end{pmatrix}\hspace{3mm}\text{and}\hspace{3mm}\bar{e}_{i}
\end{align}
Here $i$ runs over the three generations og the Standard Model. The superfields $l_i$ and $\bar{e}_i$ combine to give charged leptons and sleptons. $\nu_{i}$ give the left-handed neutrinos and sneutrinos.

\subsection{Quarks}
We place the left-handed quarks in $SU(2)_{L}$ doublets, and the right handed in $SU(2)_{L}$ singlets,
\begin{align}
    Q_{i}=\begin{pmatrix}
            u_i\\
            d_i
\end{pmatrix}\hspace{3mm}\text{and}\hspace{3mm}\bar{u}_{i},\bar{d}_{i}
\end{align}
Colour indices are omitted. Here, the superfields $u_{i}$ and $\bar{u}_{i}$ combine to give the up-type quarks and squarks, and similarly the superfields $d_{i}$ and $\bar{d}_{i}$ combine to give the down-type quarks and squarks.

\subsection{Gauge Bosons}
The gauge bosons are represented by using vector supermultiplets. In a vector supermultiplet we have one massless vector boson and two Weyl spinors of each handedness, yielding two fermionic and two bosonic degrees of freedom. One vector superfield is needed per generator of the algebra, and these are denoted
\begin{align}
    W^{a},C^{a},B^{0}
\end{align}
From these one construct the spin-1 gauge bosons of the Standard Model $g$, $W^{+}$,$W^{-}$ and $B^{0}$. The fermionic superpartner of the gluon $g$, are called gluino $\tilde{g}$, and the fermionic partners of the Electroweak gauge bosons are $\tilde{W}^{0}$, $\tilde{W}^{+}$. $\tilde{W}^{-}$ and $\tilde{B}^{0}$, named Winos and Bino.

\subsection{Higgs boson}
The Higgs particle is a scalar, so it must belong to a chiral supermultiplet. The supersymmetric version of the Standard Model Higgs $SU(2)_{L}$ doublet would mix left chiral and right chiral superfields in order to give masses to all fermions, and therefore it can not appear in the superpotential. We therefore need two Higgs doublets $H_{u}$ and $H_{d}$, indexed according to the quarks they give masses to. Further, we need two Higgs doublets to make the MSSM free of gauge anomalies. The superfield doublets are defined as
\begin{align}
    H_{u}=\begin{pmatrix}
            H_{u}^{+}\\
            H_{u}^{0}
\end{pmatrix}\hspace{5mm}H_{u}=\begin{pmatrix}
            H_{d}^{0}\\
            H_{d}^{-}
\end{pmatrix}
\end{align}
The sign indicate the electric charge. In total these doublets contain four Weyl spinors and eight bosonic degrees of freedom. The Weyl spinors combine to make \emph{higgsinos} , three bosonic degrees of freedom are used to give masses to the $W^{\pm}$ and $Z^{0}$. The remaining five degrees of freedom give rise to five scalar mass eigenstates $h^{0}$, $H^{0}$, $A^{0}$, and $H^{\pm}$.

\subsection{MSSM Lagrangian}
The MSSM Lagrangian can now be constructed from the above superfields, and consists of the following parts
\begin{align}
    \mathcal{L}_{MSSM}=\mathcal{L}_{kin}+\mathcal{L}_{V}+\mathcal{L}_{W}+\mathcal{L}_{soft}
\end{align}
The kinetic part takes the following form
\begin{align}
    \mathcal{L}_{kin}=&L_{i}^{\dagger}e^{W-B}L_{i}+Q_{i}^{\dagger}e^{C+W+\frac{1}{3}B}Q_{i}
    \\
    &+\bar{u}_{i}^{\dagger}e^{C-\frac{4}{3}B}u_{i}+\bar{d}_{i}^{\dagger}e^{C+\frac{2}{3}B}d_{i}+\bar{e}_{i}^{\dagger}e^{2B}e_{i}
    \\
    &+H_{u}^{\dagger}e^{W+B}H_{u}+H_{d}^{\dagger}e^{W-B}H_{d}
\end{align}
where
\begin{align}
    W&=\frac{1}{2}g\sigma^{a}W^{a}
    \\
    C&=\frac{1}{2}g_{s}\lambda^{a}C^{a}
    \\
    B&=\frac{1}{2}g'B^{0}
\end{align}
with $g',g$ and $g_s$ the coupling of $U(1)_{Y}$, $SU(2)_{L}$ and $SU(3)_{C}$ gauge groups. The supersymmetric field strength contribution is given by
\begin{align}
    \mathcal{L}_{V}=\frac{1}{2}\bar{\theta}\bar{\theta}\text{Tr}[W^{A}W_{A}]+\frac{1}{2}\bar{\theta}\bar{\theta}\text{Tr}[C^{A}C_{A}]+\frac{1}{4}\bar{\theta}\bar{\theta}B^{A}B_{A}+c.c
\end{align}
with the field strengths
\begin{align}
    W_{A}&=-\frac{1}{4}\bar{D}\bar{D}e^{-W}D_{A}e^{W}
    \\
    C_{A}&=-\frac{1}{4}\bar{D}\bar{D}e^{-C}D_{A}e^{C}
    \\
    B_{A}&=-\frac{1}{4}\bar{D}\bar{D}D_{A}B
\end{align}
The possible gauge invariant terms in the superpotential are
\begin{align}
    W=&\mu H_{u}H_{d}+y_{ij}^{e}L_{i}H_{d}\bar{E}_{j}+y_{ij}^{u}Q_{i}H_{u}\bar{U}_{j}+y_{ij}^{d}Q_{i}H_{d}\bar{D}_{j}
    \\
    &\mu'L_{i}H_{u}+\lambda_{ijk}L_{i}L_{j}\bar{E}_{k}+\lambda_{ijk}^{'}L_{i}Q_{j}\bar{D}_{k}+\lambda_{ijk}^{''}\bar{U}_{i}\bar{D}_{j}\bar{D}_{k}
\end{align}
The terms in the last line all violate lepton and baryon number. These violations are under strict restrictions from experiments, such as the search of proton decay $p\rightarrow e^{+}\pi^{0}$, which has not been observed. In order to circumvent these terms, a new multiplicative conserved quantity can be introduced, \emph{R-parity}. This is defined by
\begin{align}
    P_{R}=(-1)^{3(B-L)+2s}
\end{align}
where $s$ is spin, B is baryon number and L is lepton number. SM particles has R-parity of $+1$ and sparticles has R-parity of $-1$. The conservation of this quantity has the consequence that
\begin{enumerate}
    \item All sparticles must be produced in pairs
    \item The lightest supersymmetric particle (LSP) is absolutely stable
    \item Every other sparticle must decay down to the LSP
\end{enumerate}

\subsection{Radiative Electroweak Symmetry Breaking}

\subsection{Gluinos and Squarks}