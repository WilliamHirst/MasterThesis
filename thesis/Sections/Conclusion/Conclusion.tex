\newpage
\chapter*{Conclusion}
\addcontentsline{toc}{chapter}{Conclusion} 
In this thesis we have investigated IR divergences that appear in gauge theories, focusing in particular on the non-Abelian gauge theory of QCD. For theories involving massless fields these divergences are a prominent feature, and in certain regions of phase space they give rise to large logarithmic corrections to physical observables. The main focus of interest was to investigate how Wilson lines and Wilson loops can be used to resum these large corrections such that physical observables exponentiate and prevents the invalidation of perturbative expansions.

\medskip
In \cref{chap:Intro QFT} we developed the route from Green's functions to scattering amplitudes and Feynman diagrams, before going into some detail about the basic ideas and possible ways of treating divergences using regularization and renormalization. We mainly focused on the UV region of phase space, as the IR region is studied in more detail in later chapters.
After this basic introduction to QFT in \cref{chap:Intro QFT}, we went on to look at the geometrical formulation of gauge theories in \cref{chap:Geometry of gauge theories}. From this formalism the concept of Wilson lines naturally appeared as fundamental building blocks for any gauge theory. We also discussed that from the Ambrose-Singer theorem it follows that physical observables can be constructed in terms of gauge invariant Wilson loops. This is the basis why we in later chapters use Wilson lines to construct Wilson loop expectation values, and from them eikonal cross sections. Further, by using Wilson lines we showed how one can construct the Yang-Mills Lagrangian from a purely geometrical standpoint. As we are mainly interested in scattering amplitudes, we focused on introducing the properties of piecewise linear Wilson lines. These naturally appear when high-energy particles meet at a point and annihilate or scatters of one another by exchanging a gauge boson.

\medskip
In \cref{Chap:pQCD} we first introduced the QCD Lagrangian and set the stage for perturbative calculations by discussing the property of asymptotic freedom in QCD. By using the most basic experimental setup of deep inelastic scattering, we went on and studied the important concepts of factorization both in the parton model and in QCD. We also introduced parton distribution functions, which is an essential ingredient in obtaining factorization in QCD. From there we made use of Wilson lines and how these can be used to render parton distributions gauge invariant. We also derived parton-in-parton distributions and showed how these can be defined such that they incorporate the collinear divergences appearing in the partonic cross section. The significance of this is that we can use factorization theorems to group divergences into different regions and define functions that are responsible for these. Lastly we investigated the appearance of large logarithms by doing an explicit NLO calculation of a lepton pair production via the annihilation of a quark-antiquark pair. We show that higher order corrections have a significant contribution to the hadronic cross section, implying that in order to have predictive results even higher order results had to be taken into account. We also made a numerical evaluation of the NLO hadronic cross section and compared with experimental results from CMS. This is of course nothing new, but the reason we used this process is because it is the simplest one to perform resummation with. By doing the NLO calculation we found a fixed order result we could compare with the resummed Drell-Yan cross section we aimed to calculate.

\medskip
In \cref{chap:Resummation in QCD} we started by looking at the behaviour of scattering amplitudes using the eikonal approximation, showing that in this limit the amplitude naturally factorizes into one hard and one soft regime. The crucial point of this derivation was to show that not only does the soft and hard parts decouple, but IR divergences coming from the soft part exponentiates. Close to the final state production threshold gluon radiation from the highly energetic initial state quarks are restricted to be soft. This implies that Wilson lines on linear paths can be used to describe the soft radiation. We use factorization theorems to first refactorize the partonic cross section in terms of parton-in-parton distributions that are responsible for collinear divergences. These cross sections are explicitly factorized by using Mellin space techniques. Then we defined a close to threshold cross section, which is factorized into three parts. One part that describes the hard process without large corrections, one part that describes collinear radiation and one part that contains soft wide angle radiation. The contribution from the soft function is found by taking the eikonal approximation giving an eikonal partonic cross section. This eikonal cross section is then constructed from a Wilson loop expectation value. The eikonal cross section contains both soft and collinear singularities, and again we used factorization theorems to group the collinear singularities into eikonal parton distributions and an infrared safe eikonal function. 

We then took a closer look at the renormalization properties of Wilson lines and parton-in-parton distributions in the $x\rightarrow 1$ limit. A piecewise linear Wilson line that is constructed in terms of two semi-infinite Wilson lines on linear paths with an angle between them, contains cusp divergences. By using the renormalization properties of Wilson lines we calculate the one-loop cusp anomalous dimension for Wilson lines on the light-cone. This calculation is important as we show that parton-in-parton distributions in the $x\rightarrow 1$ obey an evolution equation in terms of this cusp anomalous dimension. The cusp anomalous dimension also appears in the eikonal cross section, showing that it is an important ingredient when doing resummation with Wilson lines. From there we show how one can calculate the Wilson loop expectation value to one-loop order and use the non-Abelian eikonal exponentiation theorem to find the exponentiated eikonal cross section. As this eikonal cross section contains IR divergences, we use the eikonal distributions to find the infrared safe eikonal function. In the end we find an exponentiated partonic hard function, given in terms of an integral over the cusp anomalous dimension. We solve this integral by using the one-loop running coupling and find that the resummed expression contains an all order series of leading logarithmic terms. Hence, we have not only reproduced the large logarithm from the fixed order NLO calculation in \cref{Chap:pQCD}, but showed the appearance of higher order contributions without performing any fixed order calculation. We also discuss that these results are in accordance with results in the literature, apart from constant factors that we neglected throughout as we mainly focused on the limit of large $N$ in Mellin space.

Lastly, we reconstruct the full resummed hadronic cross section and show an explicit method of how one can go about calculating the inverse Mellin transform needed for a numerical evaluation. We discuss several important features that have to be taken into account when choosing the contour in Mellin space. By using that the cross section is a real valued function, we manipulate the inverse transform and show that it reduces to an integral over a real variable.

\medskip
As future work resummation techniques can be explored further by looking at physics beyond the Standard Model, for example in Supersymmetry. Supersymmetry is one of the most promising, and most studied theories for physics beyond the Standard Model. At the Large Hadron Collider there is an extensive search programme for such new physics phenomena. So far no supersymmetric particles have been found, but there is a need for higher order calculations for the production of these sparticles to precisely evaluate the current exclusion limits. As a natural extension to the work done here, we could look at the production of final state sleptons. Sleptons are colour neutral and will, as for the leptons studied here, not contain any final state gluon radiation. Hence, the radiation part of the process should not be any different from the one we have found in this thesis. The main difference will be that we are considering the production of massive particles and the threshold variable will be a function of the mass. Of course, we also have to take into account the interaction between the sleptons and the photon, leading to a different hard function. There are simplifications that can be made, and that is to look at degenerate masses. Another extension is to look at coloured final states, where one consider the production of squarks and gluinos. This extension is highly non-trivial as the final state will also contain gluon radiation. 