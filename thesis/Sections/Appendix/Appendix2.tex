\section{The Features}
\subsection{Jet Requirements}\label{subsec:sigJets}
\begin{figure}[H]
    \renewcommand\figurename{Table}
    \centering
    \footnotesize
    \makebox[\linewidth][c]{
        \begin{subfigure}{.3\textwidth}
            $
            \begin{array}{cc}
                \hline \text { Requirement } & \text { Baseline Jets }  \\
                \hline \hline p_T-\text{cut} & p_T > 20GeV  \\
                \eta-\text{cut} & |\eta| > 2.8  \\
                \hline
            \end{array}
            $
            \caption{}
            \label{table:BLJets}
        \end{subfigure}
        \begin{subfigure}{.3\textwidth}
            $
            \begin{array}{cc}
                \hline \text { Requirement } & \text { Signal Jets }  \\
                \hline \hline p_T-\text{cut} & p_T > 60  \\
                \ \ \ or &\\
                JVT-\text{cut} & |JVT| < 0.91 \\
                \hline
            \end{array}
            $
            \caption{}
            \label{table:SGJets}
        \end{subfigure}
        
    }
    \caption[Two tables displaying the baseline \ref{table:BLJets} and signal \ref{table:SGJets} requirements of the jets applied 
    to the data as part of the preprocessing.]{Two tables displaying the baseline \ref{table:BLJets} and signal \ref{table:SGJets} 
    requirements of the jets applied to the data as part of the preprocessing. Note that signal jets are required to pass both baseline 
    and signal requirements. For a formal definition of \acf{JVT}, see \cite{the_atlas_collaboration_tagging_2014}.}
    \label{table:JetCuts}
\end{figure}
\newpage
\subsection{The Feature Distribution}\label{subsec:Dist}
\begin{figure}[H]
    \makebox[0.95\linewidth][c]{%
    \centering
    \begin{subfigure}{.405\textwidth}
        \includegraphics[width=\textwidth]{Figures/FeaturesHistograms/lep3_Pt.pdf}
        \caption{}
        \label{fig:lep3_Pt}
    \end{subfigure}
    \hfill
    \begin{subfigure}{.525\textwidth}
        \includegraphics[width=\textwidth]{Figures/FeaturesHistograms/lep2_Pt.pdf}
        \caption{}
        \label{fig:lep2_Pt}
    \end{subfigure}
    }
    \makebox[0.95\linewidth][c]{%
    \begin{subfigure}{.405\textwidth}
        \includegraphics[width=\textwidth]{Figures/FeaturesHistograms/lep2_Eta.pdf}
        \caption{}
        \label{fig:lep2_Eta}
    \end{subfigure}
    \hfill
    \begin{subfigure}{.525\textwidth}
        \includegraphics[width=\textwidth]{Figures/FeaturesHistograms/lep3_Eta.pdf}
        \caption{}
        \label{fig:lep3_Eta}
    \end{subfigure}
    }
    \caption[\acs{MC} simulated and measured data comparison showing the $p_T$ for the first, 
    second and third lepton. Similarly, the distribution over $\eta$ for the first, second and third lepton.]{
        \acs{MC} simulated and measured data comparison showing the $p_T$ for the third \ref{fig:lep3_Pt} 
        and second \ref{fig:lep2_Pt} lepton. Similarly, the distribution over $\eta$ for the 
        second \ref{fig:lep2_Eta} and third \ref{fig:lep3_Eta} lepton.}
        \label{fig:Dist2}
    \end{figure}
    \newpage
    \begin{figure}[H]
    \makebox[0.95\linewidth][c]{%
    \centering
    \begin{subfigure}{.405\textwidth}
        \includegraphics[width=\textwidth]{Figures/FeaturesHistograms/lep1_Phi.pdf}
        \caption{}
        \label{fig:lep1_Phi}
    \end{subfigure}
    \hfill
    \begin{subfigure}{.525\textwidth}
        \includegraphics[width=\textwidth]{Figures/FeaturesHistograms/lep2_Phi.pdf}
        \caption{}
        \label{fig:lep2_Phi}
    \end{subfigure}
    }
    \makebox[0.95\linewidth][c]{%
    \begin{subfigure}{.405\textwidth}
        \includegraphics[width=\textwidth]{Figures/FeaturesHistograms/lep3_Phi.pdf}
        \caption{}
        \label{fig:lep3_Phi}
    \end{subfigure}
    \hfill
    \begin{subfigure}{.525\textwidth}
        \includegraphics[width=\textwidth]{Figures/FeaturesHistograms/lep1_Mt.pdf}
        \caption{}
        \label{fig:lep1_Mt}
    \end{subfigure}
    }
    \makebox[0.95\linewidth][c]{%
    \begin{subfigure}{.405\textwidth}
        \includegraphics[width=\textwidth]{Figures/FeaturesHistograms/lep2_Mt.pdf}
        \caption{}
        \label{fig:lep2_Mt}
    \end{subfigure}
    \hfill
    \begin{subfigure}{.525\textwidth}
        \includegraphics[width=\textwidth]{Figures/FeaturesHistograms/lep3_Mt.pdf}
        \caption{}
        \label{fig:lep3_Mt}
    \end{subfigure}
    }
    \caption[\acs{MC} simulated and measured data comparison showing the $\phi$ for the first, 
    second and third lepton. Similarly, the distribution over $m_t$ for the first, second and third lepton.]{\acs{MC} simulated and measured data 
    comparison showing the $\phi$ for the first \ref{fig:lep1_Phi}, 
    second \ref{fig:lep2_Phi} and third \ref{fig:lep3_Phi} lepton. Similarly, the distribution over $M_T$
    for the first \ref{fig:lep1_Mt}, second \ref{fig:lep2_Mt} and third \ref{fig:lep3_Mt} lepton.}
\end{figure}
\newpage
\begin{figure}[H]
    \makebox[0.95\linewidth][c]{%
    \centering
    \begin{subfigure}{.405\textwidth}
        \includegraphics[width=\textwidth]{Figures/FeaturesHistograms/lep1_Charge.pdf}
        \caption{}
        \label{fig:lep1_Charge}
    \end{subfigure}
    \hfill
    \begin{subfigure}{.525\textwidth}
        \includegraphics[width=\textwidth]{Figures/FeaturesHistograms/lep2_Charge.pdf}
        \caption{}
        \label{fig:lep2_Charge}
    \end{subfigure}
    }
    \makebox[0.95\linewidth][c]{%
    \begin{subfigure}{.405\textwidth}
        \includegraphics[width=\textwidth]{Figures/FeaturesHistograms/lep3_Charge.pdf}
        \caption{}
        \label{fig:lep3_Charge}
    \end{subfigure}
    \hfill
    \begin{subfigure}{.525\textwidth}
        \includegraphics[width=\textwidth]{Figures/FeaturesHistograms/lep1_Flavor.pdf}
        \caption{}
        \label{fig:lep1_Flavor}
    \end{subfigure}
    }
    \makebox[0.95\linewidth][c]{%
    \begin{subfigure}{.405\textwidth}
        \includegraphics[width=\textwidth]{Figures/FeaturesHistograms/lep2_Flavor.pdf}
        \caption{}
        \label{fig:lep2_Flavor}
    \end{subfigure}
    \hfill
    \begin{subfigure}{.525\textwidth}
        \includegraphics[width=\textwidth]{Figures/FeaturesHistograms/lep3_Flavor.pdf}
        \caption{}
        \label{fig:lep3_Flavor}
    \end{subfigure}
    }
    \caption[\acs{MC} simulated and measured data comparison showing the charge for the first,
    second and third lepton. Similarly, the distribution over the flavor for the first, second and third lepton]{\acs{MC} 
    simulated and measured data comparison showing the charge for the first \ref{fig:lep1_Charge},
    second \ref{fig:lep2_Charge} and third \ref{fig:lep3_Charge} lepton. Similarly, the distribution over the flavor
    for the first \ref{fig:lep1_Flavor}, second \ref{fig:lep2_Flavor} and third \ref{fig:lep3_Flavor} lepton.}
\end{figure}
\newpage
\begin{figure}[H]
    \makebox[0.95\linewidth][c]{%
    \centering
    \begin{subfigure}{.405\textwidth}
        \includegraphics[width=\textwidth]{Figures/FeaturesHistograms/deltaR.pdf}
        \caption{}
        \label{fig:deltaR}
    \end{subfigure}
    \hfill
    \begin{subfigure}{.525\textwidth}
        \includegraphics[width=\textwidth]{Figures/FeaturesHistograms/met_Phi.pdf}
        \caption{}
        \label{fig:met_Phi}
    \end{subfigure}
    }
    \makebox[0.95\linewidth][c]{%
    \begin{subfigure}{.405\textwidth}
        \includegraphics[width=\textwidth]{Figures/FeaturesHistograms/mlll.pdf}
        \caption{}
        \label{fig:mlll}
    \end{subfigure}
    \hfill
    \begin{subfigure}{.525\textwidth}
        \includegraphics[width=\textwidth]{Figures/FeaturesHistograms/mll_OSSF.pdf}
        \caption{}
        \label{fig:mll_OSSF}
    \end{subfigure}
    }
    \makebox[0.95\linewidth][c]{%
    \begin{subfigure}{.405\textwidth}
        \includegraphics[width=\textwidth]{Figures/FeaturesHistograms/met_Sign.pdf}
        \caption{}
        \label{fig:met_Sign}
    \end{subfigure}
    \hfill
    \begin{subfigure}{.525\textwidth}
        \includegraphics[width=\textwidth]{Figures/FeaturesHistograms/Ht_lll.pdf}
        \caption{}
        \label{fig:Ht_lll}
    \end{subfigure}
    }
    \caption[\acs{MC} simulated and measured data comparison showing the $\Delta R$ \ref{fig:deltaR}
    and the azimuthal angle of the missing transverse energy. The distribution of the invariant mass of the three leptons and the OSSF pair. 
    The distribution over the significance of the missing transverse energy and the sum of $P_t$.]{\acs{MC} simulated and measured data 
    comparison showing the $\Delta R$ \ref{fig:deltaR} and the azimuthal
    angel \ref{fig:met_Phi} of the missing transverse energy. The distribution of the invariant mass of the
    three leptons \ref{fig:mlll} and the \acs{OSSF} pair \ref{fig:mll_OSSF}. The distribution over the significance
    of the missing transverse energy \ref{fig:met_Sign} and the sum of $P_t$ \ref{fig:Ht_lll}.}
\end{figure}
\newpage
\begin{figure}[H]
    \makebox[0.95\linewidth][c]{%
    \centering
    \begin{subfigure}{.405\textwidth}
        \includegraphics[width=\textwidth]{Figures/FeaturesHistograms/Ht_SS.pdf}
        \caption{}
        \label{fig:Ht_SS}
    \end{subfigure}
    \hfill
    \begin{subfigure}{.525\textwidth}
        \includegraphics[width=\textwidth]{Figures/FeaturesHistograms/Ht_met_Et.pdf}
        \caption{}
        \label{fig:Ht_met_Et}
    \end{subfigure}
    }
    \makebox[0.95\linewidth][c]{%
    \begin{subfigure}{.405\textwidth}
        \includegraphics[width=\textwidth]{Figures/FeaturesHistograms/njet_SG.pdf}
        \caption{}
        \label{fig:njet_SG}
    \end{subfigure}
    \hfill
    \begin{subfigure}{.525\textwidth}
        \includegraphics[width=\textwidth]{Figures/FeaturesHistograms/M_jj.pdf}
        \caption{}
        \label{fig:M_jj}
    \end{subfigure}
    }
    \makebox[0.95\linewidth][c]{%
    \centering
    \begin{subfigure}{.405\textwidth}
        \includegraphics[width=\textwidth]{Figures/FeaturesHistograms/nbjet77.pdf}
        \caption{}
        \label{fig:nbjet77}
    \end{subfigure}
    \hfill
    \begin{subfigure}{.525\textwidth}
        \includegraphics[width=\textwidth]{Figures/FeaturesHistograms/nbjet85.pdf}
        \caption{}
        \label{fig:nbjet85}
    \end{subfigure}
    }
    \caption[\acs{MC} simulated and measured data comparison showing the sum of $p_T$
    for the SS pair and the sum over all three leptons added with $E_t^{miss}$. The distribution of number of 
    signal jets and the mass of the leading dijet pair. Finally, the number of B-jets with $77\%$ and $85\%$ 
    certainty.]{\acs{MC} simulated and measured data comparison showing the sum of $p_T$
    for the \acs{SS} pair \ref{fig:Ht_SS} and the sum over all three leptons added with $E_t^{miss}$
    \ref{fig:Ht_met_Et}. The distribution of number of signal jets \ref{fig:njet_SG} and the mass 
    of the leading dijet pair \ref{fig:M_jj}. Finally, the number of B-jets with $77\%$ \ref{fig:nbjet77} and $85\%$ 
    \ref{fig:nbjet85} certainty.}
\end{figure}
\newpage
\newpage
\subsection{The Selection of Features}\label{subsec:FeatSelec}

\begin{table}[H]
    \centering
    $
    \begin{array}{cc}
        \hline \text {\textbf{Feature Name} }  & \text {\textbf{Description}} \\
        \hline \hline \text {$P_T$}  & \text {Transverse momentum} \\
        \text {$\eta$}  & \text {Pseudo rapidity} \\
        \text {$\phi$}  & \text {Azimuthal angle} \\
        \text {$M_T$}  & \text {Transverse mass} \\
        \text {$Charge$}  & \text {\ac{EM} charge} \\
        \text {$Flavour$}  & \text {Particle type} \\
        \text {$E_T^{miss}$}  & \text {Missing transverse energy} \\
        \text{$\phi(miss)$} & \text {Azimuthal angle of the missing transverse energy} \\
        \text{$M_{lll}$} &  \text {Invariant mass of the trilepton}\\
        \text{$M_{ll}(OSSF)$} & \text {Mass of the \ac{OSSF} pair} \\
        \text{Sig $E_T^{miss}$} & \text {Significance of $E_T^{miss}$} \\
        \text{$H_T(lll)$} &  \text {Sum of $p_T$ for all three leptons }\\
        \text{$H_T(SS)$} &  \text {Sum of $p_T$ for the \ac{SS} pair}\\
        \text{$H_T(lll)+E_T^{miss}$} & \text {-} \\
        \text{$\Delta R$} &  \text {Distance defined in the $\eta\phi$-space}\\
        \text{Flavor combo} &  \text {Combination of flavors for all three leptons}\\
        \text{Nr of signal Jets} &  \text {Nr of jets passing the signal criteria} \\
        \text{$M_{jj}$} & \text {Mass of the leading jet pair} \\
        \text{Nr of B-jets(77)} & \text {Number of B-jets with $77\%$ efficiency} \\
        \text{Nr of B-jets(85)} & \text {Number of B-jets with $85\%$ efficiency} \\
        \hline
    \end{array}
    $
    \caption{A summary and description of all features used in this analysis.}
    \label{table:Features}
\end{table}
\newpage