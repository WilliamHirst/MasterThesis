\section{Wilson Lines on Different Topologies}
In \cref{sec:Wilson lines and Wilson loops} we have seen that Wilson lines emerge naturally in gauge theories from a geometrical viewpoint. Because of its bi-local transformation property, it is used as a parallel transporter to render non-local terms gauge invariant. In the fibre bundle formalism we showed that its definition has a strong mathematical foundation. However, the application of Wilson lines are much broader than this, and we will exploit some of them in this chapter.

The Wilson line we defined in \cref{sec:Wilson lines and Wilson loops} is valid for any gauge theory, but as our main focus is on QCD we mostly use that the gauge fields are non-Abelian in nature. The physical consequence of involving Wilson lines in a theory becomes apparent if we expand \cref{definition:Wilson}





\section{Renormalization Group Equation for Wilson Lines}

\section{Anomalous Cusp Dimension at One Loop Order}