\chapter*{Abstract} 
This thesis explores a diverse array of \acf{ML} models as they search for chargino-neutralino pair production in 
three-lepton final states with missing transverse momentum. The study is based on a data set of $\sqrt{s} = 13$TeV proton-proton
collisions recorded with the \acs{ATLAS} detector at the \acs{LHC}, corresponding to an integrated luminosity of $139 fb^{-1}$. The \acs{ML} 
models applied in the study were three variants of \acf{DNN}, and \acf{BDT}. The \acs{DNN} variants included an ordinary 
dense \acf{NN}, \acf{PNN} and ensemble models utilizing pattern-specific pathways created by competing neurons. In the latter variant I 
included a novel layer introduced in this thesis, the \acf{SCO}. The study included an analysis of how each model attained sensitivity 
when training on a diverse data set including several orthogonal \acf{BSM} variants, i.e. different masses for the chargino and neutralino. 
Furthermore, a study was made on individual attributes of each model, for example the sparse pathways of the ensemble methods or the effect 
of the choice of parameters in the \acs{PNN}. In my studies I found that the inclusion of multiple signal variants can be beneficial during 
training of an \ac{ML} model in the case that the variants exhibit overlapping feature distributions. This is specifically true in 
the case that the model displays a strong long-term memory, as it was found that models utilizing sparse pathways do. When comparing each 
model in their ability to attain sensitivity, I found that the \acs{PNN} exhibited a bias towards high statistic signal which allowed it 
to attain impressive sensitivity in low mass regions. On the contrary, the ensemble methods which did not attain the same level of sensitivity 
on low mass signal, was able to achieve a far more balanced sensitivity for signal in both high and low mass regions. By performing 
a \acf{PCA} on the data set, I found it to improve the sensitivity of the ensemble methods and the \ac{PNN} for a majority of the mass combinations.
When comparing the models expected sensitivity to that achieved by \acs{ATLAS} I found that none of the models were able to extend the previously set 
exclusion limit on the masses of the chargino or neutralino. Via a more extensive study of the output from each model, specifically the ensemble networks which showed great sensitivity in non-excluded 
regions, I believe that more definite results could be achieved. 
