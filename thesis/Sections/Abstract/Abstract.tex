\chapter*{Abstract} 
This thesis explores a diverse array of \acl{ML} models as they search for chargino-neutralino pair production in 
three-lepton final states with missing transverse momentum. The study is based on a data set of $\sqrt{s} = 13$TeV proton-proton
collisions recorded with the ATLAS detector at the LHC, corresponding to an integrated luminosity of $139 fb^{-1}$. The machine 
learning models applied in the study were three variants of \acl{DNN}, and \acl{BDT}. The deep network variants included an ordinary 
dense neural network, \acl{PNN} and ensemble models utilizing sparse pathways. In the latter variant I included a novel layer introduced 
in this thesis, the \acl{SCO}. The study included an analysis of how each model attained sensitivity when training on a diverse data set including 
several orthogonal \acl{BSM} variants, i.e. different masses of the chargino and neutralino. Furthermore, a study was made on individual attributes of each model,
for example the sparse pathways of the ensemble methods or the effect of the choice of parameters in the \acl{PNN}. In my studies I found that the inclusion 
of multiple signal variants can be beneficial during training of a machine learning model, in the case that the variants exhibit overlapping feature distributions. 
This is specifically true in the case that the model displays a strong long-term memory, as it was found that models utilizing sparse pathways do. When comparing each 
model in their ability to attain sensitivity, I found that the \acl{PNN} exhibited a bias towards high statistic signal which allowed it to attain high sensitivity in low 
mass regions. On the contrary, the ensemble methods which did not attain the same level of sensitivity on low mass signal, was able to achieve a far more balanced sensitivity, for  
signal in both high and low mass regions. When comparing the models expected sensitivity to that achieved by ATLAS, I found that none of the models were able to extend the previously set
exclusion limit. Through a more extensive study of the output from each model, specifically the ensemble networks which showed great sensitivity in non-excluded regions, I believe that more definite results 
could be achieved. 
