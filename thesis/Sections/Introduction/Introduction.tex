\chapter*{Introduction}
\addcontentsline{toc}{chapter}{Introduction} 
The \ac{SM} of Particle Physics is one of the most successful scientific theories ever
created. It describes the interactions of leptons and quarks as well as the force
carrying particles which mediate said interactions. The model is a result of over a century of work
demanding the contributions of great minds like Paul Dirac (\emph{1902-1984}), Erwin Schrödinger (\emph{1887-1961}) and Richard Feynman (\emph{1918-1988}).
In 2012 the \ac{SM} achieved one of its crowning successes when we discovered the Higgs boson \cite{Aad_2012,the_cms_collaboration_observation_2012}. 
Much of the accolade was rightfully given to the theoretical work on the \ac{SM}, but another aspect of the discovery 
was equally important. Data analysis was and is a crucial part of any new discovery in physics. \newline
\\
In spite of the success of the \ac{SM}, there are still questions we are unable to answer.
The \ac{SM} is yet to include gravity or even explain the energy-matter density in the universe. Theoretical 
physicists are constantly at work trying to accommodate these aspects of the universe,
introducing extensions of the \ac{SM} such as \ac{SUSY} \cite{SUSY} or String Theory \cite{cole_probing_2021}. To test these theories precisely
we require larger and larger amounts of data. During the period of Run 3 the \ac{LHC} is projected to reach an integrated luminosity of approximately  
$500fb^{-1}$ with Run 4 reaching over $1000fb^{-1}$\footnote{The projections were taken from the graph created by CERN
\url{https://lhc-commissioning.web.cern.ch/schedule/images/LHC-nominal-lumi-projection.png}
(Accessed 21.04.2023).}. This is compared to the current $139fb^{-1}$ accumulated from Run 2. With the data sets in particle physics progressively growing, 
so does the demand for numerical algorithms and frameworks to analyze them. One of the most exciting tools which is playing and will play an important role in meeting 
this demand, is \acf{ML}.\newline
\\
\ac{ML} is rapidly becoming an indispensable tool in many scientific fields.
In areas ranging from cancer research \cite{SkinCancer} to predicting supernova events \cite{moller_photometric_2016}, \ac{ML} is being applied to problems
once thought as impossible to solve. Particle physics is no exception. Jet flavor classification \cite{Guest_2016}, 
separating jets from gluons \cite{PhysRevD.44.2025} or using \ac{ML} to create efficient signal regions \cite{baldi_searching_2014} are 
just some examples where \ac{ML} is a vital tool. The traditional approach for \ac{ML} in \ac{HEP} is the use of \acf{BDT} and shallow \acf{NNS}. 
Especially with the release of \verb!XGBoost! \cite{XGB} in 2014, the \ac{BDT} has been shown to be both stable and capable of achieving excellent 
performance. In later years supervised \acf{DNN} have become more and more popular, partly due to their versatility and diversity in architecture.
\newline
\\
This thesis will study \ac{ML} as it is applied to the search for new physics in proton-proton collisions produced by the \ac{LHC} and collected by 
the \acs{ATLAS} detector. Specifically, I have studied different \ac{ML} models as they search for chargino-neutralino pair production in final states 
with three leptons and missing transverse momentum. I explored a range of \ac{ML} models, such as ordinary dense \acs{NN}, \acf{PNN} \cite{PNN},
ensemble models including layers introduced in the paper by Wang et al. \cite{wang_maxout_2013} and \acl{BDT}. Additionally, I introduced a new layer, \acl{SCO}, which 
resembles the channel-out layer described in the aforementioned paper \cite{wang_maxout_2013}. Each model was studied and tested for its ability to achieve sensitivity in a 
diverse data set, including events representing different variations of new physics, particularly variations in choice of mass for the chargino and neutralino.
\newpage
\subsubsection*{Outline of the Thesis}
This thesis is divided into 4 chapters: the first two introducing relevant theory and background for the analysis; the third presenting
details on the implementation and preparation of the analysis; and the fourth presenting and discussing the results. At the end of the analysis I 
summarize the findings in my analysis in the \emph{Conclusion $\&$ Outlook} section, as well as include some additional figures and tables 
in the appendix. 
\\\newline
The \emph{first} chapter will give an introduction of the \ac{SM} as well as discuss the new physics I will be searching for. This chapter will 
introduce relevant phenomenology surrounding particle physics, give a brief description of proton-proton collisions at the \ac{LHC} and discuss 
the physics behind the data set utilized in the analysis. 
\\\newline
The \emph{second} chapter covers the necessary background in regard to \ac{ML} and data analysis in general. This chapter will introduce relevant topics
surrounding data analysis such as optimization, regularization and hyperparameters. It will explain the algorithms underlying the \ac{NN}s and 
\ac{BDT}s, and introduce techniques to improve these \ac{ML} methods. In the final parts of the chapter, I will discuss how \ac{ML} relates to 
a \acf{BSM} search and how one assesses the results.
\\\newline
The \emph{third} chapter describes the implementations of the analysis. This will include discussing the relevant frameworks, data formats and 
other tools, as well as diving a little further into the data set. Additionally, this chapter will present the preselection cuts and other preprocessing steps
used to generate the final data sets, both simulated and measured collision data, and present the comparison between the two. Finally, this chapter will present the 
models I studied in the analysis, displaying the architectures, and explaining the general strategy utilized for training and validating the models.
\\\newline
In the \emph{fourth} and final chapter, I present and discuss the results from the analysis. In the first four sections, I study four different categories of \ac{ML} models in how they 
perform on a subset of the data, as well as different attributes of each model. The fifth section compares the performance of the four models, then compares 
their results with and without a \acf{PCA} in the sixth section. Finally, I compare the three best models I found when testing on a subset of the data on how well they perform 
on the complete signal grid, then compare all three to previous results published by the \acs{ATLAS} Collaboration.



