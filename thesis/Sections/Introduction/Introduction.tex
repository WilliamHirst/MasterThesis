\chapter*{Introduction}
\addcontentsline{toc}{chapter}{Introduction} 
The best current understanding of the fundamental interaction between elementary particles are combined in the extraordinary successful theory called the Standard Model (SM). This model describes the electromagnetic, weak and strong interactions between particles. After the discovery of the Higgs boson in 2012, the last piece in this theory was in place. The only force that is not incorporated yet is gravitation, and there are several reasons to believe that there must be new physics beyond the Standard Model. In order to test the SM and search for new physics the Large Hadron Collider (LHC) was built. Because of the high energies the LHC is able to reach, one are able to study the SM at even higher precision. Unfortunately, so far there has been no signs of new physics. However, to fully appreciate high precision measurements, we need high precision predictions. To calculate observables in quantum field theory, we make a perturbative expansion in the coupling of the theory, and these mathematical expressions are often given graphical representation in the form of Feynman diagrams. Calculating these diagrams at leading order is usually trivial, but higher-order calculations are notoriously difficult, as they are ridden with divergences. The program for dealing with these divergences is what we call regularization and renormalization, which is well known. But even after renormalization has been performed there are sources for large corrections. Near the kinematical production threshold scattering cross sections gain large logarithmic corrections. This is due to the fact that close to the threshold the radiation of gauge bosons is restricted to be soft. This leads to an imbalance in the cancelling between real and virtual contributions at higher order, and in order to make reliable predictions such contributions must be resummed.

In this thesis we will focus on the theory of strong interactions, which is formulated in a field theory called Quantum Chromodynamics (QCD). This is a non-abelian gauge theory based on the symmetry group $SU(3)_{c}$. The formulation of this theory is fairly straightforward, but the physical implications are very complex. Perhaps the clearest confirmation of the complexity is that the fundamental constituents in QCD, the quarks and gluons, cannot be detected as free particles. They exist in colour-neutral states we call hadrons. This is known as confinement, and is still not theoretically understood. Another complication is that the force carriers, i.e. the gluons, are self-interacting and as a result QCD calculations tend to be very complicated. 

Another important aspect of QCD is \emph{asymptotic freedom}. It states that at very large energies the coupling between coloured particles becomes small, entering the realm of perturbation theory. However, for low energy interactions the coupling blows up and perturbation theory does not apply. This is a problem, because in real life experiments there will always be low energy radiation coming from high energy particles. In order to give reliable predictions, this part of the process must be taken account of and here enters the use of \emph{Wilson lines}. With Wilson lines the perturbation series can be re-exponentiated, such that the large coupling does not invalidate the perturbation series anymore. 

Wilson lines are fascinating objects, first introduced by Kenneth Wilson in attempting to explain confinement in QCD by considering a Wilson loop expectation value on the lattice \cite{Wilson:74}. They are path-ordered exponentials of the gauge fields and their definition follows directly from a parallel transport equation in gauge theory. They can be used to render bi-local operators gauge invariant and also to construct all terms that appear in a renormalizable and gauge invariant Lagrangian. They contain all the kinematical and dynamical information from the gauge sector and are central in taking a geometrical viewpoint on quantum field theory and in particular QCD. 

However, this is not the only usage of Wilson lines. As they are path-ordered exponentials and the coupling appears in the exponent, they can be expanded in the coupling and used in perturbation theory. Such an expansion will naturally describe gauge boson radiation. Here, we will mainly consider semi-infinite Wilson lines on linear paths, which can be used to describe radiation from highly energetic particles. Also, by constructing a special class of Wilson lines, namely Wilson lines on closed paths called Wilson loops, one can fully characterize the soft radiation. Further, by using factorization theorems in QCD one can define perturbative parton distributions $f_{i/i}(x)$. These distributions can in the $x\rightarrow 1$ limit be constructed in terms of a Wilson loop expectation value. Hence, by studying Wilson lines and Wilson loops in detail will give all we need in order to find an exponentiated cross section.




\subsubsection*{Outline of the Thesis}
This thesis is divided into four chapters. \cref{chap:Intro QFT} is meant to be a basic introduction to quantum field theory, where  the chapter is loosely divided into three parts. The first part is focused on the construction of Green's functions and the quantization procedure. In the second part the link between Green's functions and scattering amplitudes are made, and the last part focuses on explaining renormalization of quantum field theories.

In \cref{chap:Geometry of gauge theories} we take a closer look at the geometry of gauge theories. This chapter has two main parts, where the first is meant to give a review of the mathematical concepts in geometrizing gauge theories. In the second part we make use of several concepts from the geometrical formalism and construct the Yang-Mills Lagrangian from purely geometrical arguments. This is a natural introduction to Wilson lines and Wilson loops, as they are geometrical objects that governs the dynamics of gauge fields. After Wilson lines have been introduced, we go on and derive some of their properties that we will have use for in later chapters.   

In \cref{Chap:pQCD} we go into more detail about the properties of the strong force and its field theoretical description in QCD. This chapter can also be divided into two parts. In the first part important concepts such as factorization, parton distribution functions and running coupling are introduced and discussed. In introducing these concepts we use the most common experimental setup, namely deep inelastic scattering. From there we go on and define gauge invariant parton distribution functions by using Wilson lines, before we use Wilson lines to introduce perturbative parton-in-parton distributions. In the last part we take a closer look at another important process, namely the Drell-Yan process, which we follow in the remainder of the thesis. Here we make an explicit next-to-leading order (NLO) calculation by using dimensional regularization. Even after the process has been dealt with by dimensional regularization we have a collinear divergence. By using factorization theorems in QCD we show how to fully renormalize the cross section. We also show that even after renormalization and regularization the cross section has large logarithmic contributions in certain parts of phase space. Because of these large contributions the need for resummation is made clear. 

Lastly, in \cref{chap:Resummation in QCD} we make use of everything that we have introduced and discussed in the previous chapters. The chapter is meant to investigate the procedure of threshold resummation by the use of factorization theorems, Wilson lines and in particular calculations of Wilson loop expectation values. We begin with looking at how scattering amplitudes exponentiate, and how they naturally factorizes into hard and soft regimes. Then we go on and factorize the hadronic Drell-Yan cross section by using Mellin space techniques. After the cross section is fully factorized we construct Wilson loop expectation values and calculate their perturbative behaviour, which we show results in a exponentiated and resummed cross section.




