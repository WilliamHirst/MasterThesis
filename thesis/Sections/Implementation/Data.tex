\section{Tools and Data}
Every year technology for generating and measuring particle collisions is improved. 
As a consequence, the amount of data increases drastically. The ATLAS experiment
is one of the largest particle detector experiments currently operating at the 
CERN laboratory near Geneva. ATLAS alone generates approximately 1 petabyte of raw
data every second from proton-proton collisions at the \ac{LHC}. 
With amounts of data this large, data handling and storing is a big challenge. 
Therefore, taking advantage of sophisticated numerical tools and data frameworks is
pivotal if scientific development is to keep up with technological development.
\\
In this section I will cover some tools and frameworks I have used to 
complete my analysis. Large amounts of details and explanations will not be covered. 
Instead, this section will highlight which tools were used and some motivation
for choosing them. Additionally, I will cover some details regarding the data
being used, both \ac{MC} and real.
\subsection{Monte Carlo Data}
\subsection{ROOT and RDataframe}
ROOT \cite{ROOT} is at its core a large c++ library and data structure made specifically for big data
analysis and computing, as well as data visualization. Today, all ATLAS data is stored as a ROOT-file along
with more than 1 exabyte of data worldwide. ROOT has many \ac{HPC} qualities which makes it ideal for particle
physics analysis which demands heavy computations. Additionally, many particle physics-specific packages
have been developed to make it an even better tool. Any function not already in library,
can easily be added in a \ac{HPC}-effective manner through c++.
\\
All distribution plots made for this thesis were created using ROOT. ROOT has implemented a highly intuitive and
effective \ac{API} for data comparison and visualization. ROOT allows for quick and direct 
comparison between data through an advanced graphical user interface. Additionally, a lot of
functionality for creating complex stacked histogram are implemented in the ROOT \ac{API}, such
as uncertainty calculations and ordering of histograms.
\\
As for the data structure, the raw data was loaded as a ROOT-file. To easier handle the data and add
new features, I used the RDataframe structure \cite{RDataFrame}.
\subsection{Computing features in ROOT: Example}
\begin{lstlisting}
float ComputeInvariantMass(VecF_t& pt, VecF_t& eta, VecF_t& phi, VecF_t& m) {
  TLorentzVector p1;
  TLorentzVector p2;
  TLorentzVector p3;
  p1.SetPtEtaPhiM(pt[0], eta[0], phi[0], m[0]);
  p2.SetPtEtaPhiM(pt[1], eta[1], phi[1], m[1]);
  p3.SetPtEtaPhiM(pt[2], eta[2], phi[2], m[2]);
  return (p1 + p2 + p3).M();
}
\end{lstlisting}


