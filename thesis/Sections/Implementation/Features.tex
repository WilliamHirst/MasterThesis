\section{Selecting Features for the Analysis}\label{sec:Feats}
The choice of which features to study and which to neglect is crucial in a search for new physics. This is particularly true 
in the case of applying machine learning. The general motivation for including a given feature can be based on several factors. 
One reason being its ability to provide a trend which we can exploit when creating our signal regions. By this I mean
that it is a variable were there is a diversity in the distribution between the different channels. This is often physics motivated, for example 
including a feature on the transverse missing energy on the basis that we are searching for a signal with large amounts of it. 
Another reason is grounded in the \ac{MC}-simulation's ability to represent the variable. If there seems to be a clear deviation between the 
measured and simulated data, typically in some defined control regions believed to not contain any new physics, the feature could be adding noise 
to the data set, and it would not be included. In the following sections I will discuss the various features included in the data sets (both simulated and measured), 
which are also summarized in the appendix in table \ref{table:Features}. The features used in this analysis were inspired by several publications of \ac{ATLAS} I read in 
preparation of the analysis \cite{franchini_search_2019, atlas_search_2021}.