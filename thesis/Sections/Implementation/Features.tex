\section{Features Selection}
The choice of which features to study and which to neglect is crucial in a search for new physics. This is particularly true 
in the case of applying machine learning. The general motivation for including a given feature can be based on several factors. 
The first being its ability to provide a trend which we as researchers can exploit when creating ours signal search regions. By this I mean
that it is a variable were there is diversity in distribution between the different channels. The second motivation is grounded in 
physics. Often we as physicists tend to lean towards variables we know have some effect one the physics we are studying. For 
example the missing transverse energy, $E_T^{miss}$ can be directly used to either include or discard events were we do or do not expect final states
with sufficient missing energy. The final motivation is grounded in the \ac{MC}-simmulations ability to represent the variable.
If there seems to be a clear deviation between the real and \ac{MC}-data which does not stem from any new physics, we tend to discard
them from the analysis. In the following sections I will discuss the various features included in the data sets (both simulated and measured),
which are also summarized in table \ref{table:Features}.