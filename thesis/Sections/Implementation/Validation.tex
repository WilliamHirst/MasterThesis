\subsection*{Validation}\label{subsec:Validation}
As mentioned in previous sections, the comparison between \ac{SM} \ac{MC} and real data is a crucial part of the analysis, 
and we must therefore insure an adequate reconstruction of the real data before we begin the analysis. This is not only 
true for the low-level features, but for all features used in the analysis to create a new \ac{ML} variable. Therefore, 
we will in this section compare both sets of data for all features included in the analysis.
\begin{figure}[H]
    \renewcommand\figurename{Table}
    \centering
    \makebox[0.6\linewidth][c]{%
        \begin{subfigure}{.3\textwidth}
            $
            \begin{array}{cccc}
                \hline \text { Feature } & \text { $l_1$ } & \text { $l_2$ } & \text { $l_3$ } \\
                \hline\hline\text{$p_t$} & - & \ref{fig:lep2_Pt} & \ref{fig:lep3_Pt}\\
                \text{$\eta$} & - & \ref{fig:lep2_Eta} & \ref{fig:lep3_Eta}\\
                \text{$\phi$} & \ref{fig:lep1_Phi} & \ref{fig:lep2_Phi} & \ref{fig:lep3_Phi} \\
                \text{$M_T$} & \ref{fig:lep1_Mt} & \ref{fig:lep2_Mt} & \ref{fig:lep3_Mt} \\
                Charge & \ref{fig:lep1_Charge} & \ref{fig:lep2_Charge} & \ref{fig:lep3_Charge} \\
                Flavor & \ref{fig:lep1_Flavor} & \ref{fig:lep2_Flavor} & \ref{fig:lep3_Flavor} \\
                \hline
            \end{array}
            $
            \caption{}
            \label{table:Ref3L}
        \end{subfigure}
        \hfill
        \begin{subfigure}{.3\textwidth}
            $
            \begin{array}{cc}
                \hline \text { Feature } & \text { $Refrence$ }  \\
                \hline\hline\text{$\phi(miss)$} & \ref{fig:met_Phi} \\
                \text{$M_{lll}$} & \ref{fig:mlll}  \\
                \text{$M_{ll}(OSSF)$} & \ref{fig:mll_OSSF} \\
                \text{Sig $E_T^{miss}$} & \ref{fig:lep1_Charge} \\
                \text{$H_t(lll)$} & \ref{fig:Ht_lll} \\
                \text{$H_t(SS)$} & \ref{fig:Ht_SS} \\
                \text{$H_t(lll)+E_T^{miss}$} & \ref{fig:Ht_met_Et} \\
                \text{$\Delta R$} & \ref{fig:deltaR} \\
                \text{Nr of signal Jets} & \ref{fig:njet_SG} \\
                \text{$M_{jj}$} & \ref{fig:M_jj} \\
                \text{Nr of B-jets(77)} & \ref{fig:nbjet77} \\
                \text{Nr of B-jets(85)} & \ref{fig:nbjet85} \\
                \hline
            \end{array}
            $
            \caption{}
            \label{table:RefGen}
        \end{subfigure}
    }
    \caption{References to figures for all lepton (\ref{table:Ref3L}) and event (\ref{table:RefGen}) specific feature distribution.}
\end{figure}
In figure \ref{fig:Dist1}, I have drawn the event distribution for the $P_t$ (\ref{fig:lep1_Pt})
and $\eta$ (\ref{fig:lep1_Eta}) for the leading lepton, as well as the $E_t^{miss}$  (\ref{fig:met_Et}) and flavor combination 
(\ref{fig:flcomp}) of the final state. The error bars in each bin are set by default to $(\# events\ per\ bin)$. The 
distribution of the remaining features have been added in the appendix \ref{subsec:Dist}. I have added two tables with references 
to each feature not in the main thesis, in the case the reader is interested in studying specific features. Table \ref{table:Ref3L} 
contains all lepton specific features and table \ref{table:RefGen} contains all event specific features.
\\
Under each figure I drew the ratio between the measured collision data and the \ac{MC} for each bin. 
By studying the ratio-plots for each figure we observe that the ratio for all bins for all features are between 
1.2 and 0.8. Most bins even lying closer to 1. The bins displaying the largest errors are in the higher $P_t$-range.
This is exemplified in figures \ref{fig:lep1_Pt} and \ref{fig:lep1_Eta}\footnote{Due to $\eta$'s dependence on the polar angle (see 
equation \ref{eq:eta}), the larger the $P_t$ the higher the absolute value of $\eta$}. 
\\
In the $P_t$ figures we can observe that for (relatively) small $P_t$ ($<100GeV$) all events lay well within a range of $[0.9-1.1]$ ratio. 
Whereas for higher $P_t$ ($>200GeV$), the errors move closer to a range of $[0.8-1.2]$ ratio. The difficulty of simulating \ac{SM} processes 
in high $P_t$ range is a known phenomenon, and not specific to this analysis. Thankfully, a smaller portion of the data lays in the high $P_t$-range
and therefore most of the simulation exhibits a solid comparison to measured data. By studying the figures in the appendix \ref{subsec:Dist}, 
we can deduce that this trend continues thought the full feature set. Additionally to studying the ratio for each features in different bins,
we can read from the labels that there are a total of $381873$ measured collisons in the data, compared to $381860$ simulated events.
Simply put we observe that the \ac{MC} seems to adequately imitate the trends of the observed data for all features used in the analysis. 
\\
Apart from the excellent agreement between observed and simulated data, we can note a couple of other expected
but interesting points. The first being the size of each channel. $Z-jets$ is by far the largest channel followed
by the $Diboson (lll)$. Although $Z-jets$ is the largest channel, by comparing the Feynman-diagrams of each channel
(section \ref{sec:bkg}), $Diboson(lll)$ should be the hardest background to separate due to the similarities in the 
final-states of the signal\footnote{I.e. Diboson(lll) has a 3 lepton final state with large missing energy.}. Another 
point of interest is the differneces in distribution between the different \ac{SM} proccesess as displayed by the 
simulated data. By studying the $P_t$ in figure \ref{fig:lep1_Pt}, we observe that some channels echibit a distirbution  
peak for low $P_t$ (ca. 50 Gev) and then rapidly drop for higher values. This is the case for $Z-jets$ and $Dibson (llll)$.
On the other hand some processes seem to have a much slower decrease in distibution. This is true for $Diboson(lll)$, $Top-Other$
and $t\bar{t}$. The exact same trend is true also for $E_T^{miss}$, where $Z-jets$ and $Dibson (llll)$ rapidly drop for high values 
wheras $Diboson(lll)$, $Top-Other$ and $t\bar{t}$ decrease much slower. High amounts of missing energy is an indicator that 
$Diboson(lll)$, $Top-Other$ and $t\bar{t}$  could deem hard to separate from our $E_T^{miss}$ heavy signal.
\begin{figure}
    \makebox[0.95\linewidth][c]{%
    \centering
    \begin{subfigure}{.405\textwidth}
        \includegraphics[width=\textwidth]{Figures/FeaturesHistograms/lep1_Pt.pdf}
        \vspace{-1cm}
        \caption{}
        \label{fig:lep1_Pt}
    \end{subfigure}
    \hfill
    \begin{subfigure}{.525\textwidth}
        \includegraphics[width=\textwidth]{Figures/FeaturesHistograms/lep1_Eta.pdf}
        \vspace{-1cm}
        \caption{}
        \label{fig:lep1_Eta}
    \end{subfigure}
    }
    \makebox[0.95\linewidth][c]{%
    \begin{subfigure}{.405\textwidth}
        \includegraphics[width=\textwidth]{Figures/FeaturesHistograms/met_Et.pdf}
        \vspace{-0.75cm}
        \caption{}
        \label{fig:met_Et}
    \end{subfigure}
    \hfill
    \begin{subfigure}{.525\textwidth}
        \includegraphics[width=\textwidth]{Figures/FeaturesHistograms/flcomp.pdf}
        \vspace{-0.75cm}
        \caption{}
        \label{fig:flcomp}
    \end{subfigure}
    }
    \caption[\ac{MC} and real data comparison and event distribution for each channel for the $P_t$ and $\eta$ of the leading lepton 
    as well as the $E_T^{miss}$ and flavor combination from each event.]{\ac{MC} and real data comparison and event distribution for each channel over $P_t$ \ref{fig:lep1_Pt} and 
    $\eta$ \ref{fig:lep1_Eta} for the first lepton. Similarly, the distribution over the $E_T^{miss}$ \ref{fig:met_Et}
    and flavor combination of the three leptons \ref{fig:flcomp}.}
    \label{fig:Dist1}
\end{figure}