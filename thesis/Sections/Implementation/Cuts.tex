\subsection{Cuts and triggers}
To allow for deep learning and a thorough analysis one must try and keep
as much of the data as possible. At the same time, including large amounts
of irrelevant data can be both redundant and destruvtive. Therefore simple 
cuts are necessary. The cuts applied in the analysis were grouped in two 
definitions, baseline and signal. The baseline requriemtns are written in 
table \ref{table:BL} and the signal requirments are written in table \ref{table:SG}.
Both sets of requirments were taken from tha ATLAS article from 2022 \cite{franchini_search_2019}.
Then we demand that each event has exactly three signal and three baseline leptons, 
thereby removing any event with more or less. 
\begin{table}
\centering
$
\begin{array}{ccc}
    \hline \text { Requirement } & \text { Baseline electrons } & \text { Baseline muons } \\
    \hline\text{Identification} &  & \text{Loose} \\
    \eta \text { cut } & |\eta|<2.47 & |\eta|<2.7  \\
    \left|z_0 \sin (\theta)\right| \text { cut } & \left|z_0 \sin (\theta)\right|<0.5 \mathrm{~mm} & \left|z_0 \sin (\theta)\right|<0.5 \mathrm{~mm} \\
    \hline
\end{array}
$
\caption{Requirments for baseline electrons and muons.}
\label{table:BL}
\end{table}

\begin{table}
    \centering
    $
    \begin{array}{ccc}
        \hline \text { Requirement } & \text { Signal electrons } & \text { Signal muons } \\
        \hline\text{Baseline} & \text{yes} & \text{yes} \\
        \left|d_0\right| / \sigma_{d_0} \text { cut } & \left|d_0\right| / \sigma_{d_0}<5.0 & \left|d_0\right| / \sigma_{d_0}<3.0 \\
        \hline
    \end{array}
    $
    \caption{Requirments for signal electrons and muons.}
\label{table:SG}
\end{table}
In addition to the simple cuts, we must insure a good comparison between
MC- and real data. Often one finds large deviation betweem the two in the case
of either very large or very small $P_t$. The latter case can often be caused by 
poor reconstruction or missidentification. These are issues we aim to solve by checking
different triggers. Given our data set is composed of different data sets spread over
many years, different triggers are used. A list 

\begin{table}
    \centering
    $
    \begin{array}{ccc}
        \hline \text { 2015 } & \text { 2016 } & \text { 2017 + 2018 } \\
        \hline
        \text{HLT\_2e12\_lhloose\_L12EM10VH} & \text{HLT\_2e17\_lhvloose\_nod0} & \text{HLT\_2e17\_lhvloose\_nod0\_L12EM15VHI or} \\
        & & \text{ HLT\_2e24\_lhvloose\_nod0}\\
        \text{HLT\_e17\_lhloose\_mu14} & \text{HLT\_e17\_lhloose\_nod0\_mu14} & \text{HLT\_e17\_lhloose\_nod0\_mu14} \\
        \text{HLT\_mu18\_mu8noL1} & \text{HLT\_mu22\_mu8noL1} & \text{HLT\_mu22\_mu8noL1} \\
        \hline
    \end{array}
    $
    \caption{Trigger requirments for events produced in their respective years.}
\label{table:SG}
\end{table}
