\documentclass[UKenglish]{book}  
\newcommand\hmmax{0}
\newcommand\bmmax{0}

\counterwithout{footnote}{chapter}
\usepackage{Setup/style}
\DeclareMathAlphabet{\mathcal}{OMS}{cmsy}{m}{n}
\raggedbottom %%reduces the gaps between the paragraphs
\usepackage{jheppub}

\title{Search for heavy neutrinos in a 3-lepton final-state}       
\subtitle{Applications using supervised machine learning}
\subsubtitle{by}
\author{William Hirst}              
\thesistext{THESIS}
\thesistextt{for the degree of}
\thesistexttt{MASTER OF SCIENCE}


\bibliographystyle{JHEP}
%\addbibresource{bibliography.bib}         
%\DeclareUnicodeCharacter{2212}{-}
%\DeclareUnicodeCharacter{03B1}{-}

\begin{document}
\duoforside[dept={Department of Physics},
  program={Master's Program Name},
  long]                                         
%%%%%%%%%%%%%%%%%%%%%%%%%%%%%%%%%%%%%%%%%%%

\frontmatter{}
\chapter*{Abstract} 
This will be the abstract.

\chapter*{Acknowledgments}
Thank you, to nobody.
\tableofcontents{}

%%%%%%%%%%%%%%%%%%%%%%%%%%%%%%%%%%%%%%%%%%%
\mainmatter{}
\chapter*{Introduction}
\addcontentsline{toc}{chapter}{Introduction} 
The standard model (SM) is perhaps one of the most successful scientific theories ever created. It accurately explains the interactions of leptons and quarks as well as the force carrying particles which mediate said interactions. In 2012 the SM achieved one of its crowning achievements when we discovered the Higgs boson. Much of the accolade was rightfully given to the theoretical work on the SM, but another aspect of the discovery was equally important. Data analysis was and is a crucial part of any new discovery in physics. One of the most important and exiting tools is machine learning.




\subsubsection*{Outline of the Thesis}





%%%%%%%%%%%% ML %%%%%%%%%%%%%%%%%%%%%%%%%
\chapter{Introduction to Machine Learning and Data Analysis}\label{chap:Intro ML}
Machine learning is rapidly becoming an overwhelming presens in many scientific fields.
In areas ranging from cancer research to stock-trading, machine learning is being applied to problems
once thought as impossible to solve. Particle physics, like many other fields is no exception. Jet flavor classification \cite{Guest_2016}, 
separating jets from gluons \cite{PhysRevD.44.2025} or using \ac{ML} to create efficient \ac{SR} are just some examples
where \ac{ML} is a vital tool. The traditional approach for ML in high-energy physics is through the use 
of supervised learning. \ac{DNN}



\section{Discovery and significance}
\section{General search strategy}

\section{Exclusion}
\subsection{Likelihood}
\section{High- and Low-level features}



\section{The search through Machine Learning}
\subsection{Model selection}
In this analysis I have chosen to compare 4 different \ac{ML}-models, dens-\acf{NN}, \acf{PNN},
\ac{LWTA} and XGBoost. The first three methods are all types of \ac{NN}. I have deliberately 
chosen to focus on \ac{NN} given that there is far more freedom in the design of the architecture
of a \ac{NN}, than compared to a XGboost. Additionally, this was motivated by the selfish reason
being that I found the networks more interesting to study and dissect. Therefore, most of the 
analysis, comparisons and discussion is focused on the networks, while XGBoost is included 
as a loose benchmark. 
\\
The choice of the three network architectures is motivated in wanting to compare a simple 
deep network, a network ensemble and a \ac{PNN}. I would assume that the optimal architecture
would be a network which combines elements from each, but for the purpose of discussion
and research I have chosen to keep them somewhat separate. 
\subsection{Creating custom layers}\label{subsec:CustomLayer}
The field of \ac{ML} is one of the most dynamic and fastest growing fields of research
today. This means, that regardless of the brave attempt made by voluntary contributors and
the people at Google\footnote{The developers of TensorFlow}, there will always be 
new and exciting \ac{ML} tools, not yet implemented in their library. This was also 
the case in this thesis. Specifically, in the case of non-dense layers I was forced
to dive into the world of \ac{ML} development and create my own implementation. 
\subsubsection*{Max-out}
TensorFlow have already implemented a very similar layer called MaxPooling1D. This does 
exactly what max-out (see subsection \ref{subsubsec:maxout}) does, but with minor differences. 
Additionally, I wanted the freedom to experiment with the implementation of the layer. The 
implementation of both max-out and channel-out (see subsection \ref{subsubsec:channelout}) 
was done by creating a custom activation function which is called inside a dense-layer. 
\\
In the code listing, \ref{lst:max_out} I have included the code implementation of the
activation function used to create the max-out layer. The listing shows how the function
takes the input from the previous layer, defines the new shape of what will become the 
output, groups the nodes in the size defined by number of units, then returns an output 
which includes only the largest activated node from each unit. By using this function 
in a dense-layer, said layer will act as a max-out layer. 
\lstset{style=Python}
\begin{lstlisting}[caption={Python implementation for the custom activation function used to define the max-out layer.},captionpos=b, label={lst:max_out}]
def call(self, inputs: tf.Tensor) -> tf.Tensor:
    # Passing input through weight kernel and adding bias terms
    inputs = gen_math_ops.MatMul(a=inputs, b=self.kernel)
    inputs = nn_ops.bias_add(inputs, self.bias)

    num_inputs = inputs.shape[0]
    if num_inputs is None:
        num_inputs = -1
    num_competitors = self.units // self.num_groups
    new_shape = [num_inputs, self.num_groups, num_competitors]

    # Reshaping outputs such that they are grouped correctly
    inputs = tf.reshape(inputs, new_shape)
    # Finding maximum activation in each group
    outputs = tf.math.reduce_max(inputs, axis=-1,keepdims=True)

    counter = tf.where(tf.equal(inputs, outputs), outputs, 0.)
    # Reshaping outputs to original input shape
    self.counter = tf.reshape(counter, [num_inputs, self.units])

    return tf.reshape(outputs,[num_inputs, self.num_groups])   
\end{lstlisting}
\subsubsection*{Channel-out}
For the activation function defined to create the channel-out layer, I again grouped
the nodes similarly as I did for max-out. Instead of returning the largest value from each 
unit, I used TensorFlow's function, $tf.greater\_equal$ to create a tensor with booleans. The 
booleans are chosen by comparing each unit to the largest value in that unit. By multiplying 
this tensor to the original input, I am left with the desired result of a tensor containing 
the largest activated nodes along with the rest whom are now set to zero. 
\lstset{style=Python}
\begin{lstlisting}[caption={Python implementation for the custom activation function used to define the channel-out layer.},captionpos=b, label={lst:channel_out}]
def channel_out(inputs, num_units = 200, axis=None, training=None):
    # Calculate the new shape of the layer after max_out.
    shape = inputs.get_shape().as_list()
    if shape[0] is None:
        shape[0] = -1
    if axis is None:  # Assume that channel is the last dimension
        axis = -1
    num_channels = shape[axis]
    
    if num_channels % num_units:
        raise ValueError('Number of features is not a multiple of num_units')
    shape[axis] = num_units
    shape += [num_channels // num_units]
    grouped = tf.reshape(inputs, shape)
    # Calculate the largest value in each unit and discard 
    top_vals = tf.reduce_max(grouped, -1, keepdims=True)
    isMax = tf.reshape(tf.greater_equal(grouped, top_vals), [shape[0], num_channels])
    output = tf.multiply(tf.cast(isMax,tf.float32), inputs)
    return output  
\end{lstlisting}
\subsection{Model Architecture}
When choosing a network architecture, there are several ways to proceed. One way is to apply a grid search.
A grid search is simply defining a grid of parameters to test, then running through all combinations and 
choosing the highest performer. With a sufficient amount of tests, a grid search should converge towards 
an optimal architecture. Grid search is very common and there exists a large range of very complex varieties \cite{GS}.
For my analysis I chose not to perform a grid search, for several reasons. The first being interpretability.
Understanding a \ac{NN} is already hard, allowing for complex and unique architectures would only add another layer
of mysticism. The second is the size of the data set. The larger the data set, the larger the amount of data 
would be needed to adequately perform tests for each combination of parameters. Not only is this time-consuming,
but trying to mediate this issue could lead to poor performance. The third and most important reason is that 
I wanted to experiment with the architectures. By manually tuning the parameters, I was able to achieve a far 
better understanding of the final architecture. 
\begin{figure}
    \makebox[0.9\linewidth][c]{%
    \centering
    \begin{subfigure}{1.1\textwidth}
        \centering
        \includegraphics[width=\textwidth]{Figures/Illustrations/architecture.png}
    \end{subfigure}
    }
    \caption{A visual summary of the workflow and framework use for the 
    computational analysis. }
    \label{fig:arch}
\end{figure}
\subsection*{Dense Neural Network and PNN}
The purpose of the simple dense \ac{NN}, is to compare the more complex networks to what is usually considered a traditional \ac{NN}.
All layers are dense layers, meaning that all nodes in the previous layer are connected to the current layer, and likewise
the current layer is connected to the next. The general structure of the network is summarized in figure \ref{fig:arch}, with 
the network label \ac{NN}\footnote{Note that the dense network will henceforth be referred to as \ac{NN}}. The figure shows a \ac{DNN} with 
three hidden layers, all with 600 nodes each. All hidden layers utiliz the $LeakyReLU$ activation (see section \ref{subsec:activation})
with an $\alpha$ = 0.01. The architecture is designed to perform deep-training, and will train on a training set where all mass combinations 
are included\footnote{Contrary to training one network for each mass combination.}. 
\\
The \ac{PNN} architecture is like the name suggests, included to represent the model proposed by the article by Baldi et al. \cite{PNN}.
The architecture is illustrated in figure \ref{fig:arch}, with the label PNN. The figure shows a practically identical 
structure to the dense-\ac{NN}, with the only difference being in the input-layer. As was discussed in section \ref{subsec:PNN},
the \ac{PNN} includes the new physics signal free parameters\footnote{In our case, the masses of the \ac{BSM}-particles.} alongside the features
in the input layer.
\subsection*{MaxOut}
The MaxOut-network differs from the dense-network and the \ac{PNN} in that it uses an ensemble of networks. In figure \ref{fig:arch}
I have illustrated the MaxOut architecture, with the label MaxOut. The figure shows a network with 6 hidden layers, 3 MaxOut and 3 dropout.
The network alternates between drop out and MaxOut, starting with dropout and finishing with MaxOut. The MaxOut layers have 600 nodes each 
which reduce down to the 200 nodes with the largest activation in their respective groups. Each dropout layer has a dropout rate of 0.15.
\\
\subsection{Training and Validating data}
When building a \ac{ML}-model, the usual approach is to divide your data into three sets; training, validation and 
testing. The training set is used to tune the internal parameters of the model, i.e. the weights and biases of a \ac{NN} or the cuts of a \ac{DT}.
The validation set is used to tune the hyperparameters of the model, i.e. the architecture of the \ac{NN} or the maximum depth of the \ac{DT} etc.
The test set should only be used when the model is finished, and is used to benchmark the models' performance. In our case, the performance we are 
interested in, is the performance on the full \ac{MC}-bakgroundset and its comparison to the measured collision data.\footnote{See section INSERT 
THE APPROPRIATE SECTION.}. Therefore, in this analysis only two sets of data will be used, training and validation. In theory, one could even just 
use one data set (training) including all the data, but the second was added as a precaution to reduce overfitting when applied to the measured collision
data.
\\
The overarching strategy is summarized in the following points:
\begin{itemize}
    \item Shuffle the data set. 
    \item Split the data set in two, training ($80\%$) and validation ($20\%$)
    \item Scale the two data set such that the sum of the weights of the background is equal to the sum of the weights of the signal in each data set.
    \item Scale both data sets using the Standard Scalar approach (see section \ref{subsubsec:StandardScalar}) using the parameters of the training set 
          on both sets.
\end{itemize}
The first stem ensures an equal 


\section{Decision Trees and Gradient Boosting}
\subsection{Decision Trees}
\acf{DT} are, similarly to \ac{NN} some of the most popular \ac{ML} methods used today.
Contrary to \ac{NN}, \ac{DT} are relatively easy in theory. Despite the simplicity of \ac{DT}, 
they are capable of solving a large range of complex problems. In this section I will cover the use 
of \ac{DT} as applied to a supervised classification problem.
\\
The goal of decision trees is to create a flowchart-like tree structure from input to output. 
Similarly to the traditional \ac{CC} method used in particle physics, a \ac{DT} 
places \emph{rectangular-cuts} on a data set. It does so to create a collection of thresholds 
$\{c_1, c_2,...,c_{N_c}\}$, which when applied to a new data set and target, 
will sort each data point to the corresponding target. 
\\
In figure \ref{fig:DT}, I have illustrated a simple \ac{DT} classifying a 4-dimensional 
input data to one of three classifications. As is visualized in figure \ref{fig:DT}, the \ac{DT} 
applies a set of thresholds on the data to find the route applicable to a target. For each applied 
cut, the data splits into what we call \emph{branches}. Each branch represents a subset of the data.
The final subset, after a sufficient amount of cuts, ends in what we call \emph{leaves}. Each leaf contains
a label which is assigned to the subset of data of which the leaf contains.
\\  
Just like most \ac{ML}-methods there are many kinds of \ac{DT}, each with their own 
architecture and benefits. In the case of \ac{DT} the defining qualities can be summarized in
its choice of \emph{Depth} and \emph{Optimization}. When provided with a data set,
a \ac{DT} can in theory create as many cuts needed to map each individual data point, $\textbf{x}_i$, to the 
corresponding target, $t_i$. Doing so would not only be very computationally heavy, but would almost 
certainly lead to overfitting if applied to any new data. Instead, when building a \ac{DT} one defines 
a maximum depth. This means that we must define a limit to the number of cuts, which 
subsequently leads to the need for a prioritization of cuts. 
\\
Building a \ac{DT} and choosing which cuts to apply at what point, is the equivalent of training a 
network. How to decide by what standard one chooses to build the hierarchy of cuts is again the 
equivalent to choosing an optimizer. For more information on \ac{DT}, the reader is referred to the 
book by Hastie et al. \cite{huang_introduction_2014}.
\begin{figure}[H]
    \centering
    \includegraphics[width=0.6\textwidth]{Figures/Illustrations/DT.png}
    \caption{An illustration of a simple \acs{DT}, mapping a four dimensional input ($x_1,x_2,x_3,x_4$) 
    to one of three values in the target space ($y_1,y_2,y_3$), through a set of cuts $\{c_1, c_2,c_{3}\}$.}
    \label{fig:DT}
\end{figure}
\subsection{Gradient Boosting in Decision Trees}
Gradient-boosting is an algorithm which uses an ensemble of 'weak' 
classifiers in order to create one strong classifier. In the case of gradient-boosted 
trees the weak classifiers are an ensemble of shallow trees, which combine to form a classifier 
that allows for deeper learning. As is the case for most gradient-boosting 
techniques, the collecting of weak classifiers is an iterative process.
\\
We define an imperfect model $\mathcal{F}_m$, which is a collection of $m$ number of weak 
classifiers, or estimators. A prediction for the model on a given data-points, $\textbf{x}_i$, is 
defined as $\mathcal{F}_m(\textbf{x}_i)$, and the target for the aforementioned data is 
defined as $t_i$. The goal of the iterative process is to minimize some cost-function 
$\mathcal{C}$ by introducing a new estimator $h_m$ to compensate for any error, 
$\mathcal{C}(\mathcal{F}_m(\textbf{x}_i), y_i)$. In other words we define the new estimator as:
\begin{align}
    \tilde{\mathcal{C}}(\mathcal{F}_m(\textbf{x}_i), \textbf{y}_i) = h_m(\textbf{x}_i),
\end{align}
where we define $\tilde{\mathcal{C}}$ as some relation between the observed and 
predicted values such that when added to the initial prediction we minimize $\mathcal{C}$.
\\
Using our new estimator $h_m$, we can now define a new model as
\begin{align}
    \mathcal{F}_{m+1}(\textbf{x}_i) = \mathcal{F}_m + h_m (\textbf{x}_i).
\end{align}
Similarly to how we define a depth of trees, we can define the degree of boosting. We define 
this as the amount of trees used in the iterative process, or $M$. This means that the final classifier 
becomes
\begin{align}
    \mathcal{F}_M (\textbf{x}_i) = \sum_{i=0}^M h_i(\textbf{x}_i)
\end{align} 
The \verb!XGBoost! \cite{XGB} framework used in this analysis enables a (advanced) gradient-boosted algorithm, 
and was initially created for the Higgs ML challenge. Since the challenge, \verb!XGBoost! has become 
a favorite for many in the \ac{ML} community and has later won many other ML challenges. \verb!XGBoost! 
often outperforms ordinary decision trees, but what it gains in results it loses in 
interpretability. A single tree can easily be analyzed and dissected, but when the number 
of trees increases this becomes harder. For a more detailed explanation of the \verb!XGBoost! framework,
the reader is referred to \cite{XGB}.

%\section{Standard Model of Particle Physics}


\subsection{Classical Lagrangians of The Standard Model}
The Lagrangian can be considered as the fundamental basis of field theories. By considering a theory involving several types of fields on spacetime, scalar fields, spinor fields, gauge fields etc, then the Lagrangian of the field theory is a real valued function that contains the dynamics and all interactions between these fields. In classical field theory, once the Lagrangian is established, the evolution of the fields are given by differentiating the Lagrangian giving the Euler-Lagrange equations. In quantum field theory, the Lagrangian enters through the path integral, which are used to calculate correlation functions and scattering amplitudes for elementary particles. Depending on which types of fields and interactions of interest, the Lagrangian can be specified in each case. There are Lagrangians for non-interacting fields (free fields), Lagrangians for single interacting fields and several interacting fields. In general, Lagrangians containing only quadratic terms in the fields are free theories, while higher order terms leads to interactions. Interacting field theories are in general very complicated, and in case of weakly interacting theories these interactions are described using what is called Feynman diagrams, which are perturbative expansions in the coupling constants. For a given field, there are in general infinitely possible Lagrangians that one could consider, but in physics there are guiding principles that tells which one describes the reality that is observed. First, there is the restriction of symmetries, secondly the theory should be renormalizable and lastly the theory should be free of gauge anomalies. These principles restrict the possible Lagrangians that appear in the Standard Model, which contains: Yang-Mills Lagrangian, Higgs Lagrangian, Dirac Lagrangian and Yukawa terms. All these Lagrangians are Lorentz invariant, meaning invariant under local Lorentz transformation of the spacetime manifold, acting on each tangent space.

\subsection{Existence of symmetries}

\subsection{Renormalization criteria}
For a quantum field theory to predict finite results it must be renormalizable, meaning that the parameters in the Lagrangian must be renormalized in such a way that finite results are obtained and may be compared with experiments. This leads to the concept of running coupling, which we will study in more detail in (ref to section of renormalization). The action $S$ is given by the four dimensional space-time integral of the Lagrangian density, and it is a dimensionless quantity. From that observation it follows that the mass dimension of the Lagrangian is four. It can then be shown that the only renormalizable Lagrangians are the ones fulfilling mass dimension four, which greatly restricts the possible Lagrangians in the Standard Model.

\subsection{Gauge anomalies}
The fields in the Lagrangian describes classical fields, and does not describe quantum fields until the theory is quantized. This quantization procedure will be explained in more detail later using the path integral formalism. The symmetries of the classical Lagrangian does not in general hold after the quantization procedure, as the measure involved in the definition of the path integral may not be invariant under the symmetry. When the symmetry of the classical Lagrangian is not inherited in the quantum theory, it is said that the symmetry is anomolous. In the case of gauge symmetry there is an important result that says: In a four dimensional Minkowski spacetime, anomalies of gauge symmetry imply that the quantum theory violates unitarity. If unitrity does not hold, then there exists states of both negative and positive norm which mean that they can not have a probability interpretation, which violats one of the fundamental principals in quantum theory. Therefore it follows that the quantum theory must be free of gauge anomalies, and this actually restricts the representations and charges of fermions. The Standard model is free of gauge anomalies. For example, anomaly cancellation in the electroweak sector requires that leptons and quarks appear in complete multiplets of the form $(E_{L},e_{R},Q_{L},u_{R},d_{R})$, where the $L$ denotes left chiral and are doublets, while $R$ denotes right chiral and are singlets under the gauge group.


\subsection{Spontaneous Symmetry Breaking and the Higgs Mechanism}
There are several ways of breaking a symmetry, one type of symmetry breaking is to add non-invariant terms to a given Lagrangian which is invariant under a certain symmetry group. Another type which is the most relevant for describing the Standard Model is the concept of a spontaneous broken symmetry. In this case the defining Lagrangian is still exactly invariant under the symmetry $G$, however there exists a ground state, called the vacuum state $\phi_{0}$, which is not invariant under the symmetry $G$. According to Goldstone's theorem a spontaneous breaking of a continous symmetry introduces massless Goldstone bosons. Gauge fields arising from Yang-Mills theories describe massless particles, as there is no way to insert a mass term without explicitly breaking gauge invariance. By spontaneously breaking a gauge symmetry, previously massless gauge bosons acquire a mass, and this is called the Higgs mechanism. The customary approach to introducing this mechansim is by the introduction of a mexican hat potential, vevs and about degrees of freedom getting eaten. This approach is of course useful in studying Standard Model phenomonology, but talks very little about the underlying mathematical structure. Therefore a more geometric approach will be taken, and thereby motivating the mechanism in a more general sense and from there look at the specific case of the Standard Model.

\subsection{Higgs Lagrangian}
Geometrically the structure is that of a principal $G$-bundle $P(M,G)$, where $M$ is Minkowski space and $G$ is a compact Lie group. Additionally we equip the structure with a complex representation $\rho:G\rightarrow GL(V)$ with associated vector bundle $E=P\times_{\rho} V\rightarrow M$, where $V=\mathbb{C}^{n}$ as we want unitary representations of $G$ on $E$. The associated vector bundle $E$ is called the Higgs bundle, and a section $\bold{\phi}:M\rightarrow E$ are the Higgs field. The Higgs field is a scalar under Lorentz transformations, and we assume there exists a potential $V:E\rightarrow \mathbb{R}$.
\begin{align*}
    \mathcal{L}_{H}=(D_{\mu}\bold{\phi})^{\dagger}(D^{\mu}\bold{\phi})-V(\bold{\phi})
\end{align*}

%%%%%%%%%%%%%%%%%%%%%%%%%%%%%%%%%%%%%%%%%%%%%%%%%%%%%%%
or more generally, $\phi$ could describe a map
\begin{align}
    \phi:\,M\rightarrow N
\end{align}
from our manifold $M$ to some other manifold $N$, known as the \emph{target} space. For example, in a gauge theory, the fundamental field is a connection $\nabla$ on a principal $G$-bundle $P\rightarrow M$. Including matter fields, these are mathematically formulated as sections of vector bundles $E\rightarrow M$ associated to the principal $G$-bundle $P\rightarrow M$ by a choice of representation. Take scalar Quantum Electrodynamics, a theory involving a scalar field $\phi$ and a gauge field $A_{\mu}$, defined up to gauge transformations as
\begin{align}
    \phi\rightarrow \phi'&=e^{i\alpha}\phi
    \\
    A_{\mu}\rightarrow A'_{\mu}&=A_{\mu}+\partial_{\mu}\alpha
\end{align}
This is the local description of a connection $A_{\mu}$ on a principal $U(1)$-bundle, together with a section $\phi$ of a rank one complex vector-bundle $E\rightarrow M$, whose fibres are equipped with a hermitian metric. Another example is in General Relativity, where one naturally uses a Riemannian manifold, which naturally is equipped with various bundles, such as the tangent, cotangent and frame bundle. The fields we talk about in physics are sections on these bundles, which transform non-trivially under Lorentz transformations. Before we dive into technical details about bundles, and why they are the mathematical description of gauge theories, we will first connect special relativity with quantum mechanics in the classic approach of canonical quantization.
%%%%%%%%%%%%%%%%%%%%%%%%%%%%%%%%%%%%%%%%%%%%%%%%%%%%%%%%%%%
\chapter{Geometry of Gauge Theories and Wilson Lines}\label{chap:Geometry of gauge theories}

The notion of gauge invariance was first presented by Hermann Weyl in 1918 when he tried to unify electromagnetism with gravity from his own purely infinitesimal geometry. This unification was met with scepticism by other physicists as it lead to unphysical results, and Weyl first abandoned the idea. The problem in Weyl's gauge theory was that he looked at transformations of the metric, but as Weyl himself and others pointed out a decade later, the transformation had to be performed on the fields. Based on this formalism Wolfgang Pauli presented the first widely recognized gauge theory in 1941  \cite{RevModPhys.13.203}. Pauli also tried to generalize this to higher dimensional internal spaces, but could not find a way of giving mass to the gauge fields, so he figured this was a dead end and did not publish any of his results. An almost complete generalization of these concepts came about in 1954 by C.N.Yang and R.L.Mills, called Yang-Mills theories  \cite{PhysRev.96.191}. Yang and Mills encountered the same problems as Pauli related to the mass of the gauge fields, but they figured their idea was so important they published it either way\footnote{According to Yang, Pauli was furious when Yang presented his and Mills ideas and could not explain why the gauge bosons were massless.}. The problem of gauge boson masses were solved after the introduction of the \emph{Higgs mechanism}. Hence, Yang-Mills theories is the basis of one of the most successfull theories in all of physics, namely the \emph{Standard Model} of particle physics.


\medskip
In this chapter we will not take the same approach as Yang and Mills, where they generalized $U(1)$ invariance to $SU(N)$. We will instead derive Yang-Mills theories from completely geometrical concepts. To this end, we will introduce the mathematical language known as \emph{fibre bundle theory}. The theory of fibre bundles is purely mathematical and is intriguing in its own sense, but it also plays a prominent role in modern theoretical physics. Fundamental theories, like General Relativity and the Standard Model, are gauge field theories and the theory of fibre bundles provides a natural mathematical framework for these theories. The framework provides a clear separation between the kinematics and the dynamics of the theory. The kinematics is provided by the structure of the base manifold---in physics this represents spacetime---and the dynamics by the identification of a Lagrangian. In the case of the Standard Model, the internal symmetries of the Lagrangian, is made by a local construction of a fibre bundle with the fibre being the symmetry group $G$. From this construction the notion of sections, connection and curvature can be defined on the bundle, which represents physical fields in spacetime. After we have introduced the basic concepts and the structure that follows from fibre bundle theory, we will make use of them by constructing Wilson lines and Wilson loops. By using Wilson lines and Wilson loops we can construct the Yang-Mills Lagrangian.

There is of course much more to be said than we cover here, so we refer the reader to \cite{Carroll:2004st,Zee:2016fuk,Hamilton:2017gbn,Nakahara:2003nw} for a more elaborate treatment on differential geometry, fibre bundle theory and Wilson lines.

\section{Mathematical Concepts in Gauge Theories}
Before introducing fibre bundle theory and its relation with gauge theories, it is instructive to review some basic concepts from differential geometry and group theory as manifolds, tangent spaces, connections, curvature, Lie groups and Lie algebras. The objective here is not to dive into all the fundamental details, but a brief introduction to the most important concepts. We should mention that in order to follow the definition of the Wilson line and Wilson loop in \cref{sec:Wilson lines and Wilson loops}, the mathematical details of fibre bundle theory and differential geometry is strictly not needed. So the reader can jump straight to \cref{sec:Wilson lines and Wilson loops} and start from there. We will refer to the equations that we explicitly use, which can be sought out if necessary. 


\subsection{Basics in Differential
Geometry}\label{sec:Basics in differential geometry}
In this section we will briefly cover the basic concepts of a manifold, tangent space, contangent space, differential forms, Lie groups and Lie algebras, curvature, covariant derivative and parallel transport. As previously mentioned, the details are not at the level of mathematical rigour. We aim to introduce the simplest explanations and nomenclature that we will use throughout the chapter.

\subsection*{Manifold}
The concept of Manifolds is central in almost all theories of modern physics, from general relativity to quantum field theories, so we will have to define what a manifold is:

\begin{mydef}{Manifold}{}
Suppose $M$ is a topological space, then $M$ is a topological manifold if it has the following properties:
\begin{itemize}
    \item $M$ is a Hausdorff space: For every pair of points $x,y$ $\in$ $M$, there are disjoint open subsets U,V $\in$ $M$ such that $x$ $\in$ $U$ and $y$ $\in$ $V$
    \item $M$ is second countable: There exists a countable basis for the topology of $M$
    \item $M$ is locally Euclidean of dimension $n$: Every point has a neighborhood that is homeomorphic to an open subset of $\mathbb{R}^{n}$
\end{itemize}
\end{mydef}\noindent
In general the manifold might have a complicated global structure, but the defining property is to be homeomorphic to $\mathbb{R}^{n}$. The homeomorphism $\phi_{i}$ from $M$ to an open subset $U_{i}$ of $\mathbb{R}^{n}$ is called a chart
\begin{align}
    \phi_{i}:M\rightarrow U_{i}\in \mathbb{R}^{n}\,,
\end{align}
which permits us to assign co-ordinates to the manifold using those from $U_{i}$. Because a manifold in general globally differs from $\mathbb{R}^{n}$, we need to provide sets of charts $(U_i,\phi)$, called an open covering, such that all of $M$ is covered. Transitions between charts is described by smooth transition functions defined by
\begin{align}
    \phi_{i}\circ\phi^{-1}_{j}:U_{i}\rightarrow U_{j}\,,
\end{align}
which may be denoted as $\phi_{ij}$. Finally one needs to define some properties of the transition functions in case there is a overlap of the charts, so
\begin{align}
    \phi_{ii}&=\text{id}_{U_{i}}
    \\
    \phi_{ij}&=\phi^{-1}_{ji}
    \\
    \phi_{ik}&=\phi_{ij}\circ\phi_{jk}\,.
\end{align}
The above description may seem a little abstract, so what we basically mean and what is most relevant for our purposes is: a manifold $M$ is a set that can be continously parametrized. The number of independent parameters needed to specify uniquely any point of $M$ is its dimension $n$, and these parameters $x=\{x^{1},\hdots,x^{n}\}$ are called co-ordinates. Manifolds are then a generalization of the familiar space $\mathbb{R}^{n}$, in the sense that they can be viewed as smooth surfaces which locally look like $\mathbb{R}^{n}$, but in general has a completely different global structure. We demand the manifold to be smooth, i.e the transition from one set of co-ordinates to another $x^{i}=f(\tilde{x}^{i},\hdots,\tilde{x}^{n})$, is $C^{\infty}$\footnote{This means infinitely differentiable.}. A simple example of a manifold is the surface $S^{2}$ of a sphere in $\mathbb{R}^{3}$. Even if the sphere is in $\mathbb{R}^{3}$, the surface is a two dimensional manifold, because it locally looks like $\mathbb{R}^{2}$.

\subsection*{Tangent Spaces and Differential Forms}
A common type of field is what is called the tangent vector field, which assigns to each point $x\in M$ a vector $v(x)$ tangent to $M$. These vectors can be used to describe what is meant by vectors \textquote{moving} on the manifold. The space of all vectors tangent to $x\in M$ is a vector space, which is denoted as $T_{x}M$. An easy and visualizable example is again the surface of a sphere, $S^{2}$, where at every point on the surface there is a plane $T_{x}S^{2}$ that is tangent to $S^{2}$ at every point $x$.

For a coordinate $x^{\mu}$ defined on some neighborhood on the manifold, there is a tangent vector $e_{\mu}\in T_{x}M$ pointing in the direction of $x^{\mu}$ on the manifold and denotes \textquote{travel} at unit speed. These vectors can be denoted by $e_{\mu}\equiv \partial_{\mu}$, from the correspondence with the differential operator. Thus any vector in the neighborhood can be expanded in this basis, so $v=v^{\mu}\partial_{\mu}$, and these vector fields are called co-ordinate basis fields for the neighborhood.

Furthermore, to every tangent space $T_{x}M$ there is the notion of a dual space, called cotangent space $T^{*}_{x}M$. As it is dual to $T_{x}M$ it consists of linear maps $dx:T_{x}M\rightarrow \mathbb{R}$, such that
\begin{align}
    dx^{\mu}(\partial_{\nu})\equiv \delta^{\mu}_{\nu}\,,
\end{align}
and elements of the cotangent space $T^{*}_{x}M$ are called covectors or one-forms.

In the same sense that a vector field is a function that at each point in the manifold assigns a vector $v(x)$ in the tangent space $T_{x}M$, a one-form is a map $w$ which takes a point $x$ on the manifold to an element $w(x)$ of the corresponding cotangent space $T^{*}_{x}M$. In order to define $n$-forms we must first define a wedge product. For two elements of a vector space $v,u\in V$, their wedge product is defined as
\begin{align}
    v\wedge u\equiv\frac{1}{2}(vu-uv)\,,
\end{align}
which can be generalized to a wedge product for $n$-vectors, called the $n$th exterior product of $V$ and is a vector space denoted $\Lambda(V)$. A $n$-form is a map $w:M\rightarrow \Lambda^{n}(T^{*}M)$, which takes a point $x$ on the manifold $M$ to an element of $\Lambda^{n}(T^{*}M)$. We also have that a one-form can be expanded in the dual basis $dx^{\mu}$ of the cotangent space $T^{*}_{x}M$ in the following way
\begin{align}
    w=w_{\mu}dx^{\mu}\,,
\end{align}
where $w_{\mu}$ is the local co-ordinate representation of a one-form. Similarly a two-form may be expanded as
\begin{align}
    \Omega=\Omega_{\mu\nu}dx^{\mu}\wedge dx^{\nu}\,,
\end{align}
and so on for higher forms. The set of all $n$-forms on a manifold $M$ is often denoted by $\Omega^{n}(M)$.

\subsection*{Lie Groups and Lie Algebras}
In this section a brief introduction of Lie groups an Lie algebras will be given. As all gauge theories involve invariance under some symmetry operation, it has a natural formulation in terms of groups. The groups of gauge theories have the structure of a manifold, and so the concept of Lie groups and Lie algebras are fundamental in gauge theories.

%\medskip
%Recall that a group $G$ is a set of elements $g$ satisfying:
%\begin{itemize}
%    \item $\exists\, e\in G\,\,\, \text{s.t} \,\,\,eg=g,\, \forall\, g\in G$
%    \item $\forall\,\,g\in G, \,\exists\,g'\in G\,\,\,\text{s.t}\,\,\,g'g=e$
%    \item $\forall\,g,g'\in G, \,\,gg'\in G$
%\end{itemize}
Let us begin by defining a Lie group:
\medskip
\begin{mydef}{Lie Group}{example}
A Lie group $G$ is a group that is a finite-dimensional differentiable manifold, with the properties that the group operations are smooth. For group elements $g,g'\in G$ we specifically have that the product
\begin{align}
    \pi:\,G\times G&\rightarrow G\,,
    \\
    (g,g')&\mapsto g\cdot g'\,,
\end{align}
\textrm{and the inverse}
\begin{align}
    \rho:\,G&\rightarrow G\,,
    \\
    g&\mapsto g^{-1}\,,
\end{align}
\textrm{are smooth maps.}
\end{mydef}\noindent
Particularly important Lie groups in theoretical physics are the general linear groups $GL(n,V)$ of invertible $n\times n$ matrices. Some examples are the subgroups of $GL(n,V)$: the orthogonal group $O(n)$, the special orthogonal group $SO(n)$, the unitary group $U(n)$ and the special unitary group $SU(n)$. The special stands for the condition that the determinant of the matrices are one.

The concept of homomorphism is important in study of Lie groups, and in group theory in general. A homomorphism of Lie groups is given by a map $\rho:G\rightarrow H$ between elements of the groups, such that $\forall$ $g$, $g'$ $\in$ $G$
\begin{align}
    \rho(g\cdot g')=\rho(g)\cdot\rho(g')\,.
\end{align}
As physicists we are interested in groups where the elements of the groups act on physical states, say a quantum field or a wavefunction. Hence, we want to \textquote{represent} Lie group elements by linear transformations on some vector space $V$. Such a mapping is called a Lie group representation, and is a group representation where we have a homomorphism of Lie groups $\rho:G\rightarrow GL(V)$. To be more specific we are interested in the map
\begin{align}
    \rho:G\times V&\rightarrow V\,,
    \\
    (g,v)&\mapsto \rho(g)v\,,\hspace{1cm}g\in G,v\in V\,.
\end{align}
The physical states are members of a vector field, and we want our group members to act on them. With the definition of a representation this makes the transformation properties of the group to be written in terms of matrices.

Lie groups are complicated geometric objects and can be difficult to study directly, but the Lie algebra corresponding to a Lie group becomes important as it is closely related to the Lie group but easier to study.

\medskip
\begin{mydef}{Lie Algebra}{}
A Lie algebra is a vector space with a skew-symmetric bilinear map $[\,,\,]:\mathfrak{g}\times\mathfrak{g}\rightarrow\mathfrak{g}$ called the Lie bracket, satisfying the Jacobi identity for $X,Y,Z$ $\in$ $\mathfrak{g}$
\begin{align}
    [X,[Y,Z]+[Y,[Z,X]]+[Z,[X,Y]]=0\,.
\end{align}
\end{mydef}\noindent
It is because of this underlying vector space structure that the Lie algebra is easier to study than the Lie group itself. The Lie algebra encodes most of the group structure of the entire Lie group and many of the topological properties. If $G$ is a Lie group, then the Lie algebra $\mathfrak{g}$ of $G$ is the tangent space of the identity element of $G$, denoted $T_{e}G$.

Since Lie groups have the structure of a manifold, we can consider vector fields on $G$. The vector fields of interest in connection with Lie algebras are left and right-invariant vector fields\footnote{We only cover left-invariant vector fields here.}. A vector field $V$ is said to be left-invariant if $L^{*}_{g}V=V$, $\forall$ g $\in$ $G$. The set of all left-invariant vector fields on a Lie group $G$ is denoted $L(G)$, and for any two left-invariant vector fields their Lie bracket is also a left-invariant vector field. This means that $L(G)$ is isomorphic to the Lie algebra $\mathfrak{g}$, hence $L(G)$ can be considered to be the Lie algebra of $G$. So, for two left-invariant vector fields $V,W$, and $v,w\in T_{e}G$ we have that $V(e)=v$ and $W(e)=w$, therefore we can define the Lie bracket $[u,w]\in T_{e}G$ as the unique element in $T_{e}G$, such that
\begin{align}
    [v,w]\equiv[V,W](e)\,,
\end{align}
which turns $T_{e}G$ into an algebra. It is then possible to define a homomorphism of Lie algebras as a linear map $\rho:\mathfrak{g}\rightarrow\mathfrak{h}$, such that
\begin{align}
    \rho([u,w])=[\rho(u),\rho(w)]\hspace{1cm}\forall\, u,v \in \mathfrak{g}\,.
\end{align}

One important property of Lie algebras is that if we have a basis set $\{M_{1},M_{2}\dots M_{n}\}$ for $L(G)\cong T_{e}G$, then the commutator of these fields must be equal to a linear combination of the fields, i.e we may write
\begin{align}\label{eq:Lie commutator}
    [M_{\alpha},M_{\beta}]=C_{\alpha\beta}^{\gamma}M_{\gamma}\,,
\end{align}
where $C_{\alpha\beta}^{\gamma}$ are real numbers. These numbers are called the structure constants of the Lie group, i.e. they characterize the structure of the group. 

There is one important way of characterizing the Lie algebra $\mathfrak{g}$, via the exponential map. As we saw above the Lie algebra is the tangent space of the Lie group at the identity, thus one say that the Lie algebra gives a linearization of the Lie group near the identity. The exponential map can then be viewed as a delinearization, i.e. it take us back to the group. Thus we define:

\begin{mydef}{Exponential map}{}
The exponential map from the Lie algebra $\mathfrak{gl(n)}$ to the general Lie group $GL(n)$ is defined by
\begin{align}
    \exp:\mathfrak{gl}\rightarrow GL\,,
\end{align}
where
\begin{align}
    \exp(X)=\sum_{n=0}^{\infty}\frac{X^{n}}{n!}\,.
\end{align}
\end{mydef}\noindent
We observe that this is just the definition of an exponential of a matrix X. For any subgroup of $GL$, the Lie algebra of that group can be mapped into the group from the exponential map, meaning that any group that can be written in term of matrices can be constructed from the algebra in this precise manner. As physicists we often want our group elements to act on complex vector spaces, thus in order to preserve the inner product on a Hilbert space these transformations must be unitary. It can be shown that for compact Lie groups it is always possible to choose the unitary representation. If the transformation is unitary it means that one multiplies the argument of the exponential with a factor $i$, and this forces the matrices $X$ to hermitian.

\subsection*{Connection and Covariant Derivative}
In physics one typically apply differential geometry by a setup where the underlying manifold represents all of spacetime, where fields on that manifold is used to describe physical quantities. In order to describe evolution of these physical quantities one needs a precise mathematical description of calculus on manifolds. In order to see why this is needed, let us look at the problems that arise when defining the derivative of a vector field. The naive definition is
\begin{align}\label{eq:directional derivative vector field}
    \partial_{\mu}v^{\nu}=\lim_{dx^{\mu}\to 0}\frac{v^{\nu}(x+dx)-v^{\nu}(x)}{dx^{\mu}}\,,
\end{align}
which is not correct as the term $v^{\nu}(x+dx)$ and $v^{\nu}(x)$ live in different tangent spaces, meaning that the subtraction can not be made in a meaningful way. %The issue can also be seen when investigating the transformation of the partial derivative of a vector field
%\begin{align}
%    \partial_{\mu}v^{\nu}&=\left((\partial_{\mu}\tilde{x}^{\sigma})\tilde{\partial}_{\sigma}\right)\left((\tilde{\partial}_{\rho}{x}^{\nu})\tilde{v}^{\rho}\right)\nonumber
%    \\
%    &=(\partial_{\mu}\tilde{x}^{\sigma})(\tilde{\partial}_{\rho}x^{\nu})\tilde{\partial}_{\sigma}\tilde{v}^{\rho}+(\partial_{\mu}\tilde{x}^{\sigma})(\tilde{\partial}_{\sigma}\tilde{\partial}_{\rho}x^{\nu})\tilde{v}^{\rho}\,,
%\end{align}
%which are not components of a $(1,1)$ tensor, i.e. this object does not transform as a tensor and are thus not covariant. 
In order to properly define a derivative operator on a manifold, one needs to be able to compare tensors (fields) at different points. For this we need the concept of a linear connection.

\medskip
\begin{mydef}{Connection}{}
A linear connection $\nabla$ is defined as a map which sends a pair of smooth vector fields $V,U$ to a new smooth vector field:
\begin{align}
    \nabla: V\text{,}U\mapsto \nabla_{V}U\,.
\end{align}
Satisfying the following requirements:
\begin{align}
    \nabla_{V}(U+W)&=\nabla_{V}U+\nabla_{V}(W)\,,
    \\
    \nabla_{fV+U}W&=f\nabla_{V}(W)+\nabla_{U}(W)\,,
    \\
    \nabla_{V}(f)&=V(f)\,,
    \\
    \nabla_{V}(fU)&=f\nabla_{V}(U)+V(f)U\,,
\end{align}
where $f$ is a function.
\end{mydef}\noindent
The object $\nabla_{V}U$ is named the covariant derivative of $U$ with respect to $V$. From the last requirement, which is the Leibnitz rule, $\nabla$ is not a tensor as it is not linear in $U$. However, as a map $\nabla U:V\mapsto \nabla_{V}U$, which is a linear map $T_{x}M\rightarrow T_{x}M$, $\nabla U$ is a $(1,1)$ tensor known as the covariant derivative of $U$.

\medskip
In general we want to decompose vectors into components, so we choose a basis $\{e_{\mu}\}$, which is a basis in the tangent space. The conventional approach is to choose basis vectors that are tangential vectors along the coordinate lines $x^{\mu}$ in $M$, so
\begin{align}
    e_{\mu}=\partial_{\mu}\,.
\end{align}
As defined above the object $\nabla_{\partial_{\mu}}$ is a map taking $\partial_{\mu}$ to some vector field, so we define
\begin{align}
    \nabla_{\partial_{\mu}}\partial_{\nu}\equiv\nabla_{\mu}\partial_{\nu}\,.
\end{align}
This is now a new vector field, which can be expanded as a linear combination of the basis vectors
\begin{align}
    \nabla_{\mu}\partial_{\nu}=\Gamma^{\sigma}_{\nu\mu}\partial_{\sigma}\,,
\end{align}
where $\Gamma^{\sigma}_{\nu\mu}$ are connection coefficients, also known as Christoffel symbols\footnote{Also known as components of the Levi-Civita connection.}. It can be shown that they do not transform as a tensor, so therefore the indices does not describe the components of a tensor. 

We can now use our definition of the covariant derivative to see how it acts on vector fields described in terms of their components. Thus we write two vector fields in component form as $v=v^{\mu}\partial_{\mu}$ and $u=u^{\mu}\partial_{\mu}$, and then define what the covariant derivative on components are
\begin{align}\label{eq:covariant derivative in terms of components}
    \nabla_{v}u=(\nabla_{v}u)^{\mu}\partial_{\mu}\equiv(u^{\mu}_{\hspace{0.2cm};\nu}v^{\nu})\partial_{\mu}\,.
\end{align}
From the definition of the covariant derivative we can calculate the left hand side of \cref{eq:covariant derivative in terms of components},
\begin{align}
    (\nabla_{v}u)^{\mu}\partial_{\mu}&=\nabla_{v}(u^{\mu}\partial_{\mu})\nonumber
    \\
    &=v(u^{\mu}\partial_{\mu})+u^{\mu}(\nabla_{v}\partial_{\mu})\nonumber
    \\
    &=v^{\nu}\partial_{\nu}u^{\mu}\partial_{\mu}+u^{\mu}(\nabla_{(v^{\nu}\partial_{\nu})}\partial_{\mu})\nonumber
    \\
    &=\partial_{\nu}u^{\mu}v^{\nu}\partial_{\mu}+u^{\mu}(v^{\nu}\nabla_{\nu}\partial_{\mu})\nonumber
    \\
    &=\partial_{\nu}u^{\mu}v^{\nu}\partial_{\mu}+u^{\mu}(v^{\nu}\Gamma^{\sigma}_{\mu\nu}\partial_{\sigma})\,,
\end{align}
which mean we can write
\begin{align}
    (u^{\mu}_{\hspace{0.2cm};\nu}v^{\nu})\partial_{\mu}=\partial_{\nu}u^{\mu}v^{\nu}\partial_{\mu}+u^{\sigma}(v^{\nu}\Gamma^{\mu}_{\sigma\nu}\partial_{\mu})\,.
\end{align}
This is to hold for all $v^{\nu}$ and all $\partial_{\mu}$. Thus, the covariant derivative on a vector field in a co-ordinate induced basis is given by
\begin{align}\label{Covariant Derivative levi civita}
    \nabla_{\nu}u^{\mu}\equiv u^{\mu}_{\hspace{0.2cm};\nu}=\partial_{\nu}u^{\mu}+\Gamma^{\mu}_{\nu\sigma}u^{\sigma}\,,
\end{align}
i.e. if $\nabla$ is to obey the Leibnitz rule, it can be written as the partial derivative plus some linear transformation, where this linear transformation describes the correction in order to make the derivative covariant. Hence, for each direction $\mu$, the covariant $\nabla_{\mu}$ will be given by the partial derivative $\partial_{\mu}$ plus a correction specified by a set of $n$ matrices $(\Gamma_{\mu})^{\sigma}_{\hspace{0.2cm}\nu}$, where $n$ is the dimension of the manifold.

\subsection*{Parallel Transport and Curvature}
Parallel transport is the curved space generalization of keeping a vector (tensor) constant as we move it along a path\footnote{Or, as we shall see for fibre bundles a path in internal space.}. The crucial difference between flat and curved spaces is that, in a curved space, the result of parallel transporting a vector from one point to another will depend on the path taken between the two points. In flat space, the requirement that a vector is constant as we move it along a curve $x^{\mu}(\lambda)$, is that the components are constant, and is expressed as
\begin{align}
    \frac{d}{d\lambda}v^{\sigma}=\frac{dx^{\mu}}{d\lambda}\partial_{\mu}v^{\sigma}=0\,.
\end{align}
As we have shown, the partial derivative of a vector is not tensorial, and therefore the generalization is to use the covariant derivative and define the directional covariant derivative
\begin{align}
    \frac{D}{d\lambda}\equiv \frac{dx^{\mu}}{d\lambda}\nabla_{\mu}=n^{\mu}\nabla_{\mu}\,,
\end{align}
where $n^{\mu}$ follows from the parametrization $x^{\mu}=\lambda n^{\mu}$. This is now a map from $(k,l)$ tensors to $(k,l)$ tensors, defined only along the path. For a general tensor $T$, we define the parallel transport of $T$ along $x^{\mu}(\lambda)$ to be the requirement that the directional covariant derivative of $T$ is zero
\begin{align}
    n^{\mu}\nabla_{\mu}T^{\mu_{1}\dots\mu_{k}}_{\hspace{1cm}\nu_{1}\dots\nu_{k}}=0\,,
\end{align}
which in particular, for a vector field, gives that
\begin{align}
    \frac{dv^{\mu}}{d\lambda}+\Gamma^{\mu}_{\alpha\beta}\frac{dx^{\alpha}}{d\lambda}v^{\beta}=0\,.
\end{align}
Hence, the requirement that the covariant derivative of a tensor in a direction which it is parallel transported is zero, gives that the covariant derivative of a tensor measures how much the tensor changes as it is parallel transported.

\medskip
Given what we now know of covariant derivatives and parallel transportation, we can investigate what curvature is. Given the paths $\lambda_{1}$ and $\lambda_{2}$ with the same endpoint $p$, then parallel transporting along these two paths are in general not the same. Thus, a vector being parallel transported around a loop will be transformed, but the resulting transformation will depend on the total curvature around the loop. Therefore it would be more convenient to have a local description of the curvature at each point, and this is what the Riemann curvature tensor provides.

We can then consider parallel transportation around an infinitesimal loop, and as the manifold looks flat in sufficiently small regions, the loop will be specified by two infinitesimal vectors $a^{\mu}$ and $b^{\nu}$. First we parallel transport a vector $v^{\mu}$ in the direction of $a^{\mu}$, then along $b^{\nu}$, before moving backwards along $a^{\mu}$ and $b^{\nu}$, returning to the starting point. Since parallel transportation is independent of co-ordinates, there should be some tensor describing how much the vector changes when it comes back to its starting point. Instead of actually performing the calculation for the change of the vector as it is parallel transported, it is easier to look at the related operation of covariant derivatives. The commutator of two covariant derivatives compares the difference between parallel transporting a tensor along $a^{\mu}$, then along $y^{\nu}$, versus first along $b^{\nu}$ , then along $a^{\mu}$. In other words, the curvature is the measure of the failure of covariant derivatives to commute. The result of this operation on a vector is given by
\begin{align}\label{eq:commutator of covariant derivative}
    [\nabla_{\mu},\nabla_{\nu}]v^{\sigma}&=\big(\partial_{\mu}\Gamma^{\sigma}_{\nu\rho}-\partial_{\nu}\Gamma^{\sigma}_{\mu\rho}+\Gamma^{\sigma}_{\mu\alpha}\Gamma^{\alpha}_{\nu\rho}-\Gamma^{\sigma}_{\nu\alpha}\Gamma^{\alpha}_{\mu\rho}\big)v^{\rho}-2\Gamma^{\alpha}_{[\mu\nu]}\nabla_{\alpha}v^{\sigma}\,,
\end{align}
where $\Gamma_{[\mu\nu]}^{\alpha}=\Gamma_{\mu\nu}-\Gamma_{\nu\mu}=0$, as the connection coefficients are symmetric in the interchange of lower indices. The term inside the bracket of \cref{eq:commutator of covariant derivative} is known as the Riemann curvature tensor,
\begin{align}\label{Riemann curvature tensor}
    R^{\sigma}_{\rho\mu\nu}=\partial_{\mu}\Gamma^{\sigma}_{\nu\rho}-\partial_{\nu}\Gamma^{\sigma}_{\mu\rho}+\Gamma^{\sigma}_{\mu\alpha}\Gamma^{\alpha}_{\nu\rho}-\Gamma^{\sigma}_{\nu\alpha}\Gamma^{\alpha}_{\mu\rho}\,.
\end{align}
In general relativity the Riemann curvature tensor is important as the curvature describes how objects move. If we look at the $_{\mu\nu}$ components of the Riemann tensor, it has a striking resemblance with the field strength tensor $F_{\mu\nu}$ in Yang-Mills theories. This is not an accident, but it took physicists a long time to understand that Yang-Mills theories and General Relativity can both be formulated in the same mathematical language of fibre bundles\footnote{Have to emphasize that this is at the classical level. Both of these theories have to be quantized and to this day only Yang-Mills theories have given reliable physical results after quantization.}.

\medskip
Up to this point we have restricted our discussion to vector and tensor fields, but in general we are interested in other types of fields as well, like for example spinor and scalar fields. Hence, we have to extend the notion of a connection such that we can do calculus with all types of fields, and in order to do this properly we use fibre bundle theory. Naturally this leads us down a path where several mathematical concepts must be introduced, but it will prove useful to see the strength of fibre bundle theory to naturally describe several concepts we use in quantum field theory. It is important to emphasize that the material we will cover on fibre bundles is by no means a full treatment, but we will try to cover what is most important for physics. Several of the definitions and statements made might seem obscure and unmotivated, which they sometimes are, but it would take too much space to write all there is about fibre bundles. 

\subsection{Gauge Fields as Connections on Principal Fibre Bundles}\label{sec:basics in fibre bundle theory}
The defining property of a manifold $M$ is to be locally homeomorphic to $\mathbb{R}^{n}$, but in general it differs from $\mathbb{R}^{n}$ globally. Hence, we needed sets of homeomorphisms, which we called charts, to locally map $M$ to a open subset of $\mathbb{R}^{n}$. This gave a Euclidean structure to the manifold, which allowed us to use conventional multivariable calculus. A fiber bundle has the property of being locally disseomorphic to a direct product of topological spaces, thus we need disseomorphisms to define a local map. Let us jump straight into the definition of a fibre bundle, and then try to clarify some of the structure.

\begin{mydef}{Fibre Bundle}{}
A fibre bundle is a structure $(E,\pi,M,F,G)$, often denoted $E\overset{\pi}{\longrightarrow} M$, which consists of the following elements:
\begin{itemize}
    \item A smooth manifold $E$ called the \textbf{total space}
    \item A smooth manifold $M$ called the \textbf{base space}, and in physics this is spacetime.
    \item A smooth manifold $F$ called the \textbf{fibre}
    \item A surjective map $\pi:E\rightarrow M$ called the \textbf{projection}. The subset of elements $\{q\}\in E$ which are projected to a point $p\in M$ is called the fibre at $p$, given by the inverse image $\pi^{-1}(p)\equiv F_{p}\cong F$.
    \item A Lie group $G$, which acts on $F$ from the left called the \textbf{structure group}
    \item A set of open covering $\{U_{i}\}$ of $M$ with a diffeomorphism $\phi_{i}:U_{i}\times F\rightarrow \pi^{-1}(U_{i})$, such that $\pi\circ\phi_{i}(p,f)=p$. As the inverse $\phi_{i}^{-1}$ maps $\pi^{-1}(U_{i})$ to the direct product $U_{i}\times F$, $\phi_{i}$ is called a \textbf{local trivialization}
    \item A way to smoothly paste the direct products $\{U_{i}\times F\}$, such that we cover all of the total space $E$. As $\phi_{i}(p,f):F\rightarrow F_{p}$ is a disseomorphism we introduce \textbf{transition functions} $t_{ij}(p)\equiv \phi_{i,p}^{-1}\circ\phi_{j,p}:F\rightarrow F$, which we require to be elements of $G$. Then $\phi_{i}$ and $\phi_{j}$ is related by a smooth map $t_{ij}:U_{i}\cup U_{j}\rightarrow G$ as:
    \begin{align}
        \phi_{j}(p,f)=\phi_{i}(p,t_{ij}(p)f)
    \end{align}
\end{itemize}
\end{mydef}\noindent
The requirement of a local trivialization comes from the fact that fibre bundles are in general extremely complex structures, so in order to use them in practical applications we need to restrict ourselves with simpler ones. The simplest case is what is called a trivial bundle, where $E$ is isomorphic to the product $M\times F$. As it turns out we need to define a local isomorphism where $U\subset M$ such that $E|_{U}$ is locally isomorphic to $U\times F$, and the bundles of interest in gauge theories has this property. 

Another way phrasing the last requiremtnt is: if we have a overlap between charts $U_{i}\cup U_{j}\neq\emptyset$ in the base space $M$, we have two maps $\phi_{i}$ and $\phi_{j}$ on the overlap. Given a point $q$ such that $\pi(q)=p\in U_{i}\cup U_{j}$, we can assign two elements of $F$, one by $\phi^{-1}_{i}(q)=(p,f_i)$ and the other by $\phi^{-1}_{j}(u)=(p,f_j)$. Then there exists a map $t_{ij}:U_{i}\cup U_{j}\rightarrow G$ which relates $f_i$ and $f_j$ as $f_i=t_{ij}f_j$, and in order to glue the local pieces of the fibre bundle together consistently we need the transition functions to obey the following requirements
\begin{align}
    t_{ii}&=\text{id}\,,
    \\
    t_{ij}&=g^{-1}_{ji}\,,
    \\
    t_{ik}&=g_{ij}\circ g_{jk}\,.
\end{align}
For a given fibre bundle $E\overset{\pi}{\longrightarrow} M$, an important feature is that the transition functions are not unique. So let $\{U_{i}\}$ be a covering of $M$ and let $\{\phi_i\}$ and $\{\tilde{\phi_i}\}$ be two sets of local trivializations giving rise to the \emph{same} fibre bundle. Then the transition functions are given by
\begin{align}
    t_{ij}&=\phi^{-1}_{i,p}\circ\phi_{j,p}\,,
    \\
    \tilde{t}_{ij}&=\tilde{\phi}^{-1}_{i,p}\circ\tilde{\phi}_{j,p}\,,
\end{align}
and we have a homeomorphism $g_{i}:F\rightarrow F$ at each point $p\in M$ that belongs to $G$, we define it to be
\begin{align}
    g_{i}(p)\equiv \phi^{-1}_{i,p}\tilde{\phi}_{i,p}\,,
\end{align}
which must be the case if the local trivializations is to describe the same fibre bundle, but then we see that the relation between the two transition functions are
\begin{align}
    \tilde{t}_{ij}(p)=g_{i}(p)^{-1}\circ t_{ij}(p)\circ g_{j}(p)\,.
\end{align}

All of this might seem very abstract and without meaning, but let us clarify some points: $t_{ij}$ can be viewed as gauge transformations for gluing patches together, and $g_i$ can be viewed as gauge transformation within a certain patch. In physics, we will often meet the case when $U_i=U_j$, i.e. is the same set $U$, and we are just comparing two different ways of associating the fibre at $\pi^{-1}(U)$ to $U\times G$. In this case the local trivializations can be viewed as choices of gauges, and the transitions functions as gauge transformations. For example, when the base space is flat spacetime $M\cong \mathbb{R}^{4}$, and the symmetry group is $G\cong U(1)$. Then the transition functions at a spacetime point $x$ is expressed as $e^{i\alpha(x)}\in U(1)$, where $\alpha(x)$ is a spacetime dependent group parameter. Another example is when we have curved spacetime and the group is $GL(n,\mathbb{R})$, which is the case for general relativity. A choice of local trivialization is then a choice of co-ordinate system over an open patch $U$, and the transition function is just general co-ordinate transformations.

\subsection*{Sections and the Pull-back}
In order to define fields on a fiber bundle we need the concepts of \emph{sections}:

\begin{mydef}{Section}{}
A section of $E$ is a smooth map
\begin{align*}
    s:M\rightarrow E\,,
\end{align*}
that satisfies
\begin{align*}
    \pi\circ s=\text{id}_{M}\,,
\end{align*}
where $\pi$ is the projection in $E$.
\end{mydef}\noindent
The set of all sections on $M$ is denoted $\Gamma(M,F)$. As usual we want the local description, so for $U\subset M$ we have a local section defined only on $U$, where the set off all local sections is naturally denoted as $\Gamma(U,F)$. 

To define gauge fields in terms of sections we will need the \emph{pull-back}:
\begin{mydef}{Pull-back}{}
Let $\phi:M\rightarrow N$ be a smooth map of manifolds and let $\phi(q)=p$. Then we let
\begin{align}
    \phi_{*}:T_{p}M\rightarrow T_{q}N\,,
\end{align}
be the differential of $\phi$. The pull-back $\phi^{*}$ is the linear transformation taking covectors at $q$ to covectors at $p$, $\phi^{*}:N^{*}(q)\rightarrow M^{*}(p)$, defined by 
\begin{align}
    \phi^{*}(\beta)(v)\equiv\beta(\phi_{*}(v))\,,\label{eq:pullback}
\end{align}
for all covectors $\beta$ at $q$ and vectors $v$ at $p$.
\end{mydef}


\subsection*{Vector Bundles}
The fields in physics are objects in a vector space, so we need the concept of vector bundle to describe these fields in this formalism. A vector bundle $E\overset{\pi}{\longrightarrow} M$ is a fibre bundle whose fibre $F$ is a vector space. More loosely spoken, this means that we attach a vector space at each point $p$ on the base manifold $M$. If $F=\mathbb{R}^{n}$, the transition functions belong to $GL(n,\mathbb{R})$, and if $F=\mathbb{C}^{n}$, they belong to $GL(n,\mathbb{C})$.


A prime example of a vector bundle is the tangent bundle $TM$, where the fiber is $\mathbb{R}^{n}$. Let $\{U_i\}$ be an open covering of $M$, and let $q$ be a point in $TM$ such that the projection satisfies $\pi(q)=p\in U_{i}\cup U_{j}$. If we define $x^{\mu}$ to be the local co-ordinate system of $U_i$ and $y^{\mu}$ to be the local co-ordinate system of $U_j$, then the vector $V$ corresponding to $q$ can be expanded in two different ways
\begin{align}
    V=V^{\mu}\frac{\partial}{\partial x^{\mu}}=\tilde{V}^{\mu}\frac{\partial}{\partial y^{\mu}}\,,
\end{align}
where the local trivializations become
\begin{align}
    \phi_{i}^{-1}&=(p,\{V^{\mu}\})\,,
    \\
    \phi_{j}^{-1}&=(p,\{\tilde{V}^{\mu}\})\,,
\end{align}
and the fibre coordinates are related by a general linear transformation
\begin{align}
     V^{\mu}=G^{\mu}_{\,\,\nu}\tilde{V}^{\nu}\,,
\end{align}
where $G^{\mu}_{\,\,\nu}$ is the transition function, found by performing a change of frame
\begin{align}
    V^{\mu}=\frac{\partial x^{\mu}}{\partial y^{\nu}}\tilde{V}^{\mu}=G^{\mu}_{\nu}(p)\tilde{V}^{\nu}\,,
\end{align}
where $G^{\mu}_{\,\,\nu}$ is an element of the structure group $GL(n,\mathbb{R})$. Hence, the tangent bundle can be identified by the structure $(TM,\pi,M,\mathbb{R}^{n},GL(n,\mathbb{R}^{n}))$.

Sections on vector bundles pointwisely obey the usual vector multiplication and addition with scalars,
\begin{align}
    (s+s')(p)&=s(p)+s'(p)\,,
    \\
    (fs)(p)&=f(p)s(p.)\,,
\end{align}
where $p\in M$ and $f\in F$. Any vector bundle admits a global section which is called the null section $s_{0}\in \Gamma(M,E)$, that satisfies the property $\phi_{i}^{-1}(s_{0}(p))=(p,0)$ in any local trivialization. What we want in physical applications is to use that any field or wavefunction can be represented as a section of a vector bundle. For example, a $U(1)$ gauge group, a complex scalar field $\Psi(x)$ defined on $U\subset M$ is represented by a local section of a complex line bundle.

\subsection*{Principal Bundles and Gauge Transformations}
In order to describe gauge transformations on fields, we need the concept of principal bundles. A principal bundle $P(M,G,\pi)$ is a fibre bundle where the fibre $F$ is identical to the structure group $G$, also called a $G$-bundle over $M$. The action of the group $G$ on $F$ becomes simple left multiplication within $G$. It is also possible to construct a right multiplication of $G$ on $P$ as: let $\phi_{i}:U_{i}\times G\rightarrow \pi^{-1}(U_{i})$ be a local trivialization given by
\begin{align}
    \phi_{i}^{-1}(q)=(p,g_i)\,.
\end{align}
Then the right multiplication is defined as
\begin{align}
    qa=\phi_{i}(p,g_{i}a)\,,
\end{align}
for any $a\in G$ and $q\in \pi^{-1}(p)$, with property $\pi(qa)=\pi(q)=p$. Because of the associativity of the group, this is true for any local trivialization. Let $p\in U_{i}\cup U_{j}$, then
\begin{align}
    qa=\phi_{j}(p,g_{j}a)=\phi_{j}(p,t_{ji}(p)g_{i}a)=\phi_{i}(p,g_{i}a)\,,
\end{align}
which mean one can just write the action as $P\times G\rightarrow P:(q,a)\mapsto qa$, without reference to any local choices. In other words, if $q$ is a point in the fibre over $p$ then acting with a group element $a$ gives another point $qa$ in the fibre over $p$, with the property $\pi(qa)=p$ so
that both $q$ and $qa$ lie in the same copy of the fibre. This means that the group action enables us to move within each copy of the fibre, or equivalently for principal bundles each copy of $G$, but does not move you around in the base space $M$. Since the fibre is equal to the structure group, also called symmetry group in physics, principal bundles is central to the description of gauge theories, and the right action on $P$ can be identified with gauge transformations. 

Since a section on a principal bundle is a map $s:M\rightarrow P$, then the value of a section at a point $p$ corresponds to an element of the structure group $G$ through a local trivialization
\begin{align}
    s(p)=\phi_{i}(p,g_{i})\,.
\end{align}
Unless the principal bundle is a direct product $M\times F$ we need local sections. Given a local section $s_i$ on $U_i$ and a $q\in \pi^{-1}(U_i)$, we can always find a $g_{i}\in G$ such that $q=s_{i}(p)g_{i}$. Then the section itself may be represented as a canonical local trivialization
\begin{align}
    s_{i}(p)=\phi_{i}(p,e)\,,
\end{align}
where $e$ is the identity element of $G$. All other local sections can then be expressed in terms of these by the right action as
\begin{align}
    \tilde{s}_{i}(p)=\phi_{i}(p,g_{i}(p))=\phi_{i}(p,e)g_{i}(p)=s_{i}(p)g_{i}(p)\,.
\end{align}
The different $\tilde{s}_{i}$ is viewed as different gauges, while $g_{i}(p)$ is the corresponding gauge transformations between them.

\subsection*{Associated bundles and Field Transformations}
We have seen how sections on vector bundles can be used to describe fields, and how sections on principal bundles describe gauge transformations. We are now ready to see how we may associate these two concepts such that the gauge transformations act on the fields.

Given a principal bundle $P(M,G)$ and a faithful representaiton $\rho:G\rightarrow GL(n,V)$ which acts on a vector space $V$ from the left. The group action on elements $(q,v)$ in the product space $P\times_{\rho} V$ can then be defined as
\begin{align}
    (q,v)\rightarrow  (qg,\rho(g)^{-1}v)\,,
\end{align}
where $q\in P$, $g\in G$ and $v\in V$. The associated vector bundle $E_{\rho}=E\times_{\rho}V$ is then defined by identifying the points related by such a group action, so
\begin{align}
    (q,v)\sim (qg,\rho(g)^{-1}v)\,,
\end{align}
which also implies that $(qg,v)=(q,\rho(g)v)$. This is the same as saying that we change the fiber from $G$ to $V$ and use transition functions $\rho(t_{ij})$ instead of $t_{ij}$. As every element of $P$ over a point $p\in M$ can be found from $(p,e)$ by an element of $G$, the equivalence relation replaces the fibre over $p$ with V, and thus replacing a principal bundle with a vector bundle. We can introduce the projection $\pi_{E_{\rho}}:E_{\rho}\rightarrow M$, defined by acting on elements $(q,v)$ as
\begin{align}
    \pi_{E_{\rho}}((q,v))=\pi(q)\,.
\end{align}
Then $E_{\rho}$ is a fibre bundle with the same structure group $G$ as its associated principal bundle $P(M,G)$. 

Let us consider the $U(1)$ group and a complex scalar matter field $\phi:M\rightarrow \mathbb{C}$.  The relevant fibre bundles are the principal $U(1)$-bundle $P(M,U(1))$ and its associated vector bundle $E_{\rho}=P\times_{\rho}\mathbb{C}$. A local section on a principal bundle may be represented as a canonical local trivialization
\begin{align}
    s_{i}(p)=\phi_{i}(p,e)\,,
\end{align}
with $e$ the identity element of $U(1)$. Within a trivializing neighborhood on a principal bundle we can define a local identity section $\sigma_{i}(p)\equiv \phi^{-1}(e)$, which mean that if we act with $\phi^{-1}$ on the left on the local trivialization we obtain
\begin{align}
    \phi_{i}^{-1}(s_{i}(p))=(p,e)\,.
\end{align}
All other local sections may then be constructed by the right action of the group $\tilde{s}_{i}(p)=s_{i}(p)g_{i}(p)$, where $g_{i}\in U(1)$. To make $\tilde{s}_i$ act on $\phi(p)$ we use the representation $\rho:g_{i}(p)\rightarrow e^{i\alpha(p)}$ and define a base section on the associated vector bundle
\begin{align}
    \mathbf{e}=[(s_{i}(p),1)]\hspace{0.5cm}\in E_{\rho}\,,
\end{align}
where $1$ is a basis vector in the complex line bundle. Naturally any other section may then be expressed in terms of these base sections, so we define our field as a section in the following way
\begin{align}
    \phi(p)\sigma_{e}=[(s_{i}(p),\phi(p))]\,,
\end{align}
where $\phi\in \mathbb{C}$, and a local gauge transformation corresponds to
\begin{align}
    \phi^{'}(p)\sigma_{e}&=[(\tilde{s}_{i}(p),\phi(p))]\nonumber
    \\
    &=[(s_{i}(p)g_{i}(p),\phi(p))]\nonumber
    \\
    &=[(s_{i}(p),\rho(g_{i})\phi(p))]\nonumber
    \\
    &=[(s_{i}(p),e^{i\alpha(p)}\phi(p))]\nonumber
    \\
    &=e^{i\alpha(p)}[(s_{i}(p),\phi(p)]\nonumber
    \\
    &=e^{i\alpha(p)}\phi(p)\sigma_{e}\,.
\end{align}
In physics we always choose local co-ordinates, so for $x^{\mu}$ we simply write this transformation as
\begin{align}
    \phi^{'}(x)=e^{i\alpha(x)}\phi(x)\,,
\end{align}
which is the well known form of the local $U(1)$ gauge transformation in quantum electrodynamics. 

We have now seen how sections on associated bundles can be formulated as matter fields and we have shown how these are transformed under a local $U(1)$ gauge transformation. The next objective is then to investigate how one can construct gauge fields and their corresponding behaviour under gauge transformations in this language.

\subsection*{Connection and Gauge Fields on Principal Bundles}
There are several ways of defining a connection on a principal bundle, but the approach we use here is to decompose tangent spaces into vertical and horizontal ones. In \cref{Covariant Derivative levi civita} we defined a linear connection to make the derivative of vector fields covariant, there in terms of the Levi-Civita connection. Here we will take another approach, where we instead define the connection in terms of sections, from which the gauge fields and the field strength tensor follows. This is approach is very abstract, but it is necessary in order to show that the gauge fields has a geometrical basis. In \cref{sec:Wilson lines} we will use the results derived here and take a more physical approach at the level of Lagrangians, by the use of Wilson lines. 

\medskip
First, we want to investigate how we can decompose the tangent space of a principal bundle, but to do that we must describe how we may construct a tangent vector in $P(M,G)$. A fundamental vector field $\textbf{v}$ may be generated through an element $A$ of a Lie algebra $\mathfrak{g}$ of the $G$-bundle $P(M,G)$, in the following way
\begin{align}
    \textbf{v}f(q)=\frac{d}{dt}f(qe^{tA})|_{t=0}\,.
\end{align}
Now, since $e^{tA}\in G$ we have that the projection $\pi(qe^{tA})=\pi(q)=p$. This means that $qe^{tA}$ defines a curve that lies within the fibre at $p$, and thus $\textbf{v}$ is tangent to the fibre at $p$ at every point $q\in P$. 

The tangent space $T_{q}P$ at $q\in P$, can be decomposed into a horizontal and vertical subspaces in the following way
\begin{align}
    T_{q}P=V_{q}P\otimes H_{q}P\,,
\end{align}
such that every vector $\textbf{x}$ in the tangent space may also be decomposed into vertical and horizontal components $\textbf{x}=\textbf{x}^{V}+\textbf{x}^{H}$. With this decomposition we are ready to define what a connection in this language is:

\medskip
\begin{mydef}{Connection one-form}{}
A connection on a Principal $G$-bundle $P(M,G)$ is a Lie algebra $\mathfrak{g}$ valued one-form $w\in \mathfrak{g}\otimes \Omega^{1}(P)$ that projects elements in $T_{q}P$ onto $V_{q}P\cong \mathfrak{g}$, satisfying the following requirements
\begin{itemize}
    \item $w(\textbf{v})=A\hspace{1.8cm}A\in \mathfrak{g}$
    \item $R_{g}w=g^{-1}wg\hspace{1cm}g\in G$
\end{itemize}
where $R_g$ describes the right action of the group.
\end{mydef}\noindent
Given $U\subset M$ and local sections $s_{i}:U_i\rightarrow \pi^{-1}(U_i)$, a local connection $A_{i}\in \mathfrak{g}\otimes \Omega^{1}(U_i)$ is defined by the pull-back of the global connection one-form $w$,
\begin{align}
    A_{i}\equiv s_{i}^{*}w\,.
\end{align}
Local connections is what we in physics call a gauge potential. Instead of this abstract definition, in practice we use that since $A_{i}$ is a Lie algebra valued one-form it can be expanded in a dual basis $dx^{\mu}$ and in terms of Lie algebra generators $t^{a}:U\rightarrow G$, in the following way
\begin{align}
    A_{i}=(A_i)_{\mu}dx^{\mu}=(A_i)^{a}_{\mu}t^{a}dx^{\mu}\,.
\end{align}
Given two local sections $s_{i}$ and $s_j$ over patches $U_{i}$ and $U_{j}$\footnote{With the requiremnt that $U_{i}\cap U_{j}\neq\emptyset$.}, $X\in T_{q}M$ and $q\in U_{i}\cap U_{j}$, it can be shown that 
\begin{align}\label{eq:important for parallel transport}
    s_{j\,*}X=R_{t_{ij}}(s_{i\,*}X)+\big(t_{ij}^{-1}\textbf{d}t_{ij}\big)\,,
\end{align}
where $t_{ij}\in G$ are the transition functions and $\textbf{d}$ is the de-Rham-differential. If we use the connection $w$ on this equation, with the relation $w(s_{j\,*})=s_{j}^{*}w$ from \cref{eq:pullback}, we have that the local connection transform as
\begin{align}\label{gauge field transformation}
    {A_{j}}=t_{ij}^{-1}A_{i}t_{ij}+t_{ij}^{-1}\textbf{d}t_{ij}\,,
\end{align}
which in the more familiar component form is written as
\begin{align}\label{eq:gauge field connection transformation}
    A'_{\mu}=g^{-1}A_{\mu}g+g^{-1}\partial_{\mu}g\,,
\end{align}
which is the transformation for the gauge fields in quantum field theory. 

\subsection{Curvature and Field Strength}
In order to talk about curvature (or field strength) in this language, we have to define an exterior covariant derivative:

\medskip
\begin{mydef}{Exterior Covariant Derivative}{}
The exterior covariant derivative of a general vector valued n-form $\Phi(x)\in \Omega^{n}\otimes V$ \,, $x_1,\dots,x_{n+1}\in T_{q}P$, is defined as:
\begin{align}
    \textbf{d}_{w}\Phi(x_1,\dots, x_{p+1})\equiv \textbf{d}\Phi(x_{1}^{H},\dots,x_{p+1}^{H})\,,
\end{align}
where $\textbf{d}\Phi=\textbf{d}\Phi^{\alpha}\otimes e_{\alpha}$, and $x^{H}\in H_{q}P$.
\end{mydef}\noindent
The curvature is a Lie algebra valued two-form $\Omega\in\Omega^{2}\otimes \mathfrak{g}$, and is defined as the exterior covariant derivative of the one-form connection $w\in\Omega^{1}\otimes \mathfrak{g}$
\begin{align}
    \Omega\equiv\textbf{d}_{w}w\,,
\end{align}
which for $x,y\in T_{q}P$ satisfies Cartan's structure equation
\begin{align}\label{Cartan structure equation}
    \Omega(x,y)=\textbf{d}w(x,y)+\mathbf{[}w(x),w(y)\mathbf{]}\,,
\end{align}
where the bracket is a tensor product of the Lie bracket and the wedge product, i.e. the curvature can be written as
\begin{align}
    \Omega=\textbf{d}w+w\wedge w\,.
\end{align}
Just as the local connection the local curvature is given by the pull-back of a local section
\begin{align}
    F_{i}\equiv s_{i}^{*}\Omega\,,
\end{align}
and by using Cartan's structure equation the local curvature may be written in terms of the local connection as
\begin{align}
    F_{i}&=s_{i}^{*}(\textbf{d}w+w\wedge w)\nonumber\,,
    \\
    &=\textbf{d}(s_{i}^{*}w)+s_{i}^{*}w\wedge s_{i}^{*}w\nonumber\,,
    \\
    &=\textbf{d}A_{i}+A_{i}\wedge A_{i}\,.
\end{align}
where it follows from the transformation of $A_i$ that the transformation of $F_i$ is given by
\begin{align}
    F_{j}=t_{ij}^{-1}F_{i}\,t_{ij}\,.
\end{align}
Let us then choose local co-ordinates $x^{\mu}$ on a patch $U\subset M$, and as we always consider local objects we neglect the subscript $i$. The connection one-form and curvature two-form can then be expanded as
\begin{align}
    A&=A_{\mu}dx^{\mu}\,,
    \\
    F&=\frac{1}{2}F_{\mu\nu}dx^{\mu}\wedge dx^{\nu}\,,
\end{align}
and the de-Rham differential takes the form
\begin{align}
    \textbf{d}A=(\partial_{\mu}A_{\nu}-\partial_{\nu}A_{\mu})dx^{\mu}\wedge dx^{\nu}\,,
\end{align}
giving the field strength in terms of components
\begin{align}\label{Field Strength}
    F_{\mu\nu}=\partial_{\mu}A_{\nu}-\partial_{\nu}A_{\mu}+[A_{\mu},A_{\nu}]\,,
\end{align}
where the bracket is the usual Lie commutator. Since both the gauge field and the field strength are Lie algebra valued, they can be expanded in the the group generators $t^{a}$ as well. Using the commutator relation between the generators $[t^{a},t^{b}]=f^{abc}t^{c}$ the field strength takes the form
\begin{align}\label{eq:curvature two-form}
    F_{\mu\nu}^{a}=\partial_{\mu}A_{\nu}^{a}-\partial_{\nu}A_{\mu}^{a}+f^{abc}A_{\mu}^{b}A_{\nu}^{c}\,,
\end{align}
which is the well known field strength tensor for a Yang-Mills theory\footnote{In physics we also need the coupling $g$, but we will come to that later.}, and the transformation of the field strength in local co-ordinates is given by
\begin{align}\label{eq:curvature two form transformation}
    F'_{\mu\nu}=g^{-1}F_{\mu\nu}g\,,
\end{align}
which we also recognize from gauge theories in quantum field theory.

\subsection*{Horizontal Lift and Parallel Transport Equation}
In the previous section we found that the tangent space $TP$ of the principal G-bundle $P(M,G)$ can be split into a vertical and horizontal part. This splitting allows us to define a \emph{horizontal lift} of a curve in the base manifold $M$:

\medskip
\begin{mydef}{Horizontal lift}{}
Let $P(M,G,\pi)$ be a principal G-bundle and
\begin{align*}
    \gamma:[0,1]\rightarrow M\,,
\end{align*}
a curve in $M$. Then a curve 
\begin{align*}
    \tilde{\gamma}:[0,1]\rightarrow P\,,
\end{align*}
is said to be a horizontal lift of $\gamma$ from the the base space up in the fibre if the tangent vector $\tilde{\gamma}\in H_{\tilde{\gamma}(t)}P.$ 
\end{mydef}\noindent
From this is follows that for a curve $\gamma$ in $M$ and a point $p$ in the fibre, i.e. $p\in\pi^{-1}(\gamma(0))$, there exist a unique horizontal lift $\tilde{\gamma}(t)$ in $P$ such that $\tilde{\gamma}(0)=p$. Further, it follows that another horizontal lift $\tilde{\gamma'}$ of $\gamma$ can be found by applying the right group action, $\tilde{\gamma'}(0)=\tilde{\gamma}(0)g$. Then, we also have that
\begin{align}
    \tilde{\gamma'}(t)=\tilde{\gamma}(t)g\,,\hspace{1cm}\forall t\in [0,1]\,,
\end{align}
which is saying that the global right action does not change the connection on the principal fibre. 

Then let $X$ be the tangent vector at $\gamma(0)$, and by using the horizontal lift we have that 
\begin{align}
    \tilde{X}=\tilde{\gamma}_{*}X\,,
\end{align}
is tangent a tangent vector at $\tilde{\gamma}(0)=p$. By construction, this vector is horizontal and if we act with the connection $w$, we get
\begin{align}
    w(\tilde{X})=0\,,
\end{align}
which is saying that acting with the connection one-form on a horizontal lifted tangent vector yields zero.

Now, since the transition functions are elements of the group, we can use \cref{eq:important for parallel transport} to write
\begin{align}
    \tilde{X}=g_{i}^{-1}(t)\big(s_{i*}X\big)g_{i}(t)+\big(g_{i}^{-1}\mathbf{d}g_{i}\big)\,,
\end{align}
and if we apply the connection $w$ on this equation, we get
\begin{align}
    0=g_{i}^{-1}(t)w(s_{i*}X)g(t)+g_{i}^{-1}(t)\frac{dg_{i}(t)}{dt}\,,
\end{align}
and if we use that $w(s_{i*}X)=s_{i}^{*}w(X)=A_{i}(X)$\footnote{This follows from the definition of the pullback in \cref{eq:pullback}.}, we find the local \emph{parallel transport equation} for a gauge theory
\begin{align}
    \frac{dg_{i}(t)}{dt}=-A_{i}(X)g_{i}(t)\,.
\end{align}
This is a matrix equation and can be solved by the use of \emph{Chen iterated integrals} \cite{chen1977}. Given the initial condition $g_{i}(0)=e$\footnote{Here $e$ is the unit element in the gauge group $G$.}, a local solution takes the form of a functional of an arbitrary path $\gamma(t)$
\begin{align}\label{eq:parallel transporter}
    g_{i}[\gamma(t)]=\mathcal{P}\exp\Big(-\int_{\gamma(0)}^{\gamma(t)}dx^{\mu}A_{i\,\mu}(x)\Big)\,,
\end{align}
where in general $A_{i\,\mu}=A_{i\,\mu}^{a}t^{a}$\footnote{In physics we use $A_{i\,\mu}=igA_{i\,\mu}^{a}t^{a}$, where $g$ is the coupling, but more on that later.}, and $\mathcal{P}$ is a path-ordering operator which orders the gauge fields along the path in the manifold\footnote{In general the matrices does not commute, so the path ordering is needed.}. It works in a similar fashion as the time-ordering operator we have for the time-evolution operator in QM.

\subsection*{Parallel Transport and Holonomy}
The last important concept of this section is that of \emph{holonomy}. Given a principal fibre bundle $P(M,G,\pi)$, we let $\gamma_1$ and $\gamma_2$ be two curves in $M$, such that
\begin{align}
    \gamma_{1}(0)&=\gamma_{2}(0)=p_{0}\,,
    \\
    \gamma_{1}(1)&=\gamma_{2}(1)=p_{1}\,.
\end{align}
Let us consider the horizontal lift of these two curves
\begin{align}
    \tilde{\gamma}_{1}(0)&=\tilde{\gamma}_{2}(0)=q_{0}\,,
\end{align}
then it is not in general true that
\begin{align}
    \tilde{\gamma}_{1}(1)&=\tilde{\gamma}_{2}(1)=q_{1}\,.
\end{align}
Then if we consider a loop $\gamma$ in $M$
\begin{align}
    \gamma(0)=\gamma(1)\,,
\end{align}
we have that in general
\begin{align}
    \tilde{\gamma}(0)\neq\tilde{\gamma}(1)\,.
\end{align}
The loop then induces a map on the fibre at $p$
\begin{align}
    \tau_{\gamma}:\pi^{-1}(p)\rightarrow \pi^{-1}(p)\,.
\end{align}
We can then consider loops with fixed base-point in the manifold $M$, denoted $L_{p}M$. Now, $\tau_{\gamma}$ can only reach certain elements of the group $G$, but combining them with a gauge transformation we can reach all elements of $G$. Hence, the set of all elements that can be reached starting from the point $(p,q)$ in the principal fibre bundle form a subgroup of $G$, called the \emph{holonomy group} at $q$, where the projection of $q$ is the point $p$, i.e. $\pi(q)=p$. We write this as
\begin{align}
    \Phi_{q}=\big\{g\in G|\tau_{\gamma}(q)=qg,\,\gamma\in L_{p}M\big\}\,.
\end{align}

Holonomy elements are generated by considering parallel transport around a closed loop and it follows from \cref{eq:parallel transporter} that it can be written as
\begin{align}
    g_{\gamma}=\mathcal{P}\exp\big(-\oint_{\gamma}dz^{\mu}A_{\mu}(z)\big)\,.
\end{align}
Parallel transport around closed loops are special objects with many interesting features, and some of them will be discussed in more detail in \cref{sec:Wilson lines and Wilson loops}. The reason why these are important in physics follows from the following theorem:

\medskip
\begin{mytheo}{Ambrose-Singer}{}
Let $P(M,G)$ be a principal fibre bundle with connection $w$, and curvature $\Omega$. Let $\Phi_{q}$ be the holonomy group with reference point $q\in P(M,G)$, and $P(q)$ the holonomy bundle of $w$ through $q$. Then the Lie algebra of $\Phi(q)$ is equal to the Lie sub-algebra $\mathfrak{g}$, generated by all elements of the form $\Omega_{p}(v_1,v_2)$ for $p\in P(q)$ and $v_1,v_2$ horizontal vectors at $p$, where $\mathfrak{g}$ is the Lie algebra of $G$. 
\end{mytheo}\noindent
This is quite technical, but instead of tackling all the mathematical language we will instead explain what it means for physics. The theorem states that all the information contained in the curvature $\Omega$ at a point in the principal fibre bundle $P$ with connection $w$, can also be found in the holonomy group $\Phi(q)$ at that point. The implication is that---at least in principal---all physical observables can be expressed as functions of the holonomies, instead as functions of the gauge potentials. The holonomy group is not gauge invariant, so real observables need to be expressed in terms of gauge invariant functions of these holonomies. Such an object is what we in physics call a \emph{Wilson loop}. Hence, we have that the Wilson loop contains all the information about the curvature two-form we need to describe the dynamics of gauge fields.
\section{Yang-Mills Theories from Wilson Lines}\label{sec:Wilson lines and Wilson loops}
In this section we will construct the Yang-Mills Lagrangian from a purely geometric setting by using the concept of connection, covariant derivative and parallel transportation. Our basic starting point is the Dirac Lagrangian, which we require to be invariant under a local gauge transformation. To do this, we will use what we have learned from the previous sections of connections on principal fibre bundles, and in particular use the solution of the parallel transport equation to generate gauge invariant objects. The requirement of local gauge invariance introduces interactions between matter fields and gauge fields, and if these gauge fields are to describe physical fields we need to find a gauge invariant term for them in the Lagrangian. This is where the Ambrose-Singer theorem can be applied by using the gauge invariant elements of the holonomy group, i.e. the Wilson loop.

\subsection{Covariant Derivative and Wilson Lines}\label{sec:Wilson lines}
We have seen how the matter fields transform under local gauge transformation and how gauge fields naturally appear from connection one-forms. However, the main objective regarding gauge invariance in physics is that the Lagrangian (or action) remain invariant under local gauge transformations, i.e we need to define connections such that derivative terms are gauge invariant. To construct the covariant derivative for gauge theories we will make use of the important object called a \emph{Wilson line}. This object is central in almost all modern theories of physics, as it can be used as a building block for gauge invariance. Before we define the Wilson line we will introduce its use from a physical perspective.

Let us start with the Dirac Lagrangian
\begin{align}
    \mathcal{L}_{D}=\Bar{\psi}(i\slashed{\partial}-m)\psi\,,
\end{align}
which because of the partial derivative cannot describe a gauge invariant theory. The problem with the partial derivative when working with manifolds was discussed from \cref{eq:directional derivative vector field}, where we looked at the directional derivative of a vector field. The directional partial derivative of the fields $\psi$ can in the naive way be expressed in a similar fashion
\begin{align}
    n^{\mu}\partial_{\mu}\psi(x)\equiv\lim_{\epsilon\to 0}\frac{\psi(x+\epsilon n)-\psi(x)}{\epsilon}\,,
\end{align}
where the fields $\psi(x+\epsilon n)$ and $\psi(x)$ have completely different transformation properties under local gauge transformations and can therefore not be subtracted in a meaningful way. With a curved background this was solved by parallel transporting the fields using the Levi-Civita connection \cref{eq:covariant derivative in terms of components}. We are now in a position to do the same thing for a gauge theory, where the gauge fields themselves act as connections\footnote{This was the main motivation for introducing connections on principal fibre bundles.}. The parallel transporter for a gauge theory is given by \cref{eq:parallel transporter}, which we use as the definition of a Wilson line:

\medskip
\begin{mydef}{Wilson line}{}
Given a path $\gamma$ in the spacetime manifold $M^{4}$, a Wilson line is defined as the solution to the parallel transport equation:
\begin{align}\label{eq:Wilson line definition}
    \mathcal{U}_{\gamma}[y,x]=\mathcal{P}\exp\Big(ig\int_{x}^{y}dz^{\mu}A_{\mu}^{a}(z)t^{a}\Big)\,,
\end{align}
where $g$ is the coupling, $\gamma$ is the path along which one parallel transports between spacetime points $x$ and $y$ and $\mathcal{P}$ denotes the path-ordering operator.
\end{mydef}\noindent
The matter fields transform through the unitary representation of $G$\footnote{We are specifically talking about the gauge (Lie) group $SU(N)$, so we have a local $\mathfrak{su}(n)$ phase rotation.}
\begin{align}
    \psi(x)\rightarrow U(x)\psi(x)\,,
\end{align}
where
\begin{align}
    U(x)=\exp\big(ig\alpha^{a}(x)t^{a}\big)\hspace{1mm}\,.
\end{align}
The Wilson line transforms under gauge transformation as\footnote{This is not trivial to show as one has to discretize the integral in the exponent.}
\begin{align}\label{eq:Wilson line transformation}
    \mathcal{U}_{\gamma}[y,x]\rightarrow U(y)\mathcal{U}_{\gamma}[y,x]U^{\dagger}(x)\,,
\end{align}
such that we can parallel transport $\psi(x)$ in the following way
\begin{align}
    \psi(y)=\mathcal{U}_{\gamma}[y,x]\psi(x)\,,
\end{align}
which transform as
\begin{align}
    \psi(y)&\rightarrow U(y)\mathcal{U}_{\gamma}[y,x]U^{\dagger}(x)U(x)\psi(x)
    \\
    &=U(y)\mathcal{U}_{\gamma}[y,x]\psi(x)
    \\
    &=U(y)\psi(y)\,.
\end{align}
This is the result we need to compare the matter fields at different points. Hence, we define the directional covariant derivative\footnote{We usually drop the subscript $\gamma$ in applications. In this case $\gamma$ is a infinitesimal straight line along $n^{\mu}$.}:

\medskip
\begin{mydef}{Directional Covariant Derivative}{}
A directional covariant derivative is defined through the Wilson line in the following way
\begin{align}
    n^{\mu}D_{\mu}\psi(x)=\lim_{\epsilon\to 0}\frac{\psi(x+\epsilon n)-\mathcal{U}[x+\epsilon n,x]\psi(x)}{\epsilon}\,.\nonumber
\end{align}
\end{mydef}\noindent
We want to expand the Wilson line, so we can use the \emph{gradient theorem}:
\medskip
\begin{mytheo}{Gradient Theorem}{}
Given an analytic function $f$ on the continous path $\gamma\in[x,y]$, we have
\begin{align}
    \int_{x}^{y}dz^{\mu}\partial_{\mu}f(z)=f(y)-f(x)\,.
\end{align}
\end{mytheo}\noindent
Expanding the Wilson line using the gradient theorem and expanding the field $\psi(x+\epsilon n)$, we get 
\begin{align}
    \mathcal{U}[x+\epsilon n,x]&=1+ig\epsilon n^{\mu}A_{\mu}(x)+\mathcal{O}(\epsilon^{2})
    \\
    \psi(x+\epsilon n)&=\psi(x)+\epsilon n^{\mu}\partial_{\mu}\psi(x)+\mathcal{O}(\epsilon^{2})\,,
\end{align}
inserting these expansions into the definition of the covariant derivaticve, we find that 
\begin{align}
    D_{\mu}\psi(x)=\partial_{\mu}\psi(x)-igA_{\mu}(x)\psi(x)\,,
\end{align}
which is the well known covariant derivative acting on fields in the fundamental representation of the gauge group.

\medskip
The transformation property of the gauge fields follow directly from the Wilson line transformation. To write this out we use the well known identity of exponentiated matrices
\begin{align}
    e^{YXY^{-1}}=YXY^{-1}\,,
\end{align}
which applied to \cref{eq:Wilson line transformation}, will give
\begin{align}
     U(y)\mathcal{U}_{\gamma}[y,x]U^{\dagger}(x)=\mathcal{P}\exp\Big[U(y)\big(ig\int_{x}^{y}dz^{\mu}A_{\mu}^{a}(z)t^{a}\Big)U^{\dagger}(x)\Big]\,,
\end{align}
which after partial integration and application of the gradient theorem takes the form
\begin{align}
    \mathcal{P}\exp\Big[ig\int_{x}^{y}dz^{\mu}U(z)\Big(A_{\mu}^{a}(z)t^{a}+\frac{i}{g}\partial_{\mu}\Big)U^{\dagger}(z)\Big]\,,
\end{align}
meaning that the gauge fields transform as
\begin{align}
    A_{\mu}^{a}(x)t^{a}\rightarrow U(x)A_{\mu}^{a}t^{a}U^{\dagger}(x)+\frac{i}{g}U(x)\partial_{\mu}U^{\dagger}(x)\,,
\end{align}
which is the same we found in \cref{eq:gauge field connection transformation}, with the difference  of the factor $i/g$ we use in physics.

With these results we have that the gauge invariant Dirac Lagrangian can be written as
\begin{align}\label{eq:gauge invariant dirac lagrangian}
    \mathcal{L}_{D}=\Bar{\psi}(i\slashed{D}-m)\psi\,.
\end{align}
The missing piece of the Yang-Mills Lagrangian is the kinetic term\footnote{No mass terms as it would not be gauge invariant.} for the gauge fields. To find the field strength, we can do the same as we did for the Riemann curvature and consider the commutator of the covariant derivatives, see \cref{eq:commutator of covariant derivative}. We find\footnote{The commutator is not a differential operator, but merely a matrix that is understood to act on $\psi$.}
\begin{align}
    [D_{\mu},D_{\nu}]=-igF_{\mu\nu}^{a}t^{a}\,,
\end{align}
where 
\begin{align}\label{eq:field strenght in wilson chapter}
    F_{\mu\nu}^{a}=\partial_{\mu}A_{\nu}^{a}-\partial_{\nu}A_{\mu}^{a}+gf^{abc}A_{\mu}^{b}A_{\nu}^{c}\,,
\end{align}
which apart from the coupling $g$ is identical to the curvature two-form we defined in \cref{eq:curvature two-form}. From the transformation property in \cref{eq:curvature two form transformation}, the field strength is not gauge invariant. Meaning we have to use a certain combination of them that are gauge invariant to use it as a term in the Lagrangian. We could follow in a heuristic manner and postulate what this combination is, but we can show that the term appear in a natural way from geometric arguments.

We will encounter the use for Wilson lines on numerous occasions throughout this thesis, for example when we want to ensure gauge invariance of bi-local operators or when we want to describe soft gauge boson radiation in scattering amplitudes. 

\subsection{Field Strength Tensor and Wilson Loops}
In this section we will take a closer look at the definition of a Wilson line on a closed path and from it construct the kinetic term in the Lagrangian for gauge bosons.

If the solution of the parallel transport equation is defined on a path $\gamma$ that is a closed loop, we have the Wilson line operator
\begin{align}
    \Gamma_{\gamma}=\mathcal{P}\exp\Big(ig\oint_{\gamma} dz^{\mu}A_{\mu}^{a}(z)t^{a}\Big)\,.
\end{align}
If one expands this exponential, this becomes an infinite series that converges absolutely to an element $g\in G$, see \cite{TAVARES_1994}. It is important to note that this object is not gauge invariant. However, a Wilson line operator acts as an operator on Hilbert space, and because of the convergent behaviour, we can just as well consider its trace\footnote{The trace over an operator converges absolutely and is independent of the basis $\{\psi_n\}$ in Hilbert space.}:

\medskip
\begin{mydef}{Wilson loop}{}
Given a Wilson line $\Gamma_{\gamma}$ defined along a closed loop $\gamma$, a Wilson loop is then defined as its trace
\begin{align}
    W[\gamma]=\text{tr}\,\mathcal{P}\exp\Big(ig\oint_{\gamma} dz^{\mu}A_{\mu}^{a}(z)t^{a}\Big)\,.\nonumber
\end{align}
\end{mydef}\noindent
Because of the trace this is gauge invariant, which can be shown by calculating
\begin{align}
    \text{tr}[\Gamma_{\gamma}]&\rightarrow \text{tr}[U(x)\Gamma_{\gamma}U^{-1}(x)]\nonumber
    \\
    &=\text{tr}[\Gamma_{\gamma}U^{-1}(x)U(x)]\nonumber
    \\
    &=\text{tr}[\Gamma_{\gamma}]\,,
\end{align}
where $U(x)$ is the usual gauge transformation.

To show that the Wilson loop contains all the dynamical information we have to expand the Wilson loop. We can first use Stokes' theorem to write the loop integral into a surface integral
\begin{align}
    \oint_{\gamma}dz\cdot A=\int_{\Sigma}d\sigma\cdot\partial\wedge A\,.
\end{align}
We can always parametrize a path in spacetime in terms of a parameter $\lambda$ in the following way
\begin{align}\label{eq:path parametrization}
    \gamma\,:\,\,&z^{\mu}(\lambda)\,,
    \\
    dz&^{\mu}=d\lambda\frac{\partial z^{\mu}}{\partial\lambda}\,.
\end{align}
Hence, we can always parametrize a surface in spacetime in terms of two parameteres $\lambda,\lambda'$
\begin{align}
    \Sigma\,:\,\,&z^{\mu}(\lambda,\lambda')\,,
    \\
    \nonumber
    \\
    d\sigma^{\mu\nu}&=dz^{\mu}\wedge dz^{\nu}\nonumber
    \\
    &=d\lambda d\lambda'\Big(\frac{\partial z^{\mu}}{\partial\lambda}\frac{\partial z^{\nu}}{\partial\lambda'}-\frac{\partial z^{\nu}}{\partial\lambda}\frac{\partial z^{\mu}}{\partial\lambda'}\Big)\,,
\end{align}
and we can write the surface integral as
\begin{align}
    \int_{\Sigma}d\sigma\cdot\partial\wedge A=\int_{\Sigma}d\lambda d\lambda'\frac{\partial z^{\mu}}{\partial\lambda}\frac{\partial z^{\nu}}{\partial\lambda'}\big(\partial_{\mu}A_{\nu}-\partial_{\nu}A_{\mu}\big)\,,
\end{align}
where we used that $\partial \wedge A=(\partial_{\mu}A_{\nu}-\partial_{\nu}A_{\mu})/2$. We want to consider an infinitesimal loop, and to do this we discretize spacetime and define the theory on a lattice with grid spacing $\epsilon$\footnote{A square with corners $x$, $x+\epsilon n$, $x+\epsilon n+\epsilon n'$ and $x+\epsilon n'$.}. However, Minkowski space can not be discretized in a well defined manner so we have to transform to Euclidean space. This is done by performing a Wick rotation. We can then write the Euclidean Wilson loop
\begin{align}
    W_{E}=\text{tr}\,\mathcal{P}\exp\Big(ig\int_{\Sigma}d\lambda d\lambda'\frac{\partial z_{E}^{\mu}}{\partial \lambda}\frac{\partial z_{E}^{\nu}}{\partial\lambda'}\big(\partial_{\mu}A_{E\,\nu}^{a}-\partial_{\nu}A_{E\,\mu}^{a}\big)t^{a}\Big)\,.
\end{align}
The expansion is tedious and takes quite some work to bring on a nice form, so we will just state the relevant terms. The first order term vanishes identically since $\text{tr}[t^{a}]=0$. Ignoring constants the expansion to $\mathcal{O}(g^{2})$ is given by
\begin{align}
    -g^{2}\frac{\epsilon^{4}}{4}\big(\partial_{\mu}A_{E\,\nu}^{a}-\partial_{\nu}A_{E\,\mu}^{a}\big)^{2}\,,
\end{align}
and the contributions from the third and fourth order expansion take the form
\begin{align}
    -g^{3}\epsilon^{4}f^{abc}A_{E\,\mu}^{a}A_{E\,\mu}^{b}\partial^{\mu}A_{E}^{\nu\,c}-g^{4}\frac{\epsilon^{4}}{4}f^{abe}f^{ecd}(A_{E\,\mu}^{a}A_{E\,\nu}^{b})(A_{E}^{\mu\,c}A_{E}^{\nu\,d})\,.
\end{align}
Collecting terms will give
\begin{align}\label{eq:expanded wilson loop collected}
    W_{E}\approx&-g^{2}\frac{\epsilon^{4}}{4}\big(\partial_{\mu}A_{E\,\nu}^{a}-\partial_{\nu}A_{E\,\mu}^{a}\big)^{2}-g^{3}\epsilon^{4}f^{abc}A_{E\,\mu}^{a}A_{E\,\mu}^{b}\partial^{\mu}A_{E}^{\nu\,a}\nonumber
    \\
    &\hspace{1cm}-g^{4}\frac{\epsilon^{4}}{4}f^{abe}f^{ecd}(A_{E\,\mu}^{a}A_{E\,\nu}^{b})(A_{E}^{\mu\,c}A_{E}^{\nu\,d})+\mathcal{O}(\epsilon^{5})\,,
\end{align}
valid up to a constant term that is unimportant for the present discussion. If we compare this expression with \cref{eq:field strenght in wilson chapter}, this looks very much the square of the field strength. The difference lies in the powers of $g$ and of course the small increment $\epsilon$. However, eventually we would like to take the continuum limit by sending $\epsilon\rightarrow 0$. This is a highly non-trivial task, and to reproduce the correct continuum limit one has to take into account for rescaling and renormalization of all quantities in the theory. Also, the continuum limit corresponds to summing over all lattice points, therefore we have to divide by the lattice spacing to the fourth $\epsilon^{4}$ before letting $\epsilon\rightarrow 0$\footnote{For a rigorous treatment of the continuum limit we refer to \cite{CASELLE_2000}.}.

But first we use \cref{eq:field strenght in wilson chapter} to write \cref{eq:expanded wilson loop collected} as   
\begin{align}
    W_{E}\approx-g^{2}\epsilon^{4}&\frac{1}{4}F_{E\,\mu\nu}^{a}F_{E}^{\mu\nu\,a}+\mathcal{O}(\epsilon^{5})\,,
\end{align}
which after rescaling of the coupling and a Wick rotation back to Minkowski space will in the continuum limit take the form \footnote{An example for $SU(2)$ can be found in \cite{Peskin:257493}.}
\begin{align}\label{eq:gauge field kinetic term}
    W[A]=-\frac{1}{4}F_{\mu\nu}^{a}F^{\mu\nu\,a}\,,
\end{align}
which is the well known gauge invariant kinetic term for the gauge fields. The gauge invariance naturally follows as the original Wilson loop is gauge invariant.  

To construct a Yang-Mills theory of gauge fields interacting with fermions, we simply add \cref{eq:gauge field kinetic term} to the gauge invariant Dirac Lagrangian. The result takes the form
\begin{align}\label{eq:Yang-Mills Lagrangian}
    \mathcal{L}_{YM}=-\frac{1}{4}F_{\mu\nu}^{a}F^{\mu\nu\,a}+\Bar{\psi}(i\slashed{D}-m)\psi
\end{align}
which is known as the Yang-Mills Lagrangian and is what we call a \emph{non-Abelian} gauge theory. 

From the above expansion of the Wilson loop we observe that there are quartic and cubic terms in $A_{\mu}^{a}$, meaning that this is a nontrivial, interacting field theory\footnote{As opposed to Abelian gauge theories where there are no such interactions between the gauge bosons.}. It is important to note that this discussion is at the classical level, so we have to quantize the Yang-Mills theory. 

\subsection{Quantization of Yang-Mills Theories}
In this section we will quantize the Yang-Mills theory. We will use several of the concepts we used in the quantization of Abelian gauge theories, so for more detail see \cref{sec:quantization of abelian}.

When we quantized the Abelian gauge field in \cref{sec:quantization of abelian}, we saw that because of gauge invariance we are integrating over an infinite number of identical field configurations leading to a divergent path integral. To fix the problem we introduced a gauge fixing condition that ensured integration over those configurations that were different. Due to the non-commutivity between non-Abelian gauge fields this is more complicated. 

To begin with, we use that an infinitesimal transformation of non-Abelian gauge fields can be written as
\begin{align}
    (A^{\alpha})_{\mu}^{a}=A_{\mu}^{a}+\frac{1}{g}D_{\mu}\alpha^{a}\,,
\end{align}
where the covariant derivative acting on fields in the adjoint representation is given by
\begin{align}
    D_{\mu}^{ab}=\delta^{ab}\partial_{\mu}-gf^{abc}A_{\mu}^{c}\,.
\end{align}
We can do as in the Abelian case and insert the identity \cref{eq:functional identity} and write the path integral as\footnote{Due to notational simplicity we will not always write out the group indicies, but for non-Abelian field it is always understood to be there.}
\begin{align}
    \int\mathcal{D}A e^{iS[A]}=\Big(\int\mathcal{D}\alpha\Big)\int\mathcal{D}A\,e^{iS[A]}\,\delta(G[A])\,\det(\frac{\partial G[A]}{\partial\alpha})\,.
\end{align}
In contrast to the Abelian case we can not move the functional determinant out of the path integral and combine it with the infinite constant. To see this we calculate the argument inside the determinant and find
\begin{align}
    \frac{\partial G[A]}{\partial\alpha}=\frac{1}{g}\partial^{\mu}D_{\mu}\,,
\end{align}
which is not independent of $A$. To circumvent this problem we use that functional determinants can be written as a path integral, a representation introduced by Fadeev and Popov \cite{Faddeev:1967fc}. The trick of it is that we need to write the determinant as a path integral of anticommuting Grassmann fields that transform in the adjoint representation, i.e. 
\begin{align}
    \det(\partial^{\mu}D_{\mu})=\int\mathcal{D}c\mathcal{D}\Bar{c}\exp(-i\int d^{4}x\,\Bar{c}(\partial^{\mu}D_{\mu})c)\,.
\end{align}
These fields can be shown to transform under Lorentz transformation as scalars, meaning that due to their Grassmannian nature do not obey the correct spin-statistics relation. Therefore, they can not be dynamical fields that we can observe. However, they can be shown to serve as negative degrees of freedom in the sense that given their Feynman rules they can cancel the unphysical timelike and longitudinal polarization states of the gauge bosons. With this representation of the determinant one can use \cref{eq:gaussian weigh in quantization} and integrate using the delta function, leading to the Fadeev-Popov Lagrangian
\begin{align}\label{eq:Fadeev-Popov Lagrangian}
    \mathcal{L}_{FP}=-\frac{1}{4}(F_{\mu\nu}^{a})^{2}+\Bar{\psi}(i\slashed{D}-m)\psi-\frac{1}{2\xi}(\partial^{\mu}A_{\mu}^{a})^{2}-\Bar{c}^{a}(\partial^{\mu}D_{\mu}^{ac})c^{c}\,.
\end{align}

As mentioned the practical use of ghosts is to cancel unphysical polarizations appearing in gauge boson diagrams. For example, when calculating a gauge boson self energy diagram, one also have to add a ghost diagram.

In some cases it can be cumbersome and difficult to calculate these extra ghost diagrams. There is another quantization procedure that circumvent the introduction of ghosts. That is, one can choose a different class of gauges, namely the axial gauges, i.e.
\begin{align}
    G[A]=n^{\mu}A_{\mu}-\omega\,,
\end{align}
for some arbitrary directional vector $n_{\mu}$. With this choice the functional determinant can be written as
\begin{align}
    \det(\pdv{G[A]}{\alpha})=\det (n^{\mu}D_{\mu})=\det (n^{\mu}\partial_{\mu}-gn^{\mu}A_{\mu})=\det (n^{\mu}\partial_{\mu}-g\omega)\,,
\end{align}
which is independent of $A$, and can be pulled out of the path integral. Not being able to pull the functional determinant out of the path integral was the problem that led to the introduction of ghosts. The general result is that any gauge choice involving derivatives, will give rise to ghosts. The downside with axial gauges is that the propagator will become more complicated. 

As for the Abelian gauge fixing we can now make an integration over $\omega$ using a Gaussian weight, see \cref{eq:gaussian weigh in quantization}. The path integral can then be written as
\begin{align}
    \int\mathcal{D}A\,e^{iS[A]}&=N(\xi)\det(n^{\mu}\partial_{\mu})\Big(\int\mathcal{D}\alpha\Big)\int\mathcal{D}\omega\int\mathcal{D}A\,e^{iS[A]}\,\delta(n^{\mu}A_{\mu}-\omega)e^{-i\int d^{4}x\frac{\omega^{2}}{2\xi}}\nonumber
    \\
    &=N(\xi)N(\alpha)\int\mathcal{D}A\,\exp(iS[A]-i\int d^{4}x\frac{1}{2\xi}(n^{\mu}A_{\mu})^{2})
\end{align}
This is completely analogous to \cref{eq:Abelian gauge fixed action}, with the difference of $n^{\mu}$ instead of $\partial^{\mu}$ in the gauge fixing term. The gauge-fixed action is then given by
\begin{align}
    S_{GF}[A]=\frac{1}{2}\int d^{4}x\,d^{4}y\,A_{\mu}(x)\delta^{(4)}(x-y)\big(\partial^{2}g^{\mu\nu}-\partial^{\mu}\partial^{\nu}+\frac{1}{\xi}n^{\mu}n^{\nu}\big)A_{\nu}(y)\,,
\end{align}
giving the propagator equation
\begin{align}
    \big(\partial^{2}g_{\mu\nu}-\partial_{\mu}\partial_{\nu}-\frac{1}{\xi}n_{\mu}n_{\nu}\big)D_{F}^{\nu\rho}(x,y)=i\delta_{\,\mu}^{\rho}\delta^{(4)}(x-y)\,.
\end{align}
As in \cref{eq:parametrization of propagator} this can be solved exactly by exploiting that the gauge boson propagator is a second rank symmetric tensor. Using this property we have that the momentum space propagator in axial gauge is given by
\begin{align}
    \text{Axial Gauge}:\hspace{1cm}D_{\mu\nu}^{ab}(k)=\frac{-i\delta^{ab}}{k^{2}+i\epsilon}\Big(g_{\mu\nu}-\frac{k_{\mu}n_{\nu}+k_{\nu}n_{\mu}}{k\cdot n}+(n^{2}+\xi k^{2})\frac{k_{\mu}k_{\nu}}{(k\cdot n)^{2}}\Big)\,.
\end{align}

This expression looks like a huge price to pay for avoiding introducing ghosts, but we can use an even smaller set of axial gauges. The one we will mostly use is the so-called \emph{light-cone gauges}, which amounts to choosing $n^{\mu}$ as a lightlike vector, i.e. $n^{2}=0$. Also, like for Lorentz gauges we have an additional gauge choice for $\xi$. We can choose $\xi=0$, which is called the \emph{homogenous light-cone gauge}, such that the term proportional to $k
^{2}$ vanishes entirely. We are then left with the light-cone propagator
\begin{align}\label{eq:lightcone propagator}
    \text{LC Gauge}:\hspace{1cm}D_{\mu\nu}^{ab}(k)=\frac{-i\delta^{ab}}{k^{2}+i\epsilon}\Big(g_{\mu\nu}-\frac{k_{\mu}n_{\nu}+k_{\nu}n_{\mu}}{k\cdot n}\Big)\,.
\end{align}

For completeness, the gauge boson propagator in covariant gauge is given by
\begin{align}\label{eq:covariant propagator}
    \text{Lorentz Gauge}:\hspace{1cm}D_{\mu\nu}^{ab}(k)=\frac{-i\delta^{ab}}{k^{2}+i\epsilon}\Big(g_{\mu\nu}-(1-\xi)\frac{k_{\mu}k_{\nu}}{k^{2}}\Big)\,.
\end{align}

Most of the time, calculations are easier in Lorentz gauges, but for some QCD calculations it is easier to use the light-cone gauge.


\section{Wilson Line Properties}\label{sec:wilson line properties}
In \cref{sec:Wilson lines and Wilson loops} we found that Wilson lines emerged naturally in gauge theories from a geometrical viewpoint. Because of its bi-local transformation property, it is used as a parallel transporter to render non-local terms gauge invariant. It could also be used to derive a gauge invariant Lagrangian for a non-Abelian gauge theory.  However, the application of Wilson lines are much broader than this and in this section we will look at some their properties in more detail.

The Wilson line we defined in \cref{sec:Wilson lines and Wilson loops} is valid for any gauge theory, but as our main focus is on QCD we mostly use that the gauge fields are non-Abelian in nature. The physical consequence of involving Wilson lines in a theory becomes apparent if we expand \cref{eq:Wilson line definition},
\begin{align}\label{eq:Expansion n-order Wilson line}
\mathcal{U}_{\gamma}&=\mathcal{P}\exp(ig\int_{\gamma}dz^{\mu}A_{\mu}(z))\nonumber
\\
&=\sum_{n=0}^{\infty}\frac{1}{n!}(ig)^{n}\mathcal{P}\int_{\gamma}dz_{n}^{\mu_n}\dots dz_{1}^{\mu_1}A_{\mu_n}(z_n)\dots A_{\mu_1}(z_1)\,,
\end{align}
i.e. the $n$-th order expansion represents radiation of $n$ gauge bosons. The point $z_i$ at which the gauge field is radiated is integrated over, meaning that all possible configurations are taken into account. The path-ordering makes sure that the radiated field are ordered such that $A_{\mu_i}(z_i)$ is radiated between $A_{\mu_{i-1}}(z_{i-1})$ and $A_{\mu_{i+1}}(z_{i+1})$. Thus, the full exponential is a resummation of gauge boson radiation from the path. One important consequence of this is that we can \textquote{dress} a particle, say a quark, with a Wilson line and this would then correspond to resummation of gluon radiation from a quark line. The same can be done for an electron with an Abelian Wilson line, with the simplification that the gauge fields commute and the path-ordering is redundant. 

Before we discuss how we can dress a fermion with a Wilson line, we summarize some important properties of Wilson lines.

\medskip
\begin{defbox*}{}{}
A Wilson line $\mathcal{U}_{\gamma}$ defined on a path $\gamma$ with endpoints $x$ and $y$ have the following properties:

\medskip
\textbf{Hermiticity:}
The hermitian conjugate of a Wilson line is given by
\begin{align}
    \mathcal{U}_{\gamma}^{\dagger}[y,x]=\mathcal{U}_{-\gamma}[x,y]\,,
\end{align}
i.e. it gives the same line in the opposite direction.

\medskip
\textbf{Causality:}
Because of path-ordering the Wilson line is path-transitive, i.e. we can continously glue several paths together to form one. For a path $\gamma=\gamma_1+\gamma_2$ going from $x$ to $z$, then from $z$ to $y$, is the same as going directly from $x$ to $y$,
\begin{align}
    \mathcal{U}_{\gamma}[y,x]=\mathcal{U}_{\gamma_1}[z,x]\,\mathcal{U}_{\gamma_2}[y,z]\,.
\end{align}

\medskip
\textbf{Unitarity:}
If we have a Wilson line from $x$ to $y$ and back from $y$ to $x$ in the opposite direction, this composition is equal to one, i.e.
\begin{align}
    \mathcal{U}_{\gamma}[y,x]\mathcal{U}_{\gamma}^{\dagger}[y,x]=1\,.
\end{align}

\medskip
\textbf{Bi-Locality:}
A Wilson line transform in function of it's endpoints only
\begin{align}
    \mathcal{U}_{\gamma}[y,x]\rightarrow e^{ig\alpha^{a}(y)t^{a}}\mathcal{U}_{\gamma}[x,y]e^{-ig\alpha^{a}(x)t^{a}}\,.
\end{align}
\end{defbox*}
The ultimate goal going forward is to use Wilson lines in perturbation theory, so we have to evaluate the line integrals in \cref{eq:Expansion n-order Wilson line}. To that end, it is convenient to parametrize the path $\gamma$ as we did in \cref{eq:path parametrization}, i.e.
\begin{align}
    \gamma:\,z^{\mu}(\lambda)\,,\hspace{1cm} \lambda=a\dots b\,,
\end{align}
where $z(a)$ and $z(b)$ are the endpoints of the path. Then we can make the change of variable
\begin{align}
    dz^{\mu}\rightarrow d\lambda\frac{dz^{\mu}}{d\lambda}\,,
\end{align}
Such that the expansion in \cref{eq:Expansion n-order Wilson line} can be written as
\begin{align}\label{eq:path ordering of expanded line}
    \mathcal{P}&\int_{\gamma}d\lambda_{n}\dots d\lambda_1\dv{z^{\mu_n}(\lambda_n)}{\lambda_n}\dots\dv{z^{\mu_1}(\lambda_1)}{\lambda_1}A_{\mu_n}(z_n)\dots A_{\mu_1}(z_1)\nonumber
    \\
    &=n!\int_{a}^{b}d\lambda_n\int_{a}^{\lambda_n}d\lambda_{n-1}\dots\int_{a}^{\lambda_2}d\lambda_1\dv{z^{\mu_n}(\lambda_n)}{\lambda_n}\dots\dv{z^{\mu_1}(\lambda_1)}{\lambda_1}A_{\mu_n}(z_n)\dots A_{\mu_1}(z_1)\,,
\end{align}
which tells us that the $i$-th gauge boson with parameter $\lambda_i$ is radiated between the $i-1$-th and $i+1$-th. Thus, the field with highest value of $\lambda$ is written leftmost in the integral, meaning that this is the field that will be written rightmost in a Feynman diagram. This implies that we read a Wilson line from right to left, like we do for Dirac lines. 

We want to use these Wilson lines in scattering amplitudes, so there is a specific Wilson line that is relevant for us, and that is Wilson lines on linear paths. In Feynman diagrams we draw particle lines as linear paths, so these are the most relevant for our purposes. We want to integrate over the path dependence, so it is convenient to disentagle the gauge field content from the path content of the line integral. This is most easily done by using the Fourier transform
\begin{align}
    A_{\mu}(z)=\int\frac{d^{d}k}{(2\pi)^{d}}\,A_{\mu}(k)\,e^{-ik\cdot z}\,,
\end{align}
such that the $n$-th order term in the expanded Wilson line \cref{eq:nth order Wilson integral} can be written as
\begin{align}
    \mathcal{U}_{\gamma}^{(n)}=\int\frac{d^{d}k_n}{(2\pi)^{d}}\dots\frac{d^{d}k_1}{(2\pi)^{d}}\,A_{\mu_n}(k_n)\dots A_{\mu_1}(k_1)\,\mathcal{I}^{(n)}\,,
\end{align}
where we have collected all the path content in the following integral
\begin{align}\label{eq:nth order Wilson integral}
    \mathcal{I}_{A}^{(n)}=(ig)^{n}\int_{a}^{b}d\lambda_{n}\dots\int_{a}^{\lambda_2}d\lambda_1 \dv{z^{\mu_n}(\lambda_n)}{\lambda_n}\dots\dv{z^{\mu_1}(\lambda_1)}{\lambda_1}\,\exp(-i\sum_{i=1}^{n}k_{i}\cdot z_{i})\,.
\end{align}
 
We want to consider paths that are bounded from below and bounded from above. The result in \cref{eq:nth order Wilson integral} is convenient to use when we have a line that is bounded from above\footnote{Hence, the subscript.}. For a path that is bounded from below, we change the integration chaining in \cref{eq:path ordering of expanded line} and flip the order of integration\footnote{This is done to make sure that the radiation happens in the correct order.}.
We summarize the two different cases in the following integrals
\begin{align}
    \mathcal{I}_{A}^{(n)}=(ig)^{n}\int_{a}^{b}d\lambda_{n}\dots\int_{a}^{\lambda_2}d\lambda_1 \dv{z^{\mu_n}(\lambda_n)}{\lambda_n}\dots\dv{z^{\mu_1}(\lambda_1)}{\lambda_1}\,\exp(-i\sum_{i=1}^{n}k_{i}\cdot z_{i})\,,\label{eq:path bounded from above}
    \\
    \mathcal{I}_{B}^{(n)}=(ig)^{n}\int_{a}^{b}d\lambda_{1}\dots\int_{\lambda_{n-1}}^{b}d\lambda_n \dv{z^{\mu_n}(\lambda_n)}{\lambda_n}\dots\dv{z^{\mu_1}(\lambda_1)}{\lambda_1}\,\exp(-i\sum_{i=1}^{n}k_{i}\cdot z_{i})\,.\label{eq:path bounded from below}
\end{align}

\subsubsection*{Semi-Infinite Wilson Lines}
The most interesting and physical relevant paths are those that contain so-called \emph{cusps}, i.e. there are points in the path that are not smooth. They are continous, but the derivative is not. Fortunately, since Wilson lines are path-transitive we can split the path at the cusp and continue with a product of two Wilson lines. 

To explain why we are interested in paths with cusps, we take the example of annihilation of two fermions. Highly energetic particles that radiates gauge bosons can be described by a Wilson line\footnote{We will show this later.}. The point of interaction is then what we call a cusp, e.g. when we draw Feynman diagrams we have two particles lines that meet and there is a crack at the interaction. Such cusps will lead to divergences and one has to renormalize the Wilson line. This will lead to the so-called \emph{cusp anamalous dimensions} containing all the interesting dynamics. We will come back these issues in \cref{chap:Resummation in QCD}.

The idea of Wilson lines coming in from infinity and meets at a point or created at a point and goes out to infinity, naturally leads to the notion of what we call semi-infinite Wilson lines. Let us start with a Wilson line that is bounded from above, which we parametrize as
\begin{align}
    z_{i}^{\mu}=a^{\mu}+n^{\mu}\lambda_{i}\,,\hspace{1cm}\lambda=-\infty\dots 0\,,
\end{align}
which is a linear path going from $-\infty$ up to a point $a^{\mu}$ along a direction $n^{\mu}$. If we insert this parametrization into \cref{eq:path bounded from above}, we get
\begin{align}
    \mathcal{I}_{A}^{(m)}=&(ig)^{m}\,n^{\mu_1}\cdots n^{\mu_m}\,\exp(-i\sum_{i=1}^{m}a\cdot k_i)\nonumber
    \\
    &\int_{-\infty}^{0}d\lambda_m\int_{-\infty}^{\lambda_{m}}d\lambda_{m-1}\cdots\int_{-\infty}^{\lambda_{2}}d\lambda_1 \,\exp(-i\sum_{i=1}^{m}(n\cdot k_{i}+i\epsilon)\lambda_{i})\,,
\end{align}
where we have used that the integrals has the form of a Fourier transformed Heaviside $\theta$-function, which is not convergent unless we use a $i\epsilon$ prescription. The solution of a Fourier transformed Heaviside function is well known, so the innermost integral is given by
\begin{align}
    \int_{-\infty}^{\lambda_{2}}d\lambda_{1}e^{-i(n\cdot k_1+i\epsilon)\lambda_1}=\frac{i}{n\cdot k_1+i\epsilon}e^{-i(n\cdot k_1+i\epsilon)\lambda_2}\,,
\end{align}
and to reveal the pattern we can also solve the next integral
\begin{align}
    \int_{-\infty}^{\lambda_{3}}d\lambda_{2}e^{-i(n\cdot k_1+n\cdot k_2+i\epsilon)\lambda_2}=\frac{i}{n\cdot k_1+n\cdot k_2+i\epsilon}e^{-i(n\cdot k_1+n\cdot k_2+i\epsilon)\lambda_3}\,.
\end{align}
By collecting all factors that are brought down by the integration, the $n$-th integral takes the form
\begin{align}
    \mathcal{I}^{(m)}=(ig)^{m}\,n^{\mu_1}\cdots n^{\mu_m}\,\exp(-i\sum_{i=1}^{m}a\cdot k_i)\prod_{i=1}^{m}\frac{i}{n\cdot\sum_{j=1}^{i} k_j+i\epsilon}\,,
\end{align}
giving
\begin{align}\label{eq:semi-infinite Wilson line -infty-to-0}
    \mathcal{U}[0,-\infty]=\sum_{m=0}^{\infty}(ig)^{m}\int\frac{d^{d}k}{(2\pi)^{d}}\,n\cdot A(k_m)\cdots n\cdot A(k_1)\,e^{-ia\cdot\sum_{i=1}^{m} k_i}\prod_{i=1}^{m}\frac{i}{n\cdot\sum_{j=1}^{i} k_j+i\epsilon}\,.
\end{align}
Just to clarify the notation used here and later: when we have a Wilson line defined from $x$ to $y$, we use the bracket $\mathcal{U}[y,x]$. But when we have a composition of Wilson lines meeting at a point $x$ coming from $-\infty$, we typically write this as $\mathcal{U}(x)=\mathcal{U}_{\gamma_1}[x,-\infty]\,\mathcal{U}_{\gamma_2}[x,-\infty]$.

The expansion in \cref{eq:semi-infinite Wilson line -infty-to-0} gives rise to the following Feynman rules:
\begin{fmffile}{tt}
\begin{align}
\begin{gathered}
\begin{fmfgraph*}(45,40)
\fmfleft{v1}
\fmfright{v2}
\fmf{plain,tension=.5,label=$k\rightarrow$,l.side=left}{v1,v2}
\end{fmfgraph*}
\end{gathered}\hspace{0.6cm}&=\frac{i}{n\cdot k+i\epsilon}\,,\hspace{2cm}\text{Wilson line propagator}\label{eq:Wilson propagator}
\\
\begin{gathered}
\begin{fmfgraph*}(45,40)
\fmfleft{v1}
\fmfright{v2}
\fmflabel{$a^{\mu}$}{v2}
\fmf{plain,tension=.5,label=$k\rightarrow$,l.side=left}{v1,v2}
\fmfdot{v2}
\end{fmfgraph*}
\end{gathered}\hspace{0.6cm}&=e^{-ia\cdot k}\,,\hspace{2.6cm}\text{External point}\label{eq:Wilson external point}
\\
\begin{gathered}
\begin{fmfgraph*}(45,40)
\fmfleft{i}
\fmfright{o}
\fmf{plain}{i,v3}
\fmf{plain}{v3,o}
\fmffreeze   % freezing the drawn elements
\fmfright{v3,o3}   % adding two more vertices
\fmfforce{(0.5w,0.5h)}{v3}   % setting position of the first vertex
\fmfforce{(0.5w,0h)}{o3}   % setting position of the second vertex
\fmfdot{v3}   % drawing the first vertex with a dot
\fmf{gluon, label=$k\uparrow$, l.side=left}{v3,o3}   % drawing a gluon line
\end{fmfgraph*}
\end{gathered}\hspace{0.6cm}&=ig\,n^{\mu}\,t^{a}\,.\hspace{2.36cm}\text{Wilson vertex}\label{eq:Wilson vertex}
\end{align}
\end{fmffile}

The next option to study is a line starting at a point $a^{\mu}$ and moving out to $+\infty$ along $n^{\mu}$. The parametrization reads
\begin{align}
    z_{i}^{\mu}=a^{\mu}+n^{\mu}\lambda_{i}\,,\hspace{1cm}\lambda=0\dots\infty\,.
\end{align}
If we insert this parametrization into \cref{eq:path bounded from below}, we get
\begin{align}
    \mathcal{I}_{B}^{(m)}=&(ig)^{m}\,n^{\mu_1}\cdots n^{\mu_m}\,\exp(-i\sum_{i=1}^{m}a\cdot k_i)\nonumber
    \\
    &\int_{0}^{\infty}d\lambda_1\int_{\lambda_1}^{\infty}d\lambda_{2}\cdots\int_{\lambda_{n-1}}^{\infty}d\lambda_n \,\exp(-i\sum_{i=1}^{m}(n\cdot k_{i}-i\epsilon)\lambda_{i})\,,
\end{align}
where the innermost integral is 
\begin{align}
    \int_{\lambda_{n-1}}^{\infty}d\lambda_{n}e^{-i(n\cdot k_1-i\epsilon)\lambda_n}=\frac{-i}{n\cdot k_1-i\epsilon}e^{-i(n\cdot k_1-i\epsilon)\lambda_{n-1}}\,,
\end{align}
giving the Wilson line
\begin{align}\label{eq:semi-infinite Wilson line 0-to-infty}
    \mathcal{U}[\infty,0]=\sum_{m=0}^{\infty}(ig)^{m}\int\frac{d^{d}k}{(2\pi)^{d}}\,n\cdot A(k_m)\cdots n\cdot A(k_1)\,e^{-ia\cdot\sum_{i=1}^{m} k_i}\prod_{i=1}^{m}\frac{-i}{n\cdot\sum_{j=1}^{i} k_j-i\epsilon}\,.
\end{align}
We observe that the Feynman rules defined above apply if we just make the change $k\rightarrow -k$ in the propagator.  There is much more to be said of different paths and structures, but we will only use Wilson lines on linear paths so we restrict ourselves to the ones discussed here.

\subsubsection*{Eikonal Particles}
We will now take a closer look at an amplitude and see the appearance of a Wilson line from it. A highly energetic fermion will always radiate soft gauge bosons. When the momentum carried by the gauge boson is much smaller than the momentum of the fermion this will lead to infrared divergences. As previously mentioned, we will investigate and treat IR divergences in more detail in \cref{Chap:pQCD}. But even after these have been treated we will have logarithmic contributions that can become large in certain regions of phase space. This problem can be solved by using Wilson lines, where we can treat diagrams perturbatively and re-exponentiate to an exact expression.

To see an example, let us investigate the process where a massless fermion radiates a gauge boson, see \cref{fig:blob with gluon radiation amplitude}\footnote{These are gluons, but we restrain from casually mention them before we have introduced the framework of QCD.}. The amplitude for this process is given by\footnote{Remember that we read the diagram against the particle flow. We also use the notation $\slashed{\varepsilon}=\gamma
^{\mu}\varepsilon_{\mu}^{a}$.}

\begin{fmffile}{ttt}
\begin{figure}
\centering
\begin{fmfgraph*}(180,100)
\fmfleft{i}
\fmfblob{.20w}{i}
\fmfright{o}
\fmf{fermion,label=$p+k\rightarrow$, l.side=left}{i,v3}
\fmf{fermion,label=$p\rightarrow$, l.side=left}{v3,o}
\fmffreeze   % freezing the drawn elements
\fmfright{v3,o3}   % adding two more vertices
\fmfforce{(0.5w,0.5h)}{v3}   % setting position of the first vertex
\fmfforce{(0.5w,0h)}{o3}   % setting position of the second vertex
\fmfdot{v3}   % drawing the first vertex with a dot
\fmf{gluon, label=$k\downarrow$, l.side=left}{v3,o3}   % drawing a gluon line
\end{fmfgraph*}
\caption{Radiation of a gauge boson from a outgoing fermion in a non-Abelian theory.}
\label{fig:blob with gluon radiation amplitude}
\end{figure}
\end{fmffile}

\begin{align}\label{eq:original gluon amplitude Wilson}
    \mathcal{M}_{1}=\Bar{u}(p)(-igt^{a}\slashed{\varepsilon}(k))\frac{i(\slashed{p}+\slashed{k})}{(p+k)^{2}}\,\mathcal{B}\,,
\end{align}
where $\mathcal{B}$ is the blob containing all the information that is independent of the radiation. As we consider massless fermions, we have that $(p+k)^{2}=2\,p\cdot k$. If the radiated gauge boson is soft, we can make the approximation $\slashed{p}+\slashed{k}\approx \slashed{p}$. This is known as the \emph{eikonal approximation}. We can also use that the fermion is massless and obeys the massless Dirac equation, i.e. we can use that $\Bar{u}(p)\slashed{p}=0$. This allows us to substitute $\slashed{\varepsilon}\slashed{p}$ with $\{\slashed{\varepsilon},\slashed{p}\}=2\varepsilon_{\mu}(k)p_{\nu}g
^{\mu\nu}$, i.e. we have just added zero to the amplitude. With these adjustments, we can write \cref{eq:original gluon amplitude Wilson} as
\begin{align}
    \mathcal{M}^{(1)}&=gt^{a}\Bar{u}(p)\slashed{\varepsilon}(k)\frac{\slashed{p}}{2p\cdot k}\,\mathcal{B}=gt^{a}\Bar{u}(p)\frac{p\cdot\varepsilon(k)}{p\cdot k-i\epsilon}\,\mathcal{B}\,.
\end{align}
where we inserted the pole prescription as it is understood that the gauge boson momenta is to be integrated over in observables. We observe that the \textquote{new} fermion propagator looks very much like the Wilson propagator in \cref{eq:Wilson propagator}. To see that this will indeed give a description of radiation from a Wilson line, we can consider the process with two soft gauge bosons. If we use the same approximations as above, we find the amplitude
\begin{align}\label{eq:original two gluon amplitude Wilson}
    \mathcal{M}^{(2)}&=\Bar{u}(p)(-igt^{b}\slashed{\varepsilon}(k_2))\frac{i(\slashed{p}+\slashed{k_2})}{(p+k_2)^{2}}(-igt^{a}\slashed{\varepsilon}(k_1))\frac{i(\slashed{p}+\slashed{k_1}+\slashed{k_2})}{(p+k_1+k_2)^{2}}\,\mathcal{B}\nonumber
    \\
    &=g^{a}t^{b}t^{a}\Bar{u}(p)\frac{p\cdot\varepsilon(k_2)}{p\cdot k_2-i\epsilon}\frac{p\cdot\varepsilon(k_1)}{p\cdot k_1+p\cdot k_2-i\epsilon}\,\mathcal{B}\,.
\end{align}
We can now use that the polarization vector is used when Fourier expanding gauge fields, so we can make the substitution $\varepsilon_{\mu}^{a}(k)\rightarrow A_{\mu}^{a}(k)$. Further, we observe that \cref{eq:original gluon amplitude Wilson} and \cref{eq:original two gluon amplitude Wilson} is invariant under the rescaling $p^{\mu}=p\,n^{\mu}$. Lastly, we integrate over all external momenta, giving the amplitude to $\mathcal{O}(g
^{2})$
\begin{align}\label{eq:Dressed fermion amplitude}
    \mathcal{M}&=\mathcal{M}^{(0)}+\mathcal{M}^{(1)}+\mathcal{M}^{(2)}+\mathcal{O}(g^{3})\nonumber
    \\
    &=\Bar{u}(p)\Big(1+ig\int\frac{d^{d}k_1}{(2\pi)^{d}}\,n\cdot A(k_1)\frac{-i}{n\cdot k_1-i\epsilon}\nonumber
    \\
    &\hspace{1cm}+(ig)^{2}\int\frac{d^{d}k_2}{(2\pi)^{d}}\frac{d^{d}k_1}{(2\pi)^{d}}n\cdot A(k_2)\,n\cdot A(k_1)\frac{-i}{n\cdot k_1-i\epsilon}\frac{-i}{n\cdot k_1+n\cdot k_2-i\epsilon}\nonumber
    \\
    &\hspace{1cm}+\mathcal{O}(g^{3})\Big)\,\mathcal{B}\,,
\end{align}
where we observe that the term inside the bracket is the $\mathcal{O}(g^{2})$ expansion of the Wilson line given in \cref{eq:semi-infinite Wilson line 0-to-infty}. Hence, the definition of a dressed fermion, also called an \emph{eikonal fermion}, is given by\footnote{These definitions are not operator valued, but ment to hold inside matrix elements.}
\begin{align}\label{eq:eq:wilso dress fermion}
    \overline{\Psi}(x)=\bar{\psi}(x)\,\mathcal{U}[\infty,0]\,,\hspace{0.3cm}\text{and}\hspace{0.3cm}\Psi(x)=\mathcal{U}^{\dagger}[\infty,0]\psi(x)\,.
\end{align}
These are now resummed fermion lines, as all the radiation has been exponentiated. The crucial step for this to happen was to add zero, by using that $\bar{u}(p)\slashed{p}=0$. Hence, Wilson lines as a resummation of gauge boson radiation can only appear next to on-shell fermions. 

What this implies is that by taking the soft limit of a scattering process, the structure of that amplitude is fully described by a Wilson line. This feature is one reason that Wilson lines are such useful and important objects to use in scattering calculations. As a teaser to why this is important: in \cref{chap:Resummation in QCD} we will use factorization theorems to separate a cross section into hard and soft parts. This soft part is what we will use Wilson lines to construct. This is an important concept to keep in mind when we delve into factorization of cross sections. 

%There is one last observation that we will have use for later. If we have processes with emission, we have to Wick contract them. For example, if we square \cref{eq:Dressed fermion amplitude} the $\mathcal{O}(g^{2})$ is on the form\footnote{Neglecting the fermion and blob contribution.}
%\begin{align}
%    I(g^{2})\sim g^{2}t^{a}t^{b}n^{\mu}n^{\nu}\int\frac{d^{d}k}{(2\pi)}\,A_{\mu}^{a}(k)A_{\nu}^{b}(k)\frac{1}{(n\cdot k)^{2}-i\epsilon}\,.
%\end{align}
%A Wick contraction of the emitted gauge bosons will give
%\begin{align}
%    A_{\mu}^{a}(k)A^{b}_{\nu}(k)\rightarrow D_{\mu\nu}^{ab}(k)=\frac{-ig_{\mu\nu}\delta^{ab}}{k^{2}-i\epsilon}\,,
%\end{align}
%where we represented the propagator in Lorentz gauge with the Feynman choice $\xi=1$, see \cref{eq:covariant propagator}. Hence, the integral is brought on the form
%\begin{align}
%    I(g^{2})\sim g^{2}t^{a}t^{a}n^{2}\int\frac{d^{d}k}{(2\pi)^{d}}\frac{1}{k^{2}-i\epsilon}\frac{1}{(n\cdot k)^{2}-i\epsilon}
%\end{align}
%which is a scaleless integral. We studied these integrals in \cref{sec:dimensional regularization}, and argued that they are zero in dimensional regularization. These scaleless integrals can be used to extract the UV and IR poles, see \cref{eq:scaleless integral}. Hence, we can use Wilson line expansions to extract the IR divergence and cancel the UV by counterterms.
%%%%%%%%%%%%%%%%%%%%%%%%%%%%%%%%%%%%%%%%%%%%%%
%There is one last observation we can make that we will have use for in \cref{sec:DIS}. Let us square the amplitude in \cref{eq:Dressed fermion amplitude}, and remember that in general we have to sum over polarization of all final states and average over initial. We are not interested in the full structure of this expression, so we will neglect the contribution from the blob and the polarization sum over fermions. The part we want to consider is where we have one gauge boson radiation. This is given by the square of the first order term in \cref{eq:Dressed fermion amplitude}, i.e.
%\begin{align}
%    I_{g}&\sim g^{2}\sum_{\text{pol}}n^{\mu}n^{\nu}\int\frac{d^{d}k_1}{(2\pi)^{d}}\,A_{\mu}^{a}(k_1)A_{\nu}^{b}(k_1)\frac{1}{(n\cdot k_1)^{2}}
%    \\
%    &=g^{2}C_{F}n^{\mu}n^{\nu}\int\frac{d^{d}k_1}{(2\pi)^{d}}\,D_{\mu\nu}(k_1)\frac{1}{(n\cdot k_1)^{2}}
%\end{align}
%where we have used that the polarization sum yields the gauge boson Feynman propagator. Here we used the propagator in Lorentz gauge with the Feynman gauge choice $\xi=1$.

%We will now turn our attention to the theory of strong interaction and it's perturbative behaviour. We will find several uses for Wilson lines here, but the main objective will be to recognize the nature of IR-singularities and how to treat them. We will find that even after the IR singularities have been treated, there are still contributions that can become large in certain domains. To treat these contributions, we will turn our attention to Wilson lines again and their renormalization properties.


%%%%%%%%%%%%% Wilson Line Properties %%%%%%%%%%%%%%%%%%%%%%%%
%\chapter{Wilson Line Properties}
%\section{Wilson Lines on Different Topologies}
In \cref{sec:Wilson lines and Wilson loops} we have seen that Wilson lines emerge naturally in gauge theories from a geometrical viewpoint. Because of its bi-local transformation property, it is used as a parallel transporter to render non-local terms gauge invariant. In the fibre bundle formalism we showed that its definition has a strong mathematical foundation. However, the application of Wilson lines are much broader than this, and we will exploit some of them in this chapter.

The Wilson line we defined in \cref{sec:Wilson lines and Wilson loops} is valid for any gauge theory, but as our main focus is on QCD we mostly use that the gauge fields are non-Abelian in nature. The physical consequence of involving Wilson lines in a theory becomes apparent if we expand \cref{definition:Wilson}





\section{Renormalization Group Equation for Wilson Lines}

\section{Anomalous Cusp Dimension at One Loop Order}

%\chapter{Supersymmetry}
%\section{Supersymmetry}
The basis of Supersymmetry(SUSY) is that it is a symmetry relating fermionic and bosonic states. For the last decades it has been one of the most prominent extensions of the Standard model of particle physics (SM), and it has therefore been extensively studied. Theoretically, there is no doubt that SUSY plays an important role, as it has been used to prove index theorems [ref to Witten], deriving positive energy theorems, setting lower bounds on soliton masses and in construction of consistent fermionic string theories. The caveat is that despite the enormous effort in its study, there is no experimental evidence of SUSY, so the big question is if SUSY is a symmetry of nature?

\subsection{Hierarchy problem}

\subsection{Supersymmetry Algebra}
A base concept of all Quantum Field Theories is that they are invariant under spacetime transformations of the Poincare group. This is the group of all transformations on the form
\begin{align}
    x^{\mu}\rightarrow x'^{\mu}=\Lambda^{\mu}_{\nu}x^{\nu}+a^{\mu}
\end{align}

\medskip
The idea of Supersymmetry originated from attempts to look for extensions of the Poincare group. In 1967, Coleman and Mandula(ref) proved a theorem that says:
\begin{itemize}
    \item In a generic quantum field theory, under a number of reasonable and physical assumptions, like locality, causality, positivity of energy, finiteness of particle number, etc..., the only possible continous symmeties of the S-matrix are those generated by the Poincare group generators, $P_{\mu}$ and $M_{\mu\nu}$, plus some internal symmetry group G where the generators $B_{i}$ of G commutes with them
    \begin{align*}
        [P_{\mu},B_{i}]=[M_{\mu\nu},B_{i}]=0
    \end{align*}
\end{itemize}
From this one draws the conclusion that any extension of the Poincare group to include gauge symmetries is isomorphic to the symmetry group $G\times P^{\uparrow}_{+}$, or in other words the most general symmetries of the S-matrix is $G\times P^{\uparrow}_{+}$. Now since Supersymmetry is a symmetry that relates fermions and bosons, which has different space-time properties, it is indeed a space-time symmetry. Therefore the theorem by Coleman and Mandula seems to shut down the possibility of relating fermions and bosons. The next question is then: What happens if one tries to loosen some of the assumptions at the basis of this theorem?

As is well known the internal symmetry group of the SM, $G_{SM}$,  are Lie groups with Lie algebra valued generators. Therefore they obey a set of commutation relations, and are therefore bosonic in character. However, we know that there exists fermions, and there is no apparent reason to believe that nature only prefers bosonic generators, and not fermionic generators. One of the assumptions in Coleman and Mandulas theorem is that the symmetry algebra only includes commutators, but what happens if one weakens this assumption and allows for anti-commutation relations?

In 1975 Haag, Lopuszanski and Sohnius introduced the concept of graded Lie algebras to evade Coleman and Mandulas theorem, where they showed that by introducing fermionic generators in a certain way, the set of allowed symmetries could be enlarged. They also showed that Supersymmetry is the only possible such option. This makes the Poincare group becoming SuperPoincare, and then the most general symmetry of the S-matrix is $SP\times G$.

The extension restricts the possible supersymmetries acceptable in a Quantum Field Theory with interactions, as only theories with one spinorial charge $Q_{\alpha}$, known as $N=1$ supersymmetry, allows for chiral fermions which are important for phenomenology. For this reason we only consider the $N=1$ case, where a single set of four fermionic generators are introduced through a two component Weyl spinor $Q_{a}$ and its Hermitian conjugate $(Q_{a})^{\dagger}=\Bar{Q}_{\dot{a}}$. The supersymmetric extension to the Poincare algebra is then given by the following set of commutation and anti-commutation relations
\begin{align*}
    [P_{\mu},Q_{\alpha}]&=0
    \\
    [Q_{\alpha},M^{\mu\nu}]&=\frac{1}{2}(\sigma^{\mu\nu})_{\alpha}^{\beta}Q_{\beta}
    \\
    \{\}
    \\
    \{Q_{\alpha},\Bar{Q}_{\beta}\}&=2(\gamma^{\mu})_{\alpha\beta}P_{\mu}
\end{align*}
This shows that the superalgebra closes to yield the the generator of the Poincare group, which shows that it is indeed a space-time symmetry. By taking the trace of the last identity, we get that the Hamiltonian is given by
\begin{align*}
    H=P^{0}=\frac{1}{4}(\Bar{Q_{1}}Q_{1}+Q_{1}\Bar{Q_{1}}+Q_{2}\Bar{Q_{2}}+\Bar{Q_{2}}Q_{2})
\end{align*}
So by considering some state $\ket{\Psi}$ we get that the expectation value of the Hamiltonian is
\begin{align*}
    \ev{H}{\Psi}\geq 0
\end{align*}
Therefore the expectation value is zero only if the state $\ket{\Psi}$ is annihilated by all supersymmetry generators. Such a state would have to be the state with lowest possible energy, which of course is the vacuum state $\ket{0}$. From basic principles in physics, we know that if we have a symmetry it commutes with the Hamiltonian and the ground state has to be invariant. Therefore we have that
\begin{align}
    Q_{a}\ket{0}=\Bar{Q}_{\dot{a}}\ket{0}=0
\end{align}
Which means that $\ev{H}{0}=0$, i.e the ground state has zero energy if supersymmetry is preserved. The converse situation where $H\ket{0}\neq0$ means that supersymmetry is broken, which is a relation that will be useful when we are forced to break supersymmetry. The scenario with zero vacuum energy is the first indication of cancellation between fermions and bosons. In non-supersymmetric Quantum Field Theories the zero point energy of the bosonic oscillators is positive and add up to infinity, whereas fermionic oscillators add up to negative infinity. These are generally set to zero as energy is measured in terms of energy differences, but in supersymmetry they naturally cancel.
\medskip

\subsection{Irreducible representations of the Superalgebra}
As usual in physics we want the members of our group, i.e the generators, to act on physical states. Therefore we must ask the question of what kind of particles transform under the SuperPoincare group? Which is the same as asking: what are the irreducible representations of the SuperPoincare group?

From Schur's lemma we know that for any irreducible representation of a Lie group, the Casimir operator(s) are proportional to the identity. It can be shown that the Casimir operators of the SuperPoincare group is $P^{2}=P_{\mu}P^{\mu}$ and $C^{2}$(ref to KW for general expressions).

Without loss of generality one can go to a particles rest frame $P=(m,0)$, and a state conveniently labeled by the $m$, will as expected give eigenvalue $m^{2}$ when acted upon by $P^{2}$. It can further be shown that in this scenario the other Casimir operator takes the form(KW)
\begin{align}
    C^{2}=2m^{4}J_{k}J^{k}
\end{align}
where $J_k$ is an abstraction of the spin operator as it obey the same algebraic strucutre. Then a state $\ket{m,j,j_{3}}$ when acted upon by $C^{2}$ will give
\begin{align}
    C^{2}\ket{m,j,j_3}=-m^{4}j(j+1)\ket{m,j,j_3}
\end{align}
where $j=0,\frac{1}{2},1,...$ and $j_3=-j,-j+1,...,j$, since $J_k$ obeys the angular momentum commutation relation. Therefore one concludes that the irreducible representaitions of the superalgebra can be labeled by $(m,j)$, and any given set of $m$ and $j$ will give $2j+1$ states with different $j_3$.

We are now in a position to construct all the states for a given representation labeled bu $(m,j)$, and it is convenient to use the Weyl spinor representation of the $Q$ generators. It can be shown that there exists a state $\ket{\Omega}$, such that
\begin{align}
    Q_{A}\ket{\Omega}=0
\end{align}
which we call the Clifford vacuum. By using the explicit expression for $J_{k}$
\begin{align}
    J_{k}=S_{k}-\frac{1}{4m}\bar{Q}_{\dot{B}}\bar{\sigma}_{k}^{\dot{B}A}Q_{A}
\end{align}
giving
\begin{align}
    J_{3}\ket{\Omega}=S_{k}\ket{\Omega}=j_{3}\ket{\Omega}
\end{align}
meaning that $s=j$ and $s_{3}=j_{3}$ are the eigenvalues of $S^{2}$ and $S_{3}$ for the clifford vacuum. We can construct three more states from the Clifford vacuum
\begin{align}
    &\bar{Q}^{\dot{1}}\ket{\Omega}
    \\
    &\bar{Q}^{\dot{2}}\ket{\Omega}
    \\
    &\bar{Q}^{\dot{1}}\bar{Q}^{\dot{2}}\ket{\Omega}
\end{align}
Specifically, we can show that
\begin{align}
    S_{3}\bar{Q}^{\dot{A}}\ket{\Omega}&=\big(j_{3}\pm\frac{1}{2}\big)\bar{Q}^{\dot{A}}\ket{\Omega}
    \\
    S_{3}\bar{Q}^{\dot{1}}\bar{Q}^{\dot{2}}\ket{\Omega}&=j_{3}\bar{Q}^{\dot{1}}\bar{Q}^{\dot{2}}\ket{\Omega}
\end{align}
In total this means that each set of quantum numbers $m,j,j_3$ gives two states with $s_3=j_3$, and two states with $s_3=j_3\pm\frac{1}{2}$, giving two bosonic and two fermionic states. This highlights the feature of supersymmetric theories that there are an equal number of bosonic and fermionic states, with the same mass.

\subsection{The $j=0$ and $j=1/2$ irreducible representations}
For $j=0$ we must have that $j_3=0$ and as a result the Clifford vacuum $\ket{\Omega}$ must have $s=0$, and is a bosonic state. Further, there are two states $\bar{Q}^{\dot{A}}\ket{\Omega}$ with $s=\frac{1}{2}$, and $\bar{Q}^{\dot{1}}\bar{Q}^{\dot{2}}\ket{\Omega}$ with $s=0$. In total there are two bosonic states and two fermionic states, which we later will represent with what we call a \emph{scalar superfield}.
\medskip
For $j=\frac{1}{2}$ we have two Clifford vacua with $s=\frac{1}{2}$ and $s_{3}=\pm\frac{1}{2}$, which we label $\ket{\Omega;\frac{1}{2}}$ and $\ket{\Omega;-\frac{1}{2}}$. Further, we construct from each of these two new fermionic states $\bar{Q}^{\dot{1}}\bar{Q}^{\dot{2}}\ket{\Omega;\pm\frac{1}{2}}$ with $s_{3}=\mp\frac{1}{2}$. In addition we have two states with $s_3=0$, one state with $s_3=1$, and one state with $s_3=-1$. In total there are four fermionic states and four bosonic states. When talking about particles, these correspond to two fermions, one vector boson and one scalar particle. These states will later be referred to as the \emph{vector superfield}.


\subsection{Supermultiplets}
Since the single-particle fermionic and bosonic states transform into one another under $Q_{a}$ and $\bar{Q}_{\dot{a}}$, they are called \textit{superpartners}. These are combined in \textit{supermultiplets} that form irreucible representations of the superalgebra. From the relation $\{Q_{a},Q_{\dot{a}}\}\sim P$, with $P$ a one-to-one mapping, the number of bosonic and fermionic degrees of freedom in a supermultiplet must be equal, $n_B=n_F$.

As the most general symmetry group is given by the direct product $SP\times G$, the supersymmetry generators commutes with the gauge group generators, from which it follows that all particles in a supermultiplet will have the same gauge quantum numbers. However, in the Standard Model there are no fermion-boson pairs with the same mass and gauge quantum numbers, and for this reason supersymmetry must be a broken symmetry in nature's current vacuum state. The breaking of supersymmetry we alluded to when we studied the Hamiltonian in terms of the operators $Q_{a}$, and is a topic that we will return to later.
\medskip

In the simplest case of $N=1$ supersymmetry, there are only two relevant supermultiplets, which are called \textit{chiral multiplets} and \textit{vector mulitplets}. The chiral multiplets are the smallest possible irreducible representation, and contains a single Weyl fermion with $n_F=2$, and a complex scalar field with $n_B=2$\footnote{Or two real scalar fields $\phi_1$ and $\phi_2$ with $n_B=1$ each, that are assembled into a complex scalar field}. The fields in each multiplet must transform according to the same representation of any gauge symmetry. We know from the SM that the left handed fermions transform differently under the gauge group than the right handed fermions, and chiral multiplets are the only multiplets that has this property. Therefore SM fermions must be members of chiral supermultiplets, with the consequence that the superpartners of quarks and leptons are spin-0 bosons. These superpartners are of course scalars, which is called \textit{scalar fermions} or \textit{sfermions}. The SM fermions are Dirac particles\footnote{Maybe with the exception of neutrinos}, which has left and right-handed pieces that are two-component Weyl spinors. Therefore each of these have a complex scalar partner, e.g the electron $e$ has two scalar partners, $\Tilde{e}_{L}$ and $\Tilde{e}_R$. The selectrons are scalars so they do not have a handedness, thus the $L$ and $R$ are only names to indicate which component of the Dirac fermion they are superpartners of.

Vector supermultiplets contains a vector boson and a Weyl spinor, both with two degrees of freedom. The vector bosons of the SM transform in the adjoint representation, thus the members of a vector supermultiplet must belong to the adjoint representation of the gauge group. The superpartners of the SM gauge bosons are fermions and are called \textit{gauginos}, e.g the supersymmetric partner of the gauge boson \textit{gluino} are called \textit{gluino}.


\subsection{Constructing Supersymmetric Lagrangians}
The construction of supersymmetric field theories can be done in to different but equally valid set of formalisms. The first is the regular Quantum Field Theory approach, with the formulation in terms of regular spacetime dependent component fields (see Stephen Martin). This formalism is very explicit, but can tend to be extremely tedious as there are an enormous amount of possible terms that has to be written out explicitly. The second and more elegant approach is to use the language of \textit{superfields}, where there are one superfield per supermultiplet.

The superfields are constucted by defining them as functions on \textit{superspace}\footnote{Introduced by Salam and Strathdee}, which is an extension of Minkowski spacetime with a set of four anticommuting co-ordinates. From this formalism, the regular spacetime dependent Lagrangian are found by integrating out the fermionic co-ordinates, a process that has the advantage of keeping only the supersymmety-invariant terms in the final Lagrangian.

\subsection{Superspace}
In superspace the invariance of supersymmetry transformations are manifest, just as invariance of Lorentz transformations are manifest in Minkowski space. Superspace is an eight dimensional manifold that are constructed from the coset space of the super-Poincare group and the Lorentz group, $SP/L$. Co-ordinates are then given by $z^{\pi}=(x^{\mu},\theta^{a},\bar{\theta}_{\dot{a}})$, where $x^{\mu}$ are the usual Minkowski co-ordinates, and $\theta^{a}(\bar{\theta}_{\dot{a}})$ are four anti-commuting Grassmann numbers, being the parameters of the $Q$ operators in the superalgebra. These Grassmann numbers we have labeled by $a,\dot{a}$ as we want four of them and in addition we want to place them in Weyl spinors. As Grassmann numbers anti-commute, we have the following relations
\begin{align}
    \{\theta^{a},\theta^{b}\}=...=0
\end{align}
Giving the relations
\begin{align}
    \theta_{a}^{2}&=\theta_{a}\theta_{a}=-\theta_{a}\theta_{a}=0
    \\
    \theta^{2}&\equiv\theta^{a}\theta_{a}=-2\theta_{1}\theta_{2}
    \\
    \bar{\theta}^{2}&\equiv\bar{\theta}_{\dot{a}}\bar{\theta}^{\dot{a}}=-2\bar{\theta}^{\dot{1}}\bar{\theta}^{\dot{2}}
\end{align}
This has the important consequence that if we have a generic function $f(\theta)$, that can be expanded in a power series, it is given by
\begin{align*}
    f(\theta)=a+b\theta_{a}
\end{align*}
as all the higher order terms vanish. Further, Grassmann calculus becomes very simple. Differentiating $f(\theta)$ just gives
\begin{align}
    \pdv{f}{\theta_{a}}=a
\end{align}
and by defining the integrals
\begin{align}
    \int d\theta_{a}&\equiv0
    \\
    \int d\theta_{a}\,\theta_{a}&\equiv1
\end{align}
with the condition of linearity, then you can easily see that
\begin{align}
    \int d\theta_{a}\,f(\theta_{a})=\pdv{f}{\theta_{a}}=a
\end{align}
meaning that with Grassmann numbers integration and differentiation are the same operation.

Integration over multiple Grassmann numbers are
\begin{align}
    \int d^{2}\theta\,\theta\theta&=1
    \\
    \int d^{2}\bar{\theta}\,\bar{\theta}\bar{\theta}&=1
    \\
    \int d^{4}\theta\,(\theta\theta)(\bar{\theta}\bar{\theta})&=1
\end{align}
where the volume elements have been defined as
\begin{align}
    d^{2}\theta&=-\frac{1}{4}d\theta^{a}d\theta_{a}
    \\
    d^{2}\bar{\theta}&=-\frac{1}{4}d\bar{\theta}_{\dot{a}}d\bar{\theta}^{\dot{a}}
    \\
     d^{4}\theta&=d^{2}\theta d^{2}\bar{\theta}
\end{align}

Further we are interested in the differential representation of the supersymmetric generators. In order to find their expression we use the exponential map, from which we know a group element of $g\in SP$ can be written in the following way
\begin{align}
    g=\exp\big(-ix^{\mu}P_{\mu}+i\theta^{A}Q_{A}+i\bar{\theta}_{\dot{A}}\bar{Q}^{\dot{A}}-\frac{i}{2}w_{\rho\nu}M^{\rho\nu}\big)
\end{align}
where $x^{\mu}, \theta, \bar{\theta}$ and $w_{\rho\nu}$ are the parametrisation of the group, and $P_{\mu},Q_{A},\bar{Q}^{\dot{A}}$ and $M^{\rho\nu}$ are the generators. We can now parametrise the coset space $SP/L$ by simply setting $w_{\mu\nu}=0$, and the remaining parameters of $SP/L$ span superspace. In order to show the effect of supersymmetry transformations, we begin by noting that any $SP$ transformation can effectively be written as
\begin{align*}
    L(a,\alpha)=\exp\big(-ia^{\mu}P_{\mu}+i\theta^{A}Q_{A}+i\bar{\theta}_{\dot{A}}\bar{Q}^{\dot{A}}\big)
\end{align*}
beacuse one can show that
\begin{align*}
    \exp\big(-\frac{i}{2}w_{\mu\nu}M^{\mu\nu}\big)L(a,\alpha)=L(\Lambda a,S(\Lambda)\alpha)\exp\big(-\frac{i}{2}w_{\mu\nu}M^{\mu\nu}\big)
\end{align*}
which mean that all Lorentz boost does is to transform spacetime coordinates by $\Lambda(M)$ and Weyl spinors by $S(\Lambda(M))$, which is a spinor representation of $\Lambda(M)$. Thus, we can pick frames, do our thing with the transformation, and boost back to any frame we wanted. Now we can find the transformation of superspace coordintes under a supersymmetry transformation, just as we have all seen the transformation of Minkowski coordinates under Lorentz transformations. The effect of $L(a,\alpha)$ on a superspace coordinate $z^{\pi}$ is defined by the mapping $z^{\pi}\rightarrow z'^{\pi}$, given by
\begin{align*}
    \exp{iz'^{\pi}K_{\pi}}=L(a,\alpha)\exp{iz^{\pi}K_{\pi}}
\end{align*}
where $z^{\pi}=(x^{\mu},\theta_{A},\bar{\theta}^{\dot{A}})$ and $K_{\pi}=(P_{\mu},Q_{A},\bar{Q}^{\dot{A}})$. Writing this out we get
\begin{align*}
    \exp{iz'^{\pi}K_{\pi}}&=L(a,\alpha)\exp{iz^{\pi}K_{\pi}}
    \\
    &=\exp\big(-ia^{\mu}P_{\mu}+i\theta^{A}Q_{A}+i\bar{\theta}_{\dot{A}}\bar{Q}^{\dot{A}}\big)\exp{iz^{\pi}K_{\pi}}
\end{align*}
Using the Baker-Hausdorff formula, we can show that the commutator $$[(-ia^{\mu}P_{\mu}+i\theta^{A}Q_{A}+i\bar{\theta}_{\dot{A}}\bar{Q}^{\dot{A}}),iz'^{\pi}K_{\pi}]\propto P_{\mu}$$
which commutes with all operators, and therefore all the higher commutators in the expansion are zero. So effectively this means that we are left with
\begin{align*}
    \exp{iz'^{\pi}K_{\pi}}=\exp{-i(x^{\mu}+a^{\mu}-i\alpha^{A}\sigma^{\mu}_{A\dot{A}}\bar{\theta}^{\dot{A}}+i\theta^{A}\sigma^{\mu}_{A\dot{A}}\bar{\alpha}^{\dot{A}})P_{\mu}+i(\theta^{A}+\alpha^{A})Q_{A}+i(\bar{\theta}_{\dot{A}}+\bar{\alpha}_{\dot{A}})\bar{Q}^{\dot{A}}}
\end{align*}
From which we draw the conclusion that superspace coordinates transform under supersymmetry transformations as
\begin{align*}
    (x^{\mu},\theta^{A},\bar{\theta}_{\dot{A}})\rightarrow f(a^{\mu},\alpha^{A},\bar{\alpha}_{\dot{A}})=(x^{\mu}+a^{\mu}-i\alpha^{A}\sigma^{\mu}_{A\dot{A}}\bar{\theta}^{\dot{A}}+i\theta^{A}\sigma^{\mu}_{A\dot{A}}\bar{\alpha}^{\dot{A}},\theta^{A}+\alpha^{A},\bar{\theta}_{\dot{A}}+\bar{\alpha}_{\dot{A}})
\end{align*}
Now we can write down the differential representation of the supersymmetry generators by applying the standard expression for generators $X_i$ of a Lie algebra, given the functions $f_{\pi}$ for the transformation of the parameters:
\begin{align*}
    X_{i}=\pdv{f_{\pi}}{a_i}\pdv{}{z_{\pi}}
\end{align*}
which gives us
\begin{align*}
    P_{\mu}&=i\partial_{\mu}
    \\
    iQ_{A}&=\partial_{A}-i(\sigma^{\mu}\bar{\theta})_{A}\partial_{\mu}
    \\
    i\bar{Q}^{\dot{A}}&=\partial^{\dot{A}}-i(\sigma^{\mu}\theta)^{\dot{A}}\partial_{\mu}
\end{align*}


\subsection{Superfields}
In general, any superfield as a function of superspace $\Phi(x^{\mu},\theta,\bar{\theta})$, can be taylor expanded in the Grassmann valued variables, with spacetime dependent components. A general superfield can then be written as
\begin{align}
    \Phi &= a(x)+\theta\xi(x)+ \bar{\theta}\bar{\chi}(x)+\theta\theta b(x)+\bar{\theta}\bar{\theta} c(x)+\bar{\theta}(\bar{\sigma}^{\mu})\theta v_{\mu}(x)\nonumber
    \\
    &\hspace{1cm}+\bar{\theta}\bar{\theta}\theta\eta(x)+\theta\theta\bar{\theta}\bar{\psi}(x)+\theta\theta\bar{\theta}\bar{\theta}d(x)
\end{align}
where $a(x), b(x)$, $c(x)$ and $d(x)$ are complex scalar fields (or complex pseduo scalar). $\xi(x)$ and $\eta(x)$ field are left-handed Weyl spinors, $\bar{\chi}(x)$ and $\bar{\psi}(x)$ are right-handed Weyl spinors, and lastly $V_{\mu}$ is a bosonic vector field.

As we know from classic quantum field theory, we want the Lagrangian to be invariant under certain kinds of symmetry transformations, like Lorentz transformations and gauge transformations. In the case of local gauge transformations we need to introduce the covariant derivative in order to compare points living in different tangent spaces. The same idea applies for SUSY transformations, and the general supersymmetric covariant derivatives can be written as
\begin{align}
    D_{A}&\equiv\partial_{A}+i(\sigma^{\mu}\bar{\theta})_{A}\partial_{\mu}
    \\
    \bar{D}^{\dot{A}}&\equiv-\partial^{\dot{A}}-i(\sigma^{\mu}\theta)^{\dot{A}}\partial_{\mu}
\end{align}
which acts on the superfields $\Phi$.
\subsection{Chiral Supermultiplets}
A chiral supermultiplet is represented by a chiral superfield, and as by its defining name we distinguish between left-chiral and right-chiral superfields. By definition these must obey the following constraints
\begin{align}
    \bar{D}_{\dot{A}}\Phi&=0\hspace{2cm}(\text{left-chiral})
    \\
    {D}^{A}\Phi^{\dagger}&=0\hspace{2cm}(\text{right-chiral})
\end{align}
With the further requirement that $\Phi$ should be Lorentz scalars or pseudo-scalars, it can be shown that the left and right-handed chiral fields can be written in terms of their component fields as
\begin{align}
    \Phi(x,\theta,\bar{\theta})&=\phi(x)+i(\theta\sigma^{\mu}\bar{\theta})\partial_{\mu}\phi(x)-\frac{1}{4}\theta\theta\bar{\theta}\bar{\theta}\partial_{\mu}{\partial^{\mu}}\phi(x)+\sqrt{2}\theta\psi(x)
    \\
    &\hspace{1cm}-\frac{i}{\sqrt{2}}\theta\theta\partial_{\mu}\psi(x)\sigma^{\mu}\bar{\theta}+\theta\theta F(x)
\end{align}
where $\phi(x)$ and $F(x)$ are complex scalars, and $\psi_{A}$ are a left handed Weyl spinor. Taking the hermitian conjugate we will get the complex versions $A^{*}$ and $F^{*}$ together with a right handed Weyl spinor $\psi^{\dot{A}}$. Now we are in a position to compare the above with the $j=0$ irreducible representation, which had two scalar states and two fermionic states. After applying the e.o.m we see that the auxiliary field $F$ can be eliminated together with two fermionic degrees of freedom. That leaves us with two scalar states and two fermionic states, exactly the same as in the $j=0$ representation.

However, in the SM we have Dirac fermions and Weyl spinors can not alone describe them. Therefore the scalar superfields will not directly correspond to particle states for the SM, but if we construct particle representations by taking one left handed scalar superfield and one different right handed scalar superfield, we form a Dirac fermion and two scalars (and their antiparticles) after applying the e.o.m.

\subsection{Vector Supermultiplets}
The vector superfield $V(x,\theta,\bar{\theta})$ is used to represent the vector supermultiplet. A vector superfield is obtained by requiring it to be real
\begin{align}
    V^{\dagger}(x,\theta,\bar{\theta})=V(x,\theta,\bar{\theta})
\end{align}
With this constraint it can be shown by taking the the general superfields $\Phi^{\dagger}$ and $\phi$, that a general vector superfield can be written as
\begin{align*}
    V(x,\theta,\bar{\theta})&=f(x)+\theta\psi(x)+\bar{\theta}\bar{\psi}(x)+\theta\theta m(x)+\bar{\theta}\bar{\theta}m^{*}(x)
    \\
    &\hspace{0.5cm}+\theta(\sigma^{\mu})\Bar{\theta} \,V_{\mu}(x)+\theta\theta\bar{\theta}\bar{\lambda}(x)+\bar{\theta}\bar{\theta}\theta\lambda(x)+\theta\theta\bar{\theta}\bar{\theta}d(x)
\end{align*}
where $f(x)$, $d(x)$ are real scalar fields, $\psi(x)$ and $\lambda(x)$ are Weyl spinors, $m(x)$ is a complex scalar field and $V_{\mu}(x)$ is a vector field. Examples of a vector superfield are $V=\Phi^{\dagger}\Phi$, $\Phi^{\dagger}+\Phi$ and $i(\Phi^{\dagger}-\Phi)$. It is however unfortunate that this does not correspond to the wanted degrees of freedom in the $j=1/2$ representation of the superalgebra. There is of course a solution, and that is gauge freedom. Given a vector superfield $V(x,\theta,\bar{\theta})$, an abelian supergauge transformation is defined as
\begin{align}
    V'(x,\theta,\bar{\theta})=V(x,\theta,\bar{\theta})+i(\Lambda(x,\theta,\bar{\theta})-\Lambda^{\dagger}(x,\theta,\bar{\theta}))
\end{align}
Using the gauge freedom, we can choose component fields to eliminate degrees of freedom. One particular choice is the Wess-Zumino gauge, which defines
\begin{align}
    \chi(x)&=-\frac{1}{\sqrt{2}}\psi(x)
    \\
    F(x)&=-m(x)
    \\
    \alpha(x)&=\phi(x)+\phi^{*}(x)=-f(x)
\end{align}
Thus, a vector superfield in the WZ gauge can be written as
\begin{align}
    V_{WZ}(x,\theta,\bar{\theta})=\frac{1}{2}(\theta\sigma^{\mu}\theta)\big(V_{\mu}(x)+i\partial_{\mu}\alpha(x))\big)+\theta\theta\bar{\theta}\bar{\lambda}(x)+\bar{\theta}\bar{\theta}\lambda(x)+\theta\theta\bar{\theta}\bar{\theta}d(x)
\end{align}
which contains one real scalar d.o.f, three gauge field d.o.f and four fermion d.o.f, corresponding to the $j=1/2$ representation. The WZ gauge is particularly convenient for calculations as
\begin{align}
    V_{WZ}^{2}=\frac{1}{2}\theta\theta\bar{\theta}\bar{\theta}\big(V_{\mu}(x)+i\partial_{\mu}\alpha(x)\big)\big(V_{\mu}(x)+i\partial_{\mu}\alpha(x)\big)
\end{align}


\subsection{Supersymmetric Lagrangian}
A basic feature of all field theories is that the symmetry transformations of the Lagrangian $\mathcal{L}$, should leave the action $\mathcal{S}$ invariant
\begin{align}\label{eq:Action}
    \mathcal{S}=\int d^{4}x\,\mathcal{L}
\end{align}
This is fulfilled as long as the symmetry transformation leaves the Lagrangian invariant up to a total derivative
\begin{align*}
    \mathcal{L}\rightarrow \mathcal{L'}=\mathcal{L}+\partial_{\mu}K(x)
\end{align*}
where $K(x)$ is a spacetime function that vanishes on the surface of integration. For supersymmetric transformations it can be shown that the highest order component fields in $\theta$ and $\bar{\theta}$ of a superfield always transform in this way. From the Grassmann calculus we had the identity
\begin{align*}
    \int d^{4}\theta \, (\theta\theta)(\bar{\theta}\bar{\theta})=1
\end{align*}
Therefore, the highest order component can be projected out by modifying the action as
\begin{align}\label{eq:SAction}
    \mathcal{S}=\int d^{4}x\int d^{4}\theta \,\mathcal{L}
\end{align}
which ensures that the action is invariant under supersymmetry transformations. Here $\mathcal{L}$ is a function of superfields and it is not the same as in \cref{eq:Action}. In classical quantum field theories renormalizable Lagrangians has mass dimension $[\mathcal{L}]=M^{4}$, but counting dimension in \cref{eq:SAction} we see that $[\mathcal{L}]=M^{2}$. Including Grassmann variables renormalizability forbids terms with mass dimension higher than four, which means that $\mathcal{L}$ can have at most three powers of scalar superfields $\Phi$. The general supersymmetric Lagrangian can therefore be written as
\begin{align}
    \mathcal{L}=\Phi_{i}^{\dagger}\Phi_{i}+\bar{\theta}\bar{\theta}W[\Phi]+\theta\theta W[\Phi^{\dagger}]
\end{align}
where $\Phi_{i}^{\dagger}\Phi_{i}$ is the kinetic term, and $W[\Phi]$ is a holomorphic function of $\Phi$ called the $\emph{superpotential}$, given by
\begin{align*}
    W[\Phi]&=a_{i}\Phi_{i}+m_{ij}\Phi_{i}\Phi_{j}+\lambda_{ijk}\Phi_{i}\Phi_{j}\Phi_{k}
\end{align*}
where $m_{ij}^{2}$ and $\lambda_{ijk}$ are symmetric. In order to specify a supersymmetric Lagrangian with a given superfield content we only need to specify the superpotential.


\subsection{Supergauge}
As in the Standard Model we want our supersymmetric theory to be gauge invariant. This derivation follows in the same manner as in (ref to gauge chapter), but for concreteness we follow through the steps. We consider a general gauge group $G$ with Lie algebra
\begin{align}
    [t^{a},t^{b}]=if^{abc}t^{c}
\end{align}
where $t^{a}$ are the usual gauge group generators and $f^{abc}$ are the structure constants of the gauge group. As usual a group element $g$ can be represented by the unitary exponential map, which give a local gauge transformation
\begin{align}\label{eq:supergauge}
    U(x)=\exp(ig\Lambda^{a}t^{a})
\end{align}
The gauge transformation for a left handed superfield is then
\begin{align}
    \Phi\rightarrow\Phi'=U\Phi
\end{align}
where $q$ is the charge of the superfield $\Phi$ under $G$, and $\Lambda^{a}(x)$ is the parameters of the gauge transformation. As $\Phi$ is a left handed superfield, so for this transformation to make sense $\Lambda^{a}$ must be a left handed superfield. Likewise, $\Lambda^{\dagger\,a}$ must be a right handed superfield as $\Phi^{\dagger}$ is a right handed superfield. For the superpotential to be gauge invariant $W[\Phi']=W[\Phi]$, we find the following requirements
\begin{align}
    a_{i}&=0\hspace{3mm}\text{if}\hspace{3mm} a_{i}U_{ir}\neq a_{r}
    \\
    m_{ij}&=0\hspace{3mm}\text{if}\hspace{3mm} m_{ij}U_{ir}U_{js}\neq m_{rs}
    \\
    \lambda_{ijk}&=0\hspace{3mm}\text{if}\hspace{3mm} \lambda_{ijk}U_{ir}U_{js}U_{kt}\neq \lambda_{rst}
\end{align}
For the kinetic term to be invariant we must introduce a gauge compensating vector superfield $V^{a}(x)$ for each Lie algebra generator $t^{a}$. We represent the vector superfield as
\begin{align}
    V(x)=\exp(gV^{a}t^{a})
\end{align}
The kinetic term can then written as
\begin{align}
    \mathcal{L}_{kin}=\Phi^{\dagger}V\Phi
\end{align}
Under gauge transformation this will give
\begin{align}
    \mathcal{L}_{kin}\rightarrow\mathcal{L'}_{kin}=\Phi^{\dagger}e^{-ig\Lambda^{\dagger\,a}t^{a}}e^{gV'^{a}t^{a}}e^{ig\Lambda^{a}t^{a}}\Phi
\end{align}
which mean that the vector superfield must transform as
\begin{align}
    V\rightarrow V'=U^{\dagger}VU
\end{align}
On infinitesimal and component form this reduces to the standard non-abelian gauge transformations
\begin{align}
    V_{\mu}^{a}=V_{\mu}^{a}+\partial_{\mu}\alpha^{a}+gf^{abc}V_{\mu}^{b}\alpha^{c}
\end{align}

\subsection{Supersymmetric Field Strength}
As we have introduced new dynamical gauge fields we must define the field strength in order to fully construct the supersymmetric Lagrangian. We want the field strengths themselves to be superfields, so we define the supersymmetric field strengths as
\begin{align}
    W_{A}&\equiv -\frac{1}{4}\bar{D}\bar{D}e^{-V^{a}t^{a}}D_{A}e^{V^{a}t^{a}}
    \\
    \bar{W}_{\dot{A}}&\equiv-\frac{1}{4}DDe^{-V^{a}t^{a}}D_{\dot{A}}e^{V^{a}t^{a}}
\end{align}
We know that $D^{3}=\bar{D}^{3}=0$, so
\begin{align}
    \bar{D}_{\dot{A}}W_{A}&=0
    \\
    D_{A}\bar{W}_{\dot{A}}&=0
\end{align}
which mean that $W_{A}$ is a left handed superfield and $\bar{W}_{\dot{A}}$ is a right handed superfield. As in the Standard Model the gauge invariant field strength terms for Non-Abelian gauge fields is given by the trace of the field strengths, Tr$[W^{A}W_{A}]$. Since $W_{A}$ is spanned by the Lie group generators, we have that
\begin{align}
    \text{Tr}[W^{A}W_{A}]=W^{a\,A}W_{A}^{b}\text{Tr}[t^{a}t^{b}]=W^{a\,A}W_{A}^{b}\delta^{ab}T(R)=T(R)W^{a\,A}W_{A}^{a}
\end{align}
where $T(R)$ is the Dynkin index. Further, by expanding in terms of component fields it can be shown that it contains the ordinary non-abelian field strength
\begin{align}
    F_{\mu\nu}^{a}=\partial_{\mu}V_{\nu}^{a}-\partial_{\nu}V_{\mu}^{a}+gf^{abc}V_{\mu}^{b}V_{\nu}^{c}
\end{align}
Now we have all the ingredients to write down the full unbroken supersymmetric Lagrangian. Normalizing the field strength terms with the Dynkin index, the most general supersymmetric Lagrangian in terms of superfields is
\begin{align}\label{eq:superlagrangian}
    \mathcal{L}(\Phi)=\Phi^{\dagger}e^{gV^{a}t^{a}}\Phi+\bar{\theta}\bar{\theta}W[\Phi]+\theta\theta W[\Phi^{\dagger}]+\frac{1}{2}W^{a\,A}W_{A}^{a}
\end{align}
For clarity it is instructive to present this Lagrangian in terms of the spacetime dependent component fields. For details on how to expand in component fields, see \cite{Martin:1997ns}, which will give the Lagrangian
\begin{align}\label{eq:component superLagrangian}
    \mathcal{L}=-\frac{1}{4}&(F_{\mu\nu}^{a})^{2}-(D_{\mu}\phi_{i})^{*}(D^{\mu}\phi_{i})+i\bar{\psi}_{i}\bar{\sigma}^{\mu}D_{\mu}\psi_{i}+i\bar{\lambda}^{a}\bar{\sigma}^{\mu}D_{\mu}\lambda^{a}
    \\
    &-\frac{1}{2}(W_{ij}\psi_{i}\psi_{j}+h.c)-\sqrt{2}g(\phi_{i}^{*}t^{a}\psi_{i}\lambda^{a}+h.c)
    \\
    &-W_{i}W_{i}^{*}-\frac{1}{2}g(\phi_{i}^{*}t^{a}\phi_{i})^{2}
\end{align}
where $D_{\mu}$ are the usual covariant derivative defined in [ref to covariant derivative]. The functions $W_{i}$ and $W_{ij}$ are the derivatives of the superpotential in terms the component fields
\begin{align}
    W_{i}&=\pdv{W}{\phi_{i}}=m_{ij}\phi_{j}+y_{ijk}\phi_{j}\phi_{k}
    \\
    W_{ij}&=\frac{\partial^{2}W}{\partial\phi_{i}\partial\phi_{j}}=m_{ij}+y_{ijk}\phi_{k}
\end{align}
As shown in [ref to superpotential trnaformation] gauge invariance restricts the parameters in the superpotential $W$. The tadpole term is only allowed if $\phi_{i}$ is a gauge singlet, and mass terms $m_{ij}\phi_{i}\phi_{j}$ can only appear if the representations the fields transform under are conjugates of one another. The yukawa term $y_{ijk}\phi_{i}\phi_{j}\phi_{k}$ require that the fields transform under representations that combine to a gauge singlet. It is worth noting that the mass parameter $m_{ij}$ and yukawa parameter $y_{ijk}$ appear in mass terms of both scalars and fermions, highlighting that members of each supermultiplet are mass degenerate. Also, the same set of yukawa couplings enter linearly in $\phi\psi\psi$ interactions and quadratically in $\phi^{4}$ interactions, which explains the coupling relation $|\lambda_{f}|^{2}=\lambda_{s}$
needed to cancel the quadratic divergences in the Higgs mass correction.


\subsection{Supersymmetry Breaking}
As mentioned above, supersymmetry predicts that all superpartners within a supermultiplet has the same mass. This scenario has been excluded by experiments, as none of the sparticles have been observed, but their Standard Model partners have been observed in vast numbers. The only solution to this problem is that supersymmetry must badly broken. Fortunately, we have a similar problem in the Standard Model, where we use the Higgs mechanism to spontaneously break the electroweak gauge invariance such that the weak gauge bosons aquire a mass. So, if we believe that supersymmetry is an exact symmetry of a fundamental theory, it must be spontaneously broken, which happens when the vacuum state is not invariant under the symmetry. Let us go back to the discussion of the vacuum state we had in [ref to supersymmetric hamiltonian], where we found the Hamiltonian
\begin{align}
    H=\frac{1}{4}(\Bar{Q_{1}}Q_{1}+Q_{1}\Bar{Q_{1}}+Q_{2}\Bar{Q_{2}}+\Bar{Q_{2}}Q_{2})
\end{align}
We showed then that we must have $\ev{H}{\psi}\geq 0$, where $\psi$ is some state. Then we showed that if the vacuum state is supersymmetric, we must have $\ev{H}{0}=0$. So, if the converse is true and there exist a positive vacuum energy, supersymmetry must be broken. As in the Standard Model we want to find a scalar potential that does the job, and in supersymmetry we have the scalar potential
\begin{align}
    V(\phi_{i},\phi_{i}^{*})=\sum_{i=1}^{n}\big|\pdv{W[\phi_{1},\dots,\phi_{n}]}{\phi_{i}}\big|^{2}
\end{align}
By applying the equation of motion for the auxiliary field $F_{i}$, we find that
\begin{align}
    F_{i}=\pdv{W}{\phi_{i}}
\end{align}
So if the auxiliary field $F_{i}$ acquires a vacuum expectation value $\langle F_{i}\rangle>0$, supersymmetry is spontaneously broken. This mechansism is known as F-term breaking. Another way of breaking supersymmetry spontaneously is by adding a term $\mathcal{L}\sim 2kV$, where $V$ is a vector superfield. If the auxiliary field $D(x)$ acquires a vacuum expectation value $\langle D\rangle >0$, we get what we call a D-term breaking.

Unfortunately, these procedures makes phenomenologically uncacceptable predicitons where at least one of the superpartners are lighter than the Standard model partner. The Standard model is gauge anomaly free, which is a feature we want a supersymmetric theory to inherit. At tree level one show that the weighted sum over tree-level squared-mass eigenvalues, known as the $\emph{supertrace}$, vanish in theories with non-anomolous gauge symmetries. The supertrace is given by
\begin{align}
    \mathcal{S}\text{Tr}[m^{2}]=\sum_{j}(-1)^{2j}(2j+1)\text{Tr}(m_{j}^{2})
\end{align}
where $m^{2}$ is the total squared mass matrix of the Lagrangian, $j$ is the spin of the particles and $m_{j}^{2}$ is the squared mass matrix for the spin-$j$ particles. If we have a theory with only scalar superfields, yielding two fermionic and two bosonic degress of freedom each we have that
\begin{align}
    \text{Tr}[(m_{s}^{2}-2m_{f}^{2})]=0
\end{align}
with the consequence that not all scalar partners can be heavier than the known fermions. A further complication is that the renormalizable supersymmetric Lagrangian \cref{eq:component superLagrangian}, does not contain any Yukawa terms $\phi\lambda\lambda$ that turn into mass terms for the gauginos if the scalar field acquires a vacuum expectation value. Therefore, to spontaneously break supersymmetry using fields that are coupled at tree level to the supermultiplets of the known particles seems not to work. In order to circumvent these difficulties, one generally assume that there is a sector at high energy scale that are hidden and not, or minimally, charged under $G_{SM}$. Then, hidden sector fields can acquire vacuum expectation values that spontaneously breaks supersymmetry, and the effects of supersymmetry breaking can be mediated from the hidden sector to the visible sector through non-renormalizable interactions or loop-processes. Through this process phenomenologically possible superpartner masses can be generated. The most studied frameworks for spontaneous supersymmetry breaking are \emph{Planck-scale-mediated supersymmetry breaking} (PMSB) and \emph{Gauge-mediated supersymmetry breaking} (GMSB), but as we will take another approach we refer to \cite{Martin:1997ns} for more details.

\subsection{Soft Supersymmetry Breaking}
Another popular approach is to not care about any specific model dependent breaking mechanism, but add terms to the Lagrangian that explicitly break supersymmetry, and treat the coefficients of these terms as free model parameters. However, we can not just simply add any supersymmetry-breaking term to the Lagrangian, with the risk of spoiling all the attractive features of low-scale supersymmetry. Terms that we do not want to add are the terms that can lead to quadratic divergences to scalar masses, and of course no terms that breaks renormalizability. Therefore, we add so called \emph{soft}-breaking terms where the total mass dimension for the interacting terms are at most 3. Ignoring all \textquote{maybe-soft} terms (ref to Are), the allowed terms can be written in terms of their component fields as
\begin{align}
    \mathcal{L}_{soft}=&-\frac{1}{2}M\lambda^{a}\lambda^{a}-m_{ij}^{2}\phi_{i}^{*}\phi_{j}
    \\
    &-t_{i}\phi_{i}-\frac{1}{2}b_{ij}\phi_{i}\phi_{j}-\frac{1}{6}a_{ijk}\phi_{i}\phi_{j}\phi_{k}+h.c
\end{align}
Where $M$ is repeated for each gauge group, and will give mass to the gauginos, while $m_{ij}^{2}$ and $b_{ij}$ will give additional scalar mass terms. If one of the scalar field in the trilinear term acquires a vacuum expectation value, $a_{ijk}$ will contribute to a mass terms for the remaining fields. The tadpole term with parameter $t_{i}$ is only allowed if $\phi_{i}$ is a gauge singlet. In total, the soft breaking terms give masses to both the scalar and fermionic superpartners of the Standard model particles.

\subsection{Minimal Supersymmetric Standard Model}
The Minimal Supersymmetric Standard Model (MSSM) is a supersymmetric extension of the Standard Model containing a minimal field content consistent with the Standard Model. We want to construct the MSSM on the same premiss as the SM, which is a theory based on $SU(3)_{C}\times SU(2)_{L}\times U(1)_{Y}$, and therefore all the known particles must be placed in appropriate supermultiplets. Standard model fermion fields and the scalar Higgs field are contained in chiral supermultiplets, and the Standard model gauge fields are contained in vector supermultiplets. In order to construct Dirac fermions we need to use a left chiral superfield and a \emph{different} right chiral superfield, which forms two fermions-particle and antiparticle-and four scalar particles-a pair of left and right handed scalar particles, and their antiparticles.

\subsection{Leptons}
For the left-handed leptons, the supermultiplets are contained in $SU(2)_{L}$ doublets, and the right-handed leptons are contained in $SU(2)_{L}$ singlets,
\begin{align}
    L_{i}=\begin{pmatrix}
            \nu_i\\
            l_i
\end{pmatrix}\hspace{3mm}\text{and}\hspace{3mm}\bar{e}_{i}
\end{align}
Here $i$ runs over the three generations og the Standard Model. The superfields $l_i$ and $\bar{e}_i$ combine to give charged leptons and sleptons. $\nu_{i}$ give the left-handed neutrinos and sneutrinos.

\subsection{Quarks}
We place the left-handed quarks in $SU(2)_{L}$ doublets, and the right handed in $SU(2)_{L}$ singlets,
\begin{align}
    Q_{i}=\begin{pmatrix}
            u_i\\
            d_i
\end{pmatrix}\hspace{3mm}\text{and}\hspace{3mm}\bar{u}_{i},\bar{d}_{i}
\end{align}
Colour indices are omitted. Here, the superfields $u_{i}$ and $\bar{u}_{i}$ combine to give the up-type quarks and squarks, and similarly the superfields $d_{i}$ and $\bar{d}_{i}$ combine to give the down-type quarks and squarks.

\subsection{Gauge Bosons}
The gauge bosons are represented by using vector supermultiplets. In a vector supermultiplet we have one massless vector boson and two Weyl spinors of each handedness, yielding two fermionic and two bosonic degrees of freedom. One vector superfield is needed per generator of the algebra, and these are denoted
\begin{align}
    W^{a},C^{a},B^{0}
\end{align}
From these one construct the spin-1 gauge bosons of the Standard Model $g$, $W^{+}$,$W^{-}$ and $B^{0}$. The fermionic superpartner of the gluon $g$, are called gluino $\tilde{g}$, and the fermionic partners of the Electroweak gauge bosons are $\tilde{W}^{0}$, $\tilde{W}^{+}$. $\tilde{W}^{-}$ and $\tilde{B}^{0}$, named Winos and Bino.

\subsection{Higgs boson}
The Higgs particle is a scalar, so it must belong to a chiral supermultiplet. The supersymmetric version of the Standard Model Higgs $SU(2)_{L}$ doublet would mix left chiral and right chiral superfields in order to give masses to all fermions, and therefore it can not appear in the superpotential. We therefore need two Higgs doublets $H_{u}$ and $H_{d}$, indexed according to the quarks they give masses to. Further, we need two Higgs doublets to make the MSSM free of gauge anomalies. The superfield doublets are defined as
\begin{align}
    H_{u}=\begin{pmatrix}
            H_{u}^{+}\\
            H_{u}^{0}
\end{pmatrix}\hspace{5mm}H_{u}=\begin{pmatrix}
            H_{d}^{0}\\
            H_{d}^{-}
\end{pmatrix}
\end{align}
The sign indicate the electric charge. In total these doublets contain four Weyl spinors and eight bosonic degrees of freedom. The Weyl spinors combine to make \emph{higgsinos} , three bosonic degrees of freedom are used to give masses to the $W^{\pm}$ and $Z^{0}$. The remaining five degrees of freedom give rise to five scalar mass eigenstates $h^{0}$, $H^{0}$, $A^{0}$, and $H^{\pm}$.

\subsection{MSSM Lagrangian}
The MSSM Lagrangian can now be constructed from the above superfields, and consists of the following parts
\begin{align}
    \mathcal{L}_{MSSM}=\mathcal{L}_{kin}+\mathcal{L}_{V}+\mathcal{L}_{W}+\mathcal{L}_{soft}
\end{align}
The kinetic part takes the following form
\begin{align}
    \mathcal{L}_{kin}=&L_{i}^{\dagger}e^{W-B}L_{i}+Q_{i}^{\dagger}e^{C+W+\frac{1}{3}B}Q_{i}
    \\
    &+\bar{u}_{i}^{\dagger}e^{C-\frac{4}{3}B}u_{i}+\bar{d}_{i}^{\dagger}e^{C+\frac{2}{3}B}d_{i}+\bar{e}_{i}^{\dagger}e^{2B}e_{i}
    \\
    &+H_{u}^{\dagger}e^{W+B}H_{u}+H_{d}^{\dagger}e^{W-B}H_{d}
\end{align}
where
\begin{align}
    W&=\frac{1}{2}g\sigma^{a}W^{a}
    \\
    C&=\frac{1}{2}g_{s}\lambda^{a}C^{a}
    \\
    B&=\frac{1}{2}g'B^{0}
\end{align}
with $g',g$ and $g_s$ the coupling of $U(1)_{Y}$, $SU(2)_{L}$ and $SU(3)_{C}$ gauge groups. The supersymmetric field strength contribution is given by
\begin{align}
    \mathcal{L}_{V}=\frac{1}{2}\bar{\theta}\bar{\theta}\text{Tr}[W^{A}W_{A}]+\frac{1}{2}\bar{\theta}\bar{\theta}\text{Tr}[C^{A}C_{A}]+\frac{1}{4}\bar{\theta}\bar{\theta}B^{A}B_{A}+c.c
\end{align}
with the field strengths
\begin{align}
    W_{A}&=-\frac{1}{4}\bar{D}\bar{D}e^{-W}D_{A}e^{W}
    \\
    C_{A}&=-\frac{1}{4}\bar{D}\bar{D}e^{-C}D_{A}e^{C}
    \\
    B_{A}&=-\frac{1}{4}\bar{D}\bar{D}D_{A}B
\end{align}
The possible gauge invariant terms in the superpotential are
\begin{align}
    W=&\mu H_{u}H_{d}+y_{ij}^{e}L_{i}H_{d}\bar{E}_{j}+y_{ij}^{u}Q_{i}H_{u}\bar{U}_{j}+y_{ij}^{d}Q_{i}H_{d}\bar{D}_{j}
    \\
    &\mu'L_{i}H_{u}+\lambda_{ijk}L_{i}L_{j}\bar{E}_{k}+\lambda_{ijk}^{'}L_{i}Q_{j}\bar{D}_{k}+\lambda_{ijk}^{''}\bar{U}_{i}\bar{D}_{j}\bar{D}_{k}
\end{align}
The terms in the last line all violate lepton and baryon number. These violations are under strict restrictions from experiments, such as the search of proton decay $p\rightarrow e^{+}\pi^{0}$, which has not been observed. In order to circumvent these terms, a new multiplicative conserved quantity can be introduced, \emph{R-parity}. This is defined by
\begin{align}
    P_{R}=(-1)^{3(B-L)+2s}
\end{align}
where $s$ is spin, B is baryon number and L is lepton number. SM particles has R-parity of $+1$ and sparticles has R-parity of $-1$. The conservation of this quantity has the consequence that
\begin{enumerate}
    \item All sparticles must be produced in pairs
    \item The lightest supersymmetric particle (LSP) is absolutely stable
    \item Every other sparticle must decay down to the LSP
\end{enumerate}

\subsection{Radiative Electroweak Symmetry Breaking}

\subsection{Gluinos and Squarks}                

%%%%%%%%%%%%% QCD %%%%%%%%%%%%%%%%%%%%%%%%
\chapter{Perturbative Quantum Chromodynamics}\label{Chap:pQCD}
In this chapter we take a closer look at QCD as the qunatum field theory of the strong interaction. We begin by introducing the Lagrangian and its constituents leading up to a discussion on the notion of asymptotic freedom and the running of the strong coupling. Then the process of DIS is explored, where important concepts as factorization and parton distribution functions are introduced. An operator valued expression for the parton distributions is derived and we show how Wilson lines can be used to render these gauge invariant. We also introduce perturbative parton-in-parton distributions by dressing them with Wilson lines. Thereafter, we investigate the Drell-Yan process and calculate a NLO quark-antiquark annihilation process. We show that even after renormalization and regularization of this cross section that there are certain regions of phase space where it gain large logarithmic corrections.

\section{Field Theoretical Description}
In \cref{chap:Geometry of gauge theories} we introduced a geometric formulation behind non-Abelian gauge theories. We are now ready to build on this formalism and introduce the non-Abelian gauge theory of the strong interaction, namely Quantum Chromodynamics. It is formulated in terms of quark and gluon fields, where quarks are spin 1/2-fermions and gluons are spin-1 gauge bosons. The symmetry group is $SU(3)_{c}$, meaning that the fields carry a quantum number which we call colour. 

The quark fields live in the fundamental representation of the gauge group, and is represented as a triplet in colour space
\begin{align}
    \psi(x)= \begin{pmatrix}
           \psi_{1}(x) \\
           \psi_{2}(x) \\
           \psi_{3}(x)
         \end{pmatrix}\,,
\end{align}
which we know transform under local gauge transformations as
\begin{align}
    \psi(x)\rightarrow U(x)\psi(x)\,,
\end{align}
where
\begin{align}
    U(x)=e^{i\alpha^{a}(x)t^{a}}\,.
\end{align}
The generators $t^{a}$ are $3\times 3$ hermitian matrices, where the hermiticity follows by insisting that the group matrices $U$ are unitary, while keeping the group parameters $\alpha^{a}(x)$ real. The generators of $SU(3)$ are traceless, and the group index runs over eight values $a=1,\dots,3^{2}-1=8$. This has the implication that there are three different coloured quarks and there are eight gluons. However, experimentally we know that there are three families of quarks, which are represented as doublets
\begin{align}
    \begin{pmatrix}
           u \\
           d \\
         \end{pmatrix}\,,
         \begin{pmatrix}
           c \\
           s \\
         \end{pmatrix}\,,
         \begin{pmatrix}
           t \\
           b \\
         \end{pmatrix}\,,
\end{align}
giving that there are in total 36 different quarks. In a scattering calculations we will not explicitly specify whether we are dealing with an up-quark or a down-quark, just regard them as fermions with fractional electric charge and remember to sum over all of them. We will mostly work in the high energy limit where we use that the quarks are massless, and then we tend to ignore the heaviest generation of the top and bottom quarks. 

From \cref{eq:Lie commutator} we know that the generators satisfy the commutation relation
\begin{align}
    [t^{a},t^{b}]=if^{abc}t^{c}\,,
\end{align}
in which the structure constants are real and antisymmetric in the indices $a,b$ and $c$. The normalization of the structure constants fixes the trace and Casimir invariants for any representation $R$, i.e.
\begin{align}
    \text{tr}(t^{a}t^{b})=C(R)\delta^{ab}\,,\hspace{1cm}\sum_{a}t^{a}t^{a}=C_{2}(R)\,.
\end{align}
For the fundamental representation of the quarks we have that
\begin{align}
    C(\text{fund})&=T_{F}=\frac{1}{2}
    \\
    C_{2}(\text{fund})&=C_{F}=\frac{4}{3}\,,
\end{align}
and for the adjoint representation of the gluons we have that
\begin{align}
    C(\text{adj})&=C_{2}(\text{adj})=C_{A}=3\,.
\end{align}

The classical QCD Lagrangian follows directly from the Yang-Mills Lagrangian \cref{eq:Yang-Mills Lagrangian}, and the quantized from the Fadeev-Popov Lagrangian \cref{eq:Fadeev-Popov Lagrangian}. Using axial gauge we have that the gauge fixed QCD-Lagrangian is given by
\begin{align}
     \mathcal{L}_{QCD}=-\frac{1}{4}(F_{\mu\nu}^{a})^{2}+\Bar{\psi}(i\slashed{D}-m)\psi-\frac{1}{2\xi}(n^{\mu}A_{\mu}^{a})^{2}\,.
\end{align}
where it is understood that the quark fields carry both flavour index and colour index, i.e. $\psi=(\psi_{\alpha\,i})$ where $\alpha=u,d,s,c,b,t$ and $i=1,2,3$. Expanding this Lagrangian will give all Feynman rules needed for calculating Feynman diagrams. The expansion is straightforward using \cref{eq:field strenght in wilson chapter}, but tedious. We can group the different parts in the following way\footnote{These are the bare fields, but since there are too many subscripts and superscripts already we will not write this out explicitly.}
\begin{align}\label{eq:QCD splitted LAgrangian}
    \mathcal{L}_{QCD}=\mathcal{L}_{\text{GF}}+\mathcal{L}_{\text{Dirac}}+\mathcal{L}_{\text{int}}\,,
\end{align}
where
\begin{align}
    \mathcal{L}_{\text{GF}}&=-\frac{1}{4}\big(\partial_{\mu}A_{\nu}^{a}-\partial_{\nu}A_{\mu}^{a}\big)^{2}-\frac{1}{2\xi}(n^{\mu}A_{\mu}^{a})^{2}\,,
    \\
    \mathcal{L}_{\text{Dirac}}&=\Bar{\psi}(i\slashed{\partial}-m_{0})\psi\,,
    \\
    \mathcal{L}_{\text{int}}=&g_{0}\Bar{\psi}\slashed{A}^{a}t^{a}\psi-g_{0}f^{abc}(\partial_{\mu}A_{\nu}^{a})A^{\mu\,b}A^{\nu\,c}\nonumber
    \\
    &-\frac{1}{2}g_{0}^{2}f^{abe}f^{ecd}A_{\mu}^{a}A_{\nu}^{b}A^{\mu\,c}A^{\nu\,d}\,.\label{eq:QCD interaction Lagrangian}
\end{align}
The first term of \cref{eq:QCD interaction Lagrangian} describes the interaction between quarks and gluons, the second term the interaction between three gluons and the fourth term between four gluons. These terms give rise to the following vertex rules\footnote{In the three gluon vertex all momenta are pointing towards the vertex.}

\begin{fmffile}{qqg}
\begin{align}
\begin{gathered}
\begin{fmfgraph*}(80,50)
\fmfleft{i1,i2}
\fmfright{o1}
\fmfv{label=$a\hspace{0.1cm}\mu$}{o1}
\fmf{fermion,tension=1/3}{i1,v1}
\fmf{plain}{v1,v2}
\fmf{fermion}{v2,v3}
\fmf{plain}{v3,i2}
\fmf{gluon}{v1,o1}
\end{fmfgraph*}
\end{gathered}\hspace{1cm}&=ig\gamma^{\mu}t^{a}\,,\label{eq:quark-gluon vertex}
\\\nonumber\\
\begin{gathered}
\begin{fmfgraph*}(80,50)
\fmfleft{i1,i2}
\fmfright{o1}
\fmfv{label=$a\hspace{0.1cm}\mu$}{i2}
\fmfv{label=$b\hspace{0.1cm}\nu$}{i1}
\fmfv{label=$c\hspace{0.1cm}\rho$}{o1}
\fmf{gluon,label=$p$, l.side=left,tension=1/3}{i1,v1}
\fmf{gluon,label=$k$, l.side=left,tension=1/3}{i2,v1}
\fmf{gluon,label=$q$, l.side=left}{v1,o1}
\end{fmfgraph*}
\end{gathered}\hspace{1cm}&=gf^{abc}\big(g^{\mu\nu}(k-p)^{\rho}+g^{\nu\rho}(p-q)^{\mu}+g^{\rho\mu}(q-k)^{\nu}\big)\,,\label{eq:three gluon vertex}
\\\nonumber\\\nonumber\\
\begin{gathered}
\begin{fmfgraph*}(80,50)
\fmfleft{i1,i2}
\fmfright{o1,o2}
\fmfv{label=$a\hspace{0.1cm}\mu$}{i2}
\fmfv{label=$c\hspace{0.1cm}\rho$}{i1}
\fmfv{label=$d\hspace{0.1cm}\sigma$}{o1}
\fmfv{label=$b\hspace{0.1cm}\nu$}{o2}
\fmf{gluon,tension=1/3}{i1,v1}
\fmf{gluon,tension=1/3}{i2,v1}
\fmf{gluon,tension=1/3}{v1,o1}
\fmf{gluon,tension=1/3}{v1,o2}
\end{fmfgraph*}
\end{gathered}\hspace{1cm}&=-ig^{2}\big[f^{abc}f^{cde}(g^{\mu\rho}g^{\nu\sigma}-g^{\mu\sigma}g^{\nu\rho})+f^{ace}f^{bde}(g^{\mu\nu}g^{\rho\sigma}-g^{\mu\sigma}g^{\nu\rho})\nonumber\\&\hspace{1cm}+f^{ade}f^{bce}(g^{\mu\nu}g^{\rho\sigma}-g^{\mu\rho}g^{\nu\sigma})\big]\,.\label{eq:four gluon vertex}\\\nonumber
\end{align}
\end{fmffile}
The propagators for fermions and gauge bosons can be found in e.g. \cref{eq:canocical fermion propagator} and \cref{eq:covariant propagator}.

The missing piece to this Lagrangian is to renormalize it by rescaling the fields and define counterterms that remove all UV-divergences. We have already seen in \cref{sec:renormalized perturbation theory} how this works for a scalar theory. We will use that the exact same procedure applies for QCD, with the obvious difference that there are several fields and a more intricate relation between the parameters and the renormalization constants. We rescale the bare fields in the usual way 
\begin{align}
    A^{\mu}&\rightarrow\mathcal{Z}_{3}^{1/2}A^{\mu}\,,\label{eq:rescaled gauge field QCD}
    \\
    \psi&\rightarrow\mathcal{Z}_{2}^{1/2}\psi\,,
\end{align}
where we need different renormalization constants for different fields. We have neglected all subscript separating bare and renormalized fields, but keep in mind that with this rescaling it is the renormalized fields that appear in the \textquote{new} Lagrangian. We can now define the counterterms
\begin{align}\label{eq:counterterms}
    \delta_{1}&=Z_{1}-1\,,\hspace{1cm}\delta_{2}=\mathcal{Z}_{2}-1\,,\hspace{1cm}\delta_{3}=\mathcal{Z}_{3}-1\,,\hspace{1cm}
    \\\vspace{0.2cm}
    \delta_{m}&=\mathcal{Z}_{2}m_{0}-m\,,\hspace{1cm}\delta_{A^{3}}=\mathcal{Z}_{A^{3}}-1\,,\hspace{1cm}\delta_{A^{4}}=\mathcal{Z}_{A^{4}}-1\,,
\end{align}
where we defined
\begin{align}\label{eq:relation g and g0}
    \mathcal{Z}_{1}g&=\mathcal{Z}_{2}\mathcal{Z}_{3}^{1/2}g_0\,,\hspace{1cm}\mathcal{Z}_{A^{3}}g=\mathcal{Z}_{3}^{3/2}g_{0}\,,\hspace{1cm}\mathcal{Z}_{A^{4}}g^{2}=\mathcal{Z}_{3}^{2}g_{0}^{2}\,,
\end{align}
such that the Lagrangian takes the form
\begin{align}
    \mathcal{L}=\mathcal{L}_{\text{R}}+\mathcal{L}_{\text{CT}}\,,
\end{align}
where the first part is identical to \cref{eq:QCD splitted LAgrangian}, but with the renormalized fields and parameters. The counterterm part is given by
\begin{align}\label{eq:QCD counterterm Lagrangian}
    \mathcal{L}_{\text{CT}}=&-\frac{1}{4}\delta_{3}\big(\partial_{\mu}A_{\nu}^{a}-\partial_{\nu}A_{\mu}^{a}\big)^{2}-\frac{1}{2\xi}\delta_{3}(n^{\mu}A_{\mu}^{a})^{2}+\delta_{2}\Bar{\psi}(i\slashed{\partial}-\delta_{m})\psi\,,
    \\
    &+g\delta_{1}\Bar{\psi}\slashed{A}^{a}t^{a}\psi-g\delta_{A^{3}}f^{abc}(\partial_{\mu}A_{\nu}^{a})A^{\mu\,b}A^{\nu\,c}\nonumber
    \\
    &-\frac{1}{2}g^{2}\delta_{A^{4}}f^{abe}f^{ecd}A_{\mu}^{a}A_{\nu}^{b}A^{\mu\,c}A^{\nu\,d}\,.
\end{align}

The definition of the counterterms only makes sense if one gives a precise definition for the physical mass and coupling, i.e. one has to define renormalization conditions to put constraints on the counterterms. For example, the physical mass of the quarks are defined as the pole of the quark propagator at all orders, just like we did for $\phi^{4}$-theory. 

The notion of running coupling in non-Abelian gauge theories is different than in Abelian gauge theories or scalar theories. The concept of changing with scale is not different, but how it changes with scale. In \cref{eq:solution of beta function} we found that the coupling in $\phi^{4}$-theory increases with the change of scale. This is also true for an Abelian gauge theory like QED, but for a non-Abelian gauge theory like QCD this is no longer true. In QCD the coupling is large in the low energy regime, and perturbation theory breaks down, while it becomes smaller at higher energies. This phenomena is known as \emph{asymptotic freedom} and is only present in non-Abelian gauge theories. To see how this comes about, we have to look at the beta function of the theory. The beta function in QCD is found by calculating the quark self-energy (giving $\mathcal{Z}_2$), corrections to the gluon propagator (giving $\mathcal{Z}_3$) and corrections to the vertex function (giving $\mathcal{Z}_1$). This is a very long and tedious calculation, so we will not perform it explicitly here, but instead state the results in order to understand the behaviour of the strong coupling.

In \cref{eq:relation g and g0} the relation between the bare coupling and renormalized coupling is given by
\begin{align}
    g_0&=\frac{\mathcal{Z}_{1}}{\mathcal{Z}_{2}\mathcal{Z}_{3}^{1/2}}g\mu^{\epsilon}\,,
\end{align}
where we have inserted $\mu^{\epsilon}$ to make the coupling dimensionless in dimensional regularization. We can now use that the bare coupling is independent on $\mu$, giving
\begin{align}
    0=\mu\dv{g_0}{\mu}=\mu\dv{}{\mu}\Big[\frac{\mathcal{Z}_{1}}{\mathcal{Z}_{2}\mathcal{Z}_{3}^{1/2}}g\mu^{\epsilon}\Big]\,,
\end{align}
and performing the differentiation will give the differential equation
\begin{align}
    \mu\dv{g}{\mu}=-\epsilon g-g\Big[\frac{\mu}{\mathcal{Z}_1}\dv{\mathcal{Z}_1}{\mu}-\frac{\mu}{\mathcal{Z}_2}\dv{\mathcal{Z}_2}{\mu}-\frac{1}{2}\frac{\mu}{\mathcal{Z}_3}\dv{\mathcal{Z}_3}{\mu}\Big]\,.
\end{align}
The $\mu$ dependence on $\mathcal{Z}_{i}$ is only through the coupling $g$, so we can write
\begin{align}
    \frac{\mu}{\mathcal{Z}_i}\dv{\mathcal{Z}_i}{\mu}=\frac{1}{\mathcal{Z}_{i}}\pdv{\mathcal{Z}_{i}}{\mu}\mu\pdv{g}{\mu}=\frac{1}{\mathcal{Z}_{i}}\pdv{\mathcal{Z}_{i}}{\mu}\beta\,,
\end{align}
and use that $\mathcal{Z}_{i}=1+\delta_{i}$, where the relevant counterterms can be found in \cite{Schwartz:2013pla}\footnote{They use a different convention in dimensional regularization, with $d=4-\epsilon$. The difference is just a factor of two, so we adjust the answer to suit our convention.},
\begin{align}
    \delta_{1}&=-\frac{1}{\epsilon}\frac{g^{2}}{(4\pi)^{2}}(C_{F}+C_{A})
    \\
    \delta_{2}&=-\frac{1}{\epsilon}\frac{g^{2}}{(4\pi)^{2}}C_{F}
    \\
    \delta_{3}&=\frac{1}{\epsilon}\frac{g^{2}}{(4\pi)^{2}}\big(\frac{5}{3}C_{A}-\frac{4}{3}n_{f}T_{F}\big)\,,
\end{align}
where $n_f$ is the number of flavours. We observe that the counterterms are defined at $\mathcal{O}(g^{2})$. We are only interested in the one-loop correction and since all one-loop diagrams have $\mathcal{O}(g^{3})$, we can write
\begin{align}
    \frac{1}{\mathcal{Z}_{i}}\pdv{\mathcal{Z}_{i}}{\mu}=\pdv{\delta_{i}}{g}+\cdots\,,
\end{align}
and use that
\begin{align}
    \beta=-\epsilon g+\mathcal{O}(g^{2})\,,
\end{align}
leading to the following differential equation
\begin{align}\label{eq:beta one-loop}
    \mu\dv{g}{\mu}=-\epsilon g+\epsilon g^{2}\pdv{}{g}\big(\delta_{1}-\delta_{2}-\frac{1}{2}\delta_{3}\big)=-\epsilon g-\frac{g^{3}}{(4\pi)^{2}}\big(\frac{11}{3}C_{A}-\frac{4}{3}n_{f}T_{F}\big)\,.
\end{align}
We can safely send $\epsilon\rightarrow 0$, and define
\begin{align}\label{eq:zeroth order beta function}
    \beta_{0}=\frac{11}{3}C_{A}-\frac{4}{3}n_{f}T_{F}\,.
\end{align}

The solution to \cref{eq:beta one-loop} can be found by using separation of variables, giving
\begin{align}\label{eq: g running coupling one-loop}
    g^{2}(\mu)=\frac{g^{2}(\mu_0)}{1+\frac{g^{2}(\mu_{0})}{(4\pi)^{2}}\beta_{0}\ln(\frac{\mu^{2}}{\mu_{0}^{2}})}\,.
\end{align}
In scattering observables we will mostly use that $\alpha_s=g^{2}/4\pi$, so we can write \cref{eq: g running coupling one-loop} as
\begin{align}\label{eq:one-loop strong coupling}
    \alpha_{s}(\mu)=\frac{\alpha_{s}(\mu_0)}{1+\frac{\alpha_{s}(\mu_0)}{4\pi}\beta_0\ln(\frac{\mu^{2}}{\mu_{0}^{2}})}\,,
\end{align}
which is the well known one-loop coupling of QCD. The crucial point here is that: as long as $n_f<17$, $\beta_0>0$, and the coupling decreases as a function of increasing $\mu$. We know that there are six flavours of quarks, so QCD is an asymptotically free theory. This discovery was made independetly by D.Pulitzer, F.Wilczek and D.Gross, for which they received the 2004 Nobel Prize in physics.

The notion of asymptotic freedom is very counter-intuitive, as it implies that particles at an infinitesimally small separation do not attract each other (colourwise). There is a metaphor for this kind of behaviour, which is also useful to explain what we call \emph{colour confinement}. Imagine we have two particles connected by a rubber band. If we move the particles closer and closer each other, the rubber band will relax and nothing happens. The moment we try to move the particles apart, the tension in the band will increase, and the more we pull trying to separate the particles, the stronger the tension. Hence, it can work as a metaphor for why coloured particles has never been seen to exist as free states, i.e. they are confined to exist as colour neutral composite particles. We can actually extend the analogy even further. Imagine that we stretch the rubber band with such force that it snaps, i.e. we have succeeded in separating them. However, the amount of energy needed to snap the gluon binding energy is so large that a particle-antiparticle pair is created and binds with the particles we try to separate, giving new colour neutral states. However, this is only an analogy as confinement is still not theoretically understood.

The closest attempt to understand and explain confinement is due to Kenneth Wilson by the use of Wilson loops on the lattice. It is well known that the attempt to define a expectation value of a QCD current describing the potential between two colour charges fails. This procedure works in QED, and can be shown to be equivalent to calculating the beta function, i.e. it gives a measure of the running of the coupling. The problem with this approach in QCD is that it is not gauge invariant. This led Kenneth Wilson to postulate that this potential could be described by instead defining the expectation value of a Wilson loop on the lattice. Wilson's idea was that if the expectation value of a Wilson loop could be shown to be proportional to the area of the Wilson loop, it would indicate that the potential between colour charges grew with distance. Wilson was able to show analytically that on the lattice the expectation value of a Wilson loop scales as the area of the loop, which indicates that confinement is present in QCD \cite{Wilson:74}. The problem is that Wilson's arguments is valid for any gauge theory, but QED does most certainly not have this behaviour. There is also a problem with preserving confinement in the continuum limit, i.e. when the lattice spacing is removed. However, there has been a lot of progress in this field the last decades, and that is due to the development of generalized loops and loop calculus, see e.g. \cite{TAVARES_1994,MIGDAL1983199,KORCHEMSKY1986459}. We will not pursue confinement and the low energy regime any further, but it is worth mentioning that Wilson loops/lines are at the centre of both perturbative and non-perturbative QCD. 

Apart from small coupling at large energies, the other consequence of asymptotic freedom is that at a given energy scale, the coupling becomes large and invalidates perturbation theory. In theories without asymptotic freedom, like QED, this scale is so large that it is not relevant to consider, but for QCD we have that for $\Lambda_{\text{QCD}}\approx 250 $MeV the coupling becomes larger than one. The consequence of this is that we can not use perturbation theory to understand the low energy behaviour of bound states. This separates QCD into two regimes; a part where $\mu>\Lambda_{\text{QCD}}$ where perturbation theory is valid and a region where we can not calculate. This latter part needs to be extracted from experiment, and this is the basis for the \emph{factorization} framework we will investigate in more detail in this chapter. 
\section{Deep Inelastic Scattering}\label{sec:DIS}
We have given a somewhat formal introduction of Quantum Chromodynamics as the theory of strong interactions. However, as this description only applies to the fundamental constituents, namely the quarks and gluons, we need to connect this formalism to scattering amplitudes involving particles that are composited of quarks and gluons. These composited particles are what particle physicists call \emph{hadrons}. Hadrons are categorized into two families: baryons, made of an odd number of quarks – usually three quarks – and mesons, made of an even number of quarks—usually one quark and one antiquark. Protons and neutrons are examples of baryons and pions are an example of a meson. For more detail on the classification of particles, see \cite{PhysRevD.98.030001}.

\medskip
As a non-abelian gauge theory, QCD has the peculiar behaviour of asymptotic freedom, and as described above, this is due to the anti-screening of gluons. As the coupling is \textquote{small} at high energies, the intuitive approach is to use perturbation theory. However, before that can be accomplished in a meaningful way, we have to define exactly what we are to expand.

Asymptotic freedom effectively separates the strong interaction into two regimes: A perturbative regime -- $\alpha_s$ is smaller than unity -- where we can expand using standard field theory methods, and a non-perturbative regime where we are unable to calculate. Unfortunately, we can not ignore the non-perturbative regime in real-life scattering processes. Thus, we need a method where we can unite both regimes into useful observables. To achieve this we need the concept of \emph{factorization}, which allows for separation of the low and high energy regime. This separation is made at an arbitrary energy scale, called the \emph{factorization scale}, and just like the renormalization scale observables can not depend on it. This naturally leads to evolution equations that describe the behaviour of the non-perturbative part as a function of the energy scale.

\medskip
The original, and still one of the most powerful, test of perturbative QCD is the Bjorken scaling in \emph{deep inelastic scattering} (DIS). Here a hadron is probed by a highly relativistic lepton, breaking up the hadron, creating additional hadrons in the final state. Only the lepton needs to be measured in the final state, meaning the final state hadrons can be integrated out. At sufficiently high energies the DIS experiments indicated the lepton scattered of point-like particles. This led Richard Feynman to postulate that hadrons were composite objects made up of fundamental constituents, which he called partons. This was the birth of what is called the \emph{parton model} \cite{Feynman:1970fm}. 

James Bjorken was the first to formalise the parton model in DIS \cite{PhysRev.179.1547}, where he was able to predict the behaviour of the hadronic system without any fundamental Lagrangian or knowledge of the hadron structure. He parametrized the process using so-called \emph{structure functions}. In the high energy limit, he showed that these structure functions only depended on a dimensionless scaling variable. 

In \cref{sec:DIS and Parton model} we will take a closer look at this parametrization and via a calculation verify Bjorken scaling in the parton model before we in \cref{sec:QCD and Collinear factorization} include QCD effects and move on to the more precise concept of factorization. 

%%%%%%%%%%%%%%%%%%%%%%%%%%%%%%%%%%%%%%%%%%%
\subsection{DIS and the Parton Model}\label{sec:DIS and Parton model}
Protons are in the parton model envisioned as extended objects, made up of partons and glued together by their mutual interactions. Of course, the partons are the quarks and gluons of quantum chromodynamics, but we will proceed without referring to this fact just yet. 

The essential assumption of the parton model is that at high energies and momentum transfer, the electron scatters of \textquote{free} point-like particles. This seems like a strange assumption, as the strong interaction between the constituents is what make protons bound objects in the first place. To justify this assumption, we can use basic principles from special relativity. We consider the inclusive process of electron--proton scattering at high energy, where the interaction goes through an exchange of a virtual photon with momentum $q$. In the centre-of-mass frame, two key concepts happen to the proton: since it is ultra-relativistic the proton is Lorentz contracted in the direction of travel, and the internal interactions are time-dilated. 

Therefore, if the time the electron uses to traverse the proton is shorter than the dilated time of interaction, the electron effectively scatters of non-interacting partons. This is known as the impulse approximation. Also, when the momentum transfer is very high, the virtual photon becomes short-lived. Then, if the parton density is sufficiently low, the electron can only interact with one single parton. Since the partons do not interact between themselves during this period, each carries a definite fraction $\xi$ of the proton's momentum $P$. With this picture, it is possible to calculate the electron--parton interaction using perturbation theory without considering the proton as a whole.


The high energy inelastic process is therefore separated into a short-distance scattering part---the interaction between the photon and one of the partons---called the \emph{hard} part, and a long-distance---everything inside the proton---\emph{soft} part. Because of the different time scales of these two effects, one assumes that there is no quantum mechanical interference between them. The hadronic cross-section may thus be calculated by combining probabilities, rather than amplitudes. We define parton distribution functions $f_{i/h}(\xi)$ to describe the probability that a parton of type $i$ has a fraction $\xi$ of the hadrons momentum. 

\begin{figure}
    \centering
    \includegraphics[scale=0.4]{Figures/DIS.pdf}
    \caption{Kinematics of deep inelastic electron--proton scattering}
    \label{fig:Deep Inelastic Scattering}
\end{figure}
\medskip

\subsection*{Kinematics}
We consider the scattering of a high energy electron off a proton target, via the exchange of a virtual photon (see \cref{fig:Deep Inelastic Scattering}). The full treatment would involve weak interactions, but for our purposes we only consider the electromagnetic interaction. We lable the incoming and outgoing electron momenta with $l^{\mu}$ and $l'^{\mu}$ respectively, the momentum of the proton by $P^{\mu}$ and the momentum transfer by the photon $q^{\mu}=l^{\mu}-l'^{\mu}$. We will assume that the mass of the electron and the proton constituents are negligible compared to the scale of the process. The centre-of-mass energy squared is then
\begin{align}
    s=(P+l)^{2}=m_{p}^{2}+2P\cdot l\,,
\end{align}
where $m_{p}$ is the mass of the proton. It follows that the momentum transfer squared is given by
\begin{align}
    q^{2}=2E_{l}E_{l'}(\cos\theta_{ll'}-1)\leq 0\,.
\end{align}
Therefore, it is useful to instead define $Q^{2}\equiv-q^{2}\geq 0$. The invariant mass of the final state $X$ is given by
\begin{align}
    m_{X}^{2}=(P+q)^{2}=m_{p}^{2}+2P\cdot q-Q^{2}\,.
\end{align}
In order to have \emph{deep} and \emph{inelastic} scattering, we must have the requirement that $Q^{2}\gg m_{p}^{2}$---\,the momentum transfer is so large that the proton target is very much excited\,---\,and inelastically $m_{X}^{2}\gg m_{p}^{2}$. 

The two independent Lorentz invariants for the hadron system are $Q^{2}$ and $P\cdot q$, but it is convenient to define additional invariant variables
\begin{align}
    x_{B}&=\frac{Q^{2}}{2P\cdot q}\,,\label{eq:Bjorken-x}
    \\
    y&=\frac{P\cdot q}{P\cdot l}=\frac{Q^{2}}{x_{B}(s-m_{p}^{2})}\,.
\end{align}
Here $x_{B}$ is called \emph{Bjorken-x}, which we from now will only denote by $x$. Kinematically $x$ is restricted to the range $Q^{2}/s+Q^{2}\leq x \leq 1$, neglecting terms of $\mathcal{O}(m_{p}^{2}/Q^{2})$. In the parton model we will find that $x$ gives an estimate for the hadron's momentum fraction that is carried by the struck parton. In the rest frame of the proton $y$ is the fractional energy loss of the electron, but it is not an independent variable since it is given by $Q^{2}$ and $x$.



\subsection*{Factorization and Bjorken Scaling}
From the above picture, the parton model describes DIS without the strong interaction participating, as all strong effects have been absorbed into the proton. Consequently, the proton is like a black box, where we have no idea of its structure, except that we can extract a parton from it.

Before we write down the amplitude and differential cross-section for this process, we must define some relations for current matrix elements. A fermion current is $j^{\mu}(z)=\bar{\psi}(z)\gamma^{\mu}\psi(z)$, with $\psi(z)$ the fermion field. A generic current matrix element for a fermion is then given by
\begin{align}\label{eq:generic current matrix element}
    \bra{k'}j^{\mu}(z)\ket{k}=\Bar{u}(k')\gamma^{\mu}u(k)\,e^{i(k'-k)\cdot z}\,,
\end{align}
where we used $z$ as space-time variable in order to avoid confusion with the Bjorken-$x$. The relation in \cref{eq:generic current matrix element} can be derived using the expansion of the fermion fields in terms of creation and annihilation operators. Thus, a shorthand for the spinor product $\Bar{u}(k')\gamma^{\mu}u(k)$ coming out of matrix elements is just the current matrix element at space-time point $z=0$, i.e. 
\begin{align}
    \bra{k'}j^{\mu}(0)\ket{k}=\Bar{u}(k')\gamma^{\mu}u(k)\,.
\end{align}
Then we can write the amplitude for DIS as
\begin{align}
    \mathcal{M}=\bra{l'}\,j_{L}^{\mu}\,\ket{l}\bra{X}\,j_{H}^{\nu}\,\ket{P}\,D_{\mu\nu}(q)\,,
\end{align}
where $D_{\mu\nu}(q)$ is just the regular photon propagator, see \cref{eq:photon propagator without gauge choice}. The differential cross section then takes the form
\begin{align}
    d\sigma&=\frac{1}{4\,P\cdot l}\frac{d^{3}l'}{(2\pi)^{3}\,2E_{l'}}\sum_{X}\int\frac{d^{3}p_{X}}{(2\pi)^{3}\,2E_{X}}(2\pi)^{4}\delta^{(4)}(P+l-p_{X}-l)|\mathcal{M}|^{2}\nonumber\,,
\end{align}
giving
\begin{align}\label{eq:DIS differential cross section}
    E_{l'}\frac{d\sigma}{d^{3}l'}&=\frac{2}{s-m_{p}^{2}}\frac{\alpha^{2}}{Q^{4}}L_{\mu\nu}W^{\mu\nu}\,,
\end{align}
where $\alpha$ is the fine structure constant. As we only consider photon exchange, the lepton tensor is completely determined by QED
\begin{align}
    L_{\mu\nu}=\frac{1}{2}\text{tr}[\slashed{l'}\gamma_{\mu}\slashed{l}\gamma_{\nu}]=2\big(l_{\mu}l'_{\nu}+l_{\nu}l'_{\mu}-g_{\mu\nu}l\cdot l'\big)\,.
\end{align}
In contrast, the hadronic tensor contains all the information about the interaction between the electromagnetic current $j_{H}^{\mu}$ and the proton $P$
\begin{align}
    W^{\mu\nu}&=4\pi^{3}\sum_{X}\int\frac{d^{3}p_{X}}{(2\pi)^{3}\,2E_{X}}\delta^{(4)}(P+q-p_{X})\,\bra{P}\,j^{\dag\,\mu}(0)\,\ket{X}\bra{X}\,j^{\nu}(0)\,\ket{P}\nonumber
    \\
    &=\frac{1}{4\pi}\int d^{4}z\,\sum_{X}\int\frac{d^{3}p_{X}}{(2\pi)^{3}\,2E_{X}}\,e^{iz\cdot(P+q-p_{X})}\,\bra{P}\,j^{\dagger\,\mu}(0)\,\ket{X}\bra{X}\,j^{\nu}(0)\,\ket{P}\,,\nonumber
\end{align}
where we used the integral representation of the four dimensional delta function. Further, we can use the translation operator
\begin{align}\label{eq:translation operator}
    \,\bra{P}\,j^{\dagger\,\mu}(0)\,\ket{X}e^{iz\cdot(P-p_{X})}&=\,\bra{P}e^{iz\cdot\hat{P}}\,j^{\dagger\,\mu}(0)\,e^{-iz\cdot\hat{P}}\ket{X}\nonumber
    \\
    &=\bra{P}\,j^{\dagger\,\mu}(z)\,\ket{X}\,,
\end{align}
and integrate out a complete set of states by the use of the completeness relation:
\begin{align}\label{eq:complete set of states}
    \sum_{X}\int\frac{d^{3}p_{X}}{(2\pi)^{3}\,2E_{X}}\ket{X}\bra{X}=\boldsymbol{1}\,.
\end{align}
The hadronic tensor can then be written as
\begin{align}\label{eq:hadronic tensor in terms of currents}
    W^{\mu\nu}=\frac{1}{4\pi}\int d^{4}z\,e^{iq\cdot z}\bra{P}\,j^{\dagger\,\mu}(z)j^{\nu}(0)\ket{P}\,.
\end{align}

This tensor can now be decomposed in terms of tensors that governs the kinematics times scalar functions. To do this, we use that the electromagnetic current is conserved, $\partial_{\mu}j^{\mu}=0$, so that the hadronic tensor satisfies the Ward identity $q_{\mu}W^{\mu\nu}=0$. Further, using that the strong interaction is parity invariant and $W^{\mu\nu}$ is hermitian, the most general form of the hadronic tensor for unpolarized protons can be written as
\begin{align}\label{eq:1stparametrized hadronic tensor}
    W^{\mu\nu}=\Big(-g^{\mu\nu}+\frac{q^{\mu}q^{\nu}}{q^{2}}\Big)F_{1}(x,Q^{2})+\Big(P^{\mu}+\frac{1}{2x}q^{\mu}\Big)\Big(P^{\nu}+\frac{1}{2x}q^{\nu}\Big)\frac{1}{P\cdot q}F_{2}(x,Q^{2})\,.
\end{align}
The scalar functions $F_{1}$ and $F_{2}$ are the structure functions we alluded to earlier. They contain the information of the hadron structure as \textquote{seen} by the virtual photon. Combining this with the leptonic tensor we find that
\begin{align}
    L_{\mu\nu}W^{\mu\nu}=\frac{2Q^{2}}{xy^{2}}\Big[\Big(1-y+\frac{y^{2}}{2}\Big)2xF_{1}(x,Q^{2})+(1-y)\big(F_{2}(x,Q^{2})-2xF_{1}(x,Q^{2})\big)\Big]\,.
\end{align}
Plugging this into \cref{eq:DIS differential cross section}, and neglecting terms of $\mathcal{O}( m_{p}^{2}/Q^{2})$ gives the unpolarized electron--proton DIS cross section
\begin{align}\label{eq:full differential DIS}
    \frac{d^{2}\sigma}{dxdy}=\frac{4\pi\alpha^{2}s}{Q^{4}}\Big[\Big(1-y+\frac{y^{2}}{2}\Big)2xF_{1}(x,Q^{2})+(1-y)\big(F_{2}(x,Q^{2})-2xF_{1}(x,Q^{2})\big)\Big]\,.
\end{align}
To find a parton model prediction for the behaviour of the structure functions, we calculate the partonic equivalent of $\cref{eq:full differential DIS}$. Thus, we are only interested in electron--quark scattering, $e^{-}(l)q(p)\rightarrow e^{-}(l')q(p')$. By using the Mandelstam variables
\begin{align}
    \hat{s}=(l+p)^{2}\,,\hspace{0.5cm}\hat{t}=(l-l')\,,\hspace{0.5cm}\hat{u}=(p-l')\,,
\end{align}
it is straightforward to show that the spin/colour averaged amplitude takes the form
\begin{align}
    \langle\, |\mathcal{M}|^{2}\rangle = 2Q_{q}^{2}e^{4}\,\frac{\hat{s}^{2}+\hat{u}^{2}}{\hat{t}^{2}}\,,
\end{align}
where $Q_{q}$ is the fractional charge of the quark. Using the standard result for the differential cross section for massless $2\rightarrow 2$ scattering:
\begin{align}
    \frac{d\hat{\sigma}}{d\hat{t}}=\frac{1}{16\pi\hat{s}^{2}} \langle\, |\mathcal{M}|^{2}\rangle\,,
\end{align}
which after some rewriting will give the partonic differential cross section
\begin{align}
    \frac{d\hat{\sigma}}{dy}=Q_{q}^{2}\frac{4\pi\alpha^{2}\hat{s}}{Q^{4}}\Big(1-y+\frac{y^{2}}{2}\Big)\,.
\end{align}
In order to relate the hard cross section with the full cross section, we define the quark momentum as a fraction of the proton momentum,
\begin{align}
    p^{\mu}=\xi P^{\mu}\,, \hspace{0.8cm}0<\xi<1\,,
\end{align}
such that $\hat{s}=\xi s$. If we approximate with an on-shell constraint for the outgoing quark, we find that 
\begin{align}
    p'^{2}=(p+q)^{2}=2\xi P\cdot q -Q^{2}=0\,,
\end{align}
implying that $\xi=x$. The on-shell constraint fixes the momentum fraction to equal the Bjorken variable, but this is of course not a general result. The Bjorken-$x$ is a kinematical constraint defining the process, while $\xi$ is just a momentum fraction that is independent of the process. To obtain a double differential cross-section as in \cref{eq:full differential DIS}, we simply use that 
\begin{align}
    \int_{0}^{1}dx\,\delta(x-\xi)=1\,,
\end{align}
and write
\begin{align}\label{eq:partonic differential DIS}
    \frac{d^{3}\hat{\sigma}}{dxdyd\xi}=Q_{q}^{2}\frac{4\pi\alpha^{2}s}{Q^{4}}\Big(1-y+\frac{y^{2}}{2}\Big)\,\xi\,\delta(x-\xi)\,.
\end{align}
We can now use the parton model interpretation of $f_{q}(\xi)$ as a probability and convolute it with the partonic part. To find the electron--proton differential cross section we simply integrate over all possible fraction $\xi$ and sum over quark flavour
\begin{align}\label{eq:parton model factorization}
    \frac{d^{2}\sigma}{dxdy}&=\sum_{q}\int_{0}^{1}d\xi\,f_{q}(\xi)\,\frac{d^{3}\hat{\sigma}}{dxdyd\xi}\nonumber
    \\
    &=\frac{4\pi\alpha^{2}s}{Q^{4}}\Big(1-y+\frac{y^{2}}{2}\Big)\,\sum_{q}Q_{q}^{2}\,x\,f_{q}(x)\,.
\end{align}
Then if we compare \cref{eq:parton model factorization} with \cref{eq:full differential DIS} we see that the proton structure functions in this simple model is given by
\begin{align}\label{eq:proton structure functions F_2 and F_1}
    F_{2}(x)=2xF_{1}(x)&=\sum_{q}Q_{q}^{2}\,x\,f_{q}(x)\,.
\end{align}
This result shows that in the regime where $Q^{2}$ is very large, we have that the structure functions only depend on the Bjorken-$x$. This is what is called \emph{Bjorken scaling}, and the result $F_{2}=2xF_{1}$ is known as the \emph{Callan-Gross relation}. The Callan-Gross relation follows from the spin-$1/2$ nature of quarks and was later confirmed by structure function measurements. These predictions made the parton model an intriguing model to further explore the structure of hadrons.

In \cref{eq:parton model factorization} we convoluted the partonic cross section with the parton distributions resulting in a hadronic cross section. Therefore, we can also define quark structure functions that we can convolute with the parton distribution, giving the proton structure functions. We see that if we define
\begin{align}
    \hat{F}_{2}=Q_{q}^{2}\,x\,\delta(1-x)\,,\label{eq:quark structure function hat_F2}
    \\
    \hat{F}_{1}=\frac{1}{2}Q_{q}^{2}\,\delta(1-x)\,,\label{eq:quark structure function hat_F1}
\end{align}
we can write the parton model factorization formulas for the proton structure functions as,
\begin{align}
    F_{2}(x)&=\sum_{q}\int_{x}^{1}d\xi\,f_{q}(\xi)\hat{F}_{2}\Big(\frac{x}{\xi}\Big)=\sum_{q}Q_{q}^{2}\,x\,f_{q}(x)\label{eq:collinear factorization parton model F_2}\,,
    \\
    F_{1}(x)&=\sum_{q}\int_{x}^{1}\frac{d\xi}{\xi}f_{q}(\xi)\hat{F}_{1}\Big(\frac{x}{\xi}\Big)=\frac{1}{2}\sum_{q}Q_{q}^{2}f_{q}(x)\label{eq:collinear factorization parton model F_1}\,,
\end{align}
giving the same result as above. The integration bound follows from $x/\xi\leq 1$. In the same manner the differential cross section can be written in the following way:
\begin{align}
     d\sigma(x,Q^{2})&=\sum_{q}\int_{x}^{1}d\xi\,f_{q}(\xi)\,d\hat{\sigma}\Big(\frac{x}{\xi},Q^{2}\Big)\,,
    \\
\end{align}
which is known as \emph{collinear factorization} of DIS in the parton model. An important point to make is that the above factorized integrals are convolutions defined in \emph{Mellin Space}, see \cref{sec:Appendix Mellin Transform}. Mellin transformations and their properties are very useful in studying QCD, which we will lay out in more detail in \cref{chap:Resummation in QCD}.

The collinear terminology used here refers to the fact that we have only considered the case where the quark momentum is in the same direction as the proton. We should, therefore, point out that $f_{q}(\xi)$ is formally defined as
\begin{align}
    f_{q}(\xi)=\int d^{2}k_{\perp}f_{q}(\xi,k_{\perp})\,,
\end{align}
where the transverse momentum dependence has been integrated out. It should also be pointed out that this is only valid as we only measure the final state lepton. If we wanted to take into account the final state hadrons we would have to use \emph{transverse momentum distributions} (TMDs), which is much more complicated. In this chapter we will only consider processes, like DIS and Drell-Yan, where the final state hadrons are integrated out.

\medskip
We have seen that the parton model result can give predictions that have been confirmed by experiments, but it is a phenomenological model and is not a formal treatment of the strong interaction. However, the concept of factorization is exactly what makes one take a field theoretical approach to hadronic scattering processes. Including QCD into the parton model results in interactions between the partons, and the assumption of free partons does not hold anymore. Therefore, the above factorization formulas are invalid and we must make a more precise definition of factorization in QCD. To this end, we will consider the case of QCD corrections to the DIS process in the next section. 

\medskip
To end this section we will make a brief remark about the difference between structure functions and parton distribution functions. The structure functions appeared when we parametrized the hadronic tensor, which is process dependent. That is, if we considered DIS neutrino scattering, the structure functions would change as we, in that case, considered $W$ or $Z$ boson exchange. The main idea behind factorization is that, inside the structure functions, we can factorize out the proton content from the process dependent part. The factorization ansatz in \cref{eq:parton model factorization} is required to be valid for any process, meaning that the PDFs are universal. Thus, the PDFs can be extracted from DIS electron--proton scattering experiments and re-used in another experiment like DIS neutrino--proton scattering and proton--proton collisions at the LHC.

%%%%%%%%%%%%%%%%%%%%%%%%%%%%%%%%%%%%%%%%%
\subsection{Collinear Factorization in QCD}\label{sec:QCD and Collinear factorization}
In \cref{sec:DIS and Parton model} we saw that the observables in the parton model could be written on a factorized form. The crucial element of that result is the assumption that the quark momentum is a fraction of the proton's momentum, $p^{\mu}=\xi P^{\mu}$. This is in general not true, as the quark will also have transverse components. To highlight the validity of the parton model assumption it is useful to use \emph{light-cone coordinates}, see \cref{sec:Appendix Light-cone coordinates}. 

By considering the case where the proton has no transverse momentum components, the general form of the proton and parton momenta can be parametrized as
\begin{align}
    P^{\mu}=\Big(P^{+},\frac{m_{p}^{2}}{2P^{+}},0_{\perp}\Big)\,,\hspace{1cm}
    p^{\mu}=(p^{+},p^{-},p_{\perp})\,.
\end{align}
It is safe to assume that in the protons rest frame, the distribution of partons is isotropic, and that the components of the parton momentum is of the order of the proton mass. We are free to choose frame, so by choosing the frame where $P^{+}\rightarrow \infty$, the only remaining component of the proton momentum is its plus-component. It is also hard to imagine that the partons will not follow the proton along this direction---at least if we assume no gluon radiation---so we find that
\begin{align}
    P^{\mu}=(P^{+},0^{-},0_{\perp})\,,\hspace{1cm}
    p^{\mu}\approx(p^{+},0^{-},0_{\perp})\,.
\end{align}
This is what is called the \emph{infinite momentum frame} (IMF), and the partons transverse components has been neglected compared to the plus-component, $p^{+}\gg p_{\perp}\sim m_{p}$. We can now assume that the struck quark has a fraction $\xi$ of the proton's momentum, giving that in the infinite momentum frame the parton momentum is fully collinear to the proton momentum
\begin{align}
    p^{\mu}=\xi P^{\mu}\,,
\end{align}
which is the result we used for the parton model in \cref{sec:DIS and Parton model}. 

However, if the struck parton is a quark that has just emitted a gluon this is no longer necessarily true, and we will need a more general parametrization of the momenta. We can use $p^{+}=\xi P^{+}$ to parametrize the parton momentum in terms of the large plus-component $P^{+}$
\begin{align}\label{eq:light-cone parametrization}
    P^{\mu}=\Big(P^{+},\frac{m_{p}^{2}}{2P^{+}},0_{\perp}\Big)\,,\hspace{1cm}p^{\mu}=\Big(\xi P^{+},\frac{p^{2}+p_{\perp}^{2}}{2\xi P^{+}},p_{\perp}\Big)\,,
\end{align}
which reproduces the infinite momentum frame limit for $p^{2},p_{\perp}^{2},m_{p}^{2}\ll P^{+}$. In turn the photon momentum can be parametrized as
\begin{align}\label{eq:light-cone photon momenta}
    q^{\mu}=\Big(0,\frac{Q^{2}}{2xP^{+}},q_{\perp}\Big)\,,
\end{align}
where $-q^{2}=q_{\perp}^{2}=Q^{2}$ and $P\cdot q=Q^{2}/2x$, meaning that this choice reproduces the known kinematics in DIS, given in \cref{eq:Bjorken-x}. Not only does light-cone coordinates formalise the parton model assumption of fully collinear partons in the IMF, but as we will later see it is in this formalism that the parton model has it's closest relation to field theory. In the following calculations we will parametrize all momenta as in \cref{eq:light-cone parametrization}.

\medskip
In the parton model calculation we found that the structure functions scale, i.e. $F(x,Q^{2})\rightarrow F(x)$ in the Bjorken limit $Q^{2}\rightarrow \infty$. By including QCD into this picture, $F(x,Q^{2})$ will have terms that are proportional to $\ln{Q^{2}}$, i.e. we have logarithmic breaking of Bjorken scaling. The key point is that the original quark can emit a gluon before being struck by the photon and will acquire a transverse component that can not be neglected. As we shall see below, when integrating over the intermediate quark momentum, the integral over this transverse component extends up to the kinematic limit  $Q^{2}$, leading to this logarithmic dependancy of $Q^{2}$. 

Factorization in QCD is therefore the determination of at which point the emitted gluon belongs to the soft part or the hard part of the scattering. The correct way to determine this\,---\,at least in the collinear case\,---is to define a separation in terms of an energy scale $\mu_{F}$ (factorization scale). Thus, if the gluon is emitted at a scale lower than $\mu_{F}$ it is part of the PDF (soft part) and if it is emitted at a scale larger than $\mu_F$ it is part of the partonic process (hard part). The all order proof of factorization in QCD is beyond the scope of this thesis, so here we will investigate the $\mathcal{O}(\alpha_s)$ correction to DIS and see how that changes the parton model factorization.

\medskip
\subsection*{One gluon emission}
\begin{figure}
    \centering
    \includegraphics[scale=0.4]{Figures/gluonemissionDIS.pdf}
    \caption{Gluon emission from initial quark.}
    \label{fig:DISgluonemission}
\end{figure}

At NLO there are two real gluon diagrams and three virtual gluon diagrams in DIS. The real diagrams are emission from initial and final quark line, while the virtual diagrams are self energy diagrams of the initial and final state together with exchange of a gluon between the initial and final quark lines.
However, we will only consider gluon emission from the initial quark line, given in \cref{fig:DISgluonemission}. The reason for this is that we will choose to work with gluons in light-cone gauge, and in this gauge this is the only diagram that gives a logarithmic divergence. The other diagrams will give finite results that are unimportant for the current discussion and will be omitted for brevity.

\medskip
The way to proceed is to investigate corrections to the hadronic tensor and extract the structure functions from it. The proton strucure functions were in the parton model written as a convolution between the PDF and the quark structure functions, see \cref{eq:collinear factorization parton model F_2} and \cref{eq:collinear factorization parton model F_1}. The hadronic tensor can also be written on this convoluted form
\begin{align}
    W^{\mu\nu}(x,Q^{2})=\sum_{q}\int_{x}^{1}\frac{d\xi}{\xi}f_{q}(\xi)\hat{W}^{\mu\nu}\Big(\frac{x}{\xi},Q^{2}\Big)\,,
\end{align}
where $\hat{W}^{\mu\nu}$ refers to the partonic part that are calculable in perturbation theory. Thus, we will calculate $\hat{W}^{\mu\nu}$ up to $\mathcal{O}(\alpha_s)$ and extract $\hat{F}_{2}$ from it, enabling us to compare with \cref{eq:collinear factorization parton model F_2} to see the effect it has on the PDF.

In \cref{eq:1stparametrized hadronic tensor} we parametrized the hadronic tensor as
\begin{align}
    W^{\mu\nu}=\Big(-g^{\mu\nu}+\frac{q^{\mu}q^{\nu}}{q^{2}}\Big)F_{1}(x,Q^{2})+\Big(P^{\mu}+\frac{1}{2x}q^{\mu}\Big)\Big(P^{\nu}+\frac{1}{2x}q^{\nu}\Big)\frac{1}{P\cdot q}F_{2}(x,Q^{2})\,.
\end{align}
However, in higher order calculations there is an alternative parametrization that comes in handy when we want to extract the structure functions. Hence, we define normalized basis vectors (factors and terms $\mathcal{O}(m_{p}^{2}/Q^{2})$ are as usual neglected)
\begin{align}\label{eq:basis vectors}
    \hat{q}^{\mu}&\equiv\frac{q^{\mu}}{Q}\, ,\hspace{1cm}\hat{t}^{\mu}\equiv\frac{1}{Q}(q^{\mu}+2xP^{\mu})\,.
\end{align}
These basis vector satisfy: $\hat{q}\cdot\hat{t}=0$, $\hat{q}^{2}=-1$ and $\hat{t}^{2}=1$, meaning that $\hat{q}$ is spacelike and $\hat{t}$ is timelike. With respect to these basis vectors one can also define a transverse tensor
\begin{align}
    g_{\perp}^{\mu\nu}\equiv g^{\mu\nu}+\hat{q}^{\mu}\hat{q}^{\nu}-\hat{t}^{\mu}\hat{t}^{\nu}\,,
\end{align}
from which the following relations follow
\begin{align}
    \hat{t}_{\mu}g_{\perp}^{\mu\nu}&=0
    \\
    \hat{q}_{\mu}g_{\perp}^{\mu\nu}&=0
    \\
    g_{\perp}^{\mu\nu}g_{\perp\,\mu\nu}&=2\,,
\end{align}
making it consistent with \cref{eq:trasnversal tensor}. With these definitions it is straightforward to rewrite the hadronic tensor as
\begin{align}\label{eq:2ndparametrized hadronic tensor}
    W^{\mu\nu}(x,Q^{2})=-g_{\perp}^{\mu\nu}F_{1}(x,Q^{2})+\frac{\hat{t}^{\mu}\hat{t}^{\nu}}{2x}\big(F_{2}(x,Q^{2})-2xF_{1}(x,Q^{2})\big)\,,
\end{align}
from which the structure functions can be extracted by applying appropriate tensors on the hadronic tensor
\begin{align}
    F_{1}&=-\frac{1}{2}g_{\perp}^{\mu\nu}W_{\mu\nu}\,,\label{eq:extracting F_1}
    \\
    F_{2}&=x(2\hat{t}^{\mu}\hat{t}^{\nu}-g_{\perp}^{\mu\nu})W_{\mu\nu}\,.\label{eq:extracting F_2}
\end{align}
For the $F_{2}$ projection is is useful to note that $(2\hat{t}^{\mu}\hat{t}^{\nu}-g_{\perp}^{\mu\nu})g_{\mu\nu}=0$, which will cancel all terms involving $g^{\mu\nu}$ in the Dirac traces. 

The amplitude for the gluon emission in \cref{fig:DISgluonemission}, is given by
\begin{align}
    \mathcal{M}^{\mu}=-igQ_{q}t^{a}\varepsilon_{\beta}^{*}(k')\,\bar{u}(p')\,\gamma^{\mu}\frac{\slashed{k}}{k^{2}}\gamma^{\beta}\,u(p)\,.
\end{align}
We average over incoming spin and colour, sum over final spin, colour and gluon polarization,
\begin{align}
    \langle\,|\mathcal{M}|^{2}\rangle^{\mu\nu}&=\frac{1}{N_{s}N_{c}}\sum_{a,b}\sum_{\text{spin}}\sum_{\text{pol}}\big(|\mathcal{M}|^{2}\big)^{\mu\nu}\,\nonumber
    \\
    &=\frac{C_{F}}{N_s}Q_{q}^{2}g^{2}\frac{1}{k^{4}}\sum_{\text{pol}}\varepsilon_{\alpha}(k')\varepsilon_{\beta}^{*}(k')\,\text{tr}[\gamma^{\nu}\slashed{p}'\gamma^{\mu}\slashed{k}\gamma^{\alpha}\slashed{p}\gamma^{\beta}\slashed{k}]\,.
\end{align}
We can then use the gluon polarization sum in the light-cone gauge (see \cref{eq:gluon polarization sum light-cone gauge})
\begin{align}
    \sum_{\text{pol}}\varepsilon_{\alpha}(k')\varepsilon_{\beta}^{*}(k')&=-g_{\alpha\beta}+\frac{k'_{\alpha}\,n_{-\,\beta}}{k'^{+}}+\frac{k'_{\beta}\,n_{-\,\alpha}}{k'^{+}}\,,
\end{align}
which will give the following expression for the Dirac trace
\begin{align}
    \sum_{\text{pol}}\varepsilon_{\alpha}(k')\varepsilon_{\beta}^{*}(k')\,\text{tr}[\gamma^{\nu}\slashed{p}'\gamma^{\mu}\slashed{k}\gamma^{\alpha}\slashed{p}\gamma^{\beta}\slashed{k}]=&-\text{tr}[\gamma^{\nu}\slashed{p}'\gamma^{\mu}\slashed{k}\gamma^{\alpha}\slashed{p}\gamma_{\alpha}\slashed{k}]+\frac{1}{k'^{+}}\text{tr}[\gamma^{\nu}\slashed{p}'\gamma^{\mu}\slashed{k}\slashed{k}'\slashed{p}\gamma^{+}\slashed{k}]\nonumber
    \\
    &+\frac{1}{k'^{+}}\text{tr}[\gamma^{\nu}\slashed{p}'\gamma^{\mu}\slashed{k}\gamma^{+}\slashed{p}\slashed{k}'\slashed{k}]\,.\nonumber
\end{align}

These traces can be calculated using the relations given in \cref{sec:Appendix Dirac gamma matrices}, but it is tedious so we will just write out the main steps. The first trace can be simplified to
\begin{align}
    \text{tr}[\gamma^{\nu}\slashed{p}'\gamma^{\mu}\slashed{k}\gamma^{\alpha}\slashed{p}\gamma_{\alpha}\slashed{k}]=-4p\cdot k\,\text{tr}[\gamma^{\nu}\slashed{p}'\gamma^{\mu}\slashed{k}]+2k^{2}\,\text{tr}[\gamma^{\nu}\slashed{p}'\gamma^{\mu}\slashed{p}]\,,
\end{align}
and the last two traces can first be simplified by using that $k'=p-k$, giving
\begin{align}
    \text{tr}[\gamma^{\nu}\slashed{p}'\gamma^{\mu}\slashed{k}\slashed{k}'\slashed{p}\gamma^{+}\slashed{k}]&=-k^{2}\,\text{tr}[\gamma^{\nu}\slashed{p}'\gamma^{\mu}\slashed{p}\gamma^{+}\slashed{k}]\,,
    \\
    \text{tr}[\gamma^{\nu}\slashed{p}'\gamma^{\mu}\slashed{k}\gamma^{+}\slashed{p}\slashed{k}'\slashed{k}]&=-k^{2}\,\text{tr}[\gamma^{\nu}\slashed{p}'\gamma^{\mu}\slashed{k}\gamma^{+}\slashed{p}]\,,
\end{align}
where the sum of these two traces can be rewritten as
\begin{align}
    -k^{2}\text{tr}[\gamma^{\nu}\slashed{p}'\gamma^{\mu}(\slashed{k}+\slashed{p})\gamma^{+}(\slashed{p}+\slashed{k})]=&-2(k^{+}+p^{+})k^{2}\,\text{tr}[\gamma^{\nu}\slashed{p}'\gamma^{\mu}\slashed{k}]\nonumber
    \\
    &-2(k^{+}+p^{+})k^{2}\,\text{tr}[\gamma^{\nu}\slashed{p}'\gamma^{\mu}\slashed{p}]\nonumber
    \\
    &+k^{2}(k^{2}+2p\cdot k)\text{tr}[\gamma^{\nu}\slashed{p}'\gamma^{\mu}\gamma^{+}]\,.\nonumber
\end{align}
Collecting all terms and using the cyclic property of the trace, we find that the sum of all traces can be written as
\begin{align}
    \sum\text{tr}(...)=&\,\Big(4p\cdot k-2k^{2}\frac{k^{+}+p^{+}}{k^{+}-p^{+}}\Big)\text{tr}[\slashed{k}\gamma^{\mu}\slashed{p}'\gamma^{\nu}]\nonumber
    \\
    &-2k^{2}\Big(1+\frac{k^{+}+p^{+}}{k^{+}-p^{+}}\Big)\text{tr}[\slashed{p}\gamma^{\mu}\slashed{p}'\gamma^{\nu}]\nonumber
    \\
    &+\frac{k^{2}}{k^{+}-p^{+}}\big(k^{2}+2p\cdot k\big)\text{tr}[\slashed{n}_{-}\gamma^{\mu}\slashed{p}'\gamma^{\nu}]\,,\nonumber
\end{align}
where we used that $\gamma^{+}=\slashed{n}_{-}$, such that all the traces take the same form. For generic vectors, $a^{\mu}$ and $b^{\mu}$, these traces evaluate to\footnote{See \cref{sec:Appendix Dirac gamma matrices} for more detail on traces.}
\begin{align}\label{eq:tracetrace}
    \text{tr}[\slashed{a}\gamma^{\mu}\slashed{b}\gamma^{\nu}]=4(a^{\mu}b^{\nu}+a^{\nu}b^{\mu}-g^{\mu\nu}a\cdot b)\,.
\end{align}
Hence, we write the averaged amplitude as
\begin{align}\label{eq:ththehte amp}
    \langle\,|\mathcal{M}|^{2}\rangle^{\mu\nu}=\frac{C_{F}}{N_s}Q_{q}^{2}g^{2}\frac{1}{k^{4}}\sum\text{tr}(...)\,.
\end{align}


To calculate $\hat{W}^{\mu\nu}$, we must integrate over the final state particles, giving
\begin{align}
    \hat{W}^{\mu\nu}=\frac{1}{4\pi}\int d\mathcal{P}_{2}\,\langle\, |\mathcal{M}|^{2}\rangle \,,
\end{align}
where $4\pi$ comes from the normalization in \cref{eq:hadronic tensor in terms of currents}. The n-body phase space for on-shell massless particles is given by \cref{eq:n-body phase space}, so the two-body phase space takes the form
\begin{align}\label{eq:DIS differential phase space}
    d\mathcal{P}_{2}&=\int\frac{d^{4}k'}{(2\pi)^{3}}\frac{d^{4}p'}{(2\pi)^{3}}\,\delta^{+}(k'^{2})\delta^{+}(p'^{2})(2\pi)^{4}\delta^{(4)}(p+q-k'-p')\nonumber
    \\
    &=\frac{1}{4\pi^{2}}\int d^{4}k\,\delta^{+}\big((p-k)^{2}\big)\delta^{+}\big((k+q)^{2}\big)\,,
\end{align}
where we use that $\delta^{+}(k'^{2})\equiv \delta(k'^{2})\theta(k'^{+})$\footnote{In light-cone coordinates the usual Heaviside function $\theta(k^{0})$ is replaced by $\theta(k^{+})$.}.
Note that we work explicitly in four dimensions, which mean we will not use dimensional regularization in this calculation. For the Drell-Yan process (see \cref{sec:Drell-Yan Hadronic Cross Section}), we will perform a full calculation using dimensional regularization, but for physical intuition it is more useful to use momentum cutoff in DIS.  

\medskip
In the partonic system we assume that the original quark has no transverse components and move in the plus direction, the virtual\footnote{All quarks are kind of virtual, but this is common terminology in scattering processes to specify that it is intermediate, i.e. propagating and off-shell.} quark on the other hand may have large transverse components due to the radiation of a gluon from the original quark. Thus, the relevant momenta in the partonic system is given by:
\begin{align}
    p^{\mu}&=(p^{+},0^{-},0_{\perp})\,,
    \\
    k^{\mu}&=\Big(\xi p^{+},\frac{k_{\perp}^{2}-|k^{2}|}{2\xi p^{+}},k_{\perp}\Big)\,,
    \\
    q^{\mu}&=\Big(0,\frac{Q^{2}}{2xp^{+}},q_{\perp}\Big)\,,
\end{align}
where we used that $k^{2}=-|k^{2}|$ because the intermediate quark is virtual. The arguments of the delta functions in \cref{eq:DIS differential phase space} is given by
\begin{align}
    (p-k)^{2}&=-2p\cdot k-|k^{2}|=-\frac{1}{\xi}\big(k_{\perp}^{2}-(1-\xi)|k^{2}|\big)\,,
    \\
    (k+q)^{2}&=2k\cdot q-|k^{2}|-Q^{2}=2P\cdot q\Big(\xi-x-\frac{|k^{2}|+2k_{\perp}\cdot q_{\perp}}{2P\cdot q}\Big)\,,
\end{align}
and the differential is given by
\begin{align}
    d^{4}k=dk^{+}dk^{-}d^{2}k_{\perp}=\frac{1}{4\xi}d\xi dk^{2}dk_{\perp}^{2}d\theta\,,
\end{align}
with $0<\theta<\pi$. Inserting these expression into \cref{eq:DIS differential phase space}, we get
\begin{align}
     d\mathcal{P}_{2}&=\frac{1}{(4\pi)^{2}P\cdot q}\int d\xi dk^{2}dk_{\perp}^{2}d\theta \hspace{2mm}\delta\big(k_{\perp}^{2}-(1-\xi)|k^{2}|\big)\,\delta\Big(\xi-x-\frac{|k^{2}|+2k_{\perp}\cdot q_{\perp}}{2P\cdot q}\Big)\,.
\end{align}

Now, there is no a priori reason why the transverse momentum (or equivalently $|k^{2}|$) should be small, but in the second delta function the effect of these are damped by $P\cdot q\sim Q^{2}$ so we proceed by neglecting this term. This again fixes $\xi=x$ in the final integration. We are actually jumping ahead here, but with a more thorough analysis it can be shown that this is in fact the case \cite{Ellis1996QCDAC}. The first delta function fixes $k_{\perp}^{2}=(1-\xi)|k^{2}|$, so we can just use it directly to rewrite the averaged amplitude in \cref{eq:ththehte amp}, giving
\begin{align}
    \langle\, |\mathcal{M}|^{2}\rangle^{\mu\nu}=&4\pi C_{F}\,Q_{q}^{2}\,\alpha_{s}\frac{1}{|k^{2}|}\Big[\frac{1}{\xi}\frac{1+\xi^{2}}{1-\xi}\big(8k^{(\mu}p'^{\nu)}-4g^{\mu\nu}k\cdot p'\big)\nonumber
    \\
    &+\frac{1}{1-\xi}\big(8p^{(\mu}p'^{\nu)}-4g^{\mu\nu}p\cdot p'\big) + \frac{1}{1-\xi}\frac{|k^{2}|}{p^{+}}\big(8n_{-}^{(\mu}p'^{\nu)}-4g^{\mu\nu}p'^{+}\big)\Big]\,.
\end{align}
where we have evaluated the traces using \cref{eq:tracetrace} and for notational simplicity defined $a^{(\mu}b^{\nu)}=(a^{\mu}b^{\nu}+a^{\nu}b^{\mu})/2$.

Let us then finally use \cref{eq:extracting F_2} to project out $\hat{F}_{2}$,
\begin{align}
    \hat{F}_{2}=\int d\mathcal{P}_{2}\,\,x(\hat{t}^{\mu}\hat{t}^{\nu}-g_{\perp}^{\mu\nu})\,\langle\, |\mathcal{M}|^{2}\rangle_{\mu\nu}\,.
\end{align}
In the following we will only list the divergent term as the finite ones are unimportant for this discussion. For the contraction with the matrix element we use that $p'=k+q$ and the basis vector $\hat{t}$ is found from \cref{eq:basis vectors}. Putting everything together is messy, but after the effect of the delta functions the divergent part takes the form\footnote{For a more detailed derivation of this term, see \cite{Ellis1996QCDAC}.}
\begin{align}\label{eq:divergent F_2}
    \hat{F}_{2}\,\big|_{\text{div}}=Q_{q}^{2}\frac{\alpha_s}{2\pi}\,xP_{q/q}(x)\int_{Q_{0}^{2}}^{Q^{2}}\frac{d|k^{2}|}{|k^{2}|}\,,
\end{align}
where the integral is regulated over $|k^{2}|$ in terms of a IR cut-off $Q_{0}^{2}$ and a UV cut-off $Q^{2}$, that follows from the kinematics. We have also defined a function $P_{q/q}(x)$, which is known as the \emph{quark--quark splitting function}
\begin{align}\label{eq:splitting function 1}
    P_{q/q}(x)=C_{F}\frac{1+x^{2}}{1-x}\,.
\end{align}
Its form is specific to the quark-quark-gluon vertex of QCD, and it represents the probability for a quark to split into another quark with momentum fraction $x$ and a gluon with momentum fraction $1-x$. Integrating \cref{eq:divergent F_2}, we find
\begin{align}\label{eq:divergent integrated F_2}
    \hat{F}_{2}\,\big|_{\text{div}}=Q_{q}^{2}\frac{\alpha_s}{2\pi}\,xP_{q/q}(x)\,\ln{\frac{Q^{2}}{Q_{0}^{2}}}\,.
\end{align}

If we include the leading order contribution given in \cref{eq:collinear factorization parton model F_2} and all finite terms, denoted $C_{q}(x)$, we have the quark structure function at $\mathcal{O}(\alpha_s)$
\begin{align}\label{eq:quark structure function correction}
    \hat{F}_{2}=Q_{q}^{2}\,x\Big(\delta(1-x)+\frac{\alpha_s}{2\pi}\big(P_{q/q}(x)\ln{\frac{Q^{2}}{Q_{0}^{2}}}+C_{q}(x)\big)\Big)\,.
\end{align}

The natural question that arises is how to interpret the divergence of the structure function. We observe that the singularity arises when the gluon is emitted fully collinear to the quark, i.e. $k_{\perp}=0$, and is therefore referred to as a \emph{collinear singularity}. To understand what is happening we need to realize that physically $k_{\perp}^{2}\rightarrow 0$ corresponds to a long-range part of the strong interaction which is not calculable in perturbation theory. Hence, if we use the PDFs to describe physics at long-range, we can convolute the structure function $\hat{F}_{2}$ with \textquote{bare} distributions $f_{q}^{0}$ to absorb the collinear singularity. This is a similar, but still different approach to the UV-renormalization procedure we discussed in \cref{sec:Renormalization}.

To get a clear understanding of how this is done we investigate the cut-off we made in \cref{eq:divergent F_2}. By imposing the IR cut-off we effectively integrate $k_{\perp}^{2}$ from $Q_{0}^{2}$ up to $Q^{2}$. Now, the kinematics of the process justifies the upper limit of $Q^{2}$ as $k_{\perp}$ is always smaller than or equal to $Q$\footnote{This follows from the fact that the transverse momentum of the gluon can not be larger than the scale of the process.}. However, in the IR-region there is no kinematical restriction on $k_{\perp}$. The lower cut-off ensures that the gluons with transverse momentum $k_{\perp}\leq Q_{0}$ is neglected from the hard part of the scattering. We can not drop these gluons entirely, so we absorb this part of the process into the PDF. We can then renormalize the PDF up to the arbitrary energy scale $\mu_F$.

\medskip
To obtain the proton structure function we convolute the quark structure function $\hat{F}_{2}$ of \cref{eq:quark structure function correction} with the bare PDF $f_{q}^{0}$ as we did for the parton model in \cref{eq:collinear factorization parton model F_2}, 
\begin{align}
    F_{2}(x,Q^{2})=\sum_{q}Q_{q}^{2}\,x\Big(f_{q}^{0}(x)+\frac{\alpha_s}{2\pi}\int_{x}^{1}\frac{d\xi}{\xi}f_{q}^{0}(\xi)\big[P_{q/q}\big(\frac{x}{\xi}\big)\ln{\frac{Q^{2}}{Q_{0}^{2}}}+C_{q}\big(\frac{x}{\xi}\big)\big]\Big)\,.
\end{align}
Then we can absorb the collinear singularities into the bare distribution at the factorization scale $\mu_F$, or in other words, we define a renormalized distribution
\begin{align}
    f_{q}(x,\mu_{F}^{2})=f_{q}^{0}(x)+\frac{\alpha_s}{2\pi}\int_{x}^{1}\frac{d\xi}{\xi}f_{q}^{0}(\xi)\big[P_{q/q}\big(\frac{x}{\xi}\big)\ln{\frac{\mu_{F}^{2}}{Q_{0}^{2}}}+C_{q}'\big(\frac{x}{\xi}\big)\big]\,,
\end{align}
which inserted into the expression for $F_{2}$ gives the finite result
\begin{align}\label{eq:renormalized F_2}
    F_{2}(x,Q^{2})=\sum_{q}Q_{q}^{2}\,x\int_{x}^{1}\frac{d\xi}{\xi}f_{q}(\xi,\mu_{F}^{2})\big[\delta\big(1-\frac{x}{\xi}\big)+\frac{\alpha_s}{2\pi}\Big(P_{q/q}\big(\frac{x}{\xi}\big)\ln{\frac{Q^{2}}{\mu_{F}^{2}}}+D_{q}\big(\frac{x}{\xi}\big)\Big)\big]\,.
\end{align}

This result says that we can choose an arbitrary energy scale $\mu_F$ to separate the process into two parts: a hard part where $k_{\perp}$ is larger than this scale, and a soft part where $k_{\perp}$ is smaller than this scale. Thus, we \textquote{hide} the divergence inside a part that were non-perturbative in the first place. We have alluded to this separation repeatedly and made this statement in the discussion of the parton model, but now we have extended that result to the correct framework of QCD. We also note that the only difference in QCD factorization and the parton model factorization is that the PDF and the hard part acquires a dependancy on the factorization scale. Although we have only demonstrated factorization to $\mathcal{O}(\alpha_S)$ in DIS, it has been proven to all orders in perturbation theory \cite{Collins:1989gx}. Another important feature of extending to QCD is that Bjorken-scaling is violated as $F_{2}$ is explicitly dependent on $Q^{2}$.

One important aspect of the renormalization procedure we have made here is that even though the factorization specifies a prescription of dealing with the logarithmic singularities, there is still an arbitrariness in the choice of the finite terms. In \cref{eq:renormalized F_2} we have that $D(x)=C(x)-C'(x)$, where $C'(x)$ is absorbed by the PDF and $D(x)$ is the terms that remains. However, we could have made another choice and absorbed all finite terms into the PDF, and the exact choice made is referred to as a \emph{factorization scheme}. The most common scheme is the $\overline{\text{MS}}$ scheme we mentioned in \cref{sec:Renormalization}, where the absorbed term is $C'=\ln 4\pi-\gamma_{E}$.

Due to the non-perturbative nature of long distance physics in QCD, the distributions $f_{q}(x,\mu_{F}^{2})$ are not directly   calculable from first principles, and must therefore be extracted from experiments or, more recently in lattice QCD. However, what can be calculated perturbatively is the dependence on the scale $\mu_{F}^{2}$. The proton structure function $F_{2}$ is an observable, meaning that it cannot depend on the choice of scale. This leads to the following requirement
\begin{align}
    \pdv{F_2}{\ln\mu_{F}^{2}}=0\,,
\end{align}
which leads to the following differential equation for the PDFs
\begin{align}\label{eq:1st evolution equation f_q}
    \pdv{}{\ln\mu_{F}^{2}}f_{q}(x,\mu_{F}^{2})=\frac{\alpha_{s}(\mu_{F}^{2})}{2\pi}\int_{x}^{1}\frac{d\xi}{\xi}P_{q/q}\big(\frac{x}{\xi}\big)\,f_{q}(\xi,\mu_{F}^{2})\,.
\end{align}
This evolution equation is similar to the $\beta$ function equation describing the evolution of the coupling $\alpha_{s}(\mu^{2})$, and is known as the Dokshitzer-Gribov-Altarelli-Parisi (DGLAP) equation. As it stands, this equation is only valid up to $\mathcal{O}(\alpha_s)$. To get an all order evolution equation we can expand the splitting function in the coupling,
\begin{align}
    P_{q/q}(x,\alpha_{s})=\sum_{n=0}^{\infty}\Big(\frac{\alpha_s}{2\pi}\Big)^{n+1}P_{q/q}^{(n)}(x)\,,
\end{align}
and write the evolution equation as
\begin{align}\label{eq:DGLAP equation}
    \pdv{}{\ln\mu_{F}^{2}}f_{q}(x,\mu_{F}^{2})=\int_{x}^{1}\frac{d\xi}{\xi}P_{q/q}\big(\frac{x}{\xi},\alpha_{s}(\mu_{F}^{2})\big)\,f_{q}(\xi,\mu_{F}^{2})\,,
\end{align}
where the higher order information is contained inside the splitting function. We can then identify the splitting function in \cref{eq:splitting function 1} and \cref{eq:1st evolution equation f_q} as the leading order splitting function $P_{q/q}^{(0)}$. For a more rigorous derivation of this generalization see \cite{Altarelli:1977zs}. 

In order to obtain a complete discussion of deep inelastic scattering in terms of parton distribution functions, there is one more ingredient that we have not considered. That is, we also need to consider the initial scattering of gluons. The $\mathcal{O}(\alpha_s)$ contribution for initial gluon scattering is through a t-channel boson-gluon fusion process. Following a similar line of argument as for the quark initial state this will lead to the gluon structure function
\begin{align}
    \hat{F}_{2}^{g}(x,Q^{2})=\sum_{q}Q_{q}^{2}\,x\,\frac{\alpha_s}{2\pi}\Big(P_{q/g}(x)\ln{\frac{Q^{2}}{Q_{0}^{2}}}+C_{g}(x)\Big)\,,
\end{align}
which is very similar to the quark structure function \cref{eq:quark structure function correction}, apart from the fact that at zeroth order in $\alpha_s$ there is no gluon radiation. The splitting function for this process is specific for the gluon-quark-quark vertex, and is given by
\begin{align}
    P_{q/g}(x)=\frac{1}{2}(x^{2}+(1-x)^{2})\,,
\end{align}
and is interpreted as the probability for a gluon to split into a quark with momentum fraction $x$ and another quark with momentum fraction $1-x$. 

As with the quark structure function we have to convolute the gluon structure function with a bare gluon distribution $f_{g}^{0}$, define a renormalized gluon distribution and add it to \cref{eq:renormalized F_2}. This is equivalent to redefining our renormalized quark parton distribution to also absorb the singular gluon term. Thus, we define
\begin{align}
    f_{q}(x,\mu_{F}^{2})=&f_{q}^{0}(x)+\frac{\alpha_s}{2\pi}\int_{x}^{1}\frac{d\xi}{\xi}f_{q}^{0}(\xi)\big[P_{q/q}\big(\frac{x}{\xi}\big)\ln{\frac{\mu_{F}^{2}}{Q_{0}^{2}}}+C'_{q}\big(\frac{x}{\xi}\big)\big]\nonumber
    \\
    &+\frac{\alpha_s}{2\pi}\int_{x}^{1}\frac{d\xi}{\xi}f_{g}^{0}(\xi)\big[P_{q/g}\big(\frac{x}{\xi}\big)\ln{\frac{\mu_{F}^{2}}{Q_{0}^{2}}}+C'_{g}\big(\frac{x}{\xi}\big)\big]\,.
\end{align}

Instead of writing the whole final expression out, let us be even more general and define hard functions with perturbative expansions
\begin{align}
    H_{q}(z)&=\sum_{n=0}^{\infty}\Big(\frac{\alpha_s}{2\pi}\Big)^{n}H_{q}^{(n)}(z)\,,
    \\
    H_{g}(z)&=\sum_{n=1}^{\infty}\Big(\frac{\alpha_s}{2\pi}\Big)^{n}H_{g}^{(n)}(z)\,.
\end{align}
where the $\mathcal{O}(\alpha_s)$ expansion is given by
\begin{align}
    H_{q}(z)&=H_{q}^{(0)}(z)+\frac{\alpha_s}{2\pi}H_{q}^{(1)}(z)=\delta(1-z)+\frac{\alpha_s}{2\pi}\Big(P_{q/q}^{(0)}(z)\ln{\frac{Q^{2}}{\mu_{F}^{2}}}+D_{q}(z)\Big)\,,
    \\
    H_{g}(z)&=\frac{\alpha_s}{2\pi}H_{g}^{(1)}(z)=\frac{\alpha_s}{2\pi}\Big(P_{q/g}^{(0)}(z)\ln{\frac{Q^{2}}{\mu_{F}^{2}}}+D_{g}(z)\Big)\,,
\end{align}
The collinear factorization formula for $F_2$ then takes the form
\begin{align}
    F_{2}(x,Q^{2})=&\sum_{q}Q_{q}^{2}\,x\int_{x}^{1}\frac{d\xi}{\xi}f_{q}(\xi,\mu_{F}^{2})H_{q}\big(\frac{x}{\xi},\frac{Q^{2}}{\mu_{F}^{2}},\alpha_{s}(\mu_{F}^{2})\big)\nonumber
    \\
    &+\sum_{q}Q_{q}^{2}\,x\int_{x}^{1}\frac{d\xi}{\xi}f_{g}(\xi,\mu_{F}^{2})H_{g}\big(\frac{x}{\xi},\frac{Q^{2}}{\mu_{F}^{2}},\alpha_{s}(\mu_{F}^{2})\big)\,.
\end{align}
To actually have a full description of collinear factorization in QCD we would need to find the factorization of $F_{1}$ as well. To find $F_{1}$ we would have to project it out as in \cref{eq:extracting F_1}, but we will not consider the specific calculation as it follows in the same manner as for $F_{2}$. 

\medskip
If one considered even higher order corrections we would encounter the additional splitting functions $P_{g/q}(x)$ and $P_{g/g}(x)$. $P_{g/q}(x)$ decribes the splitting of a quark into a gluon with momentum fraction $x$ and a quark with momentum fraction $1-x$, and similarly $P_{g/g}(z)$ describes the splitting of a gluon into a gluon with momentum fraction $x$ and another gluon with momentum fraction $1-x$. For completeness we list all splitting functions at leading order
\begin{align}
    P_{q/q}^{(0)}(x)&=C_{F}\frac{1+x^{2}}{1-x}\,,
    \\
    P_{q/g}^{(0)}(x)&=\frac{1}{2}(x^{2}+(1-x)^{2})\,,
    \\
    P_{g/q}^{(0)}(x)&=C_{F}\frac{1+(1-x)^{2}}{x}\,,
    \\
    P_{g/g}^{(0)}(x)&=2C_{A}\Big(\frac{x}{1-x}+\frac{1-x}{x}+x(1-x)\Big)\,.
\end{align}

We observe that $P_{q/q}(x)$ and $P_{g/g}(x)$ both have singular behaviour for $x\rightarrow 1$, so we will need to regularize these as well. This is done by using so-called \emph{plus distributions}, see \cref{sec:Appendix Plus Distributions}. We will encounter plus distributions in \cref{sec:Drell-Yan Hadronic Cross Section} as well, so it is useful to write down the regulated splitting functions. The regularized splitting functions are at leading order given by \cite{Altarelli:1977zs},
\begin{align}
    P_{q/q}^{(0)}(x)&=C_{F}\Big(\Big[\frac{1+x^{2}}{1-x}\Big]_{+}+\frac{3}{2}\delta(1-x)\Big)\,,\label{eq:qq splitting function}
    \\
    P_{g/g}^{(0)}(x)&=2C_{A}\Big(\Big[\frac{1+x^{2}}{1-x}\Big]_{+}+\frac{1-x}{x}+x(1-x)\Big)+\delta(1-x)\frac{(11C_{A}-2n_{f})}{6}\,,\label{eq:gg splitting function}
\end{align}
where $n_{f}$ is the number of quarks. 

This leads us to the end of our discussion of factorization in DIS. In \cref{sec:Drell-Yan Hadronic Cross Section} we will investigate one other process where factorization has been proven, namely the Drell-Yan process. But before we move on to the Drell-Yan cross section, there are some important details about parton distribution functions we have to discuss. That is, we can actually write PDFs as operator valued matrix elements, and use Wilson lines to render these gauge invariant. Not only that, but we can use the expansion of Wilson lines to describe gluon radiation from incoming partons. That is not to say that the universal distribtuins $f_{i/h}$ can be calculated in perturbation theory, but we can define parton-in-parton distributions that are calculable in perturbation theory. This is the topic of the next few chapters.

%%%%%%%%%%%%%%%%%%%%%%%%%%%%%%%%%%%%%%%%%
\subsection{Operator Definition for PDFs}\label{sec:Operator definition for Parton Distributions}
 

The interpretation of a parton distribution function as the probability of
finding a parton inside a proton with momentum fraction $\xi$ was
crucial in finding factorized formulas for QCD scattering observables. In quantum field
theory, we have that probabilities are matrix elements squared, so we would like to define the parton distribution functions as operator-valued matrix elements. In terms of operator definitions one can use them to prove the factorization properties of QCD, and one can also use them to include soft and collinear singularities. This last part is essential for what is called \emph{re-factorization} in QCD resummation, which we will come back to in \cref{chap:Resummation in QCD}. 

\medskip
In \cref{sec:DIS and Parton model} we found an expression for the hadronic tensor in terms of the electromagnetic currents in \cref{eq:hadronic tensor in terms of currents}. Here we will construct it diagrammatically such that we can extract the so-called \emph{quark correlator}, which are the building block for the parton distribution functions. The diagrammatic construction is illustrated in \cref{fig:Hadronic tensor}. The upper diagram pulls out a quark from the proton, which subsequently fragments into the final state $X$. Mathematically this is given by the matrix element
\begin{align}
    \mathcal{A}_{i}=\bra{X}\psi_{i}(0)\ket{P}\,.
\end{align}
Assuming a gauge interaction in $\mu$, the bottom diagram will give a quark in the final state. Then we will get the matrix element
\begin{align}
    \mathcal{A}^{\nu}=Q_{q}\bar{u}_{j}^{s}(k)\big(\gamma^{\nu}\big)^{ji}\bra{X}\psi_{i}(0)\ket{P}\,.
\end{align}
The squared matrix element for this part of the process can then be written as:
\begin{align}
    \big(|\mathcal{A}|^{2}\big)^{\mu\nu}=\sum_{q}Q_{q}^{2}\big[\gamma^{\mu}(\slashed{k}+m)\gamma^{\nu}\big]^{ji}\bra{P}\overline{\psi}_{j}(0)\ket{X}\bra{X}\psi_{i}(0)\ket{P}\,,
\end{align}
where we have used the spin sum rule
\begin{align}
    \sum_{s}u^{s}(k)\bar{u}^{s}(k)=\slashed{k}+m\,.
\end{align}
\begin{figure}
    \centering
    \includegraphics[scale=0.4]{Figures/HadronicTensor.pdf}
    \caption{Diagramatic construction of hadronic tensor}
    \label{fig:Hadronic tensor}
\end{figure}

To write down the hadronic tensor we integrate over the final states $X$ and $k$ and impose momentum conservation through a delta function. As usual, we use the exponential representation of the delta function, and the integral over $k$ will be made by using the on-shell condition, see \cref{eq:n-body phase space}
\begin{align}
    \int\frac{d^{3}k}{(2\pi)^{3}}\frac{1}{2k^{0}}=\int\frac{d^{4}k}{(2\pi)^{4}}2\pi\delta(k^{2}-m^{2})\theta(k^{0})\,,
\end{align}

Taking the quark to be massless, we find the hadronic tensor to take the form
\begin{align}
    W^{\mu\nu}&=4\pi^{3}\sum_{q}Q_{q}^{2}\sum_{X}\int\frac{d^{3}p_{X}}{(2\pi)^{3}2E_{X}}\int\frac{d^{3}k}{(2\pi)^{3}2k^{0}}\delta^{(4)}(P+q-k-p_{X})\nonumber
    \\
    &\hspace{0.8cm}\times\big(\gamma^{\mu}\slashed{k}\gamma^{\nu}\big)^{ji}\bra{P}\overline{\psi}_{j}(0)\ket{X}\bra{X}\psi_{i}(0)\ket{P}\nonumber
    \\
    &=4\pi^{3}\sum_{q}Q_{q}^{2}\sum_{X}\int\frac{d^{3}p_{X}}{(2\pi)^{3}2E_{X}}\int\frac{d^{4}k}{(2\pi)^{4}}2\pi\delta(k^{2})\theta(k^{0})\int\frac{d^{4}z}{(2\pi)^{4}}\,e^{iz\cdot(P-q-k-p_{X})}\nonumber
    \\
    &\hspace{0.8cm}\times\big(\gamma^{\mu}\slashed{k}\gamma^{\nu}\big)^{ji}\bra{P}\overline{\psi}_{j}(0)\ket{X}\bra{X}\psi_{i}(0)\ket{P}\nonumber
    \\
    &=\frac{1}{2}\sum_{q}Q_{q}^{2}\sum_{X}\int\frac{d^{3}p_{X}}{(2\pi)^{3}2E_{X}}\int d^{4}p\,\delta((p+q)^{2})\theta(p^{0}+q^{0})\nonumber
    \\
    &\hspace{0.8cm}\int\frac{d^{4}z}{(2\pi)^{4}}\,e^{iz\cdot(P-q-k-p_{X})}\big(\gamma^{\mu}(\slashed{p}+\slashed{q})\gamma^{\nu}\big)^{ji}\bra{P}\overline{\psi}_{j}(0)\ket{X}\bra{X}\psi_{i}(0)\ket{P}\,,
\end{align}
where we in the last step used that $k=p+q$, where $q$ is the photon momentum. By using the translation operator \cref{eq:translation operator} and the completeness relation \cref{eq:complete set of states}, we find
\begin{align}\label{eq:intermediate hadronic tensor}
    W^{\mu\nu}&=\frac{1}{2}\sum_{q}Q_{q}^{2}\int d^{4}p\,\delta((p+q)^{2})\theta(p^{0}+q^{0})\,\Phi_{ij}(p)\big(\gamma^{\mu}(\slashed{p}+\slashed{q})\gamma^{\nu}\big)^{ji}\nonumber
    \\
    &=\frac{1}{2}\sum_{q}Q_{q}^{2}\int d^{4}p\,\delta((p+q)^{2})\theta(p^{0}+q^{0})\,\text{tr}\Big(\Phi(p)\gamma^{\mu}(\slashed{p}+\slashed{q})\gamma^{\nu}\Big)\,,
\end{align}
where the quark correlator in momentum space is defined as
\begin{align}\label{eq:quark correlator}
    \Phi_{ij}(p)=\int\frac{d^{4}z}{(2\pi)^{4}}\,e^{-ip\cdot z}\bra{P}\overline{\psi}_{j}(z)\psi_{i}(0)\ket{P}\,.
\end{align}

We want to simplify the hadronic tensor further, which is easiest by using light-cone coordinates, see \cref{sec:Appendix Light-cone coordinates}. We can then parametrize the proton, quark and photon momenta as in \cref{eq:light-cone parametrization} and \cref{eq:light-cone photon momenta}. With this parametrization, we can expand the delta function in \cref{eq:intermediate hadronic tensor} in the following way
\begin{align}
    \delta((p+q)^{2})&=\delta(p^{2}+2p\cdot q+q^{2})\nonumber
    \\
    &=\delta(2p^{+}q^{-}+2p^{-}q^{+}-2p_{\perp}\cdot q_{\perp}-Q^{2})\nonumber
    \\
    &\approx\delta(2p^{+}q^{-}-Q^{2})\nonumber
    \\
    &=\delta(2P\cdot q\,\xi-2P\cdot q\,x)\nonumber
    \\
    &=\frac{1}{2P\cdot q}\delta(\xi-x)\,,
\end{align}
where we in the first step used that the quarks are massless, and in the third line we used that $p_{\perp}\ll Q^{2}$. To rewrite the argument we also used the Bjorken variable $x=Q^{2}/2P\cdot q$. We observe that this delta function fixes the momentum fraction to be equal the Bjorken-$x$, just as in the parton model calculation.  

The second simplification we can make is to use the fact that in the infinite momentum frame, the incoming photon hits the quark head-on. Then the outgoing quark will move in the $k^{-}$ direction, giving
\begin{align}
    \slashed{k}&=\slashed{p}+\slashed{q}= \gamma^{+}q^{-}+\mathcal{O}\Big(\frac{1}{P^{+}}\Big)\,,
\end{align}
and with these simplifications the hadronic tensor takes the form
\begin{align}\label{eq:simplified hadronic tensor}
    W^{\mu\nu}&=\frac{1}{2}\sum_{q}Q_{q}^{2}\int d^{4}p\,\delta^{+}((p+q)^{2})\,\text{tr}\Big[\Phi(p)\gamma^{\mu}(\slashed{p}+\slashed{q})\gamma^{\nu}\Big]\nonumber
    \\
    &=\frac{1}{4}\sum_{q}Q_{q}^{2}\int dp^{+}dp^{-}d^{2}p_{\perp}\,\frac{1}{P\cdot q}\,\text{tr}\big[\Phi(p)\gamma^{\mu}\gamma^{+}q^{-}\gamma^{\nu}\big]\delta(\xi-x)\nonumber
    \\
    &=\frac{1}{4}\sum_{q}Q_{q}^{2}\int d\xi dp^{-}d^{2}p_{\perp}\,\frac{P^{+}q^{-}}{P\cdot q}\,\text{tr}\big[\Phi(p)\gamma^{\mu}\gamma^{+}\gamma^{\nu}\big]\delta(\xi-x)\nonumber
    \\
    &=\frac{1}{4}\sum_{q}Q_{q}^{2}\,\text{tr}\big[\Phi(x)\gamma^{\mu}\gamma^{+}\gamma^{\nu}\big]\,,
\end{align}
where we have defined the fourier transformed integrated quark correlator
\begin{align}\label{eq:integrated quark correlator}
    \Phi_{ij}(x)&=\int dp^{-}d^{2}p_{\perp} \Phi_{ij}(x,p^{-},p_{\perp})\nonumber
    \\
    &=\int\frac{dz^{-}}{2\pi}\,e^{-ixP^{+}z^{-}}\bra{P}\overline{\psi}_{j}(0^{+},z^{-},0_{\perp})\psi_{i}(0)\ket{P}\,.
\end{align}

To find the quark parton distribution we use the light-cone contraction $\gamma^{\mu}\gamma^{+}\gamma_{\mu}=-2\gamma^{+}$, such that the trace in \cref{eq:simplified hadronic tensor} can be written as
\begin{align}
    \text{tr}[\Phi(x)\gamma^{\mu}\gamma^{+}\gamma^{\nu}]&=g^{\mu\nu}\text{tr}[\Phi(x)\gamma^{\mu}\gamma^{+}\gamma_{\mu}]\nonumber
    \\
    &=-2g^{\mu\nu}\text{tr}[\Phi(x)\gamma^{+}]\,.
\end{align}
Further, we use that $g^{\mu\nu}=g_{\perp}^{\mu\nu}-\big(n_{+}^{\mu}n_{-}^{\nu}+n_{+}^{\nu}n_{-}^{\mu}\big)$, see \cref{eq:trasnversal tensor}, and insert it into \cref{eq:simplified hadronic tensor}. This will give the following expression for the hadronic tensor
\begin{align}
    W^{\mu\nu}=-\frac{1}{2}g_{\perp}^{\mu\nu}\sum_{q}Q_{q}^{2}\,\text{tr}[\Phi(x)\gamma^{+}]+\big(n_{+}^{\mu}n_{-}^{\nu}+n_{+}^{\nu}n_{-}^{\mu}\big)\sum_{q}Q_{q}^{2}\,\text{tr}[\Phi(x)\gamma^{+}]\,,
\end{align}
and if we compare this expression to \cref{eq:2ndparametrized hadronic tensor}, we find that we must have
\begin{align}
    F_{1}(x)=\frac{1}{2}\sum_{q}Q_{q}^{2}\,\text{tr}[\Phi(x)\gamma^{+}]\,.
\end{align}
However, we already know from \cref{eq:collinear factorization parton model F_1} that this structure function is given by
\begin{align}
    F_{1}(x)=\frac{1}{2}\sum_{q}Q_{q}^{2}f_{q}(x)\,,
\end{align}
from which it follows that the operator definition for the integrated quark PDF is given by
\begin{align}\label{eq:unpolarized quark PDF}
    f_{q/P}(x)=\int\frac{dz^{-}}{4\pi}\,e^{-ixP^{+}z^{-}}\bra{P}\overline{\psi}(0^{+},z^{-},0_{\perp})\gamma^{+}\psi(0)\ket{P}\,,
\end{align}
where the subscript $f_{i/P}$ is the common notation for integrated PDFs of a parton with flavour $i$ inside the proton. From now we will not make this statement as it is implicit in the notation. 

Having found an expression for the parton distribution function in terms of matrix elements, let us take a closer look at how we can make these gauge invariant and subsequently used as a motivation for deriving perturbative distributions.



%%%%%%%%%%%%%%%%%%%%%%%%%%%%%%%%%%%%%%%%%%%%%%%%%%%%%%%%%%%%%%
\subsection{Gauge Invariant Parton Distributions}
The quark correlator in \cref{eq:quark correlator} is not gauge invariant, which can be made clear if we look at the gauge transformation of the fields. In general, fermionic fields transform under local gauge transformations as
\begin{align}
    \psi'(x)&=e^{ig\alpha^{a}(x)t^{a}}\psi(x)
    \\
    \overline{\psi}'(x)&=\overline{\psi}(x)e^{-ig\alpha^{a}(x)t^{a}}\,,
\end{align}
and as a consequence the quark correlator in momentum space transform as
\begin{align}
    \Phi'(p)=\int\frac{d^{4}z}{(2\pi)^{4}}\,e^{-ip\cdot z}\bra{P}\overline{\psi}(z)e^{-i\alpha^{a}(z)t^{a}}e^{ig\alpha^{a}(0)t^{a}}\psi(0)\ket{P}\,,
\end{align}
where the fields are defined at different space-time points and the exponentials inside the matrix element do not cancel. 

A similar problem appeared when trying to define a directional derivative of the fermionic fields in \cref{sec:Wilson lines and Wilson loops}. We used a Wilson line to parallel transport the fields such that they could be subtracted in a meaningful way, giving rise to a directional covariant derivative that ensured gauge invariance. The same procedure does not apply here, but in \cref{sec:Wilson lines and Wilson loops} we had that the Wilson line transformed under gauge transformation as
\begin{align}\label{eq:Wilson line transformation in operator section}
    \mathcal{U'}[y,x]=e^{ig\alpha^{a}(y)t^{a}}\mathcal{U}[y,x])e^{-ig\alpha^{a}(x)t^{a}}\,,
\end{align}
which is exactly the transformation we need to cancel the exponentials. So, let us define the following Wilson line
\begin{align}
    \mathcal{U}[z,0]=\mathcal{P}\exp\Big(-ig\int_{0}^{z}dy^{\mu}A_{\mu}(y)\Big)\,,
\end{align}
where it is understood that $A_{\mu}=A_{\mu}^{a}t^{a}$. We can then use the transformation property of the Wilson line to write down the gauge invariant quark correlator as
\begin{align}
    \Phi(p)=\int\frac{d^{4}z}{(2\pi)^{4}}\,e^{-ip\cdot z}\bra{P}\overline{\psi}(z)\mathcal{U}[z,0]\psi(0)\ket{P}\,.
\end{align}

The integrated quark correlator in \cref{eq:integrated quark correlator} has fields that are fixed along the $z^{-}$ direction, which mean that the quark correlator takes the form
\begin{align}
    \Phi(x)&=\int\frac{dz^{-}}{2\pi}\,e^{-ixP^{+}z^{-}}\bra{P}\overline{\psi}(0^{+},z^{-},0_{\perp})\mathcal{U}[z^{-},0]\psi(0)\ket{P}\,,
    \\
    \mathcal{U}[z^{-},0]&=\mathcal{P}\exp\Big(-ig\int_{0}^{z^{-}}dy^{-}A^{+}(0^{+},y^{-},0_{\perp})\Big)\,,
\end{align}
and it follows that the gauge invariant formulation of the quark parton distribution is 
\begin{align}\label{eq:gauge invaiant unpolarized quark PDF}
    f_{q/P}(x)&=\int\frac{dz^{-}}{2\pi}\,e^{-ixP^{+}z^{-}}\bra{P}\overline{\psi}(0^{+},z^{-},0_{\perp})\gamma^{+}\mathcal{U}(z^{-}\,;0)\psi(0)\ket{P}\,.
\end{align}

We observe that if we choose the light-cone gauge, $A^{+}=0$, the Wilson line is $\mathcal{U}[z^{-},0]=1$. Hence, we reduce the parton distribution to the one defined in \cref{eq:unpolarized quark PDF}, and as long as one stays in light-cone gauge the Wilson line can be neglected. 

For a complete treatment we will also give the operator definition of the gluon distribution. The calculation is very similar to the one we made for the quark distribution, but this is a higher order effect through boson-gluon fusion via a quark, so we will not make the explicit derivation here. If one starts in the light-cone gauge the Wilson line can be ignored, and the integrated gluon PDF can be found to be \cite{Ji_2005,Dominguez_2011},
\begin{align}\label{eq:gauge dependent gluon PDF}
    f_{g/P}(x)=\int\frac{dz^{-}}{2\pi}x P^{+}e^{-ixP^{+}z^{-}}\bra{P}A_{i}^{a}(z^{-})A_{i}^{a}(0)\ket{P}\,,
\end{align}
where $i=-,1,2$ as we have set $A^{+}=0$. 

We know from \cref{sec:Wilson lines and Wilson loops} that the gauge fields transform under gauge transformation as
\begin{align}
    A_{\mu}^{a}t^{a}\rightarrow \frac{i}{g}e^{ig\alpha^{a}t^{a}}D_{\mu}e^{-ig\alpha^{a}t^{a}}\,,
\end{align}
meaning that in the current form the gluon distribution is not gauge invariant. Because of the derivative it would not help if we naively insert a Wilson line as we did for the quark distribution. Let us instead investigate the field strength tensor $F_{\mu\nu}$, with the transformation
\begin{align}
    F_{\mu\nu}\rightarrow e^{ig\alpha^{a}t^{a}}F_{\mu\nu}e^{-ig\alpha^{a}t^{a}}\,,
\end{align}
which we observe transform in a similar fashion as the Wilson line in \cref{eq:Wilson line transformation in operator section}. However, it is not valid to just replace the gauge fields with the field strength. We observe that if we did, and inserted the same Wilson line as for the quark distribution, it would still not be gauge invariant. This is not surprising as the quark fields are defined in the fundamental representation, and the gluon fields in the adjoint representation of $SU(3)$. Therefore, we need to use a Wilson line in the adjoint representation, which we define as
\begin{align}
    \mathcal{U}_{ab}^{A}[z,0]=\mathcal{P}\exp{-ig\int_{0}^{z}dy^{\mu}A_{\mu}^{c}(y)(t^{c})_{ab}}\,,
\end{align}
where $(t^{c})_{ab}=-if^{abc}$ are the generators in the adjoint representation. If we insert this Wilson line between two field strength tensors, we would have a gauge invariant operator. We can then use the field strength and relate it to the gauge fields in the following standard way
\begin{align}\label{eq:}
    F_{\mu\nu}^{a}&=\partial_{\mu}A_{\nu}^{a}-\partial_{\nu}A_{\mu}^{a}+gf^{abc}A_{\mu}^{b}A_{\nu}^{c}\,,
\end{align}
and use that $A^{+}=0$, which gives
\begin{align}
    F_{+i}^{a}&=\partial_{+}A_{i}^{a}\,.
\end{align}
Inserting for $A_{i}$ into \cref{eq:gauge dependent gluon PDF} and integrating by parts yields the gauge invariant gluon PDF
\begin{align}
    f_{g/P}(x)=\int\frac{dz^{-}}{2\pi}\frac{1}{x P^{+}}e^{-ixP^{+}z^{-}}\bra{P}F_{a}^{+i}(z^{-})\mathcal{U}_{ab}^{A}[z^{-},0]F_{b}^{+i}(0)\ket{P}\,,
\end{align}
where the factor of $xP^{+}$ is common for even spin particles, i.e for bosons. There are several additional subtleties when dealing with gluon PDFs that we have not covered, so for more details see \cite{Dominguez_2011}.

%%%%%%%%%%%%%%%%%%%%%%%%%%%%%%%%%%%%%%%%%%%%%%%%%%
\subsection{Parton-in-Parton Distributions}\label{sec:lightcone parton in parton distributions}
In this section we will define one last object that we will have use for in \cref{chap:Resummation in QCD}, namely \emph{parton-in-parton} distributions. As we shall see, these will be useful when we want to renormalize our hadron--hadron cross-section and when we want to refactorize the cross-section in the so-called \emph{threshold region}, making it eligible for resummation. 


To define the parton-in-parton distribution we start from the quark PDF \cref{eq:unpolarized quark PDF} and insert a complete set of final states in the following way
\begin{align}
    f_{q/P}(x)&=\int\frac{dz^{-}}{4\pi}\,e^{-ixP^{+}z^{-}}\bra{P}\overline{\psi}(0^{+},z^{-},0_{\perp})\gamma^{+}\psi(0)\ket{P}\nonumber
    \\
    &=\int\frac{dz^{-}}{4\pi}\,e^{-iz^{-}(P_{n}^{+}-P^{+}+xP^{+})}\sum_{n}\bra{P}\overline{\psi}(0)\ket{n}\gamma^{+}\bra{n}\psi(0)\ket{P}\nonumber
    \\
    &=\frac{1}{2}\sum_{n}\bra{P}\overline{\psi}(0)\ket{n}\gamma^{+}\bra{n}\psi(0)\ket{P}\,\delta(P_{n}^{+}-(1-x)P^{+})\,,
\end{align}
where we used the translation operator to pick out the momentum $P_{n}^{+}-P^{+}$ in the exponential, see \cref{eq:translation operator}. It is understood that the matrix element is an average and sum over spin, as $f_{q/P}$ is by construction unpolarized. Let us then consider the case where $P=q$, i.e. a quark. In that case we must also include an average and sum over colour. Then we can naively write down the quark-in-quark distribution as
\begin{align}\label{eq:expanded matrix element quark-in-quark}
    f_{q/q}(x)&=\frac{1}{4N_c}\sum_{colour}\sum_{spin}\sum_{n}\bra{q}\overline{\psi}(0)\ket{n}\gamma^{+}\bra{n}\psi(0)\ket{q}\,\delta(p_{n}^{+}-(1-x)p^{+})\,,
\end{align}
where we have explicitly written out the average and sum over colour and spin. To evaluate the matrix elements we use that if a quark operator $\psi$ act on a quark state $\ket{q}$, we get
\begin{align}\label{eq:quark operator on quark state}
    \psi(z)\ket{q}=e^{-ip\cdot z}u(p)\ket{0}\,,
\end{align}
which mean that the matrix elements in \cref{eq:expanded matrix element quark-in-quark} only have a contribution if $n=0$. This scenario corresponds to a quark that just \textquote{travels} along without changing. Thus, we conclude that the expression we have written down is the leading order expansion of $f_{q/q}$, where there is no gluon radiation from the quark line. It is therefore important to point out that the only way a quark can \textquote{change} is by radiating a gluon, and in this scenario there is no gluon to make that change. Hence, by using \cref{eq:quark operator on quark state}, we find the leading order result for the quark-in-quark distribution,
\begin{align}
    f_{q/q}^{(0)}(x)&=\frac{1}{4p^{+}}\sum_{s}u_{i}^{s}(p)\bar{u}_{j}^{s}(p)\gamma_{ji}^{+}\,\delta(1-x)\nonumber
    \\
    &=\frac{1}{4p^{+}}\text{tr}[\slashed{p}\gamma^{+}]\delta(1-x)\nonumber
    \\
    &=\delta(1-x)\,,
\end{align}
which states that in the absence of interaction the quark remains itself. For general partons, we can therefore write 
\begin{align}
    f_{i/j}^{(0)}(x)=\delta_{ij}\delta(1-x)\,,
\end{align}
where $\delta_{ij}$ is inserted to make sure that the parton does not change without a gauge interaction. If we assume that $f_{i/j}$ can be calculated in perturbation theory, the expansion take the form
\begin{align}
    f_{i/j}(x,\mu^{2})=\delta_{ij}\delta(1-x)+\sum_{n=1}^{\infty}\Big(\frac{\alpha_s}{2\pi}\Big)^{n}f_{i/j}^{(n)}(x,\mu^{2})\,.
\end{align}

Before we proceed to the general treatment of radiation from quark lines, we can actually \textquote{guess} the first order correction. In \cref{sec:QCD and Collinear factorization} we calculated a diagram where a gluon was emitted from an incoming quark line, see \cref{fig:DISgluonemission}. The divergent part of the diagram was found to be  
\begin{align}
    \hat{F}_{2}\big|_{\text{div}}=Q_{q}^{2}\,\frac{\alpha_s}{2\pi}x\,P_{q/q}^{(0)}(x)\ln{\frac{Q^{2}}{Q_{0}^{2}}}\,,
\end{align}
where the quark--quark splitting function naturally appeared as the process involved a quark emitting a gluon and continued on as another quark. Thus, if $f_{q/q}^{(1)}$ describes a quark emitting a gluon and continuing on as another quark, the natural guess would be
\begin{align}\label{eq:momentum cut-off quark-in-quark}
    f_{q/q}^{(1)}(x)\propto P_{q/q}^{(0)}(x)\,,
\end{align}
i.e. it has to be proportional to the splitting functions as it describes exactly what the splitting function describes.

For the general treatment, we need to implement gauge interactions systematically. But we already know from \cref{sec:Wilson lines and Wilson loops} that this can be done by dressing the quark line with a Wilson line. Therefore, we can use our expression for the gauge invariant parton distributions, see \cref{eq:gauge invaiant unpolarized quark PDF}, and expand the Wilson line. To do this, we can first use the Wilson line relation
\begin{align}
    \mathcal{U}[z,0]=\mathcal{U}^{\dagger}[+\infty,z]\mathcal{U}[+\infty,0]\,,
\end{align}
and use \cref{eq:eq:wilso dress fermion} to write
\begin{align}
    \Psi(z)=\mathcal{U}[\infty,0]\psi(z)\,,
    \\
    \overline{\Psi}(z)=\bar{\psi}(z)\mathcal{U}^{\dagger}[\infty,z]\,.
\end{align}

With these definitions the quark-in-quark distribution can be defined as 
\begin{align}
    f_{q/q}(x)=\int\frac{dz^{-}}{4\pi}e^{-ixp^{+}z^{-}}\bra{q}\overline{\Psi}(z^{-})\gamma^{+}\Psi(0)\ket{q}\,,
\end{align}
also commonly rewritten by inserting a complete set of states, giving
\begin{align}\label{eq:Collins definition of parton-in-parton}
    f_{q/q}(x)=\int\frac{dz^{-}}{4\pi}e^{-ixp^{+}z^{-}}\sum_{n}\bra{q}\overline{\Psi}(z^{-})\ket{n}\gamma^{+}\bra{n}\Psi(0)\ket{q}\,.
\end{align}

The expansion of a semi-infinite Wilson line is given in \cref{eq:path bounded from above} and \cref{eq:path bounded from below}, from which \cref{eq:Collins definition of parton-in-parton} can be calculated in perturbation theory. We will not perform this calculation explicitly, but we can find its pole structure by comparing with the amplitude in \cref{eq:Dressed fermion amplitude}. When squaring this amplitude, one finds that the $\mathcal{O}(g
^{2})$ term describes a scaleless integral. Scaleless integrals has the pole structure given in \cref{eq:scaleless integral}. The UV-divergence can be removed by counterterms, leaving the IR-divergence in $1/\epsilon$. An explicit calculation to $\mathcal{O}(\alpha_s)$ in dimensional regularization gives \cite{Collins:1989gx},
\begin{align}\label{eq:quark in quark PDF to one-loop}
    f_{q/q}(x)=\delta(1-x)-\frac{\alpha_s}{2\pi}\Big(\frac{4\pi\mu^{2}}{\mu_{F}^{2}}\Big)^{\epsilon}\frac{\Gamma(1-\epsilon)}{\Gamma(1-2\epsilon)}\frac{1}{\epsilon}P_{q/q}^{(0)}(x)+\mathcal{O}(\alpha_{s}^{2})\,,
\end{align}
where the $1/\epsilon$ factor appear as a consequence of a collinear singularity. From these considerations it follows that the first order correction $f_{q/q}^{(1)}$ has the structure given in \cref{eq:momentum cut-off quark-in-quark}. We could also have gluons or quarks radiating from a gluon in the initial state, and the same arguments would apply with the difference in which splitting function that would appear in the expression. We should emphasize that there are several ways of defining these parton-in-parton distributions. The choice made in \cref{eq:Collins definition of parton-in-parton} are lightcone distributions with a fixed momentum fraction. In \cref{sec:Resummation Drell-yan} we will define distributions at fixed energy instead, which make them more suitable for resummation. The most important point here is that one can use these parton-in-parton distributions to absorb collinear singularities that appear when gluons radiate from quark lines in scattering processes. This is a very important feature when doing resummation that will be used on several occasions. 

   






 


\section{Drell-Yan Cross Section in QCD}\label{sec:Drell-Yan Hadronic Cross Section}
In this section we will make another calculation that historically has been important in the study of QCD, namely the Drell-Yan process.  

In 1970, the first observation of a $\mu^{+}\mu^{-}$ in hadron-hadron collision was observed \cite{Christenson:1970um}. By applying the parton model Drell and Yan were the first to give a theoretical prediction of this process \cite{Drell:1970wh}. In the modern framework of QCD, the impulse approximation of the parton model is as in DIS replaced by the more precise concept of factorization. Then it can be proven that the hadronic Drell-Yan cross section can be written as a convolution of a perturbative calculable partonic cross section and universal process independent parton distribution functions \cite{Collins:1989gx}. Without specifying the kinematics, the hadronic cross section can be written as
\begin{align}
  \sigma_{h_1\,h_2}(P_1,P_2)=\sum_{i,j}\int dx_1\,dx_2\,f_{i/h_1}(x_1,\mu_{F})f_{j/h_1}(x_1,\mu_{F})\,\hat{\sigma}_{ij}(x_1P_1,x_2P_2,\mu_{F})\,,
\end{align}
which is the form we expected from the parton model, and as in DIS, the parton distribution functions and the partonic cross section has acquired an dependancy on the factorization scale. 

In the simplest case, a Drell-Yan process is the annihilation of a quark-antiquark pair into a virtual photon, which subsequently produces a pair of leptons with invariant mass $Q^{2}$. As leptons are blind to the strong interaction, there will not be any final state gluon radiation. The consequence is that all radiation comes from the initial state quark-antiquark pair, and therefore we have a clear probe of the behaviour of coloured particles. Consequently, the Drell-Yan process is an effective way of studying the internal structure of hadrons.

The main objective of this section is to investigate the divergences appearing in higher-order calculations, and how to deal with them using renormalization techniques. The calculation will proceed through the use of dimensional regularization, where the poles will manifest themselves in $1/\epsilon$. Further, close to the threshold, the finite result has contributions that become large, and these must eventually be handled using resummation techniques. To this end, we will focus on the next-to-leading order (NLO) calculation, where the initial state quarks emit a gluon.

\subsection{LO Drell-Yan Cross Section}
For the sake of completeness we will first sketch the leading order result. At this order we write the partonic process as $q(p)+\bar{q}(p')\rightarrow l^{-}(k)+l^{+}(k')$, where the corresponding Feynman diagram is given in \cref{fig:Drell-Yan LO}.
\begin{figure}
  \centering
  \includegraphics[scale=0.4]{Figures/DrellYanLO}
  \caption{Drell-Yan process at leading order.}
  \label{fig:Drell-Yan LO}
\end{figure}\noindent
Here $p$ and $p'$ denote the momenta of the incoming quark and antiquark, and likewise $k$ and $k'$ denote the momentum of the outgoing leptons. We can directly write down the amplitude from the basic Feynman rules\footnote{To see where all these term comes from, see \cref{eq:QED feynman rule} for the quark-photon vertex, \cref{eq:photon propagator without gauge choice} for the photon propagator and \cref{sec:canonical quantization free theories} for the appearance of spinors.}
\begin{align}
    i\mathcal{M}=i\delta_{ij}\frac{Q_{q}e^{2}}{q^{2}}\big[\bar{v}(p')\gamma^{\mu}u(p)\big]\big[\bar{u}(k)\gamma_{\mu}v(k')\big]\,,
\end{align}
where $\delta_{ij}$ is a colour conserving factor for the quark vertex, and $Q_{q}$ is the fractional charge of the quarks. As usual we are only interested in the spin and colour averaged amplitude,
\begin{align}
    \langle\,|\mathcal{M}|^{2}\rangle=\frac{1}{N_{c}N_{s}^{2}}\frac{Q_{q}^{2}e^{4}}{\hat{s}^{2}}H^{\mu\nu}L_{\mu\nu}\,,
\end{align}
where $N_c$ and $N_s$ denote the number of colours and spin, and we  have defined hadronic and leptonic tensors
\begin{align}
    H^{\mu\nu}&=\text{Tr}\big[\slashed{p'}\gamma^{\mu}\slashed{p}\gamma^{\nu}\big]\,,
    \\
    L_{\mu\nu}&=\text{Tr}\big[\slashed{k}\gamma_{\mu}\slashed{k'}\gamma_{\nu}\big]\,,
\end{align}
and the contraction yields
\begin{align}
    H^{\mu\nu}L_{\mu\nu}=\hat{s}^{2}(1+\cos\theta^{2})\,,
\end{align}
where $\theta$ is the centre of mass scattering angle, and $\hat{s}$ is the partonic centre of mass energy, $\hat{s}=(p+p')^{2}$. The differential cross section is given by
\begin{align}
    d\hat{\sigma}_{q\bar{q}}^{(0)}=\frac{1}{2\hat{s}}\langle\,|\mathcal{M}|^{2}\rangle d\mathcal{P}^{(2)}\,,
\end{align}
where $d\mathcal{P}^{(2)}$ is the two-body phase space of the final state leptons, see \cref{eq:n-body phase space}. The integrated cross section is given by
\begin{align}
    \hat{\sigma}^{(0)}=\frac{4\pi Q_{q}^{2}\alpha^{2}}{3N_{c}\hat{s}}\,.
\end{align}
We can write this as a differential cross section by using that
\begin{align}
    1=\int dQ^{2}\delta(\hat{s}-Q^{2})\,,
\end{align}
which is merely a statement that the invariant mass of the $q\bar{q}$ pair that annihilates into the photon, matches the invariant mass of the photon. This identity can also be written on the form
\begin{align}
    1=\frac{1}{\hat{s}}\int dQ^{2}\delta(1-\frac{Q^{2}}{\hat{s}})\,,
\end{align}
giving the differential cross section in $Q^{2}$
\begin{align}
    \frac{d\hat{\sigma}_{q\bar{q}}^{(0)}}{dQ^{2}}=\frac{4\pi Q_{q}^{2}\alpha^{2}}{3N_{c}\hat{s}\,Q^{2}}\delta(1-z)=\hat{\sigma}_{0}\,\delta(1-z)\,,
\end{align}
where we defined the partonic Born cross-section $\hat{\sigma}_0=(4\pi Q_{q}^{2}\alpha^{2})/(3N_{c}\hat{s}Q^{2})$, and the partonic threshold variable $z=Q^{2}/\hat{s}$. For later purposes, we want to make some remarks about the structure of the leading order diagram. The photon and leptons do not feel the strong interaction, which means that even at higher orders, the final state part of the amplitude decouples from the initial part. Further, the electromagnetic corrections are negligible compared to the strong corrections so any photon radiation is neglected. To find the higher-order corrections, we only have to consider the corrections for the production of a virtual photon. The leptonic part of the hard process contributes with a factor $\alpha/3Q
^{2}$, where $Q^{2}$ is the invariant mass of the final state leptons.

\begin{figure}
    \centering
    \includegraphics[scale=0.3]{Figures/radiativecorrDY.pdf}
    \caption{NLO diagrams contributing to the Drell-Yan process. Top: Real gluon emission. Bottom: Virtual correction.}
    \label{fig:NLO drell-yan}
\end{figure}
%%%%%%%%%%%%%%%%%%%%%%%%%%%%%%%%%%%%%%%%%%%%%%%%%%%%%%%%%%%%%%%%%%
\subsection{NLO Drell-Yan Cross Section}\label{sec:NLO drell yan calculation}
Let us now go beyond the leading order approximation and calculate the $\mathcal{O}(\alpha_s)$ correction to the Drell-Yan process. The contributing diagrams is illustrated in \cref{fig:NLO drell-yan}, where the upper diagrams corresponds to real gluon emission, and the lower diagram is the virtual correction. As we will see, the difficulty due to divergences in the integral is substantial. To treat these divergences we will use dimensional regularization, see \cref{sec:dimensional regularization} for more detail. One important feature to keep in mind is that the UV-divergences is treated by evaluating $\epsilon>0$ and the IR-divergences for $\epsilon<0$. 

Now, by counting order in $g_s$ in \cref{fig:NLO drell-yan}, we see that the virtual correction has one order higher than the real emission and can not be squared in order to contribute to NLO. Therefore, one must multiply the virtual correction amplitude with the leading order amplitude found from \cref{fig:Drell-Yan LO}, which contributes to wanted order in $\alpha_s$.
The real contribution corresponds to the emission of gluons that are on mass-shell, i.e. there are no undetermined loop momenta that has to be integrated over. However, the phase space integration contains two kinds of singularities. The first is when the gluon is emitted collinearly to the emitting quark. The second singularity appears if the gluon momentum is soft, $k\rightarrow 0$. Both of these divergences are what is called infrared-singularities. Virtual corrections emerge when the initial quarks exchange a gluon. This gluon is virtual, and the integral is over undetermined loop momenta, leading to ultraviolet-divergences. To handle these divergences, we will use dimensional regularization, and work in $d=4-2\epsilon$ dimensions.

\subsection*{Differential cross section}
By applying the QCD Feynman rules, the amplitude for real gluon emission can be written as
\begin{align}
    i\mathcal{M}=&Q_{q}\,e\,g_{s}\,\mu^{(4-d)/2}\,(t^{a}_{ij})\varepsilon_{\alpha}^{*}(k)\varepsilon_{\mu}^{*}(q)\big[\bar{v}(p')A^{\mu\alpha}u(p)\big]\,,
\end{align}
where we have made the usual substitution $g_{s}\rightarrow \mu^{\epsilon}g_s$. We have also used the abbreviation
\begin{align}
    A^{\mu\alpha}=\gamma^{\mu}\frac{i(\slashed{p}-\slashed{k})}{(p-k)^{2}}\gamma^{\alpha}+\gamma^{\alpha}\frac{i(\slashed{p}-\slashed{q})}{(p-q)^{2}}\gamma^{\mu}\,,
\end{align}
where $k$ is the gluon momenta and $q$ is the massive photon momenta. The spin and colour averaged amplitude is given by
\begin{align}\label{eq:ugly NLO averaged amplitude DY}
    \langle\,|\mathcal{M}|^{2}\rangle&=\mathcal{C}\frac{Q_{q}^{2}e^{2}g_{s}^{2}\mu^{(4-d)}}{N_{s}^{2}}\text{Tr}[\slashed{p'}A^{\mu\alpha}\slashed{p}A_{\alpha\mu}]\nonumber
    \\
    &=C_{F}\frac{Q_{q}^{2}e^{2}g_{s}^{2}\mu^{(4-d)}}{N_{c}\,N_{s}^{2}}\,2(d-2)\Big(2\hat{s}\frac{Q^{2}}{\hat{t}\hat{u}}+2(d-4)+(d-2)\Big[\frac{\hat{t}}{\hat{u}}+\frac{\hat{u}}{\hat{t}}\Big]\Big)\nonumber
    \\
    &=C_{F}\frac{Q_{q}^{2}e^{2}g_{s}^{2}\mu^{2\epsilon}}{N_{c}\,N_{s}^{2}}\,8(1-\epsilon)\Big(2\hat{s}\frac{Q^{2}}{\hat{t}\hat{u}}-2\epsilon+(1-\epsilon)\Big[\frac{\hat{t}}{\hat{u}}+\frac{\hat{u}}{\hat{t}}\Big]\Big)\,,
\end{align}
where the terms with $d=4-2\epsilon$ appears because of the modified Dirac algebra in $d$-dimensions \cref{sec:Appendix Dirac gamma matrices}. The color factor for this process follows from the QCD group structure\footnote{See \cite{Peskin:257493} for a more elaborate treatment of the colour sums.} 
\begin{align}
    \mathcal{C}=\frac{1}{N_{c}}\text{Tr}[t^{a}t^{a}]=\frac{C_{F}}{N_C}\,,
\end{align}
and the partonic Mandelstam variables is defined as
\begin{align}\label{eq:partonic Mandelstam}
    \hat{s}&=(p+p')^{2}=(q+k)^{2}\,,
    \\
    \hat{t}&=(p-q){2}=(p'-k)^{2}\,,
    \\
    \hat{u}&=(p-k)^{2}=(p'-q)^{2}\,.
\end{align}
The partonic differential cross section is then given by
\begin{align}
    \frac{d\hat{\sigma}_{q\bar{q}}^{r}}{dQ^{2}}=\frac{1}{2}\Big(\frac{\alpha}{3Q^{2}}\Big)\int\,d\mathcal{P}^{(2)}\,\langle\,|\mathcal{M}|^{2}\rangle\,,
\end{align}
where the bracket represents lepton part appearing at all order. The differential phase space is over the massless gluon momenta $k$ and the virtual photon momenta $q$. In $d$-dimensions, the differential phase space can be written as
\begin{align}
    d\mathcal{P}=\frac{1}{8\pi}\Big(\frac{4\pi}{Q^{2}}\Big)^{\epsilon}\frac{(1-z)^{1-2\epsilon}\,z^{\epsilon}}{\Gamma(1-\epsilon)}\big(y(1-y))\big)^{-\epsilon}\,dy\,,
\end{align}
where $y=\frac{1}{2}(1+\cos\theta)$, $\Gamma$ is the Euler-Gamma function and z is the threshold variable $z=Q^{2}/\hat{s}$. Notice that the integral is to be taken over the dimensionless quantity $y$, so we should find a way of rewriting the averaged amplitude in terms $y$ as well. We already have the partonic centre of mass-energy, $\hat{s}$ in terms of $z$, so by writing out $\hat{t}$ and $\hat{u}$ we find
\begin{align}
    \hat{t}&=-\frac{Q^{2}}{z}(1-z)(1-y)\,,
    \\
    \hat{u}&=-\frac{Q^{2}}{z}(1-z)y\,.
\end{align}
With these definitions, the averaged amplitude in \cref{eq:ugly NLO averaged amplitude DY} takes the form
\begin{align}
     \langle\,|\mathcal{M}|^{2}\rangle&=C_{F}\frac{Q_{q}^{2}e^{2}g_{s}^{2}\mu^{2\epsilon}}{N_{c}\,N_{s}^{2}}\,8(1-\epsilon)\Big(\frac{2z}{(1-z)^{2}(1-y)y}\nonumber
     \\
     &\hspace{0.2cm}+(1-\epsilon)\Big[\frac{1-y}{y}+\frac{y}{1-y}\Big]-2\epsilon\Big)\,.
\end{align}
At this point it is worth commenting the different terms in this expression: we have that $(1-z)$ is the momentum fraction of the emitted gluon, and if the gluon is soft, $z=1$, we have a singularity. Further, if the gluon is emitted collinearly to the emitting quarks, we have that $y=1$, resulting in another singularity. Hence, we have both a collinear and soft divergence.

However, if we keep $\epsilon$ finite, the singularities can be extracted by using the Euler-Beta integral. The singularities will then manifest themselves in $\epsilon$ singularities. Writing out all terms, the differential cross-section for real emission takes the form
\begin{align}
    \frac{d\hat{\sigma}_{q\bar{q}}^{r}}{dQ^{2}}&=\frac{1}{2\hat{s}}\Big(\frac{\alpha}{3\pi Q^{2}}\Big)\Big[C_{F}\frac{Q_{q}^{2}e^{2}g_{s}^{2}\mu^{2\epsilon}}{N_{c}\,N_{s}^{2}}\Big]\frac{1}{8\pi}\Big(\frac{4\pi}{Q^{2}}\Big)^{\epsilon}\frac{(1-z)^{1-2\epsilon}\,z^{\epsilon}}{\Gamma(1-\epsilon)}8(1-\epsilon)\nonumber
    \\
    &\hspace{0.3cm}\times\int_{0}^{1}dy\,\big(y(1-y)\big)^{-\epsilon}\Big(\frac{2z}{(1-z)^{2}(1-y)y}+(1-\epsilon)\Big[\frac{1-y}{y}+\frac{y}{1-y}\Big]-2\epsilon\Big)\nonumber
    \\
    &=C_{F}\hat{\sigma}_{0}\frac{\alpha_s}{2\pi}\Big(\frac{4\pi\mu^{2}}{Q^{2}}\Big)^{\epsilon}\frac{(1-\epsilon)}{\Gamma(1-\epsilon)}(1-z)^{1-2\epsilon}z^{\epsilon}\nonumber
    \\
    &\hspace{0.3cm}\times\Big[\frac{2z}{(1-z)^{2}}B(-\epsilon,-\epsilon)+2(1-\epsilon)B(-\epsilon,2-\epsilon)-2\epsilon B(1-\epsilon,1-\epsilon)\Big]\nonumber
    \\
    &=C_{F}\hat{\sigma}_{0}\frac{\alpha_s}{\pi}A(\epsilon)\frac{z^{\epsilon}}{\epsilon}\Big(-2z(1-z)^{-1-2\epsilon}-(1-z)^{1-2\epsilon}\Big)\,,
\end{align}
where we for notational simplicity have neglected several factors that are finite for $\epsilon\rightarrow 0$. The integral over $y$ was performed by the Euler-Beta integral
\begin{align}
    B(a,b)=\int_{0}^{1}dt\,t^{a-1}(1-t)^{b-1}=\frac{\Gamma(a)\Gamma(b)}{\Gamma(a+b)}\,,
\end{align}
and we defined the prefactor $A(\epsilon)$ as
\begin{align}
    A(\epsilon)=\Big(\frac{4\pi\mu^{2}}{Q^{2}}\Big)^{\epsilon}\frac{\Gamma(1-\epsilon)}{\Gamma(1-2\epsilon)}\,.
\end{align}

We still have a singularity as $z\rightarrow 1$. To make all singularities manifest in $\epsilon$, we can expand the terms inside the bracket by the use of plus distributions, see \cref{sec:Appendix Plus Distributions}. We write this as
\begin{align}\label{eq:DRell Yan plus distribution example}
    (1-z)^{-1-2\epsilon}&=-\frac{1}{2\epsilon}\delta(1-z)+\Big[\frac{1}{1-z}\Big]_{+}-2\epsilon\Big[\frac{\ln(1-z)}{1-z}\Big]_{+}+\mathcal{O}(\epsilon^{2})\,,
    \\
    z^{\epsilon}&=1+\epsilon \ln z+\mathcal{O}(\epsilon^{2})\,.
\end{align}
With these considerations, the final result for real gluon emission takes the form
\begin{align}
    \frac{d\hat{\sigma}_{q\bar{q}}^{r}}{dQ^{2}}=C_{F}&\hat{\sigma}_{0}\frac{\alpha_s}{\pi}A(\epsilon)\Big(\frac{1}{\epsilon^{2}}\delta(1-z)-\frac{1}{\epsilon}\Big[\frac{1+z^{2}}{1-z}\Big]_{+}\,,\nonumber
    \\
    &+2(1+z^{2})\Big[\frac{\ln(1-z)}{(1-z)}\Big]_{+}-\frac{1+z^{2}}{1-z}\ln z\Big)\,.
\end{align}

We have now managed to bring all singularities on $\epsilon$ form. The term $1/\epsilon^{2}$ is an infrared singularity due to simultaneously soft and collinear emission, and the term $1/\epsilon$ is due to collinear emission. 

To any given order in perturbation theory, we must consider all possible diagrams at that specific order and add them at the end. Therefore, we must find the virtual contribution and see if some of the singular terms cancel. The only virtual contribution comes from the lower diagram in \cref{fig:NLO drell-yan}. There are two additional diagrams at this order that could contribute, the quark self-energy diagrams, but it can be shown that these exactly cancel each other. The virtual contribution contains both infrared and ultraviolet-singularities. To separate these singularities, we do as in \cref{eq:scaleless integral} and name them $\epsilon_{UV}$ and $\epsilon_{IR}$. 

The amplitude for the virtual correction is given by
\begin{align}\label{eq:DY virtual amplitude}
    \mathcal{M}^{\mu}=&t_{ik}^{a}t_{kj}^{a}\mu^{(4-d)}\int\frac{d^{d}k}{(2\pi)^{2}}\bar{v}(p')(ig_s\gamma^{\alpha})\frac{i(\slashed{p'}+\slashed{k})(ie\gamma^{\mu})i(\slashed{k}-\slashed{p}))}{(p'+k)^{2}(p-k)^{2}}(ig_s\gamma^{\beta})u(p)\,D_{\alpha\beta}(k)\nonumber
    \\
    &=eg_s^{2}C_{F}\mu^{(d-4)}\int\frac{d^{d}k}{(2\pi)^{2}}\bar{v}(p')\gamma^{\alpha}\frac{(\slashed{p'}+\slashed{k})\gamma^{\mu}(\slashed{k}-\slashed{p})}{k^{2}(p'+k)^{2}(p-k)^{2}}\gamma^{\beta}u(p)\nonumber
    \\
    &=\bar{v}(p')(\Gamma^{\mu})u(p)\,,
\end{align}
where we have defined the vertex correction $\Gamma^{\mu}$. The integrand can be simplified by the Dirac algebra in $d$-dimensions and the massless Dirac equation. Further, by applying the Feynman parameter method introduced in \cref{sec:dimensional regularization}, the integral can be turned into a Euler-Beta integral. 

The propagator factors in \cref{eq:DY virtual amplitude}, are the same as the ones we derived in \cref{eq:virtual gluon feynman parametrization}. The numerator is straightforward to manipulate, but tedious. The resulting vertex correction is given by
\begin{align}
    \Gamma^{\mu}=\gamma^{\mu}&C_{F}\frac{\alpha_s}{4\pi}\Big(\frac{4\pi\mu^{2}}{Q^{2}}\Big)^{\epsilon}(-1)^{\epsilon}\Gamma(1+\epsilon)\Gamma(1-\epsilon)\frac{\Gamma(1-\epsilon)}{\Gamma(1-2\epsilon)}\nonumber
    \\
    &\Big(\frac{1}{\epsilon_{UV}}-\frac{2}{\epsilon_{IR}^{2}}-\frac{4}{\epsilon_{IR}}-8+\mathcal{O}(\epsilon)\Big)\nonumber
    \\
    &=\gamma^{\mu}C_{F}\frac{\alpha_s}{4\pi}A(\epsilon)\Big(\frac{1}{\epsilon_{UV}}-\frac{2}{\epsilon_{IR}^{2}}-\frac{4}{\epsilon_{IR}}-8+\frac{2\pi^{2}}{3}+\mathcal{O}(\epsilon)\Big)\,,
\end{align}
where we expanded the factors
\begin{align}
    (-1)^{\epsilon}\Gamma(1+\epsilon)\Gamma(1-\epsilon)=1-\frac{\pi^{2}}{3}\epsilon^{2}+\mathcal{O}(\epsilon^{3})\,.
\end{align}
To regulate the UV-singularities we have to add a counterterm. We are not considering electroweak corrections, so for our vertex correction the only reasonable counterterm to add is the quark self-energy. In dimensional regularization this contribution yields a scaleless integral, which we encountered in \cref{eq:scaleless integral}. To remove the UV-divergence we add\footnote{As discussed in \cref{sec:renormalized perturbation theory}, this is always allowed for a renormalizable theory.}
\begin{align}
    \delta_{\Gamma}\gamma^{\mu}=-\gamma^{\mu}C_{F}\frac{\alpha_s}{4\pi}A(\epsilon)\Bigg(\frac{1}{\epsilon_{UV}}-\frac{1}{\epsilon_{IR}}\Bigg)\,.
\end{align}
This results in the UV-finite vertex correction
\begin{align}\label{eq:UV-finite vertex}
    \Gamma_{R}^{\mu}&=\gamma^{\mu}C_{F}\frac{\alpha_s}{4\pi}A(\epsilon)\Big(-\frac{2}{\epsilon_{IR}^{2}}-\frac{3}{\epsilon_{IR}}-8+\frac{2\pi^{2}}{3}+\mathcal{O}(\epsilon)\Big)\nonumber
    \\
    &=\gamma^{\mu}\mathcal{F}\,.
\end{align}

The differential cross section for the virtual correction is obtained by considering the interference between the leading order \cref{fig:Drell-Yan LO} and the vertex correction \cref{fig:NLO drell-yan}. This corresponds to making the substitution 
\begin{align}
    e\gamma^{\mu}\rightarrow\Gamma^{\mu}_{R}=e\gamma^{\mu}\mathcal{F}
\end{align}
which leads to the differential cross section
\begin{align}
    \frac{d\hat{\sigma}_{q\bar{q}}^{v}}{dQ^{2}}&=\frac{d\hat{\sigma}_{q\bar{q}}^{(0)}}{dQ^{2}}2\text{Re}(\mathcal{F})
    \\
    &=C_{F}\hat{\sigma}_{0}\frac{\alpha_s}{\pi}A(\epsilon)\Big(-\frac{1}{\epsilon^{2}}-\frac{3}{2\epsilon}-4+\frac{\pi^{2}}{3}\Big)\delta(1-z)\,.
\end{align}

The real and virtual contributions are now on the same form, such that we can easily add them, giving the NLO result
\begin{align}\label{eq:final NLO partonic qq}
    \Big(\frac{d\hat{\sigma}_{q\bar{q}}}{dQ^{2}}\Big)_{\text{NLO}}&=\frac{d\hat{\sigma}_{q\bar{q}}^{r}}{dQ^{2}}+\frac{d\hat{\sigma}_{q\bar{q}}^{v}}{dQ^{2}}\nonumber
    \\
    &=C_{F}\hat{\sigma}_{0}\frac{\alpha_s}{\pi}A(\epsilon)\Big[-\frac{1}{\epsilon}\Big(\Big[\frac{1+z^{2}}{1-z}\Big]_{+}+\frac{3}{2}\delta(1-z)\Big)+\Big(\frac{\pi^{2}}{3}-4\Big)\delta(1-z)\nonumber
    \\
    &\hspace{1.5cm}+2(1+z^{2})\Big[\frac{\ln(1-z)}{1-z}\Big]_{+}-\frac{1+z^{2}}{(1-z)}\ln z\Big]\,.
\end{align}

We observe that in the sum over real and virtual contributions, some of the divergences have canceled. This is a manifestation of the Kinoshita-Lee-Nauenberg (KLN) theorem \cite{Lee:1964is, Kinoshita:1975bt}. The KLN-theorem states that in the sum of all diagrams, the result is to be free of IR-divergences. But we still have a collinear divergence, which we soon will show how to deal with.  

In general, we can write the partonic cross section as the perturbative expansion
\begin{align}
    \frac{d\hat{\sigma}}{dQ^{2}}&=\sum_{n=0}^{\infty}\Big(\frac{\alpha_s}{\pi}\Big)^{n}\frac{d\hat{\sigma}^{(n)}}{dQ^{2}}\,,\nonumber
    \\
    &=\frac{d\hat{\sigma}^{(0)}}{dQ^{2}}+\frac{\alpha_s}{\pi}\frac{d\hat{\sigma}^{(1)}}{dQ^{2}}+\cdots\nonumber
    \\
    &=\hat{\sigma}_{0}\big(\omega^{(0)}+\frac{\alpha_s}{\pi}\omega^{(1)}+\cdots\big)\,,
\end{align}
where we separated out the Born cross section and defined a hard function $\omega^{(n)}$. With this notation, we have that the hard functions up to $\mathcal{O}(\alpha_s)$ are given by
\begin{align}
    \omega^{(0)}&=\delta(1-z)\,,\nonumber
    \\
    \omega^{(1)}&=-A(\epsilon)\frac{1}{\epsilon}P_{q/q}^{(0)}+A(\epsilon)\Big[\Big(\frac{\pi^{2}}{3}-4\Big)\delta(1-z)\nonumber
    \\
    &\hspace{1.5cm}+2(1+z^{2})\Big[\frac{\ln(1-z)}{1-z}\Big]_{+}-\frac{1+z^{2}}{1-z}\ln z\Big]\,,
\end{align}
where we have inserted the splitting function $P_{q/q}^{(0)}$, see \cref{eq:qq splitting function}.

As mentioned above some of the IR-divergences has canceled, but there still remain a collinear singularity in $1/\epsilon$. The reason this collinear singularity is still present and did not cancel according to the KLN theorem is that we are considering massless particles. To treat the remaining singularity we can use the factorization property of QCD. In DIS we defined \textquote{bare} parton distributions to cancel the divergence. However, we have already seen that we can define parton-in-parton distributions that contains collinear divergences, see \cref{eq:Collins definition of parton-in-parton}. Hence, we can use these parton-in-parton distributions instead, which we will show how to do in the next section.

%recall the underlying assumption behind the factorization theorem in QCD. The hadronic cross section could be factorized in a hard scattering part convoluted with the universal parton distribution functions. The hard part describes the physics on short scale, and the parton distribution functions describes the physics on long scale. The crucial point is that the only observable quantity is the hadronic cross section, meaning that we can renormalize the cross section by refactorizing the partonic cross section using parton-in-parton distributions.

\subsection{Renormalization of NLO cross section}\label{sec:renormalizing drell-yan}
According to the QCD factorization theorem, the hadronic Drell-Yan cross section can be written as\footnote{This is proven to hold up to corrections of $\mathcal{O}(1/Q^{2})$, see \cite{Collins:1989gx}. In \cref{sec:NLO drell yan calculation} we only considered the $q\bar{q}$ process, but to be general we will use generic subscripts.}
\begin{align}\label{eq:Factorized hadronic Dre-Yan}
   \frac{ d\sigma_{h_1h_2}}{dQ^{2}}=\sigma_{0}W(\tau,Q,\mu,\alpha_s)\,,
\end{align}
where\footnote{This is really a Mellin convolution, see \cref{sec:Appendix Mellin Transform}.}
\begin{align}\label{eq:hadronic convolution}
    W(\tau,Q,\mu,\alpha_s)=\sum_{i,j}Q_{q}^{2}\int \frac{dx_{1}}{x_1}\frac{dx_{2}}{x_2}f_{i/h_1}(x_1,\mu)f_{j/h_2}(x_2,\mu)w_{ij}\Big(z,Q,\mu,\alpha_{s}(\mu)\Big)\,,
\end{align}
where we set the factorization and renormalization scale to be the same, $\mu_F=\mu_R=\mu$. We have also factored out the hadronic Born cross section by using that $x_{1}x_{2}\bar{\sigma_{0}}=Q_{q}^{2}\sigma_{0}$, where
\begin{align}
    \sigma_{0}=\frac{4\pi\alpha^{2}}{3N_{c}sQ^{2}}
\end{align}
and defined the hadronic and partonic threshold variables as
\begin{align}
    \tau=\frac{Q^{2}}{s}\,,\hspace{1cm}z=\frac{\tau}{x_{1}x_{2}}\,.
\end{align}

The singular part of the fixed order calculation needs to be separated out of the hard scattering part, such that we can absorb it into the parton distribution functions. To this end, we define the partonic analogue to \cref{eq:hadronic convolution}
\begin{align}\label{eq:refactorized partonic drell-yan}
   h_{ij}(z,Q,\mu,\alpha_{s}(\mu),\epsilon)=&\sum_{k,l}\int_{z}^{1}\frac{dy_{1}}{y_{1}}\frac{dy_{2}}{y_{2}}\,f_{k/i}(y_1,\mu,\epsilon)\,f_{l/j}(y_2,\mu,\epsilon)\nonumber
   \\
   &\times\omega_{ij}\Big(\frac{z}{y_1},\frac{z}{y_2},Q,\mu,\alpha_{s}(\mu^{2})\Big)
\end{align}
which is a refactorization of the hard scattering function using the parton-in-parton densities we discussed in \cref{sec:lightcone parton in parton distributions}. We can extract the singular part by making the following perturbative expansion
\begin{align}\label{eq:perturbative expansion DY refactorization}
  h_{ij}=h_{ij}^{(0)}+\frac{\alpha_s}{\pi}h_{ij}^{(1)}+\mathcal{O}(\alpha_{s}^{2})
  \\
  \omega_{ij}=\omega_{ij}^{(0)}+\frac{\alpha_s}{\pi}w_{ij}^{(1)}+\mathcal{O}(\alpha_{s}^{2})
  \\
  f_{i/j}=f_{i/j}^{(0)}+\frac{\alpha_s}{\pi}f_{i/j}^{(1)}+\mathcal{O}(\alpha_{s}^{2})\,,
\end{align}
where $h_{ij}^{(1)}$ is the singular hard function we calculated at NLO in \cref{sec:NLO drell yan calculation}, i.e.
\begin{align}
  h_{ij}^{(1)}&=-A(\epsilon)\frac{1}{\epsilon}P_{q/q}^{(0)}+A(\epsilon)\Big(\frac{\pi^{2}}{3}-4\Big)\delta(1-z)\nonumber
    \\
    &\hspace{1.5cm}+2(1+z^{2})\Big[\frac{\ln(1-z)}{1-z}\Big]_{+}-\frac{1+z^{2}}{1-z}\ln z\,,
\end{align}
and the $\mathcal{O}(\alpha_s)$ parton-in-parton density is given in \cref{eq:quark in quark PDF to one-loop}. Let us expand $A(\epsilon)$, giving
\begin{align}
    A(\epsilon)=\Big(\frac{4\pi\mu^{2}}{Q^{2}}\Big)^{\epsilon}\frac{\Gamma(1-\epsilon)}{\Gamma(1-2\epsilon)}=1-\epsilon(\gamma_{E}-\ln{4\pi})+\epsilon\ln\Big(\frac{\mu^{2}}{Q^{2}}\Big)+\mathcal{O}(\epsilon^{2})\,.
\end{align}
By using the $\overline{\text{MS}}$ scheme we remove $\gamma_E$ and $\ln(4\pi)$, giving
\begin{align}
    h_{ij}^{(1)}(z,\epsilon)&=C_F\Big[\Big(\frac{\pi^{2}}{3}-4\Big)\delta(1-z)+2(1+z^{2})\Big[\frac{\ln(1-z)}{1-z}\Big]_{+}-\frac{1+z^{2}}{1-z}\ln z\Big]\nonumber
    \\
    &\hspace{0.5cm}-\frac{1}{\epsilon}\Big(1+\epsilon\ln{\frac{\mu^{2}}{Q^{2}}}\Big)P_{q/q}^{(0)}(z)\,,
    \\
    f_{i/j}^{(1)}(z,\epsilon)&=-\frac{1}{2\epsilon}\Big[1+\epsilon\ln\big(\frac{\mu^{2}}{\mu_{F}^{2}}\big)\Big]P_{q/q}^{(0)}(z)\,.
\end{align}
If we insert the perturbative expansions in \cref{eq:perturbative expansion DY refactorization} into \cref{eq:refactorized partonic drell-yan}, we find that
\begin{align}
    h_{ij}^{(0)}(z)=\omega_{ij}^{(0)}(z)=\delta_{ij}\delta(1-z)\,,
\end{align}
and the first order correction is given by
\begin{align}
  \omega_{ij}^{(1)}(z)&=h_{ij}^{(1)}(z,\epsilon)+\frac{1}{2\epsilon}\Big[1+\epsilon\ln\Big(\frac{\mu^{2}}{\mu_{F}^{2}}\Big)\Big]\sum_{k}\int_{z}^{1}\frac{dy_1}{y_1}P_{k/i}^{(0)}(y_1)h_{kj}^{(0)}(\frac{z}{y_1})\nonumber
  \\
  &\hspace{1cm}+\frac{1}{2\epsilon}\Big[1+\epsilon\ln\Big(\frac{\mu^{2}}{\mu_{F}^{2}}\Big)\Big]\sum_{l}\int_{z}^{1}\frac{dy_2}{y_1}P_{l/j}^{(0)}(y_2)h_{il}^{(0)}(\frac{z}{y_2})\nonumber
  \\
  &=h_{ij}^{(1)}(z,\epsilon)+\frac{1}{\epsilon}P_{ij}^{(0)}(z)+\ln\Big(\frac{\mu^{2}}{\mu_{F}^{2}}\Big)P_{ij}^{(0)}(z)\nonumber\,,
\end{align}
where we used the delta function to evaluate the integrals.
We observe that the $1/\epsilon$ singularity will cancel in this sum, leading to the infrared safe hard function in the $\overline{MS}$-scheme
\begin{align}\label{eq:infrared safe hard function}
    \omega_{q\bar{q}}^{(1)}(z)&=P_{qq}^{(0)}\ln{\frac{Q^{2}}{\mu_{F}^{2}}}+C_{F}\Big[\Big(\frac{\pi^{2}}{3}-4\Big)\delta(1-z)\nonumber
    \\
    &\hspace{0.3cm}+2(1+z^{2})\Big[\frac{\ln(1-z)}{1-z}\Big]_{+}-\frac{1+z^{2}}{1-z}\ln z\Big]\,.
\end{align}
Hence, regularizing soft divergences gives rise to logarithmic plus distributions and regularizing collinear divergences gives rise to a logarithm of the factorization scale.

We can remove this last part by choosing $\mu_F=Q$, which also removes the splitting function that contains further plus distributions. This is the most common choice in the literature, which we also adopt. With this choice the partonic scattering function up to $\mathcal{O}(\alpha_s)$ takes the form
\begin{align}\label{eq:NLO drell yan hard function}
    \omega_{q\bar{q}}(z,\alpha_{s}(\mu))&=\delta(1-z)+\frac{\alpha_{s}}{\pi}C_{F}\Big[\Big(\frac{\pi^{2}}{3}-4\Big)\delta(1-z)\nonumber
    \\
    &\hspace{0.3cm}+2(1+z^{2})\Big[\frac{\ln(1-z)}{1-z}\Big]_{+}-\frac{1+z^{2}}{1-z}\ln z\Big]\,.
\end{align}

The result in \cref{eq:NLO drell yan hard function} contains no singularities, but the plus distributions is a source of potential large corrections. When we are close to threshold, there is little phase space left for the emission of gluons. In this case, most of the energy is used to produce the final state leptons, leading to a suppression of real soft gluon emission\footnote{These are soft because near threshold most of the energy is used to produce the final state leptons.}. These plus distributions appeared when we summed real and virtual contributions. Thus, if real contributions are suppressed there will be an imbalance in the cancellation. These corrections appear to all order in perturbation theory, so to $n$-th order they take the form
\begin{align}
    \alpha_{s}^{n}\Big[\frac{\ln^{2n-1}(1-z)}{1-z}\Big]_{+}\,.
\end{align}
As they are large in the threshold regime, they will spoil the convergent behaviour of the perturbative expansion. Therefore, they must be treated to all orders in order to make reliable predictions. The technique to do this is called \emph{threshold resummation}, which we will lay out in more detail in the next chapter.


\subsection{Numerical Evaluation of the Hadronic Drell-Yan Cross Section}
In this section we will compare the LO and the full NLO differential cross section for the Drell-Yan process to experimental results from the CMS experiment \cite{Sirunyan:2018owv}. Our basic result is shown in \cref{fig:numerical LO and NLO cross section} where the LO result is given by the dashed line, the full NLO result in the solid line and the data from CMS is given by the red dots with errorbars \cite{Sirunyan:2018owv}. The numerical evaluation is based on the analytical $\mathcal{O}(\alpha_s)$ result in \cref{eq:NLO drell yan hard function} and performed by evaluating \cref{eq:Factorized Dre-Yan Cross Section} for both the LO and LO$+$NLO cross section. The analytical cross section has been obtained at a centre of mass energy $\sqrt{s}=13$ TeV and we consider production of leptons with invariant mass between 50 and 300 GeV. At LO there is only the $q\bar{q}$ channel that contributes, but at NLO we also have a contribution from a gluon initiated process. To be in accordance with our analytical calculation we have chosen to neglect this contribution in the numerical evaluation as well and only focused on the colour singlet process. Also, the plus distribution in \cref{eq:NLO drell yan hard function} have been omitted as they are to be resummed in \cref{chap:Resummation in QCD}. We use the CT10 NLO \cite{Lai:2010vv} PDF set, and we choose the factorization scale and renormalization scale to be equal the invariant mass of the final state leptons, $\mu=Q$. We have also included the $Z$-boson resonance by using an effective coupling as discussed in \cite{Banfi:2011dm}. 

The main theoretical uncertainty in our analysis comes from the dependence of the calculated cross section on the renormalization and factorization scale. This uncertainty is found by varying the scale between $Q/2$ and $2Q$. By calculating the uncertainty in this way the error-band should give a estimate of how higher order effects should affect the cross section. There are also theoretical uncertainties coming from the PDFs, but we have chosen not to take these into account.

We can see from \cref{fig:numerical LO and NLO cross section} that the full NLO calculation we have performed is much closer to the true cross section found by the CMS experiment than the LO result. This is of course not surprising as the true cross section is an all order process and taking into account higher and higher order in the perturbative expansion should give better and better predictions. However, considering the simplifications we have made by neglecting gluon initiated processes and omitting the plus distribution, the result is surprisingly good. Another point to make is that the dependancy on the renormalization scale is shown to decrease when including QCD effects, implying that even higher order corrections are needed in order to reduce this dependancy even further. Also, according to our uncertainty estimation the full NLO result should lie inside the error-band coming from the LO calculation. But we observe that this is far off, especially at the tails. We suspect that this is due to the fact that at LO we have a pure electroweak process, and QCD effects are only manifest at higher orders. 

\begin{figure}[H]
    \centering
    \includegraphics[width=\linewidth]{Figures/dsigmadmll.pdf}
    \caption{Differential cross section of neutral Drell-Yan production as function of final-state invariant mass. Shown are LO (dashed) and LO$+$NLO (solid) results with corresponding scale errors (pink bands). The figure also shows the differential cross section (red dots) measured by the CMS experiment \cite{Sirunyan:2018owv}.}
    \label{fig:numerical LO and NLO cross section}
\end{figure}


%%%%%%%%%%%%% Resummation %%%%%%%%%%%%%%%%%%%%%%%%
\chapter{Resummation using Wilson Lines}\label{chap:Resummation in QCD}
In this chapter we will take a closer look at how we can deal with the large logarithms in the threshold region. First we will take a classical approach to eikonal exponentiation in an Abelian theory, following a similar line as Steven Weinberg \cite{Weinberg:1972rt}. The generalization to non-Abelian theories is not straightforward. This is of course due to the non-commutivity between non-Abelian gauge fields. The proof exist in the literature, and the result is known as the \emph{non-Abelian eikonal exponentiation theorem} \cite{Gatheral:1983cz,Frenkel:1984pz,Laenen:2008gt}. We will not cover this generalization as it would take us to far from our main purpose, but the above references can be sought out for more detail. 

After we have seen how amplitudes exponentiate, we will move on to discuss several factorization properties of cross sections in QCD. First, we will introduce the Mellin space formalism where we take a Mellin transform of the hadronic cross section and show how it factorizes. Then we go on and use factorization theorems to fully factorize the cross section in the threshold regime by using parton-in-parton distributions. After the cross section has been fully factorized into hard, soft and collinear parts we will use Wilson lines to construct an eikonal cross section that governs the soft radiation. Then we will briefly discuss the renormalization properties of Wilson lines and make an explicit calculation of the cusp anomalous dimension to one-loop order. 

With the introduction of the cusp anomalous dimension we discuss the renormalization group equation for parton-in-parton distributions and show that it is given in terms of the cusp anomalous dimension. The non-Abelian exponentiation theorem will then be used to find an exponentiated eikonal cross section. After we have found the exponentiated cross section we will take the discussion to the level of the hadronic cross section again. We will discuss possible ways of evaluating the inverse Mellin transform in order to obtain a cross section in real space. 


\section{Exponentiation}\label{sec:exponentiation}
In \cref{sec:Drell-Yan Hadronic Cross Section}, we dealt with a fixed order calculation for the Drell-Yan process. Due to soft gluon radiation from the hard quark lines, we identified large logarithmic contributions to the cross-section in the threshold region, i.e. $z\rightarrow 1$. As these large effects appear at all orders, they spoil the convergent behaviour of the series expansion in the strong coupling. To perform an all-order calculation in the full theory of QCD is an impossible task, but with certain approximations, these large contributions can be resummed.

Resummation of large logarithmic contributions was first demonstrated in \cite{1961AnPhy..13..379Y}, where the radiation of photons in QED was considered. The resummation was done by using the eikonal approximation, i.e. the photons are restricted to be soft. The crucial feature of the eikonal approximation is that scattering amplitudes exponentiate, with the consequence that one can compute the logarithm of the amplitude via a set of simplified rules. Therefore, we can access all order information from low-order perturbative calculations, which simplifies calculations substantially. Using this approximation, a scattering amplitude $\mathcal{M}$ describing multiple soft photon radiation can be written on the form
\begin{align}\label{eq:Soft Abelian amplitude}
    \mathcal{M}=\mathcal{M}_{0}\exp(\sum W_{c})\,,
\end{align}
where $\mathcal{M}_{0}$ is the amplitude without soft photon radiation, and the exponential has a sum over all connected Feynman diagrams $W_{c}$, involving soft photon emission only. Since QED is an Abelian gauge theory, the emitted photons will not interact and the amplitude factorizes, leading to an exponentiation of the amplitude. 

For a non-Abelian gauge theory, such as QCD, this simple factorization no longer applies as the gluons interact with each other. Nevertheless, in \cite{Sterman:1981jc}, it was observed that one could achieve exponentiation by using the eikonal approximation also in the case of QCD. The formal proof of this observation was later provided in \cite{Gatheral:1983cz, Frenkel:1984pz}, where the structure of the exponent is much more complicated due to the non-commutativity of the colour matrices. The non-abelian analogue of \cref{eq:Soft Abelian amplitude} can then be written on schematic form as
\begin{align}
    \mathcal{M}=\mathcal{M}_{0}\exp(\sum \Bar{c}_{W} W)\,,
\end{align}
where the soft gluons are emitted from two hard partons connected by a colour singlet vertex, which is exactly the case considered in the Drell-Yan process. The sum in the exponent involves Feynman diagrams $W$, called \emph{webs} in the literature, and $\Bar{c}_{W}$ are modified colour factors which in general differ from the usual colour factors $c_W$ using standard Feynman rules. We will encounter these webs later on, but we will not cover the details of their properties\footnote{This is an extensive topic, so for more detail about webs, see e.g. \cite{White:2015wha,Berger_2002,Laenen:2008gt,article}.}.

We will now show how the structure in \cref{eq:Soft Abelian amplitude} appears by considering the all order process of photon emission from a final state fermion-antifermion creation process. We will not do this by explicitly using Wilson lines, but we will again see that this it is equivalent.

%%%%%%%%%%%%%%%%%%%%%%%%%%%%%%%%%%%%%%%%%%%%%%%%
\subsection*{Eikonal Exponentiation in QED}\label{sec:Eikonal}
\begin{figure}
    \centering
    \includegraphics[scale=0.3]{Figures/photonvertexloop.pdf}
    \caption{Leading order and $\mathcal{O}(\alpha)$ correction to virtual photon decay.}
    \label{fig:photon diagrams}
\end{figure}
In this section we will look at the exponentiation of soft photon emission. 
Let us start by considering the amplitude of an off-shell photon decaying to a massless fermion-antifermion pair, see \cref{fig:photon diagrams}. The leading order amplitude is given by
\begin{align}
    \mathcal{M}_{0}=\Bar{u}(p)\gamma^{\mu}v(p')\,,
\end{align}
where coupling factors have been neglected for simplicity. If we now consider the correction, where a photon is emitted from the final state, we get the amplitude
\begin{align}
    \mathcal{M}_{1}=\int\frac{d^{d}k}{(2\pi)^{d}}\Bar{u}(p)\gamma^{\alpha}\frac{(\slashed{p}-\slashed{k})}{(p-k)^{2}}\gamma^{\mu}\frac{(\slashed{k}-\slashed{p'})}{(k-p')^{2}}\gamma^{\beta}v(p')D_{\alpha\beta}(k)\,,
\end{align}
where $D_{\alpha\beta}(k)$ is the photon propagator. 

As we showed in \cref{sec:wilson line properties} the eikonal approximation corresponds to taking the soft limit $k\rightarrow 0$, such that we can neglect $\slashed{k}$ in the numerator and $k^{2}$ in the denominator of the fermion propagators. If we also assume that the fermions are massless, $p^{2}=0$, we get the much simpler eikonal amplitude
\begin{align}\label{eq:factorized eikonal 1 photon amplitude}
    \mathcal{M}_{1}&=\int\frac{d^{d}k}{(2\pi)^{d}}\Bar{u}(p)\gamma^{\alpha}\Big(-\frac{\slashed{p}}{2p\cdot k}\Big)\gamma^{\mu}\Big(\frac{\slashed{p'}}{2p\cdot k}\Big)\gamma^{\beta}v(p')D_{\alpha\beta}(k)\nonumber
    \\
    &=\int\frac{d^{d}k}{(2\pi)^{d}}\big[\bar{u}(p)\gamma^{\mu}v(p'){\big]}\Big(-\frac{p^{\alpha}}{p\cdot k}\Big)\Big(\frac{p'^{\beta}}{p\cdot k}\Big)D_{\alpha\beta}(k)\nonumber
    \\
    &=\mathcal{M}_{0}\int\frac{d^{d}k}{(2\pi)^{d}}\Big(-\frac{p^{\alpha}}{p\cdot k}\Big)\Big(\frac{p'^{\beta}}{p\cdot k}\Big)D_{\alpha\beta}(k)\,,
\end{align}
where we in the second step used the Dirac algebra $\{\gamma^{\mu},\gamma^{\nu}\}=2g^{\mu\nu}$ and the massless Dirac equation, $\bar{u}(p)\slashed{p}=0$ and $\slashed{p'}v(p')=0$. We observe that the tree-level amplitude has been factored out, and contains no divergences. This result is actually an example of factorization of soft physics from hard physics, where the soft part is described by the integral and the hard part is the leading order amplitude $\mathcal{M}_{0}$. The physical reason for this factorization is that the momentum of the soft photon is to low to resolve the inner structure of the hard process. 

Another important observation is that we can define the terms inside the brackets as an eikonal Feynman rule\footnote{This is of course closely related to the Wilson line rules we derived in \cref{sec:wilson line properties}, with the distinction that this is an Abelian theory.}

\begin{fmffile}{ee}
\begin{align}
\begin{gathered}
\begin{fmfgraph*}(45,40)
\fmfleft{i}
\fmfright{o}
\fmf{plain}{i,v3}
\fmf{plain}{v3,o}
\fmffreeze   % freezing the drawn elements
\fmfright{v3,o3}   % adding two more vertices
\fmfforce{(0.5w,0.5h)}{v3}   % setting position of the first vertex
\fmfforce{(0.5w,0h)}{o3}   % setting position of the second vertex
\fmfdot{v3}   % drawing the first vertex with a dot
\fmf{photon, label=$k\downarrow$, l.side=left}{v3,o3}   % drawing a gluon line
\end{fmfgraph*}
\end{gathered}\hspace{0.6cm}&=\frac{p^{\mu}}{p\cdot k}\,.\hspace{2.36cm}\text{Eikonal vertex}\label{eq:Wilson vertex}
\end{align}
\end{fmffile}
This actually means that we can think of the factors multiplying the tree-level amplitude in \cref{eq:factorized eikonal 1 photon amplitude} as a new type of Feynman diagrams, which are \emph{subdiagrams} of the full amplitude. 
\begin{figure}
    \centering
    \includegraphics[scale=0.4]{Figures/LadderDiagram.pdf}
    \caption{Ladder diagram of $n$ soft photon emission.}
    \label{fig:Ladder diagram}
\end{figure}

The next step is to generalize to the emission of $n$ soft photons, so let us start with the diagram \cref{fig:Ladder diagram}, where none of the photon lines cross each other, called a \emph{ladder diagram}. This amplitude is given by
\begin{align}\label{eq:nth amplitude}
    \mathcal{M}^{(n)}=\Big(\prod_{i=1}^{n}\int\frac{d^{d}k_i}{(2\pi)^{d}}\Big)\bar{u}(p)\mathcal{E}^{\alpha_1\dots\alpha_n}(p,k_{i})\gamma^{\mu}\mathcal{E}^{\beta_1\dots\beta_n}(p',k_{i})v(p')D_{\alpha_1\beta_1}(k_1)\dots D_{\alpha_n\beta_n}(k_n)\,,
\end{align}
where we have collected the product of fermion propagators and the gamma matrices from the corresponding vertices into $\mathcal{E}$ in the following way
\begin{align}
    \mathcal{E}^{\alpha_1\dots\alpha_n}(p,k_{i})=\frac{\gamma^{\alpha_1}(\slashed{p}-\slashed{k_1})\cdots\gamma^{\alpha_n}(\slashed{p}-\slashed{k_1}- \dots -\slashed{k_n})}{(p-k_1)^{2}\cdots(p-k_1-\dots k_n)^{2}}\,,
\end{align}
and by taking the eikonal approximation this can be simplified to 
\begin{align}
    \mathcal{E}^{\alpha_1\dots\alpha_n}(p,k_{i})=\frac{\bar{u}(p)\gamma^{\alpha_1}\slashed{p}\gamma^{\alpha_2}\slashed{p}\cdots\gamma^{\alpha_n}\slashed{p}}{(-2p\cdot k_1)\cdots (-2p\cdot(k_1+\cdots+k_n)}\,,
\end{align}
which can be further simplified by permuting the gamma matrices using the Dirac algebra and use that $\bar{u}(p)\slashed{p}=0$, giving
\begin{align}
    \bar{u}(p)\mathcal{E}^{\alpha_1\dots\alpha_n}&=\frac{(-1)^{n}\bar{u}(p)p^{\alpha_1}\cdots p^{\alpha_n}}{p\cdot k_1\cdots p\cdot(k_1+\dots +k_n)}\,,
    \\
    \mathcal{E}^{\beta_1\dots\beta_n}v(p')&=\frac{p'^{\beta_1}\cdots p'^{\beta_n}\,v(p')}{p'\cdot k_1\cdots p'\cdot(k_1+\dots +k_n)}\,,
\end{align}
where the factors of $2$ cancels from those in the Dirac algebra, see \cref{eq:Dirac algebra}. 
Combining these two expressions into \cref{eq:nth amplitude}, we get
\begin{align}\label{eq:nth amplitude rewritten}
    \mathcal{M}^{(n)}=\mathcal{M}_{0}\Big(\prod_{i=1}^{n}\int\frac{d^{d}k_i}{(2\pi)^{d}}\Big)\Big[\frac{(-1)^{n}p^{\alpha_1}\cdots p^{\alpha_n}\,p'^{\beta_1}\cdots p'^{\beta_n}\,D_{\alpha_1\beta_1}(k_1)\cdots D_{\alpha_n\beta_n}(k_n)}{p\cdot k_1\cdots p\cdot(k_1+\dots +k_n)\,p'\cdot k_1\cdots p'\cdot(k_1+\dots +k_n)}\Big]\,.
\end{align}

This reveals an intricate dependency on the photon momenta $k_i$, as they are coupled along both the fermion and anti-fermion line. However, we have only considered the ladder diagram where the photons are emitted in the order $k_1\dots k_n$. Therefore, we must sum over all diagrams to this order, meaning we must sum over all permutations of the emitted photon momenta. We start by fixing the order of photon emission on the anti-fermion line, and sum over permutations on the fermion line. If we let $\pi$ denote a permutation of $(1,2,\dots,n)$, which maps to $(\pi_1,\pi_2,\dots \pi_n)$, the sum over diagrams correspond to making the following substitution
\begin{align*}
    \frac{1}{p\cdot k_1\cdots p\cdot(k_1+\dots +k_n)}\rightarrow \sum_{\pi}\frac{1}{p\cdot k_{\pi_1}\cdots p\cdot(k_{\pi_1}+\dots +k_{\pi_n})}\,.
\end{align*}
Then we can make use of the so-called \emph{eikonal identity}\footnote{This identity can be proven by using the Wilson lines we considered in \cref{sec:wilson line properties}, but for a standard derivation we refer the reader to \cite{Peskin:257493}.}
\begin{align}
    \sum_{\pi}\frac{1}{p\cdot k_{\pi_1}\cdots p\cdot(k_{\pi_1}+\dots +k_{\pi_n})}=\prod_{i=1}^{n}\frac{1}{p\cdot k_i}\,.
\end{align}
If this seems mysterious, let us show a simple example that highlights how this works. For the simple case of $n=2$, we have that
\begin{align}
    \frac{1}{p\cdot k_1\,p\cdot(k_1+k_2)}+\frac{1}{p\cdot k_2\,p\cdot(k_1+k_2)}=\frac{1}{p\cdot k_1\,p\cdot k_2}\,,
\end{align}
which shows the structure.

So by using the eikonal identity we substantially simplify the dependence on the photon momenta for the fermion line, as they have decoupled from each other. We could hope to do the same on the anti-fermion line, but that is not possible as they have already been fixed. However, we can exploit that $k_i$ are dummy variables inside the integral. The integrand has a symmetric Lorentz structure under the permutation of any two momenta, which means that we can make the replacement
\begin{align}
    \Big(&\prod_{i=1}^{n}\int\frac{d^{d}k_i}{(2\pi)^{d}}\Big)\frac{1}{p'\cdot k_1\cdots p'\cdot(k_1+\dots +k_n)}\nonumber
    \\
    &=\frac{1}{n!}\Big(\prod_{i=1}^{n}\int\frac{d^{d}k_i}{(2\pi)^{d}}\Big)\sum_{\pi}\frac{1}{p'\cdot k_{\pi_1}\cdots p'\cdot(k_{\pi_1}+\dots +k_{\pi_n})}\nonumber
    \\
    &=\frac{1}{n!}\prod_{i=1}^{n}\int\frac{d^{d}k_i}{(2\pi)^{d}}\frac{1}{p'\cdot k_i}\,,
\end{align}
where we have used that there are $n!$ such permutations. 

Substituting these simplifications into \cref{eq:nth amplitude rewritten}, we get
\begin{align}
    \mathcal{M}^{(n)}&=\mathcal{M}_{0}\frac{1}{n!}\prod_{i=1}^{n}\int\frac{d^{d}k_i}{(2\pi)^{d}}\big(\frac{p^{\alpha_i}}{p\cdot k_i}\big)\big(\frac{-p'^{\beta_i}}{p'\cdot k_i}\big)D_{\alpha_i\beta_i}(k_i)\nonumber
    \\
    &=\mathcal{M}_{0}\frac{1}{n!}\Big[\int\frac{d^{d}k}{(2\pi)^{d}}\big(\frac{p^{\alpha}}{p\cdot k}\big)\big(\frac{-p'^{\beta}}{p'\cdot k}\big)D_{\alpha\beta}(k)\Big]^{n}\,.
\end{align}
This looks very much like the $n$th term in a Taylor expansion of an exponential. By using the eikonal approximation we have found the remarkable result that the sum over all $n$ photon graphs is given by the one-loop graph to the $n$th power! If we take the sum over all diagrams for any number of single soft photon emission, we find the all order amplitude
\begin{align}\label{eq:exponentiated amplitude}
    \mathcal{M}&=\sum_{n=1}^{\infty}\mathcal{M}^{(n)}=\mathcal{M}_{0}\exp\Big(\int\frac{d^{d}k}{(2\pi)^{d}}\big(\frac{p^{\alpha}}{p\cdot k}\big)\big(\frac{-p'^{\beta}}{p'\cdot k}\big)D_{\alpha\beta}(k)\Big)\,,
\end{align}
which demonstrates that the hard part factorizes from the soft part also in the all order case, with the addition that the soft part exponentiates. From the structure of \cref{eq:exponentiated amplitude}, we can write the amplitude on the factorized form
\begin{align}
    \mathcal{M}=\mathcal{H}\mathcal{S}\,,
\end{align}
where $\mathcal{H}$ is a hard scattering function and $\mathcal{S}$ is a soft scattering function. The hard function is finite, while the soft function contains all the soft and collinear divergences. It is important to note that we have only considered single photon emissions, but to be general one should in fact consider the case where the emitted photons could connect off the external lines, via fermion loops. This would extend the result in \cref{eq:exponentiated amplitude} to include the sum over all connected diagrams including loops in the exponent, see \cite{1961AnPhy..13..379Y}. 

As already mentioned the complexity increases substantially for a non-abelian gauge theory, and while the result has existed in literature for a long time \cite{Gatheral:1983cz,Frenkel:1984pz}, the proof is restricted to involve only two coloured external particles. This proof covers Drell-Yan like processes, but if the final state contains coloured particles as well this proof does not hold. This scenario has been studied in great detail and recent work has been done to improve it to include several coloured particles \cite{White:2015wha,Laenen:2008gt}. The proof is too advanced to go into detail on here, but the main idea is to use a path integral approach with Wilson lines as a source for creating particles from the vacuum. From there one uses the so-called replica method from statistical mechanics to show that the theory can be written as $N$ replicas leading to the structure of webs.



%\section{Threshold Factorization}
In \cref{sec:Drell-Yan Hadronic Cross Section} we used the QCD factorization theorem and wrote the hadronic Drell-Yan cross section on the form\footnote{The difference here is that we have not factored out the Born cross section at this point.}
\begin{align}\label{eq:resummation factorized drell-yan Cross Section}
   \frac{ d\sigma_{h_1h_2}}{dQ^{2}}=\sum_{i,j}\int dx_{1}dx_{2}\,f_{i/h_1}(x_1,\mu)f_{j/h_2}(x_2,\mu)\,\hat{\sigma}_{ij}\Big(z,Q,\mu,\alpha_s(\mu^{2})\Big)\,.
\end{align}
As we have said many times before, these parton distribution functions $f_{i/h_{1}}$ are not perturbative and have to be extracted from experiment. We made an $\mathcal{O}(\alpha_s)$ calculation of the hard partonic cross section in \cref{sec:Drell-Yan Hadronic Cross Section}, where we found both collinear and soft divergences. By summing over all diagrams we found that some of these canceled, and the last divergence were treated by considering a refactorization using parton-in-parton distributions. However, the final result contained terms that give large logarithmic corrections near threshold. The way to treat these large logarithms is by using the non-Abelian eikonal exponentiation theorem, see (\ar{ref to nonA exponentiation}). 

Let us first consider the partonic equivalent to \cref{eq:resummation factorized drell-yan Cross Section} using parton-in-parton distributions
\begin{align}
    \frac{d\hat{\sigma}_{ij}}{dQ^{2}}=\sum_{kl}\int dy_{1}dy_{2}f_{k/j}(x_1,\mu)f_{l/j}(x_2,\mu)\,\hat{\sigma}_{kl}(z,Q,\mu,\alpha_s(\mu^{2}))\,
\end{align}

%%%%%%%%%%%%%%%%%%%%%%%%%%%%%%%%%%%%%%%%%%%%%%%%%%%%%%%%%%%%
\section{Hadronic Cross Section and Inverse Mellin}
In this section we will recover the hadronic cross section in Drell-Yan $x$-space, and discuss how we can evaluate the inverse Mellin transform. 

From \cref{eq:hadronic cross section in Mellin} and \cref{eq:LL result} the Mellin transformed hadronic cross section is given by \footnote{We have set $\mu=Q$, and for generality inserted the hard function $H_{ij}$.}
\begin{align}
    \tilde{\sigma}_{h_1h_2}(N)&=\int_{0}^{1}d\tau \tau^{N-1}\,\frac{1}{\sigma_{0}}\frac{d\sigma_{h_1h_2}}{dQ^{2}}\nonumber
    \\
    &=\sum_{i,j=q,\bar{q}}H_{ij}\big((Q,\alpha_{s}(Q)\big)\tilde{f}_{i/h_1}(N,Q)\tilde{f}_{j/h_2}(N,Q)\,\exp\big(\Me{E}_{ij}(N,Q,\alpha_s)\big)
\end{align}
where the hard function has been exponentiated. We can now use the inverse Mellin transform \cref{eq:Appendix Inverse Mellin} to write\footnote{We recover the fractional charge of the quarks.}
\begin{align}
    \frac{d\sigma_{h_1h_2}}{dQ^{2}}=&\,\sigma_{0}\sum_{i,j=q,\bar{q}}Q_{q}^{2}H_{ij}\big((Q,\alpha_{s}(Q)\big)\nonumber
    \\
    &\frac{1}{2\pi i}\int_{c-i\infty}^{c+i\infty}dN\,\tau^{-N}\tilde{f}_{i/h_1}(N,Q)\tilde{f}_{j/h_2}(N,Q)\exp\big(\Me{E}_{ij}(N,Q,\alpha_s)\big)
\end{align}

There exists numerical packages to evaluate parton distributions in Mellin space, see \cite{Vogt_2005}. However, to use the $x$-space formalism we can do several manipulations by using the convolution properties of the Mellin transform.


\subsection{The Inverse Mellin Transform}
The inverse Mellin transform as defined in \cref{sec:Appendix Mellin Transform}, is for a general function given by
\begin{align}\label{eq:h}
    h(x)=\frac{1}{2\pi i}\int_{c-i\infty}^{c+i\infty}dN\,x^{-N}\,\Me{h}(N)
\end{align}
where 
\begin{align*}
    \Me{h}(N)=\int_{0}^{\infty}dx\,x^{N-1}\,h(x)
\end{align*}
since $h(x)$ is a real valued function
\begin{align}\label{eq:h_real}
    \Me{h}^{*}(N)=\int_{0}^{\infty}dx\,x^{N^{*}-1}\,h(x)=\Me{h}(N^{*})
\end{align}
The Mellin inversion integral \cref{eq:h} can be splitted into two, one part for the lower bound and one part for the upper bound. In the lower bound term a change of variable will be made $N\rightarrow N^{*}$, which makes the integration bounds change accordingly $c-i\infty\rightarrow c+i\infty$.
\begin{align}
    h(x)&=\frac{1}{2\pi i}\Bigl(\int_{c-i\infty}^{c}dN\,x^{-N}\,\Me{h}(N)+\int_{c}^{c+i\infty}dN\,x^{-N}\,\Me{h}(N)\Bigr)\nonumber
    \\
    &=\frac{1}{2\pi i}\Bigl(\int_{c+i\infty}^{c}dN^{*}\,x^{-N^{*}}\,\Me{h}(N^{*})+\int_{c}^{c+i\infty}dN\,x^{-N}\,\Me{h}(N)\Bigr)\nonumber
    \\
    &=\frac{1}{2\pi i}\Bigl(-\int_{c}^{c+i\infty}dN^{*}\,x^{-N^{*}}\,\Me{h}^{*}(N)+\int_{c}^{c+i\infty}dN\,x^{-N}\,\Me{h}(N)\Bigr)\,,
    \intertext{and by choosing the parametrization of the Mellin variable to be $N=c+ze^{i\phi}$, with $z$ real, the integral will take the form}
    &=\frac{1}{2\pi i}\int_{0}^{\infty}dz\,\left(e^{i\phi}\,x^{-N}\Me{h}(N)-e^{-i\phi}\,x^{-N^{*}}\Me{h}^{*}(N)\right)\nonumber
    \\
    &=\frac{1}{2\pi i}\int_{0}^{\infty}dz\,2i\,\text{Im}\left(e^{i\phi}\,x^{-N}\Me{h}(N)\right)\nonumber
    \\
    &=\frac{1}{\pi}\int_{0}^{\infty}dz\,\text{Im}\left(e^{i\phi}\,x^{-N}\Me{h}(N)\right)\,,
\end{align}
where the relation used in the third last step is
\begin{align*}
    \Me{h}(N)-\Me{h}^{*}(N)&=2i\,\text{Im}(\Me{h}(N))\,.
\end{align*}
Writing out the parametrization, the integral is equal to
\begin{align}\label{eq:Inversed Mellin Integral}
    h(x)=\frac{1}{\pi}\int_{0}^{\infty}dz\,\text{Im}\left(e^{i\phi}\,x^{-c-z\,\exp(i\phi)}\Me{h}(c+z\,\exp(i\phi))\right)\,.
\end{align} 
\section{Mellin Space Factorization}\label{sec:Resummation Drell-yan}
In the NLO calculation \cref{sec:renormalizing drell-yan}, we found that the final hard scattering function was infrared safe. However, in the cancellation of infrared divergences, we found the emergence of logarithmic distributions. These logarithmic distributions appear at every order and ruin the convergence of the perturbative expansion in the threshold regime. To handle these large corrections, we have to use threshold resummation techniques. Threshold resummation was first derived for the Drell-Yan process \cite{Sterman:1986aj,CATANI1989}. These papers laid the groundwork for resummation of large contributions in many hard QCD processes. 

\subsection{Phase Space Factorization}
A fundamental ingredient for resummation in QCD, is that the eikonal approximation leads to exponentiation. Near the threshold, most of the available energy goes to producing the final state particles. Therefore, the emitted gluons are soft, and the eikonal approximation corresponds to taking the partonic threshold limit. However, the derivation of eikonal exponentiation in \cref{sec:Eikonal} was at the level of amplitudes. Eventually, we want the full cross-section to exponentiate, including the underlying hard process. The problem is that the hard part contains phase space integrals, which may lead to an intricate dependency between the final state gluon momenta. If the cross-section is to exponentiate the gluon momenta must disentangle from each other. To make this problem explicit, let us consider the Drell-Yan process where the differential phase space for the emission of $n$ soft gluons can be written as
\begin{align}
    d\mathcal{P}^{(n)}&=\frac{d^{4}q}{(2\pi)^{4}}\,\prod_{i=1}^{n}\frac{d^{3}k_{i}}{(2\pi)^{3}}\frac{1}{2k_{i}^{0}}\,2\pi\delta^{+}(q^{2}-Q^{2})(2\pi)^{4}\delta^{(4)}\Big(p_1+p_2-q-\sum_{i}k_i\Big)\nonumber
    \\
    &=\prod_{i=1}^{n}\frac{d^{3}k_{i}}{(2\pi)^{2}}\frac{1}{2k_{i}^{0}}\,\delta\Big(\Big[p_1+p_2-\sum_{i}k_i\Big]^{2}-Q^{2}\Big)\,,
\end{align}
where we began with the general expression for clarity and applied the delta function. Here $k_i$ is the gluon momenta, $q$ is the photon momenta, $p_1$ and $p_2$ is the incoming quarks and $Q^{2}$ is the final state invariant mass. In the soft gluon region we can make the approximation 
\begin{align}\label{eq:n-phase space exponentiation}
    d\mathcal{P}^{(n)}&\approx\prod_{i=1}^{n}\frac{d^{3}k_{i}}{(2\pi)^{2}}\frac{1}{2k_{i}^{0}}\delta\Big(\hat{s}-2\sum_{i}k_{i}^{0}\sqrt{\hat{s}}-Q^{2}\Big)\nonumber
    \\
    &=\prod_{i=1}^{n}\frac{d^{3}k_{i}}{(2\pi)^{2}}\frac{1}{2k_{i}^{0}}\frac{1}{\hat{s}}\delta\Big(1-z-\sum_i \omega_{k_i}\Big)\,,
\end{align}
where we have defined the fractional energy of the $i$th gluon as $\omega_{k_i}=2k_i^{0}/\sqrt{\hat{s}}$, and $z$ is the usual partonic threshold variable, $z=Q^{2}/\hat{s}$. We observe that the delta function forces the emitted gluon momenta to depend on each other, spoiling the factorization of the phase space.

There is a neat solution to this problem via the Laplace transform. In general, the Laplace transform of a delta function is given by
\begin{align}
    \mathcal{L}[\delta(x-y)](N)=e^{-Ny}\,.
\end{align}
where $N\in\mathbb{C}$. The goal now is to try to disentagle the delta function, so if we take the inverse Laplace of the delta function in \cref{eq:n-phase space exponentiation} we obtain the expression
\begin{align}\label{eq:inverse Laplace}
    \delta(1-z-\sum_i \omega_{k_i})=\frac{1}{2\pi i}\int_{c-i\infty}^{c+i\infty}dN\, e^{N(1-z-\sum_i \omega_{k_i})}\,.
\end{align}
However, we are only interested in the threshold regime $z\rightarrow 1$, so let us consider the following Taylor series
\begin{align}
    \ln z=\sum_{n=1}^{\infty}(-1)^{n}\frac{(1-z)^{n}}{n}\,,
\end{align}
leading to the approximate identity, $e^{N(1-z)}\approx e^{-N\ln z}=z^{-N}$. With this approximation \cref{eq:inverse Laplace} takes the form
\begin{align}
    \delta(1-z-\sum_i \omega_{k_i})=\frac{1}{2\pi i}\int_{c-i\infty}^{c+i\infty}dN\, z^{-N}\prod_{i=1}^{n}e^{-N\omega_{k_i}}\,,
\end{align}
which is an inverse Mellin transform, see \cref{sec:Appendix Mellin Transform}. The consequence of this result is made clear if we take the Mellin transform of the differential phase space
\begin{align}
    \int_{0}^{1}dz\,z^{N-1}\,d\mathcal{P}_{n}=\prod_{i=1}^{n}\frac{d^{3}k_{i}}{(2\pi)^{2}}\frac{1}{2k_{i}^{0}}\frac{1}{\hat{s}}e^{-N\omega_{k_i}}\,,
\end{align}
giving that the phase space has taken a factorized form, enabling it to exponentiate.

\subsection{Hadronic Cross Section in Mellin Space}
Factorization of the phase space is not the only advantage of using the Mellin transform. In \cref{sec:Drell-Yan Hadronic Cross Section}, we wrote the factorized Drell-Yan cross section as
\begin{align}\label{eq:Factorized Dre-Yan Cross Section}
   \frac{ d\sigma_{h_1h_2}}{dQ^{2}}=\sigma_{0}\sum_{i,j}\int \frac{dx_{1}}{x_1}\frac{dx_{2}}{x_2}f_{i/h_1}(x_1,\mu)f_{j/h_2}(x_2,\mu)\,\omega_{ij}\Big(z,Q,\mu,\alpha_s(\mu^{2})\Big)\,.
\end{align}

Instead of working with the convoluted integrals, we consider the Mellin transform in the hadronic threshold variable $\tau=x_1x_2z$. In this expression it is only the hard function that is calculable in perturbation theory and is the main function of interest. Hence, let us divide out the Born cross section and define the Mellin space hadronic cross section\footnote{We will recover the Born cross section when we return to the resummed hadronic cross section. We have also neglected the fractional charge of the quarks for simplicity.} 
\begin{align}\label{eq:hadronic cross section in Mellin}
    \tilde{\sigma}_{h_1h_2}(N)&=\int_{0}^{1}d\tau \tau^{N-1}\,\frac{1}{\sigma_{0}}\frac{d\sigma_{h_1h_2}}{dQ^{2}}\nonumber
    \\
    &=\sum_{i,j}\tilde{f}_{i/h_1}(N,\mu)\tilde{f}_{j/h_2}(N,\mu)\,\tilde{\omega}_{ij}\Big(N,Q,\mu,\alpha_s(\mu^{2})\Big)\,,
\end{align}
where
\begin{align}
    \tilde{\omega}_{ij}(N)=\int_{0}^{1}dz\,z^{N-1}\,\omega_{ij}(z)\,,
    \\
    \tilde{f}_{i/h_1}(N)=\int_{0}^{1}dx\,x^{N-1}\,f_{i/h_1}(x)\,.
\end{align}
Here we observe that one of the advantages of working in Mellin space is that the convolution has turned into simple products. If this transformation seems obscure, see \cref{eq:Appendix Mellin convolution transform}.

To see how the threshold logarithms manifest themselves in Mellin space, we take the limit $z\rightarrow 1$ of \cref{eq:NLO drell yan hard function}, giving
\begin{align}
    \omega_{q\bar{q}}=\delta(1-z)+\frac{\alpha_s}{\pi}C_{F}\Big(4\Big[\frac{\ln(1-z)}{1-z}\Big]_{+}+\Big(\frac{\pi^{2}}{3}-4\Big)\delta(1-z)\Big)+\mathcal{O}(\alpha_{s}^{2})\,.
\end{align}
By using the Mellin transforms given in \cref{sec:Appendix Mellin Transform}, we find the Mellin space expression
\begin{align}\label{eq:Mellin space NLO of omegaqbarq}
    \Me{\omega}_{q\bar{q}}=1+\frac{\alpha_s}{\pi}C_{F}2\ln^{2}\bar{N}+\mathcal{O}(\alpha_{s}^{2})\,,
\end{align}
where $\bar{N}=Ne^{\gamma_{E}}$, and we omitted constant terms that has no relevance compared to the logarithm. We observe that the threshold limit $z\rightarrow 1$, corresponds to $N\rightarrow\infty$. Hence, the constant factors are unimportant for large $N$. 

To $n$-th power of the logarithm, we have that the plus distributions give terms of the form
\begin{align}
    \int_{0}^{1}dz z^{N-1}\Big[\frac{\ln^{n}(1-z)}{(1-z)}\Big]_{+}=\frac{(-1)^{n+1}}{n+1}\ln^{n+1}(\bar{N})+\mathcal{O}(\ln^{n-1}(\bar{N}))\,,
\end{align}
after Mellin transformation. These are the logarithms we now want to resum, and try to reproduce after the resummation procedure has been performed. In order to make progress, we will use the factorization property of QCD to refactorize the cross section by using parton-in-parton distributions. We also used this property in the fixed order Drell-Yan calculation, where we had that the hard function $\omega_{ij}$ were collinear divergent. By refactorizing the hard functions by using parton-in-parton distributions we showed that the hard function could be rendered infrared safe as the parton-in-parton distributions were responsible for these. This is a very important feature that we will exploit heavily in the sections to come. 

\subsection*{Kinematics}
Let us briefly look at some of the kinematics that is used. We choose the quarks to be fully collinear to the incoming protons, i.e.
\begin{align}
    p_{1}&=x_{1}P_1\,,
    \\
    p_{2}&=x_{2}P_2\,.
\end{align}
We work in the centre of mass frame of $P_1$ and $P_2$, with $P_{1}^{0}=P_{2}^{0}=E=Q/2$. The hadrons have the following momenta
\begin{align}
    P_{1}&=\frac{Q}{2}(1,1,0)
    \\
    P_{2}&=\frac{Q}{2}(1,-1,0)\,,
\end{align}
such that
\begin{align}
    s=(P_1+P_2)^{2}=Q^{2}\,.
\end{align}
In the threshold limit $\tau\rightarrow 1$, we have that $\tau$ coincides with $z$ when $x_{1}\rightarrow 1$ and $x_{2}\rightarrow 1$. Hence, we have that the momenta of the quarks are $p_1=P_1$ and $p_2=P_2$. Also, if a gluon radiates from an incoming quark, it follows that the energy of that gluon is given by
\begin{align}
    Q^{2}&=(p_1+p_2-k)^{2}=s-2\sqrt{s}k^{0}
    \\
    &\rightarrow k^{0}=(1-\tau)Q/2\,.
\end{align}

\newpage
\section{Threshold Factorization}\label{sec:threshold factorization}
In this section we will use factorization theorems to find a fully factorized form of the hard function $\Me{w}_{ij}(N)$. 

First we want to extract the universal collinear singularities we encountered in the fixed order NLO calculation. We do the same as in \cref{sec:renormalizing drell-yan} and define a partonic analogue to the hadronic cross section \cref{eq:hadronic cross section in Mellin},
\begin{align}\label{eq:partonic analogue to hadronic in mellin }
    \tilde{\sigma}_{ij}(N)=\tilde{f}_{i/i}(N,\mu,\epsilon)\tilde{f}_{j/j}(N,\mu,\epsilon)\,\Me{\omega}_{ij}(N,Q,\mu,\alpha_{s})\,,
\end{align}
where $f_{i/i}$ are the light-cone distributions responsible for the collinear singularities. Thus, the partonic function $\Me{\omega}_{ij}$ is with this refactorization infrared safe.  
%\begin{align}
%    \frac{d\Me{\sigma}_{ij}}{dQ^{2}}(w,\epsilon)&=\sigma_{0}H_{ij}(\alpha_{s}(Q))\int \frac{dx_{1}}{x_1}\frac{dx_{2}}{x_2}dw_s\,J_{i/i}(x_1,Q,\epsilon)J_{j/j}(x_2,Q,\epsilon)\,S_{ij}(w_s,Q)\nonumber
%    \\
%    &\hspace{2cm}\delta(1-\tau-(1-x_1)-(1-x_2)-w_s)\,.
%\end{align}
%We define $w=1-\tau$, $w_1=1-x_1$, $w_2=1-x_2$ and $w_s=2\sum_{i}k_{i}^{0}/Q$, where $\sum_{i}k_{i}^{0}$ is total the energy of the emitted gluons. We call these weights, and they become small in the threshold region. The delta function result from taking the threshold limit of $\delta(Q^{2}-(p_{1}^{\mu}+p_{2}^{\mu}-\sum_{i}k_{i}^{\mu})^{2})$, taking only linear terms of $w_{i}$ into account. This is needed as we want the collinear and soft divergences to decouple, so we demand that all the weights are independent and additive, i.e. $w=w_1+w_2+w_s$\footnote{For more detail on near threshold factorization, see \cite{Sterman:1986aj,Kidonakis:1997gm,Contopanagos:1996nh}.}.
We can go even further with the factorization theorem. 

As we saw in \cref{sec:NLO drell yan calculation}, the cancellation of soft divergences are responsible for the large logarithms. Hence, we want to factorize out these soft parts and use the non-Abelian eikonal exponentiation theorem to resum them. We begin with defining a threshold analogue to \cref{eq:hadronic cross section in Mellin}  \cite{Sterman:1986aj}
\begin{align}\label{eq:threshold factorization Mellin}
    \Me\sigma_{ij}(N)&=H_{ij}(Q,\alpha_{s}(Q))\Me{J}_{i/i}(N,Q,\epsilon)\Me{J}_{j/j}(N,Q,\epsilon)\Me{S}_{ij}(N,Q,\mu,\alpha_{s}(\mu))\,,
\end{align}
where $H_{ij}(Q)$ is a hard function that is free of singular distributions. We have also defined jet functions $J_{i/i}(N)$. We specifically defined these as functions of $Q$ and not $\mu$, which we will comment on later. These contain additional collinear divergences, that will cancel the collinear singularities contained in the light-cone distributions $f_{i/i}(x,\mu,\epsilon)$. Lastly, we have a soft function $S_{ij}(N)$ that is responsible for all wide angle soft radiation. The derivation of this formula is quite technical, but we will try to give the main arguments behind it\footnote{For a more comprehensive explanation, see \cite{Collins:1989gx}.}.
\begin{figure}
    \centering
    \includegraphics[scale=0.2]{Figures/An Example Diagram.pdf}
    \caption{Diagram that has no IR-divergences.}
    \label{fig:Gluon connected to propagator}
\end{figure}

The first step is to decouple $H$ from $S$. Diagrams such as \cref{fig:Gluon connected to propagator} does not generate IR-divergences. This is true to all order, i.e. a soft line can only generate an IR divergence if it is connected to an on-shell external line \cite{sterman_1993}. Thus, gluons that connect the soft and the hard part do not contribute to the IR-divergences, i.e. the soft and hard part cannot be connected directly. The physical interpretation of such a decoupling is that soft gluons correspond to large length scales, while the hard part takes place at small length scales. Therefore the soft gluons are unable to resolve the internal structure of the hard process. This fact also follows from the eikonal approximation we made in \cref{sec:wilson line properties}, where we showed that the eikonal approximation led to the concept of particles dressed with Wilson lines. This was only possible if the soft emission was connected to an on-shell external line.

The second step of the argument is to decouple the jets from each other. By definition all jets move in different directions, so lines in different jets are proportional to different momenta. We have two jets that meet at the hard interaction, but before that they cannot combine. Thus their collinear divergences will not mix with each other \cite{Sterman78}. Again, based on length scales the small momentum of the gluons in the $J_i$ cannot resolve the inner structure of the hard process. The decoupling of the jets from the soft part is more subtle, but since jets have large total momentum their substructure cannot be resolved by the soft gluons in $S$. However, close to the threshold there are energy restrictions on the gluons, leading to the large logarithmic corrections. Hence, the soft function does not completely decouple and we have to take it into account. This soft function can be constructed by taking the eikonal approximation of the partonic process, i.e. we can build it out of Wilson lines. We will come back to this procedure in the next section.

Close to the threshold we have that \cref{eq:partonic analogue to hadronic in mellin } and \cref{eq:threshold factorization Mellin} must be equal, giving the fully factorized hard function
\begin{align}\label{eq:partonic hard function ratio in mellin}
    \Me{\omega}_{ij}(N,Q,\mu,\alpha_s(\mu))=\frac{\Me{J}_{i/i}(N,Q,\epsilon)\Me{J}_{j/j}(N,Q,\epsilon)}{\tilde{f}_{i/i}(N,\mu,\epsilon)\tilde{f}_{j/j}(N,\mu,\epsilon)}H_{ij}(Q,\alpha_{s}(\mu))\Me{S}_{ij}(N,Q,\mu,\alpha_{s}(\mu))\,.
\end{align}

Since the partonic function is defined to be infrared safe, the ratio of the jet and parton distributions must cancel the collinear divergences. The ratio of distributions might seem strange and it may not be obvious at this point how to evaluate them. But we will show later how this is done using renormalization group equations. 

The next step going forward is to make use of what we know of Wilson lines in order to construct the soft function. We will do this by constructing an eikonal cross section, i.e. a cross section where the radiation is restricted to be soft. Then we will show that the soft function can be replaced from \cref{eq:partonic hard function ratio in mellin} by the eikonal cross section. The motivation behind this is to use the non-Abelian eikonal exponentiation theorem to calculate the eikonal cross section.  


































  
\section{Factorization of Soft Gluons}\label{sec:factorization soft gluons}
In this section we will take a closer look at how we can organize the soft contributions coming from $\tilde{S}$ by constructing the eikonal cross section.

Near threshold all radiation is restricted to be soft compared with the hard scattering function. This naturally leads to an eikonal approximation for the cross section, i.e. we can construct this cross section by using Wilson lines. To this end, we consider the following Wilson lines\footnote{This definition is slightly different in appearance to the ones we derived in \cref{sec:wilson line properties}, but all the rules are equivalent.}
\begin{align}
    \mathcal{U}_{p}[x,\infty]=\mathcal{P}\exp{ig\int_{\infty}^{0}ds\,p\cdot A(x+ps)}\,,
    \\
    \mathcal{U}_{-p}[\infty,x]=\mathcal{P}\exp{-ig\int_{0}^{\infty}ds\,p\cdot A(x-ps)}\,,
\end{align}
where we have parametrized the path as $z^{\mu}=x^{\mu}+p^{\mu}s$, where $p$ is the light-like momentum of the incoming massless quark (or anti-quark) and $s$ is the proper time. From this we can construct the Drell-Yan Wilson line\footnote{If the notation $\mathcal{U}(0)$ is confusing it just mean that the composition are connected such that the two Wilson lines meet at space-time point 0.}
\begin{align}
    \mathcal{U}_{DY}(0)=\mathcal{U}_{-p_2}[\infty,0]\,\mathcal{U}_{p_1}[0,\infty]\,,
\end{align}
where $\mathcal{U}_{p_1}[\infty,0]$ and $\mathcal{U}_{-p_2}[\infty,0]$ are the Wilson lines evaluated along the classical trajectories of the incoming massless quark and anti-quark. The classical trajectory means that the particles are so energetic that they will not recoil as the soft gluons are emitted, such that they move along a straight line. This is necessary as we want to use Wilson lines on linear paths. From this we can construct the expectation value\footnote{There is an implicit average and sum over colour in this expectation value.}
\begin{align}\label{eq:1st DY wilson loop}
    \mathcal{W}_{DY}(0)=\bra{0}\Bar{\mathcal{T}}\,\mathcal{U}_{DY}^{\dagger}(0)\mathcal{T}\,\mathcal{U}_{DY}(0)\ket{0}\,,
\end{align}
where $\mathcal{T}$ and $\Bar{\mathcal{T}}$ are the time and anti-time ordering operators. 

The main idea from here is to construct a Wilson loop expectation value. To this end, we rewrite \cref{eq:1st DY wilson loop} by inserting a complete set of final states
\begin{align}
    \mathcal{W}_{DY}(0)=\sum_{n}\bra{0}\Bar{\mathcal{T}}\,\mathcal{U}_{DY}^{\dagger}(0)\ket{n}\bra{n}\mathcal{T}\,\mathcal{U}_{DY}(0)\ket{0}\,,
\end{align}
and the eikonal cross section is then constructed by inserting a energy conserving delta function, i.e.
\begin{align}
    \sigma_{ij}^{(\text{eik})}(\tau,Q)=\sum_{n}\delta((1-\tau)Q/2-E_{n})\bra{0}\Bar{\mathcal{T}}\,\mathcal{U}_{DY}^{\dagger}(0)\ket{n}\bra{n}\mathcal{T}\,\mathcal{U}_{DY}(0)\ket{0}\,,
\end{align}
where $E_{n}=\sum_{n}k_{n}^{0}$ is the total energy of the emitted gluons, which is restricted to be $(1-\tau)Q/2$. 

By using the Fourier representation of the delta function, we find
\begin{align}\label{eq:eikonal cross section wilson loop}
    \sigma_{ij}^{(\text{eik})}(\tau,Q,\mu,\alpha_s,\epsilon)&=\frac{Q}{2}\sum_{n}\int_{-\infty}^{\infty}\frac{dy^{0}}{2\pi}\,e^{iy^{0}((1-\tau)Q/2-\sum_{n}k_{n}^{0})}\bra{0}\Bar{\mathcal{T}}\,\mathcal{U}_{DY}^{\dagger}(0)\ket{n}\bra{n}\mathcal{T}\,\mathcal{U}_{DY}(0)\ket{0}\nonumber
    \\
    &=\frac{Q}{2}\int_{-\infty}^{\infty}\frac{dy^{0}}{2\pi}\,e^{iy^{0}(1-\tau)Q/2}\,\mathcal{W}_{DY}(y)\,,
\end{align}
where we have explicitly included the dimensional regulator $\epsilon$ as an argument in the eikonal cross section, as it contains divergences. We also used the translation property $\mathcal{U}_{DY}(y)=e^{iP\cdot y}\mathcal{U}_{DY}(0)e^{-iP\cdot y}$, where $y^{\mu}=(y^{0},\Vec{0})$, to define
\begin{align}
    \mathcal{W}_{DY}(y)&=\bra{0}\Bar{\mathcal{T}}\,\mathcal{U}_{DY}^{\dagger}(y)\mathcal{T}\,\mathcal{U}_{DY}(0)\ket{0}\equiv\bra{0}\mathcal{P}\exp\Big(ig\oint_{\gamma_{DY}}dx^{\mu}A_{\mu}(x)\Big)\ket{0}
\end{align}
which is an expectation value of a gauge invariant Wilson loop integrated over the path $\gamma_{DY}$, see \cref{fig:DYWilsonLoop}\footnote{This might not seem as a Wilson loop, but the Wilson lines go out to infinity where they combine.}. 
%%%%%%%%%%% figure %%%%%%%%%%%%%%%%
\begin{figure}
    \centering
    \includegraphics[scale=0.3]{Figures/DrellYanWilsonLoop.pdf}
    \caption{Integration contour for the eikonal approximation in the Drell-Yan process, and $y=(y^{0},\Vec{0})$.}
    \label{fig:DYWilsonLoop}
\end{figure}
%%%%%%%%%%%%%%%%%%%%%%%%%%%%%%%%%%

We can now do the same as we did in \cref{sec:threshold factorization} and use factorization properties of cross sections to define eikonal distributions responsible for the collinear divergences in $\sigma_{ij}^{(\text{eik})}$. 
So the next object to consider is the eikonal analogue to the light-cone distributions $f_{i/i}$. In the limit $x\rightarrow 1$, parton-in-parton distributions can be shown to take the form \cite{Korchemsky:1988si}\footnote{Again, there is an implicit average and sum over colour in the expectation value.}
\begin{align}
    f_{i/i}^{(\text{eik})}(x,\mu,\epsilon)&=\frac{Q}{2}\int_{-\infty}^{\infty}\frac{dy^{-}}{2\pi}\,e^{iy^{-}(1-x)Q/2}\bra{0}\Bar{\mathcal{T}}\{\mathcal{U}_{-p_1}[y,\infty]\}\mathcal{T}\{\mathcal{U}_{p_1}[0,\infty]\}\ket{0}\nonumber
    \\
    &=\frac{Q}{2}\int_{-\infty}^{\infty}\frac{dy^{-}}{2\pi}\,e^{iy^{-}(1-x)Q/2}\mathcal{W}_{\gamma_{p_1}}(y)\,,
\end{align}
where the path $\gamma_{p_1}$ is the $p_1$ part of $\gamma_{DY}$. 

By using these eikonal distributions the eikonal cross section can be written as
\begin{align}\label{eq:1st eikonal cross section}
    \sigma_{ij}^{(\text{eik})}(w,Q,\mu,\alpha_s,\epsilon)&=\int dw_1 dw_2 dw'\,f_{i/i}^{(\text{eik})}(w_1,\mu,\epsilon)\,f_{i/i}^{(\text{eik})}(w_2,\mu,\epsilon)\,\omega_{ij}^{(\text{eik})}(w',Q,\mu,\alpha_s)\nonumber
    \\
    &\hspace{1cm}\delta(w-w_1-w_2-w')\,.
\end{align}
where we defined the energy fractions $w=1-\tau$, $w_1=1-x_1$, $w_2=1-x_2$ and $w'=1-z$. 

In \cref{sec:threshold factorization} we were working in Mellin space, so by taking the Mellin transform of \cref{eq:1st eikonal cross section} we obtain 
\begin{align}\label{eq:eikonal approcximation of partonic}
    \Me{\sigma}^{(\text{eik})}(N,Q,\mu,\alpha_s,\epsilon)=\Me{f}_{i/i}^{(\text{eik})}(N,\mu,\epsilon)\,\Me{f}_{j/j}^{(\text{eik})}(N,\mu,\epsilon)\,\Me{\omega}_{ij}^{(\text{eik})}(N,Q,\mu,\alpha_s)\,,
\end{align}
where we have used that the Mellin transform of the delta function is
\begin{align}
    \int_{0}^{1}d\tau\tau^{N-1}\,\delta(1-\tau-(1-x_1)-(1-x_2)-(1-z))=e^{-N(1-x_1+1-x_2+1-z)}\,,
\end{align}
where $e^{-N(1-x)}=x^{N-1}$ in the large $N$ limit. The result in \cref{eq:eikonal approcximation of partonic} is the eikonal approximation of \cref{eq:partonic analogue to hadronic in mellin }. 

We can also make an eikonal approximation of the near threshold cross section \cref{eq:threshold factorization Mellin}, which can be constructed in a similar fashion as we have done for \cref{eq:eikonal approcximation of partonic}, given by
\begin{align}\label{eq:eikonal approcximation of near threshold}
    \Me{\sigma}^{(\text{eik})}(N,Q,\mu,\alpha_s,\epsilon)=\Me{J}_{i/i}^{(\text{eik})}(N,\mu,\epsilon)\,\Me{J}_{j/j}^{(\text{eik})}(N,\mu,\epsilon)\,\Me{S}_{ij}(N,Q,\mu,\alpha_s)\,,
\end{align}
where we have used that the soft function by definition contains the soft contributions, i.e. $S_{ij}=S_{ij}^{(\text{eik})}$. Then we can use that \cref{eq:eikonal approcximation of partonic} and \cref{eq:eikonal approcximation of near threshold} must be equal near threshold, giving
\begin{align}\label{eq:partonic eikonal with soft function}
    \Me{\omega}_{ij}^{(\text{eik})}(N,Q,\mu,\alpha_s)=\frac{\Me{J}_{i/i}^{(\text{eik})}(N,\mu,\epsilon)\,\Me{J}_{j/j}^{(\text{eik})}(N,\mu,\epsilon)}{\Me{f}_{i/i}^{(\text{eik})}(N,\mu,\epsilon)\,\Me{f}_{j/j}^{(\text{eik})}(N,\mu,\epsilon)}\Me{S}_{ij}(N,Q,\mu,\alpha_s)\,.
\end{align}

Apart from contributions from hard virtual gluons, \cref{eq:partonic eikonal with soft function} is the eikonal approximation of \cref{eq:partonic hard function ratio in mellin}. This means that we have an expression for the soft function $\Me{S}$, so if we solve for the soft function in \cref{eq:partonic eikonal with soft function} and insert it into \cref{eq:partonic hard function ratio in mellin}, we find that the hard partonic function can be written as
\begin{align}\label{eq:partonic and eikonal plus ration}
    \Me{\omega}_{ij}(N,Q,\mu,\alpha_s(\mu))=&\Big[\frac{\Me{J}_{i/i}(N,Q,\epsilon)\Me{J}_{j/j}(N,Q,\epsilon)}{\tilde{f}_{i/i}(N,\mu,\epsilon)\tilde{f}_{j/j}(N,\mu,\epsilon)}\Big]\Big[\frac{\Me{f}_{i/i}^{(\text{eik})}(N,\mu,\epsilon)\,\Me{f}_{j/j}^{(\text{eik})}(N,\mu,\epsilon)}{\Me{J}_{i/i}^{(\text{eik})}(N,\mu,\epsilon)\,\Me{J}_{j/j}^{(\text{eik})}(N,Q,\epsilon)}\Big]\nonumber
    \\
    &H_{ij}(Q,\alpha_{s}(\mu))\,\Me{\omega}_{ij}^{(\text{eik})}(N,Q,\mu,\alpha_s)\,,
\end{align}
where both of these ratios are defined such that they cancel each others collinear divergences. This expression looks daunting, but as previously mentioned the ratios can be simplified by using the renormalization group properties of the distributions, which we will do in \cref{sec:RGE for parton in parton}. The eikonal function $\omega_{ij}^{(\text{eik})}$ is the main function of interest, as it contains the parts where soft contributions cancel to give the large logarithms.

We have managed to bring the hard partonic function $\Me{w}_{ij}$ on a fully factorized form in terms of the eikonal function $\Me{\omega}_{ij}^{(\text{eik})}$. It might not have been obvious what the point of this whole refactorization is, but the idea is that we now have an expression where the IR-divergences are grouped into different terms responsible for different regions of phase space. Before we go into details of how to find the eikonal function $\Me{w}_{ij}
^{\text{(eik)}}$ we will take a closer look at the renormalization properties of Wilson lines and the renormalization group equations for the distributions in \cref{eq:partonic and eikonal plus ration}. The reason for taking this slight detour is to introduce the cusp anomalous dimension, which we will have use for later. %The procedure to calculate it is very complicated, but we will later try to explain how it can be done.

%But first we will in the next couple of sections look at renormalization properties of parton-in-parton distributions in the $x\rightarrow 1$ limit and the renormalization properties of Wilson lines.

\subsection*{Overview of IR-divergences}
We have tried to point out as we went along where the divergences in the different expressions above are, but let us try to make that more clear.

First of all, the hadronic cross section $\sigma_{h_{1}h_{2}}$ is of course finite. On the other hand, we observed in a fixed order calculation at NLO that the partonic function $\omega_{ij}$ has collinear divergences. To single this contribution out we defined a partonic analog to the hadronic cross section \cref{eq:partonic analogue to hadronic in mellin } in Mellin space, in such a way that the parton-in-parton distributions $f_{i/i}$ were responsible for these collinear divergences, rendering $\omega_{ij}$ IR-finite. The consequence of this definition is that the parton-in-hadron distributions $f_{i/h}$ do not contain any singularities, which is important as one wants to take these from experimental measurements. 

From there we went on to write down a near threshold form of the partonic cross section \cref{eq:threshold factorization Mellin}, where all collinear singularities are included in the jet subprocesses $J_{i/i}$. The hard subprocess $H$ includes only lines that are off-shell and does not contain any large logarithms. The soft subprocess $S_{ij}$ is defined to include all wide angle soft radiation.

Finally we made an eikonal cross section $\sigma_{ij}^{(\text{eik})}$ in \cref{eq:eikonal cross section wilson loop}, which contains collinear singularities due to the light-like momenta $p_1$ and $p_2$ of the incoming partons\footnote{Or rather due to the light-like directional vectors $n_{1}$ and $n_{2}$ along the momenta $p_1$ and $p_2$.}. We factorized this cross section in \cref{eq:eikonal approcximation of partonic} and \cref{eq:eikonal approcximation of near threshold} such that $f_{i/i}^{(\text{eik})}$ and $J_{i/i}^{(\text{eik})}$ are responsible for these collinear divergences. The eikonal cross section $\sigma_{ij}
^{(\text{eik})}$ also contains soft divergences, but according to the KLN-theorem these cancel in the sum over all final states, leading to large logarithms. Hence, the soft function $S_{ij}$ and the eikonal function $\omega_{ij}
^{(eik)}$ are free of IR-divergences. 
\section{Renormalization of Wilson Lines}
In this section we will look at the renormalization properties of Wilson lines, and in particular find the cusp anomalous dimension. Later we will see that the cusp anomalous dimension appear in evolution equations for parton distributions and in the exponent of the exponentiated eikonal cross section. Hence, it is a fundamental ingredient in resummation with Wilson lines. 

There are two kinds of cusp anomalous dimensions, one is for Wilson lines on light-cone and one for Wilson lines off light-cone. We are considering the case of massless quarks, so the Wilson lines on light-cone is those of main focus, i.e. we need the on light-cone cusp anomalous dimension. We will show one way of calculating it in the next section, but first we will show how it appears.


In order to find the behaviour of Wilson lines at different scales, we can use the basic principle that the bare definition must be independent on the renormalization scale, i.e. the bare Wilson line satifies
\begin{align}\label{eq:bare wilson line}
    \mu\dv{}{\mu}\mathcal{U}_{\gamma}^{0}=0\,.
\end{align}
The bare Wilson line is given in terms of the bare coupling $g_0$ and the bare gauge field. Let us then rescale the field as in \cref{eq:rescaled gauge field QCD}, and use \cref{eq:counterterms} to define the relation
\begin{align}
    \mathcal{Z}_{3}^{1/2}g_{0}=\mathcal{Z}_{g}g\,,
\end{align}
giving the renormalized Wilson line\footnote{The scale factor $\mu$ is as usual hidden in g, i.e. we always make the substitution $g(\mu)\rightarrow\mu^{d-4}g$.}
\begin{align}
    \mathcal{U}_{\gamma}(g,\mu)=\mathcal{P}\exp{ig\mathcal{Z}_{g}\int_{\gamma}dz^{\mu}A_{\mu}(z)}\,.
\end{align}
A smooth Wilson with no cusps is completely renormalized as long as the coupling and the field is renormalized \cite{POLYAKOV1980171,DOTSENKO1980527}. Hence, by applying \cref{eq:bare wilson line} we would find a regular Callan-Symanzik equation. 

However, since we are studying a quark--antiquark pair that meets at a point and annihilates we are in interested in paths with cusps. These cusps will contribute with additional UV-divergences, so-called cusp divergences. A cusp in a Wilson line is characterized by two directional vectors $n_{1}^{\mu}$ and $n_{2}^{\mu}$ and the cusp divergence is a function of the angle $\chi$ between these two vectors. In Minkowski space this angle is defined as
\begin{align}\label{eq:Minkowski space angle}
    \cosh \chi=\frac{n_{1}\cdot n_2}{\sqrt{n_{1}^{2}n_{2}^{2}}}\,,
\end{align}
and the Wilson line will acquire a dependency on the regulator $\epsilon$ in dimensional regularization, i.e. $\mathcal{U}_{\gamma}(g,\mu,\epsilon)$. If both vectors are off light-cone, i.e. $n_{1}^{2}\neq 0$ and $n_{2}^{2}\neq 0$, the cusp divergences can be treated multiplicatively by introducing a multiplicative factor $\mathcal{Z}_{\text{cusp}}$ \cite{DOTSENKO1980527,Korchemsky:1987wg}
\begin{align}
    \widetilde{\mathcal{U}}_{\gamma}(g,\mu)=\mathcal{Z}_{\text{cusp}}(\chi,g,\mu,\epsilon)\,\mathcal{U}_{\gamma}(g,\mu,\epsilon)
\end{align}
giving the Callan-Symanzik equation
\begin{align}
    \Big(\mu\pdv{}{\mu}+\beta(g)\pdv{}{g}\Big)\,\ln\widetilde{\mathcal{U}}_{\gamma}(g,\mu)=\Gamma_{\text{cusp}}(\chi,g)\,,
\end{align}
where the cusp anomalous dimension is given by
\begin{align}
    \Gamma_{\text{cusp}}(\chi,g)=\lim_{\epsilon\to 0}\frac{\mu}{\mathcal{Z}_{\text{cusp}}}\dv{}{\mu}\mathcal{Z}_{\text{cusp}}(\chi,g,\epsilon)=\lim_{\epsilon\to 0}\dv{}{\ln\mu}\ln\mathcal{Z}_{\text{cusp}}(\chi,g,\epsilon)\,.
\end{align}
In regular UV-renormalization we have that the renormalization factor removes the $\epsilon$ dependence via counterterms, see \cref{sec:Renormalization}. Hence, we can calculate a Wilson line with cusps in perturbation theory and use $\mathcal{Z}_{\text{cusp}}$ to pull out the divergent part.  

If one or both vectors are on the light-cone, we can no longer use the multiplicative renormalization technique. This follows from the fact that for light-like vectors, \cref{eq:Minkowski space angle} blows up, and creates additional divergences. In dimensional regularization these additional divergences are double poles, i.e. of the form $1/\epsilon^{2}$. In \cref{sec:NLO drell yan calculation} we found that by adding the real and virtual gluon emission the double pole vanished and the cross section acquired large logarithmic dependancy. Thus, treating the $1/\epsilon
^{2}$ divergence would give a way of managing these large contributions by the renormalization properties of Wilson lines. 

Now, there is a relation between the on light-cone and off light-cone cusp anomalous dimension that we can use. In \cite{KORCHEMSKAYA1992169} it was found that the relation between the two in the limit of large $\chi$, is given by
\begin{align}\label{eq:off lightcone and on lightcone}
    \lim_{\chi\rightarrow\infty}\Gamma_{\text{cusp}}(\chi,g)=\chi\Gamma_{\text{cusp}}(g)+\mathcal{O}(\chi^{0})\,.
\end{align}
In this large limit, it follows from \cref{eq:Minkowski space angle} that
\begin{align}\label{eq:large minkowski limit}
    \chi=\ln\Big(\frac{2n_{1}\cdot n_{2}}{\sqrt{n_{1}^{2}n_{2}^{2}}}\Big)\,.
\end{align}
We observe that if we differentiate \cref{eq:off lightcone and on lightcone} with respect to $\ln n_{1}\cdot n_2$ we remove the troublesome denominator that blows up for light-like vectors. Therefore, we can write the on light-cone cusp anomalous dimension as\footnote{If the $\epsilon\rightarrow 0$ limit seem sketchy it is ment to be happen after the differentiation has been performed.}
\begin{align}\label{eq:on lightcone cusp anomalous}
    \Gamma_{\text{cusp}}(g)=\lim_{\epsilon\to 0}\dv{}{\ln n_1\cdot n_2}\dv{}{\ln\mu}\ln\mathcal{Z}_{\text{cusp}}(\chi,g,\epsilon)\,,
\end{align}
which is the expression we will use after we have calculated $\Gamma(\chi,g)$ in the next section.
%We can then integrate over $n_{1}\cdot n_2$, giving the modified Callan-Symanzik equation
%\begin{align}
%    \Big(\mu\pdv{}{\mu}+\beta(g)\pdv{}{g}\Big)\,\ln\widetilde{\mathcal{U}}_{\gamma}(g,\mu)=\Gamma_{\text{cusp}}(g)\ln n_{1}\cdot n_{2}+\Gamma(g)\,,
%\end{align}
%where $\Gamma(g)$ is some integration constant.

Wilson lines with endpoints will also have their own renormalization factors, and a corresponding endpoint anomalous dimension \cite{KORCHEMSKY1986459}. But we will only consider semi-infinite Wilson lines with endpoint at infinity. These contains IR-divergnces, which we will treat with an exponential regulator that suppress such contributions. 

\subsection{One-Loop Cusp Anomalous Dimension}
To calculate the one-loop cusp anomalous dimension, we consider the case of two semi-infinite Wilson lines bounded from below, see \cref{eq:semi-infinite Wilson line 0-to-infty}. We denote these as
\begin{align}
    \mathcal{U}_{\gamma_1}[\infty,0]&=\mathcal{P}\exp\Big(ig\int_{0}^{\infty}d\lambda_1\,n_{1}\cdot A(\lambda_1 n_1)\Big)\,,
    \\
    \mathcal{U}_{\gamma_2}[\infty,0]&=\mathcal{P}\exp\Big(ig\int_{0}^{\infty}d\lambda_2\,n_{2}\cdot A(\lambda_2 n_2)\Big)\,.
\end{align}

To construct the geometry of the diagrams in \cref{fig:one loop cusp anomalous dimension}, we use that Wilson lines are path-transitive and can be written as the composition\footnote{We could have used $\mathcal{W}_{DY}$ to calculate the cusp anomalous dimension, but that calculation is more complicated.}
\begin{align}\label{eq:wedged wilson line}
    \mathcal{U}_{\wedge}(0)=\mathcal{U}_{\gamma_1}[\infty,0]\mathcal{U}_{\gamma_2}[\infty,0]\,.
\end{align}
Expanding \cref{eq:wedged wilson line} to $\mathcal{O}(g^{2})$, we find
\begin{align}\label{eq:two wilson expansion}
    \mathcal{U}_{\wedge}(0)&=1+igt^{a}n_{1}^{\mu}\int_{0}^{\infty}d\lambda_1\,A_{\mu}^{a}(\lambda_{1}n_1)+igt^{b}n_{2}^{\mu}\int_{0}^{\infty}d\lambda_2\,A_{\mu}^{b}(\lambda_{2}n_2)\nonumber
    \\
    &\hspace{1cm}-g^{2}t^{a}t^{b}n_{1}^{\mu}n_{2}^{\nu}\int_{0}^{\infty}d\lambda_1\int_{0}^{\infty}d\lambda_2\,A_{\mu}^{a}(\lambda_1n_1)A_{\nu}^{b}(\lambda_2n_2)\,,
\end{align}
which follows from the expansions we discussed in \cref{sec:wilson line properties}. But in \cref{sec:wilson line properties} we integrated over $\lambda$ directly by Fourier transforming the fields, giving \cref{eq:semi-infinite Wilson line 0-to-infty}. In momentum space, we have IR-divergences when $n\cdot k\rightarrow 0$ and $k^{2}\rightarrow 0$. These originate from the Wilson line propagator and after the gauge fields have been Wick contracted to give the gauge field propagator. However, it is easier to work in coordinate space for this calculation. The IR-divergence in coordinate space originates from $\lambda\rightarrow\infty$, so to treat it we insert an exponential regulator in the exponent of the Wilson lines
\begin{fmffile}{wilsonone}
\begin{figure}
\centering
\begin{fmfgraph*}(150,100)
\fmfleft{i1} 
\fmfright{o1,o2}
\fmf{fermion}{i1,v1}
\fmf{fermion}{i1,v2}
\fmf{plain}{v1,o1}
\fmf{plain}{v2,o2}
\fmflabel{$\lambda_1 n_1$}{v2}
\fmflabel{$\lambda_2 n_2$}{v1}
\fmffreeze
\fmf{gluon}{v1,v2}
\end{fmfgraph*}
\hspace{1cm}
\begin{fmfgraph*}(150,100)
\fmfleft{i1} 
\fmfright{o1,o2}
\fmf{fermion}{i1,o1}
\fmf{plain,label=$\lambda_{1}n$,l.side=left}{i1,v2}
\fmf{fermion}{v2,v3}
\fmf{plain, label=$\lambda_2 n$,l.side=left}{v3,o2}
\fmffreeze
\fmf{gluon,left,tension=0}{v2,v3}
\end{fmfgraph*}
\caption{Wilson line diagrams contributing to the one-loop cusp anomalous dimension $\Gamma(\chi,g)$.}
\label{fig:one loop cusp anomalous dimension}
\end{figure}
\end{fmffile}
%%%%%%%%%%%%%%
\begin{align}\label{eq:IR regularized Wilson line}
    \mathcal{U}^{\delta}[\infty,0]=\mathcal{P}\exp(ig\int_{0}^{\infty}d\lambda\,n\cdot A(\lambda n)e^{-\delta\lambda\sqrt{-n^{2}}})\,,
\end{align}
which was proposed for Wilson line calculations in \cite{article}. The idea here is that $\delta\sqrt{-n^{2}}>0$, so that the exponential factor smoothly cuts off the $\lambda\rightarrow\infty$ contribution. This is guaranteed to yield an IR-finite result for the integral in \cref{eq:IR regularized Wilson line}, and all the remaining poles are of UV origin, i.e. $\lambda\rightarrow 0$.  

\subsubsection*{One-Loop Calculation}
To calculate the full cusp anomalous dimension $\Gamma_{\text{cusp}}(\chi,g)$, we would have to calculate both diagrams in \cref{fig:one loop cusp anomalous dimension}. But as we can see, the diagram on the right-hand does not depend on the cusp angle as the radiation is from the same line, so to find $\Gamma_{\text{cusp}}(g)$ we only focus on the left-hand diagram. 

To calculate this contribution we consider the expectation value
\begin{align}\label{eq:wedge wilson loop}
    \mathcal{W}_{\wedge}=\bra{0}\mathcal{T}\mathcal{U}_{\wedge}(0)\ket{0}\,
\end{align}
where the $\mathcal{T}$ is the time-ordering operator. Expanding \cref{eq:wedge wilson loop} to $\mathcal{O}(g^{2})$ using the expansion in \cref{eq:two wilson expansion}, will give
\begin{align}
    \mathcal{W}_{\wedge}&=1+\mathcal{W}_{\wedge}^{(1)}\nonumber
    \\
    &=1-g^{2}t^{a}t^{b}n_{1}^{\mu} n_{2}^{\mu}\int_{0}^{\infty}d\lambda_{1}\int_{0}^{\infty}d\lambda_{2}\,D_{\mu\nu}^{ab}(\lambda_{1}n_1-\lambda_{2}n_2)\,,
\end{align}
where we Wick contracted the emitted gluons to give the propagator. The propagator in coordinate space is given by \cite{article},
\begin{align}
    D_{\mu\nu}^{ab}(x-y)=-\mathcal{N}\frac{g_{\mu\nu}\delta^{ab}}{(-(x-y)^{2}+i\epsilon)^{d/2-1}}\,,
\end{align}
where
\begin{align}
    \mathcal{N}=\frac{\Gamma(d/2-1)}{4\pi^{d/2}}\,.
\end{align}
Here we should keep in mind that the Feynman prescription $i\epsilon$ and the regulator $\epsilon$ are not the same when we expand in $d=4-2\epsilon$. 

Let us then insert the IR-regulator given in \cref{eq:IR regularized Wilson line}, giving the expression
\begin{align}\label{eq:amplitude radiation wilson line cusp}
    \mathcal{W}_{\wedge}^{(1)}&=-g^{2}t^{a}t^{b}n_{1}^{\mu} n_{2}^{\mu}\int_{0}^{\infty}d\lambda_{1}\int_{0}^{\infty}d\lambda_{2}\,D_{\mu\nu}^{ab}(\lambda_{1}n_1-\lambda_{2}n_2)\,e^{-\delta(\lambda_1\sqrt{-n_{1}^{2}}+\lambda_2\sqrt{-n_{2}^{2}})}\nonumber
    \\
    &=g^{2}C_{F}\mathcal{N}(\epsilon)n_{1}\cdot n_{2}\int_{0}^{\infty}d\lambda_{1}\int_{0}^{\infty}d\lambda_{2}\,\frac{e^{-\delta(\lambda_1\sqrt{-n_{1}^{2}}+\lambda_2\sqrt{-n_{}^{2}})}}{(-(\lambda_{1}n_{1}-\lambda_{2}n_{2})^{2})^{1-\epsilon}}\,.
\end{align}
To evaluate the integrals we can make the change of variables
\begin{align}
    \lambda_1&=\frac{\alpha x}{\sqrt{-n_{1}^{2}}}\,,
    \\
    \lambda_{2}&=\frac{\alpha(1-x)}{\sqrt{-n_{2}^{2}}}\,,
\end{align}
where $x\in[0,1]$ and $\alpha\in[0,\infty)$, giving the Jacobian
\begin{align}
    \mathcal{J}=\frac{\alpha}{\sqrt{n_{1}^{2}n_{2}^{2}}}\,.
\end{align}
With these changes we get the following integral
\begin{align}
    I&=\int_{0}^{\infty}d\lambda_{1}\int_{0}^{\infty}d\lambda_{2}\,\frac{e^{-\delta(\lambda_1\sqrt{-n_{1}^{2}}+\lambda_2\sqrt{-n_{}^{2}})}}{(-(\lambda_{1}n_{1}-\lambda_{2}n_{2})^{2})^{1-\epsilon}}\nonumber
    \\
    &=\frac{1}{\sqrt{n_{1}^{2}n_{2}^{2}}}\int_{0}^{1}dx \frac{1}{(x^{2}+(1-x)^{2}+2x(1-x)\cosh\gamma)^{1-\epsilon}}\int_{0}^{\infty}d\alpha\,e^{-\delta\alpha}\alpha^{-1+2\epsilon}\,,
\end{align}
where we defined 
\begin{align}\label{eq:gamma minkowski}
    \cosh\gamma=-\frac{n_{1}\cdot n_{2}}{\sqrt{n_{1}^{2}n_{2}^{2}}}\,,
\end{align}
and by inserting this back into \cref{eq:amplitude radiation wilson line cusp}, we get
\begin{align}
    \mathcal{W}_{\wedge}^{(1)}=-g^{2}C_{F}\mathcal{N}(\epsilon)\int_{0}^{1}dx\frac{\cosh\gamma}{(x^{2}+(1-x)^{2}+2x(1-x)\cosh\gamma)^{1-\epsilon}}\int_{0}^{\infty}d\alpha\,e^{-\delta\alpha}\alpha^{-1+2\epsilon}\,.
\end{align}

Let us evaluate the $\alpha$ integral by another change of variable $y=\alpha\delta$, giving
\begin{align}
    \int_{0}^{\infty}d\alpha\,e^{-\delta\alpha}\alpha^{-1+2\epsilon}=\delta^{-2\epsilon}\int_{0}^{\infty}dy\,e^{-y}y^{-1+2\epsilon}=\delta^{-2\epsilon}\,\Gamma(2\epsilon)\,,
\end{align}
where we used the integral representation of the Gamma function. This gamma function has the expansion as $\epsilon\rightarrow 0$
\begin{align}
    \Gamma(2\epsilon)=\frac{1}{2\epsilon}+\mathcal{O}(\epsilon^{0})\,.
\end{align}

The $x$ integral can be rewritten in terms of the hypergeometric function $_{2}F_{1}$. However, we want the expansion in the limit $\epsilon\rightarrow 0$, so it is inconvenient to use this representation. Let us instead set $\epsilon=0$ in this integral, giving
\begin{align}
    \int_{0}^{1}dx\frac{\cosh\gamma}{(x^{2}+(1-x)^{2}+2x(1-x)\cosh\gamma)}=\gamma\coth\gamma\,.
\end{align}
This integral would not be convergent without the definition of $\gamma$ in \cref{eq:gamma minkowski}. But we want our result in terms of $\chi$, and from \cref{eq:Minkowski space angle} these are related in the following way
\begin{align}
    \cosh\gamma&=-\cosh\chi=\cosh(\chi+i\pi)\,,
\end{align}
giving
\begin{align}
    \gamma=\chi+i\pi
\end{align}
and
\begin{align}
    \coth(\chi+i\pi)=\coth\chi\,.
\end{align}
Also, we can safely neglect the terms that are non singular for $\epsilon\rightarrow 0$, i.e. $\delta^{-2\epsilon}\rightarrow 1$ and $\mathcal{N}(\epsilon)\rightarrow 1/4\pi^{2}$. This removes the IR regulator $\delta$ from the expression in a smooth way. Using all these relations, we find that the $\mathcal{O}(g^{2})$ expansion of the Wilson loop expectation value takes the form
\begin{align}\label{eq:one loop wedge wilson loop}
    \mathcal{W}_{\wedge}^{(1)}&=-g^{2}\,C_{F}\,\mathcal{N}(\epsilon)\,\delta^{-2\epsilon}\,\Gamma(2\epsilon)\,\gamma\coth\gamma\nonumber
    \\
    &=-g^{2}\,C_{F}\,\frac{1}{4\pi^{2}}\,\frac{1}{2\epsilon}\,(\chi+i\pi)\coth\chi\,.
\end{align}
As mentioned above, this is only one contribution to the cusp anomalous dimension $\Gamma_{cusp}(\chi,g)$. But the other contribution does not depend on the cusp angle, so when we perform the differentiation with respect to $\ln n_1\cdot n_2$ it will not contribute to $\Gamma_{cusp}(g)$ that we are interested in.

\subsubsection*{Cusp Anomalous Dimension $\Gamma_{\text{cusp}}(g)$}
In \cref{eq:on lightcone cusp anomalous} we had that the cusp anomalous dimension for Wilson lines on light-cone could be written as
\begin{align}
    \Gamma_{\text{cusp}}(g)=\lim_{\epsilon\to 0}\dv{}{\ln n_1\cdot n_2}\dv{}{\ln\mu}\ln\mathcal{Z}_{\text{cusp}}(\chi,g,\epsilon)\,.
\end{align}
To find $\Gamma_{\text{cusp}}(g)$ we can now use that $\mathcal{Z}_{\text{cusp}}$ is used to cancel the $\epsilon$ divergence from the Wilson line in \cref{eq:one loop wedge wilson loop}. We can also introduce the dependence on the scale $\mu$ in the usual way $g^{2}\rightarrow \mu^{2\epsilon}g^{2}$, giving the cusp factor 
\begin{align}
    \mathcal{Z}_{\text{cusp}}=1+g^{2}\mu^{2\epsilon}C_{F}\frac{1}{4\pi^{2}}\frac{1}{2\epsilon}(\chi+i\pi)\coth\chi\,.
\end{align}
Performing the differentiation and keeping only terms to $\mathcal{O}(g^{2})$, we find
\begin{align}
    \Gamma_{\text{cusp}}(g)&=\lim_{\epsilon\to 0}\dv{}{\ln n_1\cdot n_2}\mu\dv{}{\mu}\ln\Big(1+g^{2}\mu^{2\epsilon}C_{F}\frac{1}{4\pi^{2}}\frac{1}{2\epsilon}(\chi+i\pi)\coth\chi\Big)\nonumber
    \\
    &=\dv{}{\ln n_1\cdot n_2}\Big(g^{2}C_{F}\frac{1}{4\pi^{2}}(\chi+i\pi)\coth\chi\Big)\nonumber
    \\
    &=\frac{g^{2}}{4\pi^{2}}C_{F}\,,
\end{align}
where we in the last differentiation used that $\chi$ is given by \cref{eq:large minkowski limit} in the large limit, and that $\coth\chi=1$ in this limit. At first sight the this might seem a little fishy as the derivative of the logarithm gives the argument in the denominator. But if we expand this denominator it will give a $\mathcal{O}(g^{4})$ term and we are only considering the $\mathcal{O}(g^{2})$ correction. As usual we use that $\alpha_s=g^{2}/4\pi$, giving
\begin{align}\label{eq:one-loop cusp anomalous dimension}
    \Gamma_{\text{cusp}}(\alpha_s)=\frac{\alpha_s}{\pi}C_{F}\,,
\end{align}
which is the well known one-loop cusp anomalous dimension for a Wilson line in the fundamental representation \cite{Korchemsky:1987wg}.  


\section{Exponentiation of Parton-In-Parton Distributions}\label{sec:RGE for parton in parton}
In \cref{sec:QCD and Collinear factorization}, we discussed the renormalization group equation for the parton distribution functions $f_{i/P}(x,\mu)$, i.e. the DGLAP equation. Now we would like to discuss the renormalization group equations for the parton-in-parton disttribution functions $f_{i/i}$. In \cref{sec:lightcone parton in parton distributions} we derived the parton-in-parton distributions, given by
\begin{align}
    f_{i/i}(x)=\int\frac{dy^{-}}{4\pi}e^{-ixp^{+}y^{-}}\bra{q}\overline{\Psi}(y^{-})\gamma^{+}\Psi(0)\ket{q}\,,
\end{align}
where the product in the matrix element are eikonal fermions\footnote{Or particles dressed with Wilson lines.}. In \cite{Korchemsky:1988si} it was found that in the limit $x\rightarrow 1$, the parton-in-parton distributions obeys the evolution equation
\begin{align}\label{eq:RGE partoninparton}
    \mu\dv{}{\mu}f_{i/i}(x,\mu)=\int_{x}^{1}\frac{dz}{z}P_{i/i}\big(\frac{x}{z},\alpha_s\big)f_{i/i}(z,\mu)+\mathcal{O}((1-x)^{0})\,,
\end{align}
where the splitting functions has the asymptotic behaviour 
\begin{align}\label{eq:asymptotic splitting function}
    P_{i/i}(z,\alpha_s)=2\Gamma_{\text{cusp}}^{(i)}(\alpha_s)\Big[\frac{1}{1-z}\Big]_{+}+2C^{(i)}(\alpha_s)\delta(1-z)+\mathcal{O}((1-z)^{0})\,,
\end{align}
which is true to all order \cite{Korchemsky:1988si}, and the one-loop $\Gamma_{\text{cusp}}^{(q,\bar{q})}$ is the one we found in \cref{eq:one-loop cusp anomalous dimension}. We can verify this behaviour by taking the limit $z\rightarrow 1$ of the splitting functions we found in \cref{eq:qq splitting function} and \cref{eq:gg splitting function}, giving
\begin{align}
    P_{q/q}(z)&=2C_{F}\frac{\alpha_s}{\pi}\Big(\Big[\frac{1}{1-z}\Big]_{+}+\frac{3}{4}\delta(1-z)\Big)+\mathcal{O}(\alpha_{s}^{2})\,,
    \\
    P_{g/g}(z)&=2\frac{\alpha_s}{\pi}\Big(C_{A}\Big[\frac{1}{1-z}\Big]_{+}+\frac{\beta_{0}}{4}\delta(1-z)\Big)+\mathcal{O}(\alpha_{s}^{2})\,,
\end{align}
where $\beta_0$ is the one-loop beta coefficient, see \cref{eq:beta one-loop}, and we observe that the cusp anomalous dimension for $q,\bar{q}$ corresponds to the one we found in \cref{eq:one-loop cusp anomalous dimension}. We also observe that the cusp anomalous dimension for $i=g$ is given by
\begin{align}
    \Gamma_{\text{cusp}}^{(g)}(\alpha_s)=\frac{\alpha_s}{\pi}C_{A}+\mathcal{O}(\alpha_{s}^{2})\,,
\end{align}
which is just a matter of making the calculation we did in \cref{eq:one-loop cusp anomalous dimension} by using Wilson lines in the adjoint representation giving the Casimir invariant $C_{A}$. We can also read of the one-loop expression for $C^{(i)}$, given by
\begin{align}
    C^{(q,\bar{q})}(\alpha_s)&=\frac{\alpha_s}{\pi}\frac{3}{4}C_{F}+\mathcal{O}(\alpha_{s}^{2})\,,\label{eq:C for quarks}
    \\
    C^{(g)}(\alpha_s)&=\frac{\alpha_s}{\pi}\frac{\beta_{0}}{4}+\mathcal{O}(\alpha_{s}^{2})\,,\label{eq:C for gluons}
\end{align}

In order to solve \cref{eq:RGE partoninparton} we take the Mellin transform, giving the large $N$ equation\footnote{Remember that $z\rightarrow 1$ corresponds to to large N.}
\begin{align}
    \mu\dv{}{\mu}f_{i/i}(N,\mu)=P_{i/i}(N,\alpha_{s})f_{i/i}(N,\mu)+\mathcal{O}(1/N)\,.
\end{align}
where the moments of the splitting function takes the form
\begin{align}
    P_{i/i}(N,\alpha_s)&=2\Gamma_{\text{cusp}}^{(i)}(\alpha_s)\int_{0}^{1}dz\,z^{N-1}\Big[\frac{1}{(1-z)_{+}}\Big]+2C^{(i)}(\alpha_s)\int_{0}^{1}dz\,z^{N-1}\delta(1-z)\nonumber
    \\
    &=-2\Gamma_{\text{cusp}}^{(i)}(\alpha_s)\ln\bar{N}+2C^{(i)}(\alpha_s)\,,
\end{align}
where we neglect constant terms from the Mellin transform, see \cref{eq:App mellin of ln plus dist}. These constant terms would reproduce the constant terms we neglected in \cref{eq:Mellin space NLO of omegaqbarq}, so we do the same here. Hence, the evolution equation in Mellin space take the form
\begin{align}
    \dv{}{\ln\mu}\ln f_{i/i}(N,\mu)=-2\Gamma_{\text{cusp}}^{(i)}\ln\bar{N}+2C^{(i)}(\alpha_s)\,.
\end{align}
with the solution
\begin{align}\label{eq:RGE parton in parton}
    f_{i/i}(N,\mu)=\exp{-\int_{0}^{\mu^{2}}\frac{d\mu'^{2}}{\mu'^{2}}\Big(\Gamma_{\text{cusp}}^{(i)}(\alpha_{s}(\mu'))\ln\bar{N}-C^{(i)}(\alpha_{s}(\mu'))\Big)}\,.
\end{align}
where we have chosen the initial condition $f_{i/i}(N,\mu=0)=1$.

For the eikonal distributions $f_{i/i}^{(\text{eik})}$ there is a slight modification. We want them to be sum of plus distributions, so from \cref{eq:asymptotic splitting function} we must have that $C^{i}=0$. The solution can then be written as
\begin{align}\label{eq:RGE parton in parton eikonal}
    f_{i/i}^{(\text{eik})}=\exp{-\int_{0}^{\mu^{2}}\frac{d\mu'^{2}}{\mu'^{2}}\Gamma_{\text{cusp}}^{(i)}(\alpha_{s}(\mu')\ln\bar{N}}\,.
\end{align}
Notice that the choice of the lower boundary diverges for $\mu\rightarrow 0$, which will be used later to cancel divergences coming from $\Me{\sigma}^{\text{(eik)}}(N,\epsilon)$. The equations for $J_{i/i}$ and $J_{i/i}^{(\text{eik})}$ are completely analogous. 

\subsection{Hard Virtual Gluons}
With the solutions in \cref{eq:RGE parton in parton} and \cref{eq:RGE parton in parton eikonal} we are now ready to compute the ratios of distributions we had in \cref{eq:partonic and eikonal plus ration}. But first, we mentioned in \cref{sec:factorization soft gluons} that the main difference between the partonic function $\omega_{ij}$ and its eikonal approximation $\omega_{ij}^{(\text{eik})}$ are contributions $G_{ij}(Q,\mu)$ from hard virtual gluons \cite{KORCHEMSKY1993433}. So in general, we can write the relation between the two as
\begin{align}\label{eq:relation w and eikonal w}
    \Me{\omega}_{ij}(N,Q,\mu,\alpha_{s}(\mu))=H_{ij}(Q,\alpha_{s}(\mu))\,G_{ij}(Q,\mu)\,\Me{\omega}_{ij}^{(\text{eik})}(N,Q,\mu,\alpha_s)\,,
\end{align}
where $H(Q)$ is the same hard process without large logarithms. If we compare this expression with \cref{eq:partonic and eikonal plus ration}, we find that the hard virtual contributions are entirely describes by the ratios 
\begin{align}\label{eq:hard virtual gluons}
    G_{ij}(Q,\mu)=\Big[\frac{\Me{J}_{i/i}(N,Q,\epsilon)\Me{J}_{j/j}(N,Q,\epsilon)}{\tilde{f}_{i/i}(N,\mu,\epsilon)\tilde{f}_{j/j}(N,\mu,\epsilon)}\Big]\Big[\frac{\Me{f}_{i/i}^{(\text{eik})}(N,\mu,\epsilon)\,\Me{f}_{j/j}^{(\text{eik})}(N,\mu,\epsilon)}{\Me{J}_{i/i}^{(\text{eik})}(N,Q,\epsilon)\,\Me{J}_{j/j}^{(\text{eik})}(N,Q,\epsilon)}\Big]\,,
\end{align}
and as mentioned in \cref{sec:factorization soft gluons}, the two ratios are free of collinear divergences and so is $G(Q,\mu)$.  We can now use the solutions in \cref{eq:RGE parton in parton} and \cref{eq:RGE parton in parton eikonal} to compute the ratios, giving
\begin{align}\label{eq:ratio pdf and eikonal pdf}
    \frac{\Me{f}_{i/i}^{(\text{eik})}(N,\mu,\epsilon)\,\Me{f}_{j/j}^{(\text{eik})}(N,\mu,\epsilon)}{\tilde{f}_{i/i}(N,\mu,\epsilon)\tilde{f}_{j/j}(N,\mu,\epsilon)}=\exp{-2\int_{0}^{\mu^{2}}\frac{d\mu'^{2}}{\mu'^{2}}C^{(i)}(\alpha_{s}(\mu'))}\,,
\end{align}
where the logarithm $\ln\bar{N}$ accompanied by the cusp anomalous dimension has canceled. We have also used that $C^{(i)}=C^{(j)}$ for $i=q$ and $j=\bar{q}$, giving two times $C^{(i)}$\footnote{We choose to use generic subscripts even if we really mean $i=q$. In this way it would be easier to generalize to cases where we also have $i=g$.}. The ratio of jet distributions are equivalent, with the difference of $Q$ instead of $\mu$, i.e.
\begin{align}\label{eq:ratio jets and eikonal jets}
    \frac{\Me{J}_{i/i}(N,Q,\epsilon)\Me{J}_{j/j}(N,Q,\epsilon)}{\Me{J}_{i/i}^{(\text{eik})}(N,Q,\epsilon)\,\Me{J}_{j/j}^{(\text{eik})}(N,Q,\epsilon)}=\exp{2\int_{0}^{Q^{2}}\frac{d\mu'^{2}}{\mu'^{2}}C^{(i)}(\alpha_{s}(\mu'))}\,.
\end{align}
If we insert \cref{eq:ratio pdf and eikonal pdf} and \cref{eq:ratio jets and eikonal jets} into \cref{eq:hard virtual gluons}, we get after reshuffling the ratios that the contribution from hard virtual gluons can be written as
\begin{align}
    G_{ij}(Q,\mu)=\exp{2\int_{\mu^{2}}^{Q^{2}}\frac{d\mu'^{2}}{\mu'^{2}}C^{(i)}(\alpha_{s}(\mu'))}\,,
\end{align}
where we observe that by choosing $Q$ as an argument in the jet distributions, we have a contribution from this expression. But we have already set $Q=\mu$ on several occasions, so for the simplified result we do the same here, giving that $G_{ij}(Q,Q)=1$. With this choice we have from \cref{eq:relation w and eikonal w} that the partonic function $\Me{\omega}_{ij}$ is given by
\begin{align}\label{eq:partonic and eikonal relation}
    \Me{\omega}_{ij}(N,Q,\mu,\alpha_{s}(\mu))=H_{ij}(Q,\alpha_{s}(\mu))\,\Me{\omega}_{ij}^{(\text{eik})}(N,Q,\mu,\alpha_{\mu})\,.
\end{align}

We have reduced the problem of resumming large logarithmic contributions to finding the eikonal function $\Me{\omega}_{ij}^{(\text{eik})}(N)$. Hence, we will in the next section turn our attention back to the eikonal cross section $\Me{\sigma}_{ij}^{\text{eik}}$ and how to calculate it. 

%%%%%%%%% Eikonal cross section %%%%%%%%%%%%%%%%%
\section{The Eikonal Cross Section}
In order to calculate the eikonal cross section $\Me{\sigma}_{ij}^{\text{eik}}$, we look at the Wilson loop expectation value $\mathcal{W}_{DY}$ we found in \cref{sec:factorization soft gluons}. We will first make an $\mathcal{O}(g^{2})$ expansion of the expectation value, and then use the non-Abelian eikonal exponentiation theorem to find the exponentiated form. Then we will use the factorized form of the eikonal cross section in \cref{eq:eikonal approcximation of partonic} to find $\Me{\omega}_{ij}
^{\text{(eik)}}$. 

So let us start from the expectation value
\begin{align}
    \mathcal{W}_{DY}(y)&=\bra{0}\Bar{T}\,\mathcal{U}_{DY}^{\dagger}(y)T\,\mathcal{U}_{DY}(0)\ket{0}\,,
\end{align}
where a one-loop contribution is illustrated in \cref{fig:DYwilsonloopcutpropagator}. By
using the momentum space expansion of Wilson lines derived in \cref{sec:wilson line properties}, we find that to $\mathcal{O}(g^{2})$
\begin{align}\label{eq:order g2 DY wilson loop}
    \mathcal{W}_{DY}=1+g^{2}C_{F}\int\frac{d^{d}k}{(2\pi)^{d}}2\pi\delta^{+}(k^{2})\frac{p_{1}\cdot p_{2}}{p_1\cdot k\,p_2\cdot k}(e^{-iy^{0}k^{0}}-1)\,,
\end{align}
where we have used that $p_{1}^{2}=p_{2}^{2}=0$. We have also used that in diagrams such as \cref{fig:DYwilsonloopcutpropagator}, we have to use a cut gluon propagator \cite{Korchemsky:1992xv}
\begin{align}
    D_{\mu\nu\,+}^{ab}(k)=-\delta^{ab}g_{\mu\nu}2\pi\delta^{+}(k^{2})\,.
\end{align}
To rewrite this further, we can use that the ratio of momenta is invariant under rescaling of $p_1$ and $p_2$. With light-cone coordinates (see \cref{sec:Appendix Light-cone coordinates}), we have that
\begin{align}
    \frac{p_{1}\cdot p_{2}}{p_1\cdot k\,p_2\cdot k}&=\frac{n_{1}\cdot n_{2}}{n_{1}\cdot k\,n_{2}\cdot k}=\frac{1}{k^{+}k^{-}}\label{eq:k+k- relation in calculation}
    \\
    k^{2}&=2k^{+}k^{-}-k_{\perp}^{2}
    \\
    k^{0}&=(k^{+}+k^{-})/\sqrt{2}\label{eq:energy relation lightcone momenta}
\end{align}
where we used that $n_1$ and $n_2$ are light-like vectors. Let us also use the measure $d^{d}k=dk^{+}dk^{-}d^{d-2}k_{\perp}$, such that with the above rewritings the expansion can be rewritten as
\begin{align}\label{eq:W(Drell) to one-loop}
    \mathcal{W}_{DY}&=1+\frac{\alpha_{s}}{\pi}C_{F}\int\frac{d^{d-2}k_{\perp}}{(2\pi)^{d-2}}\int dk^{+}dk^{-}\,2\pi\delta(2k^{+}k^{-}-k_{\perp}^{2})\frac{(e^{-iy^{0}(k^{+}+k^{-})/\sqrt{2}}-1)}{k^{+}k^{-}}\,,
\end{align}
where we pulled out a factor of $(2\pi)^{2}$ to give the coupling. On this form, it is not obvious how to treat these integrals as the limits on the $k^{+}$ and $k^{-}$ are unspecified. Instead we use the non-Abelian exponentiation theorem, where there are restrictions on the form of the exponent to preserve the exponentiation conditions.
%%%%%%%%%%%%%%% figure %%%%%%%%%%%%%%%
\begin{figure}
    \centering
    \includegraphics[scale=0.3]{Figures/DrellYanLoop.pdf}
    \caption{A one-loop contribution to the Drell-Yan eikonal cross section.}
    \label{fig:DYwilsonloopcutpropagator}
\end{figure}
%%%%%%%%%%%%%%%%%%%%%%%%%%%%%%%%%%%%%%
But before we give the procedure to find the eikonal cross section, we take a look at the structure of the expansion in \cref{eq:W(Drell) to one-loop}. We observe that the factor in front of the integral looks very much like the one-loop cusp anomalous dimension in \cref{eq:one-loop cusp anomalous dimension}. So if we use that the scale of the coupling is $\alpha_{s}(k_{\perp})$, we can write the expansion as\footnote{It is understood that it is the cusp anomalous dimension for particles in the fundamental representation.}
\begin{align}\label{eq:W(Drell) to one-loop number two}
    \mathcal{W}_{DY}&=1+\int\frac{d^{2-2\epsilon}k_{\perp}}{(2\pi)^{1-2\epsilon}}\Gamma_{\text{cusp}}(\alpha_{s}(k_{\perp}))\int dk^{+}dk^{-}\,\delta(2k^{+}k^{-}-k_{\perp}^{2})\frac{(e^{-iy^{0}(k^{+}+k^{-})/\sqrt{2}}-1)}{k^{+}k^{-}}\,,
\end{align}
where we have inserted for $d=4-2\epsilon$ and it is understood that one has to use the running coupling at one-loop order, e.g. the one-loop in \cref{eq:one-loop strong coupling}. Further, by using the non-Abelian exponentiation theorem we can write $\mathcal{W}_{DY}$ as
\begin{align}\label{eq:relation wDY and WDY}
    \mathcal{W}_{DY}=1+\sum_{n=1}^{\infty}\mathcal{W}_{DY}^{(n)}=\exp\Big(\sum_{n=1}^{\infty}W_{DY}^{(n)}\Big)\,,
\end{align}
where $W_{DY}$ are the webs we alluded to in \cref{sec:exponentiation}\footnote{For more details on webs, see \cite{White:2015wha,article}.}. For Drell-Yan, we have that $\mathcal{W}_{DY}^{(1)}=W_{DY}^{(1)}$, which we will use for our calculation. This is not true in general, but we are only interested in the one-loop result.

From the non-Abelian eikonal exponentiation theorem, the Mellin transformed eikonal cross section can on the most general form be written as \cite{laenen2000power}\footnote{The expression in \cite{laenen2000power} is for joint resummation, i.e. threshold and low transverse momentum of the final state. We are only considering threshold resummation, so we adjust the expression to our purpose.}
\begin{align}\label{eq:general result of eikonal cross section}
    \Me{\sigma}_{ij}^{\text{(eik)}}(N,Q,\epsilon)=\exp\Big(&2\int\frac{d^{4-2\epsilon}k}{\Omega_{1-2\epsilon}}\,\theta\Big(\frac{Q}{\sqrt{2}}-k^{+}\Big)\theta\Big(\frac{Q}{\sqrt{2}}-k^{-}\Big)\nonumber
    \\
    &W_{DY}\big(k^{2},\frac{n_{1}\cdot k\,n_{2}\cdot k}{n_{1}\cdot n_2},\mu,\alpha_{s},\epsilon\big)\Big(e^{-Nk^{0}/Q}-1\Big)\Big)\nonumber
    \\
    &=\exp\big(\Me{E}^{\text{(eik)}}(N,\epsilon)\big)\,,
\end{align}
where $W_{DY}$ is the web, and the invariance under rescaling of the momentum is applied. The theta functions are used to cut off the $k^{+}$ and $k^{-}$ integrals such that they are UV-finite and resctricts the $k_{\perp}$ integral to have the maximum value of $Q^{2}$. Without this restriction, the exponentiation conditions would not be valid \cite{Laenen:2004pm}. The appearance of $N$ in the exponent can be understood from the taking the Mellin transform of \cref{eq:order g2 DY wilson loop}, using the saddle point approximation $y^{0}\approx -iN/Q$ as discussed in \cite{KORCHEMSKY1993433}. The angular factor can in $d$-dimension be found in \cref{eq:d-dimensional sphere area}. 

We can now use that $W_{DY}^{(1)}=\mathcal{W}_{DY}^{(1)}$, and use \cref{eq:W(Drell) to one-loop number two} to write the exponent as
\begin{align}
    \Me{E}^{\text{(eik)}}(N,Q,\epsilon)&=2\int\frac{d^{2-2\epsilon}k_{\perp}}{\Omega_{1-2\epsilon}}\Gamma_{\text{cusp}}(\alpha_{s}(k_{\perp}))\int dk^{+}dk^{-}\,\theta\Big(\frac{Q}{\sqrt{2}}-k^{+}\Big)\theta\Big(\frac{Q}{\sqrt{2}}-k^{-}\Big)\nonumber
    \\
    &\hspace{2cm}\times\delta(2k^{+}k^{-}-k_{\perp}^{2})\frac{1}{k^{+}k^{-}}\big(e^{-N(k^{+}+k^{-})/\sqrt{2}Q}-1\big)\nonumber
    \\
    &=4\int\frac{d^{2-2\epsilon}k_{\perp}}{\Omega_{1-2\epsilon}}\frac{\Gamma_{\text{cusp}}(\alpha_{s}(k_{\perp}))}{k_{\perp}^{2}}\int\frac{dk^{+}}{2k^{+}}\,\theta\Big(\frac{Q}{\sqrt{2}}-k^{+}\Big)\theta\Big(\frac{Q}{\sqrt{2}}-\frac{k_{\perp}^{2}}{2k^{+}}\Big)\nonumber
    \\
    &\hspace{2cm}\times\Big(e^{-N\big(k^{+}+\frac{k_{\perp}^{2}}{2k^{+}}\big)/\sqrt{2}Q}-1\Big)
\end{align}
where we have applied the delta function over $k^{-}$. Because of the theta functions, the lower and upper limits of the $k^{+}$ integral are finite. Hence, the exponent take the form 
\begin{align}
    \Me{E}^{\text{(eik)}}(N,Q,\epsilon)&=4\int\frac{d^{2-2\epsilon}k_{\perp}}{\Omega_{1-2\epsilon}}\frac{\Gamma_{\text{cusp}}(\alpha_{s}(k_{\perp}))}{k_{\perp}^{2}}\int_{k_{\perp}^{2}/\sqrt{2}Q}^{Q/\sqrt{2}}\frac{dk^{+}}{2k^{+}}\Big(e^{-N\big(k^{+}+\frac{k_{\perp}^{2}}{2k^{+}}\big)/\sqrt{2}Q}-1\Big)\,.
\end{align}


%To treat the $k^{+}$ and $k^{-}$ integrals we will instead of using \cref{eq:k+k- relation in calculation}, use that $k^{+}k^{-}=(k^{2}+k_{\perp}^{2})/2$ which follows from \cref{App.eq:light-cone momenta squared}. Then by a change of variable using \cref{eq:minus ligh-cone momenta} and \cref{eq:energy relation lightcone momenta}, the exponent can be written as
%\begin{align}
%    \Me{E}^{\text{(eik)}}(N,Q,\epsilon)&=2\int\frac{d^{2-2\epsilon}k_{\perp}}{\Omega_{1-2\epsilon}}\int dk^{2}\int\frac{dk^{+}}{2k^{+}}\theta\Big(\frac{Q}{\sqrt{2}}-k^{+}\Big)\theta\Big(\frac{Q}{\sqrt{2}}-\frac{k_{\perp}^{2}+k^{2}}{2k^{+}}\Big)\nonumber
%    \\
%    &\hspace{1cm}W_{ij}\big(k^{2},k^{2}+k_{\perp}^{2},\mu,\alpha_{s},\epsilon\big)\Big(e^{-N\big(k^{+}+\frac{k_{\perp}^{2}+k^{2}}{2k^{+}}\big)/\sqrt{2}Q}-1\Big)\nonumber
%    \\
%    &=2\int\frac{d^{2-2\epsilon}k_{\perp}}{\Omega_{1-2\epsilon}}\int_{0}^{Q^{2}-k_{\perp}^{2}} dk^{2}\,W_{ij}\big(k^{2},k^{2}+k_{\perp}^{2},\mu,\alpha_{s},\epsilon\big)\nonumber
%    \\
%    &\hspace{0.5cm}\int_{(k_{\perp}^{2}+k^{2})/\sqrt{2}Q}^{Q/\sqrt{2}}\frac{dk^{+}}{2k^{+}}\Big(e^{-N\big(k^{+}+\frac{k_{\perp}^{2}+k^{2}}{2k^{+}}\big)/\sqrt{2}Q}-1\Big)\,,
%\end{align}
%where the theta functions has acted to give finite limits to the $k^{2}$ and $k^{+}$ integrals. If that step is not obvious, set the upper limit of $k^{+}=Q/\sqrt{2}$ inside the other step function and solve for $k^{2}$, giving $k^{2}=Q^{2}-k_{\perp}^{2}$. Similarly for the lower bound of $k^{+}$.

One of the $k^{+}$ integrals are straightforward, i.e.
\begin{align}
    \int_{k_{\perp}^{2}/\sqrt{2}Q}^{Q/\sqrt{2}}\frac{dk^{+}}{2k^{+}}=-\ln\Big(\sqrt{\frac{k_{\perp}^{2}}{Q^{2}}}\Big)\,,
\end{align}
while the other is more tricky, it can be shown that for large $N$ this behaves as a zeroth order modified bessel function of the second kind
\begin{align}
    K_{0}(z)=\int_{0}^{\infty}\frac{dt}{2t}e^{-t-\frac{z^{2}}{4t}}\,.
\end{align}

The actual rewriting is not pretty, but with a change of variable $t=Nk^{+}/\sqrt{2}Q$, this integral can in the large $N$ limit be represented as
\begin{align}
    \int_{k_{\perp}^{2}/\sqrt{2}Q}^{Q/\sqrt{2}}\frac{dk^{+}}{2k^{+}}\,e^{-N\big(k^{+}+\frac{k_{\perp}^{2}}{2k^{+}}\big)/\sqrt{2}Q}=K_{0}\Big(2N\sqrt{\frac{k_{\perp}^{2}}{Q^{2}}}\Big)\,,
\end{align}
up to terms of $\mathcal{O}(e^{-N})$.

After these considerations, we can write the exponent as
\begin{align}\label{eq:med eiko exponent}
   \Me{E}^{\text{(eik)}}(N,Q,\epsilon)&=4\int\frac{d^{2-2\epsilon}k_{\perp}}{\Omega_{1-2\epsilon}}\frac{\Gamma_{\text{cusp}}(\alpha_{s}(k_{\perp}))}{k_{\perp}^{2}}\,\Big[K_{0}\Big(2N\sqrt{\frac{k_{\perp}^{2}}{Q^{2}}}\Big)+\ln\Big(\sqrt{\frac{k_{\perp}^{2}}{Q^{2}}}\Big)\Big]\,.
\end{align}
We observe that the logarithm inside the bracket is divergent for $k_{\perp}\rightarrow 0$, but if we use the following expansion of the bessel function for $z$ small
\begin{align}
    K_{0}(z)=-\ln\big(\frac{ze^{\gamma_{E}}}{2}\big)-\frac{z^{4}}{4}\big[\ln\big(\frac{ze^{\gamma_{E}}}{2}\big)-1\big]+\mathcal{O}(z^{4})\,,
\end{align}
and only keep the first term, we see that the term inside the bracket in \cref{eq:med eiko exponent} is given by
\begin{align}
    K_{0}\Big(2N\sqrt{\frac{k_{\perp}^{2}}{Q^{2}}}\Big)+\ln\Big(\sqrt{\frac{k_{\perp}^{2}}{Q^{2}}}\Big)&\approx -\ln\bar{N}-\ln\Big(\sqrt{\frac{k_{\perp}^{2}}{Q^{2}}}\Big)+\ln\Big(\sqrt{\frac{k_{\perp}^{2}}{Q^{2}}}\Big)\nonumber
    \\
    &=-\ln\bar{N}\,,
\end{align}
i.e. the logarithm that diverges for $k_{\perp}\rightarrow 0$ cancels in the sum. There is still a collinear divergences, but we will soon see how to treat it.   





%To proceed from here there are some general considerations about the renormalization properties of webs that can be used, see \cite{Berger_2002,laenen2000power}. But we choose the less general path and just consider the one-loop calculation. We have partially performed the calculation of $W_{\text{DY}}^{(1)}$, so if we look at the structure of \cref{eq:W(Drell) to one-loop number two} and use the relation in \cref{eq:relation wDY and WDY}, the exponent can be written on the form
%\begin{align}
%    \Me{E}_{ij}^{\text{(eik)}}(N,Q,\epsilon)&=4\int\frac{d^{2-2\epsilon}k_{\perp}}{\Omega_{1-2\epsilon}}\frac{\Gamma_{\text{cusp}}^{(i)}(\alpha_{s}(k_{\perp}))}{k_{\perp}^{2}}\nonumber
%    \\
%    &\hspace{1cm}\int_{0}^{Q^{2}-k_{\perp}^{2}}dk^{2}\,\Big(K_{0}\Big(2N\sqrt{\frac{k_{\perp}^{2}+k^{2}}{Q^{2}}}\Big)+\ln\Big(\sqrt{\frac{k_{\perp}^{2}+k^{2}}{Q^{2}}}\Big)\Big)\,.
%\end{align}
%We can observe that for $k^{2}+k_{\perp}^{2}\rightarrow 0$ the logarithm diverges. However, the bessel function has the expansion for low $z$
%\begin{align}
%    K_{0}(z)=-\ln\big(\frac{ze^{\gamma_{E}}}{2}\big)-\frac{z^{4}}{4}\big[\ln\big(\frac{ze^{\gamma_{E}}}{2}\big)-1\big]+\mathcal{O}(z^{4})\,,
%\end{align}
%and by keeping only the first term in this expansion, the sum inside the bracket will in this limit take the form
%\begin{align}
%    K_{0}\Big(2N\sqrt{\frac{k_{\perp}^{2}+k^{2}}{Q^{2}}}\Big)+\ln\Big(\sqrt{\frac{k_{\perp}^{2}+k^{2}}{Q^{2}}}\Big)&\approx -\ln\bar{N}-\ln\Big(\sqrt{\frac{k_{\perp}^{2}+k^{2}}{Q^{2}}}\Big)+\ln\Big(\sqrt{\frac{k_{\perp}^{2}+k^{2}}{Q^{2}}}\Big)\nonumber
%    \\
%    &=-\ln\bar{N}\,,
%\end{align}
%and thus the divergence of the $k^{2}$ integral is effectively canceled in the sum. The rest of the integral over $k^{2}$ will give terms that are finite. These are not of leading logarithmic order, so we collect them in a function $B(Q,k_{\perp},\alpha_{s}(k_{\perp}))$. 

To further rewrite the expoenent, we set $\epsilon=0$ and use that the theta functions in \cref{eq:general result of eikonal cross section} restricts the $k_{\perp}$ integral to maximum value of $Q
^{2}$\footnote{This had to be true for the exponentiation conditions to be valid.}. By using polar coordinates $d^{2}k_{\perp}=k_{\perp}dk_{\perp}d\Omega_{1}$, we can write
\begin{align}
    \int\frac{d^{2}k_{\perp}}{\Omega_{1}}=\frac{1}{2}\int_{0}^{Q^{2}}dk_{\perp}^{2}\,,
\end{align}
and we arrive at the result
\begin{align}
    \Me{E}_{ij}^{\text{(eik)}}(N,\epsilon)=&2\int_{0}^{Q^{2}}\frac{dk_{\perp}^{2}}{k_{\perp}^{2}}\Gamma_{\text{cusp}}^{(i)}(\alpha_{s}(k_{\perp}))\Big[K_{0}\big(2N\frac{k_{\perp}}{Q}\big)+\ln(\frac{k_{\perp}}{Q})\big)\Big]\,,
\end{align}
which is only valid up to large logarithms. 

The eikonal cross section in \cref{eq:general result of eikonal cross section} can then be written as
\begin{align}
    \Me{\sigma}_{ij}^{\text{(eik)}}(N,Q,\epsilon)=\exp\Big(2\int_{0}^{Q^{2}}\frac{dk_{\perp}^{2}}{k_{\perp}^{2}}\Gamma_{\text{cusp}}^{(i)}(\alpha_{s}(k_{\perp}))\Big[K_{0}\big(2N\frac{k_{\perp}}{Q}\big)+\ln(\frac{k_{\perp}}{Q})\Big]\Big)\,.
\end{align}

As previously mentioned there is still a collinear divergence in this expression. However, we factorized $\Me{\sigma}_{ij}^{\text{(eik)}}(N,\epsilon)$ in such a way that $\Me{w}_{ij}^{\text{(eik)}}(N)$ was to be free of these divergences. Hence, by using \cref{eq:eikonal approcximation of partonic} we divide by the eikonal parton distributions \cref{eq:RGE parton in parton eikonal}
\begin{align}
    \Me{w}_{ij}^{\text{(eik)}}(N,Q,\mu,\alpha_s)=\frac{\Me{\sigma}_{ij}^{\text{(eik)}}(N,Q,\epsilon)}{\tilde{f}_{i/i}^{\text{(eik)}}(N,\mu,\epsilon)\tilde{f}_{j/j}^{\text{(eik)}}(N,\mu,\epsilon)}\,,
\end{align}
giving the exponent
\begin{align}
    \hat{\Me{E}}_{ij}^{\text{(eik)}}(N,Q,\mu)=&2\int_{0}^{Q^{2}}\frac{dk_{\perp}^{2}}{k_{\perp}^{2}}\Gamma_{\text{cusp}}^{(i)}(\alpha_{s}(k_{\perp}))\Big[K_{0}\big(2N\frac{k_{\perp}}{Q}\big)+\ln(\frac{k_{\perp}}{Q})\Big]\nonumber
    \\
    &+2\int_{0}^{\mu^{2}}\frac{d\mu'^{2}}{\mu'^{2}}\Gamma_{\text{cusp}}^{(i)}(\alpha_{s}(\mu'))\ln\bar{N}\,.
\end{align}

If we add $(\ln\bar{N}-\ln\bar{N})$ inside the bracket of the first line and choose $\mu'=k_{\perp}$, we can group these terms as
\begin{align}
    \hat{\Me{E}}_{ij}^{\text{(eik)}}(N,Q,\mu)=&2\int_{0}^{Q^{2}}\frac{dk_{\perp}^{2}}{k_{\perp}^{2}}\Gamma_{\text{cusp}}^{(i)}(\alpha_{s}(k_{\perp}))\Big[K_{0}\big(2N\frac{k_{\perp}}{Q}\big)+\ln(\bar{N}\frac{k_{\perp}}{Q})\Big]\nonumber
    \\
    &-2\int_{\mu^{2}}^{Q^{2}}\frac{dk_{\perp}^{2}}{k_{\perp}^{2}}\Gamma_{\text{cusp}}^{(i)}(\alpha_{s}(k_{\perp}))\ln\bar{N}\,,
\end{align}
and by choosing $\mu=Q$ can remove the last term. Hence, the eikonal function is given by
\begin{align}\label{eq:final result eikonal hard function}
    \Me{w}_{ij}^{\text{(eik)}}(N,Q,\mu,\alpha_s)=\exp\Big(2\int_{0}^{Q^{2}}\frac{dk_{\perp}^{2}}{k_{\perp}^{2}}\Gamma_{\text{cusp}}^{(i)}(\alpha_{s}(k_{\perp}))\Big[K_{0}\big(2N\frac{k_{\perp}}{Q}\big)+\ln(\bar{N}\frac{k_{\perp}}{Q})\Big]\Big)\,,
\end{align}
and we can see that when $k_{\perp}\rightarrow 0$ the integral is finite as the bracket perfectly cancels.

We should mention that in the general treatment, one should keep the distinction between the renormalization scale $\mu$, factroization scale $\mu_{F}$ and $Q$. But for simplicity we have chosen them all to be the same.  


\section{Logarithmic Corrections in Drell-Yan}
In this section we will use \cref{eq:final result eikonal hard function} to show how we can reproduce the large logarithm found in the NLO calculation \cref{eq:Mellin space NLO of omegaqbarq}, and also show that we find higher order logartithms without doing any higher order loop calculations. 

For the current discussion we are only interested in the large logarithmic corrections, so we neglect $H(Q,\alpha_s)$ in \cref{eq:partonic and eikonal relation}\footnote{From factorization theorems this does not include logarithmic corrections.}. Then from \cref{eq:final result eikonal hard function} it follows that the partonic function $\Me{w}_{q\bar{q}}$ has exponentiated, i.e
\begin{align}
    \Me{w}_{q\bar{q}}(N,Q,\alpha_{s}(Q))=\exp\Big(2\int_{0}^{Q^{2}}\frac{dk_{\perp}^{2}}{k_{\perp}^{2}}\Gamma_{\text{cusp}}^{(q)}(\alpha_{s}(k_{\perp}))\Big[K_{0}\big(2N\frac{k_{\perp}}{Q}\big)+\ln(\bar{N}\frac{k_{\perp}}{Q})\Big]\Big)\,.
\end{align}
The lower limit does not give a well defined result, but to produce large logarithms it is standard to evaluate these from $Q^{2}/\bar{N}$ up to $Q^{2}$ \cite{KORCHEMSKY1993433}\footnote{They actually solve a renormalization group equation for the Wilson line, where the integral is evaluated from $Q^{2}/\bar{N}$ up to $Q^{2}$.}. We can justify this by the approximation $Q^{2}/N\approx 0$ as $N\rightarrow\infty$. With this change of lower limit, we can make the change of variable $x=k_{\perp}/Q$, giving the exponent
\begin{align}
    \Me{E}_{q\bar{q}}(N,Q,\alpha_s)=4\int_{1/\bar{N}}^{1}\frac{dx}{x}\Gamma_{\text{cusp}}^{(q)}(\alpha_{s}(Qx))\Big[K_{0}\big(2Nx\big)+\ln(\bar{N}x)\Big]\,,
\end{align}
which can be simplified even further by looking at the large $N$ behaviour of the Bessel function. For large values of the argument, the modified Bessel function has the following expansion
\begin{align}
    K_{0}(z)=\big(\frac{\pi}{z}\Big)^{1/2}e^{-z}\big(1-\mathcal{O}(z^{-1})\big)\approx0\,,
\end{align}
and the exponent can be simplifies to
\begin{align}\label{eq:integral for drell-yan logarithms}
    \Me{E}_{q\bar{q}}(N,Q,\alpha_s)=4\int_{1/\bar{N}}^{1}\frac{dx}{x}\Gamma_{\text{cusp}}^{(q)}(\alpha_{s}(Qx))\ln(\bar{N}x)\,,
\end{align}
valid up to constant terms that are negligible in the large $N$ limit.

This integral can now be solved by using the one-loop cusp anomalous dimension \cref{eq:one-loop cusp anomalous dimension}, and the one-loop running coupling \cref{eq: g running coupling one-loop}
\begin{align}
    \Gamma_{\text{cusp}}(\alpha_{s}(Qx))&=\frac{\alpha_{s}(Qx)}{\pi}C_{F}\,,
    \\
    \alpha_{s}(Qx)&=\frac{\alpha_{s}(Q)}{1+\frac{\alpha_{s}(Q)}{2\pi}\beta_{0}\ln x}\,.
\end{align}
Inserting these expression into \cref{eq:integral for drell-yan logarithms}, and with another change of variable $y=\ln x$ gives the LL (leading logarithmic) result
\begin{align}\label{eq:LL result}
    \Me{E}_{q\bar{q}}^{(\text{LL})}(N,Q,\alpha_s)&=4\int_{1/\bar{N}}^{1}\frac{dx}{x}\frac{\alpha_{s}(Q)}{\pi}C_{F}\Big(1+\frac{\alpha_{s}(Q)}{2\pi}\beta_{0}\ln x\Big)^{-1}\ln\bar{Nx}\nonumber
    \\
    &=A_{q\bar{q}}\big[2\bar{\lambda}+(1-2\bar{\lambda})\ln(1-2\bar{\lambda}))\big]\,,
\end{align}
giving
\begin{align}\label{eq: omega LL result}
    \Me{w}_{q\bar{q}}^{(\text{LL})}(N,Q,\alpha_{s}(Q))=\exp\Big(A_{q\bar{q}}\big[2\bar{\lambda}+(1-2\bar{\lambda})\ln(1-2\bar{\lambda})\big]\Big)\,,
\end{align}
where we have defined $A_{q\bar{q}}=C_{F}/\alpha_{s}\pi b_{0}^{2}$ and $\bar{\lambda}=\alpha_s b_{0}\ln\bar{N}$, where $b_{0}=\beta_{0}/4\pi$. This result is in agreement with the LL correction for singlet ($q\bar{q}$) annihilation in Drell-Yan \cite{Catani:1996,Catani:1998}. %In order to obtain the NLL on this form we would have to use the two-loop coupling, but we will not consider that case here. 

At first sight \cref{eq: omega LL result} does not have the same form as \cref{eq:Mellin space NLO of omegaqbarq}, but if we expand the logarithm
\begin{align}
    \ln(1-2\bar{\lambda})=-2\bar{\lambda}-\bar{\lambda}^{2}-\frac{2}{3}\bar{\lambda}^{3}-\frac{1}{2}\bar{\lambda}^{4}+\cdots\,,
\end{align}
we find the LL terms
\begin{align}
    \Me{E}_{q\bar{q}}^{(\text{LL})}(N,Q,\alpha_s)=\frac{\alpha_{s}}{\pi}2C_{F}\ln^{2}\bar{N}+\Big(\frac{\alpha_{s}}{\pi}\Big)^{2}\frac{\beta_{0}}{3}C_{F}\ln^{3}\bar{N}+\Big(\frac{\alpha_{s}}{\pi}\Big)^{3}\frac{\beta_{0}^{2}}{32}\ln^{4}\bar{N}+\mathcal{O}(\alpha_{s}^{4})\,,
\end{align}
where the first term are the LL for the corresponding NLO calculation we found in \cref{eq:Mellin space NLO of omegaqbarq}\footnote{With the very important distinction that it has been exponentiated.}, and the other terms are the LL for even higher order calculations. If we wanted to compare with fixed order calculations to higher orders, we could have expanded the exponential in \cref{eq: omega LL result} and found an expanded form of $\Me{w}_{q\bar{q}}(N)$. This expansion would give fixed NLL order terms as well, but these can be found in \cite{MAGNEA1991703}. A last point to make is that in order to obtain resummed NLL terms, we would have to use the coupling to two-loop order and the cusp anomalous dimension up to two loop order, but we did not consider this scenario here.


\section{Hadronic Cross Section and Inverse Mellin}
We have managed to bring the partonic function on an exponentiated form, but the correct observable is the hadronic cross section. So in this section we will recover the hadronic Drell-Yan cross section in $x$-space, and discuss how the inverse Mellin transform can be evaluated. 

From \cref{eq:hadronic cross section in Mellin} and \cref{eq:LL result} the Mellin transformed hadronic cross section is given by
\begin{align}
    \tilde{\sigma}_{h_1h_2}(N)&=\int_{0}^{1}d\tau \tau^{N-1}\,\frac{1}{\sigma_{0}}\frac{d\sigma_{h_1h_2}}{dQ^{2}}\nonumber
    \\
    &=\sum_{i,j=q,\bar{q}}\tilde{f}_{i/h_1}(N,Q)\tilde{f}_{j/h_2}(N,Q)\,\exp\big(\Me{E}_{ij}(N,Q,\alpha_s)\big)\,,
\end{align}
where the partonic function has been exponentiated. We can now use the inverse Mellin transform \cref{eq:Appendix Inverse Mellin} to write\footnote{Where we have recovered the fractional charge of the quarks.}
\begin{align}
    \frac{d\sigma_{h_1h_2}}{dQ^{2}}=&\,\sigma_{0}\sum_{i,j=q,\bar{q}}Q_{q}^{2}\,
    \frac{1}{2\pi i}\int_{c-i\infty}^{c+i\infty}dN\,\tau^{-N}\tilde{f}_{i/h_1}(N,Q)\tilde{f}_{j/h_2}(N,Q)\exp\big(\Me{E}_{ij}(N,Q,\alpha_s)\big)\,.
\end{align}

There exists numerical packages to evaluate parton distributions in Mellin space, see \cite{Vogt_2005}. However, the standard method is to use parton distributions sets from the LHAPDF code existing is $x$-space \cite{whalley2005les,Buckley_2015}.  To use the $x$-space formalism we can do several manipulations by using the convolution properties of the Mellin transform, given in \cref{sec:Appendix Mellin Transform}. But let us first derive an expression for the inverse Mellin transform in terms of an integral over a real variable.

\subsection{The Inverse Mellin Transform}
The inverse Mellin transform for a general function is given in terms of an integral over a complex variable $N$, see \cref{sec:Appendix Mellin Transform}, and reads
\begin{align}\label{eq:h}
    h(x)=\frac{1}{2\pi i}\int_{c-i\infty}^{c+i\infty}dN\,x^{-N}\,\Me{h}(N)\,,
\end{align}
where 
\begin{align*}
    \Me{h}(N)=\int_{0}^{\infty}dx\,x^{N-1}\,h(x)\,.
\end{align*}
Since $h(x)$ is a real valued function
\begin{align}\label{eq:h_real}
    \Me{h}^{*}(N)=\int_{0}^{\infty}dx\,x^{N^{*}-1}\,h(x)=\Me{h}(N^{*})\,.
\end{align}

The Mellin inversion integral \cref{eq:h} can be splitt in two, one part for the lower bound and one part for the upper bound. To manipulate the integral, we make a change of variable $N\rightarrow N^{*}$ in the term that is integrated from $c-i\infty$ up to $c$. This makes the integration bound change accordingly $c-i\infty\rightarrow c+i\infty$, and we find
\begin{align}
    h(x)&=\frac{1}{2\pi i}\Bigl(\int_{c-i\infty}^{c}dN\,x^{-N}\,\Me{h}(N)+\int_{c}^{c+i\infty}dN\,x^{-N}\,\Me{h}(N)\Bigr)\nonumber
    \\
    &=\frac{1}{2\pi i}\Bigl(\int_{c+i\infty}^{c}dN^{*}\,x^{-N^{*}}\,\Me{h}(N^{*})+\int_{c}^{c+i\infty}dN\,x^{-N}\,\Me{h}(N)\Bigr)\nonumber
    \\
    &=\frac{1}{2\pi i}\Bigl(-\int_{c}^{c+i\infty}dN^{*}\,x^{-N^{*}}\,\Me{h}^{*}(N)+\int_{c}^{c+i\infty}dN\,x^{-N}\,\Me{h}(N)\Bigr)\,,
    \intertext{and by choosing the parametrization of the Mellin variable to be $N=c+ze^{i\phi}$, in terms of real $z$, the integral will take the form}
    h(x)&=\frac{1}{2\pi i}\int_{0}^{\infty}dz\,\left(e^{i\phi}\,x^{-N}\Me{h}(N)-e^{-i\phi}\,x^{-N^{*}}\Me{h}^{*}(N)\right)\nonumber
    \\
    &=\frac{1}{2\pi i}\int_{0}^{\infty}dz\,2i\,\text{Im}\left(e^{i\phi}\,x^{-N}\Me{h}(N)\right)\nonumber
    \\
    &=\frac{1}{\pi}\int_{0}^{\infty}dz\,\text{Im}\left(e^{i\phi}\,x^{-N}\Me{h}(N)\right)\,,
\end{align}
where we used \cref{eq:h_real} and the relation
\begin{align*}
    \Me{h}(N)-\Me{h}^{*}(N)&=2i\,\text{Im}(\Me{h}(N))\,.
\end{align*}

Writing out the parametrization, the final integral is given by
\begin{align}\label{eq:Inversed Mellin Integral}
    h(x)=\frac{1}{\pi}\int_{0}^{\infty}dz\,\text{Im}\left(e^{i\phi}\,x^{-c-z\,\exp(i\phi)}\Me{h}(c+z\,\exp(i\phi))\right)\,.
\end{align}


\subsection{Hadronic Cross Section in $x$-space}
In order to rewrite the inverse Mellin in $x$-space it is advantageous to use that \cref{eq:Factorized Dre-Yan Cross Section} is understood to be a Mellin convolution. Let us begin with \cref{eq:Factorized Dre-Yan Cross Section}, and use \cref{eq:Appendix Mellin convolution} to write it as\footnote{For simplicity we have abbreviated some of the arguments in the integrand.}
\begin{align}
    \frac{ d\sigma_{h_1h_2}}{dQ^{2}}=\sigma_{0}\sum_{i,j=q,\bar{q}}\int_{0}^{1}dz\,dx_{1}\,dx_{2}\,\delta\big(\tau-x_{1}x_{2}z\big) f_{i/h_1}(x_1)f_{j/h_2}(x_2)\,\omega_{ij}(z)\,,
\end{align}
and using the delta function property
\begin{align}\label{eq:delta_property}
    \int dx\,\delta(g(x))=\abs{\frac{\partial g}{\partial x}}_{x=x^{*}}^{-1}\int dx\,\delta(x-x^{*})\,,
\end{align}
we find that
\begin{align}\label{eq:rewritten convolution to inverse}
    \frac{ d\sigma_{h_1h_2}}{dQ^{2}}=\sigma_{0}\sum_{i,j=q,\bar{q}}\int_{\tau}^{1}\frac{dz}{z}\int_{\tau/z}^{1}\frac{dx_1}{x_1}f_{i/h_1}(x_1)f_{j/h_2}\Big(\frac{\tau}{x_1z}\Big)\,\omega_{ij}(z)\,,
\end{align}
where the limits have changed after acting with the delta function, i.e. we have
\begin{align}
    x_{2}=\frac{\tau}{x_1z}\leq 1\,,
    \hspace{0.7cm}
    \frac{\tau}{z}\leq x_1\leq 1\,,
    \hspace{0.7cm}
    \tau\leq z\leq 1\,.
\end{align}

%Then by Parseval's convolution theorem for the inverse Mellin transform, we have that
%\begin{align}
%    \int_{\tau}^{1}\frac{dz}{z}\int_{\tau/z}^{1}\frac{dx_1}{x_1}f_{i/h_1}(x_1)f_{j/h_2}\Big(\frac{\tau}{x_1z}\Big)\,\omega_{ij}(z)=\frac{1}{2\pi i}\int_{c-i\infty}^{c+i\infty}dN\,\tau^{-N}\tilde{f}_{i/h_1}(N)\tilde{f}_{j/h_2}(N)\Me{\omega}_{ij}(N)\,,
%\end{align}
%where it follows that we can evaluate the left hand side with $x$-space parton distributions, while $\omega(z)$ is evaluated as an inverse Mellin.

To actually perform the inverse transform is a nontrivial exercise, so let us go through some of the general details. The difficulty originates from the singularity structure of the integrand. These singularities can be divided into two regions in the complex $N$-plane, see \cref{fig:MP contour}. The first ones are positioned in the left part of the complex plane. These are the result of poles in the parton distribution functions $\Me{f}_{i/h}$. In general, the functional form of parton distributions functions at the initial scale $\mu_{0}$ is given by
\begin{align}
    xf_{i/h}(x,\mu_{0})=\sum_{n}A_{n}x^{\gamma_n}(1-x)^{\delta_n}\,,
\end{align}
where $A_{n}$, $\gamma_n$ and $\delta_{n}$ are obtained from a fitting procedure to hard scattering data, and for small $x$ and $\gamma_n<0$ this becomes singular. In Mellin space, this is transformed to
\begin{align}
    \Me{f}_{i/h}(N,\mu_{0})=\sum_{n}A_{n}\beta(N+\gamma_n,1+\delta_n)\,,
\end{align}
where $\beta(a,b)$ is the Euler-beta function in which the singular behaviour is well known. This first region can be avoided by choosing the contour in \cref{fig:MP contour}, as derived in \cite{Catani:1996}. We observe that the singularities from the parton distributions are avoided by choosing a constant $C_{MP}$ that lies to the right of the rightmost singularity of $\Me{f}_{i/h}$. 

The second region is more troublesome, as it is in the right part of the complex $N$-plane. This singularity originates from the Landau pole that arises for small couplings, i.e. the non-perturbative regime of QCD. The Landau pole manifest itself in the exponent of \cref{eq:LL result}, and is due to the expression $\ln(1-2\bar{\lambda})$. Hence, for $\bar{\lambda}=1/2$ we have
\begin{align}
    N_{L}=e^{-\gamma_{E}}e^{\frac{1}{2b_{0}\alpha_s}}\,.
\end{align}

This problem has been extensively studied in the litterature, and the most common approach is the so-called \emph{Minimal Prescription} \cite{Catani:1996}. This method states that the contour has to be chosen such that the constant $C_{MP}$, satisfies
\begin{align}
    C_{f}<C_{MP}<N_{L}\,,
\end{align}
where $C_{f}$ is the rightmost pole of $\Me{f}_{i/h}$. In \cref{fig:MP contour} we see the possible contours, where $C_{0}$ is the vertical line and $C_{1}$ is the bent contour. As argued in \cite{Catani:1996}, the choice $C_{MP}=2$ and $C_1$ with $\phi >\pi/2$, will make the integral converge faster than the choice $C_{MP}=2$ and $C_{0}$ with $\phi=\pi/2$.
\begin{figure}
    \centering
    \includegraphics[scale=0.4]{Figures/CMP.pdf}
    \caption{Possibilities to choose the contour for the Mellin inversion as proposed in \cite{Catani:1996}. $C_{0}$ is the vertical contour and $C_{1}$ is the contour with an angle.}
    \label{fig:MP contour}
\end{figure}

So if the hadronic cross section is to be calculated with parton distributions in Mellin space, the following integral must be implemented
\begin{align}
    \frac{d\sigma_{h_1h_2}}{dQ^{2}}=&\,\sigma_{0}\sum_{i,j=q,\bar{q}}Q_{q}^{2}\,\frac{1}{\pi}\int_{0}^{\infty}dy\,\text{Im}\Big[e^{i\phi}\tau^{-N}\tilde{f}_{i/h_1}(N,Q)\tilde{f}_{j/h_2}(N,Q)\,\Me{\omega}_{ij}(N,\alpha_{s}(Q))\Big]\,,
\end{align}
where we used \cref{eq:Inversed Mellin Integral} to write the cross section as an integral over a real variable and $N=C_{MP}+ye^{i\phi}$. 

However, to use the $x$-space formalism the hadronic cross section is obtained by calculating
\begin{align}\label{eq:next to final hadronic}
    \frac{ d\sigma_{h_1h_2}}{dQ^{2}}=\sigma_{0}\sum_{i,j=q,\bar{q}}Q_{q}^{2}\int_{\tau}^{\infty}\frac{dz}{z}\int_{\tau/z}^{1}\frac{dx_1}{x_1}f_{i/h_1}(x_1,Q)f_{j/h_2}\Big(\frac{\tau}{x_1z},Q\Big)\,\omega_{ij}(z,Q,\alpha_{s}(Q))\,,
\end{align}
where the parton distributions are calculated in $x$-space, while the hard function is found by the inverse transform
\begin{align}
    \omega_{ij}(z,Q,\alpha_{s}(Q))=\frac{1}{\pi}\int_{0}^{\infty}dy\,\text{Im}\big(e^{i\phi}z^{-N}\,\Me{\omega}_{ij}(N,\alpha_{s}(Q))\big)\,.
\end{align}
We observe that the upper limit of the $z$ integral in \cref{eq:next to final hadronic} has changed. The reason for this change is that in the minimal prescription, the resummed cross section does not vanish for $z>1$ due to the Landau pole \cite{Catani:1996}. 

In \cref{eq:next to final hadronic}, the parton distributions are calculated in $x$-space, but there are problems that might occur in the hard function $\omega_{ij}$. Close to threshold, the resummed exponent can give large oscillations, see \cite{Catani:1996,KULESZA:2002} for more details. One possibility to dampen the oscillations is by a simple rewriting of the integrand. We do this by using \cref{eq:Appendix derivative of function}, to write
\begin{align}
    N\Me{f}_{i/h}(N)=\int_{0}^{1}dx\,x^{N-1}\mathcal{F}_{i/h}(x)\,,
\end{align}
where we used that parton distributions vanish for $x=1$, and defined
\begin{align}
    \mathcal{F}_{i/h}(x)=-x\frac{d}{dx}f_{i/h}(x)\,,
\end{align}
giving the modified hadronic cross section
\begin{align}
    \frac{ d\sigma_{h_1h_2}}{dQ^{2}}=\sigma_{0}\sum_{i,j=q,\bar{q}}Q_{q}^{2}\int_{\tau}^{1}\frac{dz}{z}\int_{\tau/z}^{1}\frac{dx_1}{x_1}\mathcal{F}_{i/h_1}(x_1,Q)\mathcal{F}_{j/h_2}\Big(\frac{\tau}{x_1z},Q\Big)\,\mathcal{S}_{ij}(z,Q,\alpha_{s}(Q))\,,
\end{align}
where
\begin{align}
    \mathcal{S}_{ij}(z,Q,\alpha_{s}(Q))=\frac{1}{2\pi i}\int_{C_{MP}}dN\,z^{-N}\frac{\Me{w}_{ij}(N,Q,\alpha_{s}(Q))}{N^{2}}\,.
\end{align}
The derivatives of the parton distributions can be performed numerically for the common sets \cite{Martin:2009,Pumplin:2002}. As mentioned the point of this rewriting is to dampen the behaviour of the exponent in $\omega_{q\bar{q}}$, but as argued in \cite{Catani:1996} gluon initiated processes should have even higher powers of $N$ as dampening factors. This would subsequently lead to higher order derivatives of the parton distributions.


 




%%%%%%%%%%%%% Conclusion %%%%%%%%%%%%%%%%%%%%%%%%%%
\newpage
\chapter*{Conclusion $\&$ Outlook}
\addcontentsline{toc}{chapter}{Conclusion $\&$ Outlook} 
In this thesis I have applied a range of \ac{ML} models to the search for chargino-neutralino production resulting in 
a three lepton final state with missing transverse momentum. Two data sets were utilized during the analysis; simulated \ac{MC} data, including 
both the \ac{SM} background and the \ac{BSM} signal, and  measured proton-proton collisions at $\sqrt{s} = 13TeV$ produced at the \ac{LHC} and detected 
at \ac{ATLAS}. The models applied and studied during my analysis were a set of \ac{NN} variants, in addition to a \ac{BDT} which was used to 
create a benchmark for my analysis. The network variants encompassed a diverse array of approaches including an ordinary dense \ac{NN}, ensemble networks employing 
\ac{LWTA} layers and a \ac{PNN}.
\\
The simulated signal set included and studied in the analysis consisted of a set of orthogonal \ac{BSM} variants, specifically different masses for the chargino ($\tilde{\chi}^\pm_1$) 
and neutralino ($\tilde{\chi}_1$). In my analysis I tested two approaches for dealing with a diverse signal set; training one model per variant, and training one model on a larger set 
of variants. Comparing the two, I found the latter to achieve a higher sensitivity as a consequence of a couple of factors. By including an assortment of variations of new physics, 
the \ac{ML} models were able to avoid overfitting longer, which allowed for deeper learning. Furthermore, I found that the models were able to exploit overlapping feature trends between 
the variations, which resulted in more training data for all mass combinations.  
\\
I studied three variants of \ac{LWTA} layers; channel-out, maxout, and \ac{SCO}. The first two were taken from the paper by Wang et al. \cite{wang_maxout_2013}, and the third layer was introduced
in this present thesis. Each layer, reduces the number of nodes during a forward propagation, similarly to the dropout layer, but does so by comparing activation with other nodes in the layer, and 
dropping all but the largest node. To study the implementation and effect of these layers, I constructed a set of figures to visualize the activation and dropping of nodes, before and after training. 
When dissecting the figures, I observed that the \ac{LWTA} layers (specifically the maxout layer) were able to build an ensemble of networks by means of trend specific paths. In other words, after training 
the model, the data chose different paths through the network dependent on if it was background or signal. Moreover, by comparing two sets of signal with different mass combinations, I found that the model 
was also able to differentiate between different variations of signal. This indicates a strong long-term memory, which allows the model to target a larger set of signal and is an important attribute
in the \ac{LWTA} layers.
\\
The \ac{PNN} applied in the thesis was inspired by the network introduced in the paper by Baldi et al. \cite{PNN}. The purpose of the \ac{PNN} was to motivate the model to differentiate between the mass combination 
in the signal set, through including the parameters as a feature. To study the effect of the \ac{PNN}, I drew the distribution of a subset of mass combinations, where all the events were given the same parameters.
The idea was that the events which were given the correct parameters would outperform those which were not. By repeating this test, but assigning the data another set of parameters I found that \ac{PNN} did in fact 
perform best when predicting on data which were assigned the correctly, although not by much. For example, compared to when the events were given the wrong parameters, signal events with masses equal to $\tilde{\chi}_1=50$ 
and $\tilde{\chi}_2=250$GeV, improved effectiveness by $3\%$ when applying a rigid cut on the output of 0.975. This indicates that the \ac{PNN} was able to discriminate between different variations of signal, even if only by a small degree.
\\
When studying and comparing the performance of the \ac{ML} models, two sets of the signal were used, original ($\tilde{\chi}_1\in[0-400]GeV$ and  $\tilde{\chi}_2\in[400-800]GeV$) and full statistics ($\tilde{\chi}_1\in[0-400]GeV$ 
and  $\tilde{\chi}_2\in[200-800]GeV$), where the first was a subset of the second. When comparing the achieved sensitivity, or significance of the models on each mass combination in the original signal set, I found that the 
maxout model outperformed all other \ac{LWTA} models, achieving a higher significance in (24/30) combinations. Although, the introduced \ac{SCO} layer did not outperform maxout model, it did achieve a higher significance in 6 
combinations. Most likely, by modifying the \ac{SCO} layer during prediction, the performance of the layer would improve.
\\
Four models were chosen when comparing performance on the original signal set; ordinary dense \ac{NN}, maxout model, \ac{PNN} and a \ac{BDT} implemented using the default settings of XGBoost \cite{XGB}. In the comparison I found that 
all three network variants were able to outperform the \ac{BDT}, although I believe this to be the cause of little attention towards the architecture of the \ac{BDT}. Out of the three network variants, the maxout model was able to achieve 
the highest significance for most mass combinations (24/30), but mostly in the higher mass range ($\tilde{\chi}_2>600GeV$). In the events with lower masses and higher statistics ($\tilde{\chi}_2<600GeV$), the \ac{PNN} outperformed all others, 
almost doubling significance achieved by the maxout model. Shared among all models was the fact that they all found processes with high amounts of missing energy difficult to separate from the signal, i.e. $Diboson(lll)$, $t\bar{t}$ and $top other$.
\\
Before applying the models to the full statics of the signal data, I applied a \ac{PCA} to study if it could improve performance. When requiring the conservation of $99.99\%$ of the variation from the original feature set, 5 features were removed.
Training the dense \ac{NN}, maxout model and \ac{PNN} with this new data set, I found it to improve the sensitivity of the two latter models. The comparison was based on the choice to weight the importance of all mass combinations equally.
\\
Finally, in my analysis I compared the performance of my three best performing models (maxout model and \ac{PNN} with a \ac{PCA}, and the dense \ac{NN}) on the full statistics signal set. Additionally, I compared the results to the expected exclusion 
limits ($Z>1.64$) set by \ac{ATLAS} in 2021 \cite{atlas_search_2021}. The calculated significance for the models were done using a flat uncertainty of $20\%$, $10\%$ and $<1\%$. Based on the comparison I found that none of the \ac{ML} models were 
able to extend the limits set by \ac{ATLAS}, with the exception of the \ac{PNN} when utilizing $<1\%$ uncertainty. For an uncertainty of $20\%$, the \ac{PNN} was able to achieve a limit which mirrors \ac{ATLAS} for smaller masses ($\tilde{\chi}_2<250GeV$)
and set a limit past that achieved by the other networks, for both the chargino and neutralino mass. When decreasing the uncertainty, the maxout model and dense \ac{NN} were able to extend the limit past that achieved by the \ac{PNN} for higher masses, 
but never surpassing the limit by \ac{ATLAS}. From my analysis I found that where the \ac{PNN} exhibits bias towards higher statistic signal, the ordinary dense \ac{NN} and maxout model are able to achieve a more balanced sensitivity. Especially the 
maxout model, with an impressive long-term memory is able to uphold a strong performance for lower statistics signal and smaller differences in significance ($\Delta Z \approx 10$). Due to the fact that future analysis will need sensitivity in high mass regions, 
I believe the \ac{LWTA} layers to be interesting candidates for future models, in regard to their ability to exploit high statistics combinations while maintaining a strong performance on lower statistics signal. 


\backmatter{}
%%%%%%%%%%%%% Appendix %%%%%%%%%%%%%%%%%%%%%%%%

\begin{appendices}
\numberwithin{equation}{section}

\appendix
\chapter{Appendix A}
\renewcommand{\thechapter}{A}
\renewcommand{\theequation}{\thechapter.\arabic{equation}}
\section{Expected Significance Results}\label{sec: Sensitivity}
\subsection{The Ensemble Models Applied to the Original Signal Set}\label{appendix:Ensembles}
\begin{figure}[H]
    \makebox[\linewidth][c]{%
    \centering
    \begin{subfigure}{.5\textwidth}
        \includegraphics[width=\textwidth]{Figures/MLResults/NN/SUSY/Grid/StochChannelOutGridSig.pdf}
        \vspace{-1cm}
        \caption{}
        \label{fig:StochChannelOutGridSig}
    \end{subfigure}
    \begin{subfigure}{.5\textwidth}
        \includegraphics[width=\textwidth]{Figures/MLResults/NN/SUSY/Grid/ChannelOutGridSig.pdf}
        \vspace{-1cm}
        \caption{}
        \label{fig:ChannelOutGridSig}
    \end{subfigure}
    }
    \caption{A grid displaying the expected significance on the original signal set using the signal region 
    created by the \acs{SCO} \ref{fig:StochChannelOutGridSig} and a channel-out network \ref{fig:ChannelOutGridSig}.}
    \label{fig:SCOCO}
\end{figure}

\subsection{Results from the PCA}\label{appendix:PCA}
\begin{figure}[H]
    \makebox[\linewidth][c]{%
    \centering
    \begin{subfigure}{.5\textwidth}
        \includegraphics[width=\textwidth]{Figures/MLResults/NN/SUSY/Grid/NNPCAGridSig.pdf}
        \caption{}
        \label{fig:NNPCAGridSig}
    \end{subfigure}
    \begin{subfigure}{.5\textwidth}
        \includegraphics[width=\textwidth]{Figures/MLResults/NN/SUSY/Grid/MaxOutPCAGridSig.pdf}
        \caption{}
        \label{fig:MaxOutPCAGridSig}
    \end{subfigure}
    }
    \caption{A grid displaying the expected significance on the original signal set using the signal region 
    created by the ordinary dense \acs{NN} \ref{fig:NNPCAGridSig} and a maxout network \ref{fig:MaxOutPCAGridSig}. A \ac{PCA} 
    analysis has been applied to the data being utilized in this result.}
\end{figure}

\begin{figure}[H]
    \makebox[\linewidth][c]{%
    \centering
    \begin{subfigure}{.65\textwidth}
        \includegraphics[width=\textwidth]{Figures/MLResults/NN/SUSY/Grid/PNNPCAGridSig.pdf}
    \end{subfigure}
    }
    \caption{A grid displaying the expected significance on the original signal set using the signal region 
    created by the \acs{PNN} network. A \acs{PCA} analysis has been applied to the data being utilized in this result.}
    \label{fig:PNNPCAGridSig}
\end{figure}

\begin{figure}[H]
    \makebox[\linewidth][c]{%
    \centering
    \begin{subfigure}{.65\textwidth}
        \includegraphics[width=\textwidth]{Figures/MLResults/NN/SUSY/Comparison/NNPCANetworkComp.pdf}
    \end{subfigure}
    }
    \caption['Pie-plot' comparing sensitivity on the original signal set, where the figure shows the comparison between a model training on data 
    with and without a \acs{PCA}.]{'Pie-plot' comparing sensitivity on the original signal set, where the figure shows the comparison between a model trained on data 
    with and without a \acs{PCA}. The size of each 'slice' represents the relative size of the significance and the color around each 
    point displays the method with the largest sensitivity for the respective combination.}
    \label{fig:NNPCAComp}
\end{figure}

\subsection{Comparing Models Trained on Original and the Complete Signal Grid}\label{appendix:BigVsSmall}
\begin{figure}[H]
    \makebox[\linewidth][c]{%
    \centering
    \begin{subfigure}{.65\textwidth}
        \includegraphics[width=\textwidth]{Figures/MLResults/NN/SUSY/Comparison/BigVsLittleSetMaxOutNetworkComp.pdf}
    \end{subfigure}
    }
    \caption['Pie-plot' comparing sensitivity achieved by the maxout model on the original signal set, where the figure shows the comparison between a model trained 
    on the original signal grid, and the complete signal grid.]{'Pie-plot' comparing sensitivity achieved by the maxout model on the original signal set, where the figure 
    shows the comparison between a model trained on the original signal grid, and the complete signal grid. A \ac{PCA} analysis has been applied to the data being utilized 
    in this result. The size of each 'slice' represents the relative size of the significance and the color around each 
    point displays the method with the largest sensitivity for the respective combination.}
    \label{fig:BigVsLittleSetMaxOut}
\end{figure}
\begin{figure}[H]
    \makebox[\linewidth][c]{%
    \centering
    \begin{subfigure}{.65\textwidth}
        \includegraphics[width=\textwidth]{Figures/MLResults/NN/SUSY/Comparison/BigVsLittleSetPNNNetworkComp.pdf}
    \end{subfigure}
    }
    \caption['Pie-plot'comparing sensitivity achieved by the \acs{PNN} model on the original signal set, where the figure shows the comparison between a model trained 
    on the original signal grid, and the complete signal grid.]{'Pie-plot' comparing sensitivity achieved by the \acs{PNN} model on the original signal set, where the figure 
    shows the comparison between a model trained on the original signal grid, and the complete signal grid. A \ac{PCA} analysis has been applied to the data being utilized 
    in this result. The size of each 'slice' represents the relative size of the significance and the color around each 
    point displays the method with the largest sensitivity for the respective combination.}
    \label{fig:BigVsLittleSetPNN}
\end{figure}
\newpage


\appendix
\renewcommand{\thechapter}{B}
\chapter{Appendix B}
%\renewcommand{\theequation}{\thechapter.\arabic{equation}}
\section{The Features}
\subsection{Jet Requirements}\label{subsec:sigJets}
\begin{figure}[H]
    \renewcommand\figurename{Table}
    \centering
    \footnotesize
    \makebox[\linewidth][c]{
        \begin{subfigure}{.3\textwidth}
            $
            \begin{array}{cc}
                \hline \text { Requirement } & \text { Baseline Jets }  \\
                \hline \hline p_T-\text{cut} & p_T > 20GeV  \\
                \eta-\text{cut} & |\eta| > 2.8  \\
                \hline
            \end{array}
            $
            \caption{}
            \label{table:BLJets}
        \end{subfigure}
        \begin{subfigure}{.3\textwidth}
            $
            \begin{array}{cc}
                \hline \text { Requirement } & \text { Signal Jets }  \\
                \hline \hline p_T-\text{cut} & p_T > 60  \\
                \ \ \ or &\\
                JVT-\text{cut} & |JVT| < 0.91 \\
                \hline
            \end{array}
            $
            \caption{}
            \label{table:SGJets}
        \end{subfigure}
        
    }
    \caption[Two tables displaying the baseline \ref{table:BLJets} and signal \ref{table:SGJets} requirements of the jets applied 
    to the data as part of the preprocessing.]{Two tables displaying the baseline \ref{table:BLJets} and signal \ref{table:SGJets} 
    requirements of the jets applied to the data as part of the preprocessing. Note that signal jets are required to pass both baseline 
    and signal requirements. For a formal definition of \acf{JVT}, see \cite{the_atlas_collaboration_tagging_2014}.}
    \label{table:JetCuts}
\end{figure}
\newpage
\subsection{The Feature Distribution}\label{subsec:Dist}
\begin{figure}[H]
    \makebox[0.95\linewidth][c]{%
    \centering
    \begin{subfigure}{.405\textwidth}
        \includegraphics[width=\textwidth]{Figures/FeaturesHistograms/lep3_Pt.pdf}
        \caption{}
        \label{fig:lep3_Pt}
    \end{subfigure}
    \hfill
    \begin{subfigure}{.525\textwidth}
        \includegraphics[width=\textwidth]{Figures/FeaturesHistograms/lep2_Pt.pdf}
        \caption{}
        \label{fig:lep2_Pt}
    \end{subfigure}
    }
    \makebox[0.95\linewidth][c]{%
    \begin{subfigure}{.405\textwidth}
        \includegraphics[width=\textwidth]{Figures/FeaturesHistograms/lep2_Eta.pdf}
        \caption{}
        \label{fig:lep2_Eta}
    \end{subfigure}
    \hfill
    \begin{subfigure}{.525\textwidth}
        \includegraphics[width=\textwidth]{Figures/FeaturesHistograms/lep3_Eta.pdf}
        \caption{}
        \label{fig:lep3_Eta}
    \end{subfigure}
    }
    \caption[\acs{MC} simulated and measured data comparison showing the $p_T$ for the first, 
    second and third lepton. Similarly, the distribution over $\eta$ for the first, second and third lepton.]{
        \acs{MC} simulated and measured data comparison showing the $p_T$ for the third \ref{fig:lep3_Pt} 
        and second \ref{fig:lep2_Pt} lepton. Similarly, the distribution over $\eta$ for the 
        second \ref{fig:lep2_Eta} and third \ref{fig:lep3_Eta} lepton.}
        \label{fig:Dist2}
    \end{figure}
    \newpage
    \begin{figure}[H]
    \makebox[0.95\linewidth][c]{%
    \centering
    \begin{subfigure}{.405\textwidth}
        \includegraphics[width=\textwidth]{Figures/FeaturesHistograms/lep1_Phi.pdf}
        \caption{}
        \label{fig:lep1_Phi}
    \end{subfigure}
    \hfill
    \begin{subfigure}{.525\textwidth}
        \includegraphics[width=\textwidth]{Figures/FeaturesHistograms/lep2_Phi.pdf}
        \caption{}
        \label{fig:lep2_Phi}
    \end{subfigure}
    }
    \makebox[0.95\linewidth][c]{%
    \begin{subfigure}{.405\textwidth}
        \includegraphics[width=\textwidth]{Figures/FeaturesHistograms/lep3_Phi.pdf}
        \caption{}
        \label{fig:lep3_Phi}
    \end{subfigure}
    \hfill
    \begin{subfigure}{.525\textwidth}
        \includegraphics[width=\textwidth]{Figures/FeaturesHistograms/lep1_Mt.pdf}
        \caption{}
        \label{fig:lep1_Mt}
    \end{subfigure}
    }
    \makebox[0.95\linewidth][c]{%
    \begin{subfigure}{.405\textwidth}
        \includegraphics[width=\textwidth]{Figures/FeaturesHistograms/lep2_Mt.pdf}
        \caption{}
        \label{fig:lep2_Mt}
    \end{subfigure}
    \hfill
    \begin{subfigure}{.525\textwidth}
        \includegraphics[width=\textwidth]{Figures/FeaturesHistograms/lep3_Mt.pdf}
        \caption{}
        \label{fig:lep3_Mt}
    \end{subfigure}
    }
    \caption[\acs{MC} simulated and measured data comparison showing the $\phi$ for the first, 
    second and third lepton. Similarly, the distribution over $m_t$ for the first, second and third lepton.]{\acs{MC} simulated and measured data 
    comparison showing the $\phi$ for the first \ref{fig:lep1_Phi}, 
    second \ref{fig:lep2_Phi} and third \ref{fig:lep3_Phi} lepton. Similarly, the distribution over $M_T$
    for the first \ref{fig:lep1_Mt}, second \ref{fig:lep2_Mt} and third \ref{fig:lep3_Mt} lepton.}
\end{figure}
\newpage
\begin{figure}[H]
    \makebox[0.95\linewidth][c]{%
    \centering
    \begin{subfigure}{.405\textwidth}
        \includegraphics[width=\textwidth]{Figures/FeaturesHistograms/lep1_Charge.pdf}
        \caption{}
        \label{fig:lep1_Charge}
    \end{subfigure}
    \hfill
    \begin{subfigure}{.525\textwidth}
        \includegraphics[width=\textwidth]{Figures/FeaturesHistograms/lep2_Charge.pdf}
        \caption{}
        \label{fig:lep2_Charge}
    \end{subfigure}
    }
    \makebox[0.95\linewidth][c]{%
    \begin{subfigure}{.405\textwidth}
        \includegraphics[width=\textwidth]{Figures/FeaturesHistograms/lep3_Charge.pdf}
        \caption{}
        \label{fig:lep3_Charge}
    \end{subfigure}
    \hfill
    \begin{subfigure}{.525\textwidth}
        \includegraphics[width=\textwidth]{Figures/FeaturesHistograms/lep1_Flavor.pdf}
        \caption{}
        \label{fig:lep1_Flavor}
    \end{subfigure}
    }
    \makebox[0.95\linewidth][c]{%
    \begin{subfigure}{.405\textwidth}
        \includegraphics[width=\textwidth]{Figures/FeaturesHistograms/lep2_Flavor.pdf}
        \caption{}
        \label{fig:lep2_Flavor}
    \end{subfigure}
    \hfill
    \begin{subfigure}{.525\textwidth}
        \includegraphics[width=\textwidth]{Figures/FeaturesHistograms/lep3_Flavor.pdf}
        \caption{}
        \label{fig:lep3_Flavor}
    \end{subfigure}
    }
    \caption[\acs{MC} simulated and measured data comparison showing the charge for the first,
    second and third lepton. Similarly, the distribution over the flavor for the first, second and third lepton]{\acs{MC} 
    simulated and measured data comparison showing the charge for the first \ref{fig:lep1_Charge},
    second \ref{fig:lep2_Charge} and third \ref{fig:lep3_Charge} lepton. Similarly, the distribution over the flavor
    for the first \ref{fig:lep1_Flavor}, second \ref{fig:lep2_Flavor} and third \ref{fig:lep3_Flavor} lepton.}
\end{figure}
\newpage
\begin{figure}[H]
    \makebox[0.95\linewidth][c]{%
    \centering
    \begin{subfigure}{.405\textwidth}
        \includegraphics[width=\textwidth]{Figures/FeaturesHistograms/deltaR.pdf}
        \caption{}
        \label{fig:deltaR}
    \end{subfigure}
    \hfill
    \begin{subfigure}{.525\textwidth}
        \includegraphics[width=\textwidth]{Figures/FeaturesHistograms/met_Phi.pdf}
        \caption{}
        \label{fig:met_Phi}
    \end{subfigure}
    }
    \makebox[0.95\linewidth][c]{%
    \begin{subfigure}{.405\textwidth}
        \includegraphics[width=\textwidth]{Figures/FeaturesHistograms/mlll.pdf}
        \caption{}
        \label{fig:mlll}
    \end{subfigure}
    \hfill
    \begin{subfigure}{.525\textwidth}
        \includegraphics[width=\textwidth]{Figures/FeaturesHistograms/mll_OSSF.pdf}
        \caption{}
        \label{fig:mll_OSSF}
    \end{subfigure}
    }
    \makebox[0.95\linewidth][c]{%
    \begin{subfigure}{.405\textwidth}
        \includegraphics[width=\textwidth]{Figures/FeaturesHistograms/met_Sign.pdf}
        \caption{}
        \label{fig:met_Sign}
    \end{subfigure}
    \hfill
    \begin{subfigure}{.525\textwidth}
        \includegraphics[width=\textwidth]{Figures/FeaturesHistograms/Ht_lll.pdf}
        \caption{}
        \label{fig:Ht_lll}
    \end{subfigure}
    }
    \caption[\acs{MC} simulated and measured data comparison showing the $\Delta R$ \ref{fig:deltaR}
    and the azimuthal angle of the missing transverse energy. The distribution of the invariant mass of the three leptons and the OSSF pair. 
    The distribution over the significance of the missing transverse energy and the sum of $P_t$.]{\acs{MC} simulated and measured data 
    comparison showing the $\Delta R$ \ref{fig:deltaR} and the azimuthal
    angel \ref{fig:met_Phi} of the missing transverse energy. The distribution of the invariant mass of the
    three leptons \ref{fig:mlll} and the \acs{OSSF} pair \ref{fig:mll_OSSF}. The distribution over the significance
    of the missing transverse energy \ref{fig:met_Sign} and the sum of $P_t$ \ref{fig:Ht_lll}.}
\end{figure}
\newpage
\begin{figure}[H]
    \makebox[0.95\linewidth][c]{%
    \centering
    \begin{subfigure}{.405\textwidth}
        \includegraphics[width=\textwidth]{Figures/FeaturesHistograms/Ht_SS.pdf}
        \caption{}
        \label{fig:Ht_SS}
    \end{subfigure}
    \hfill
    \begin{subfigure}{.525\textwidth}
        \includegraphics[width=\textwidth]{Figures/FeaturesHistograms/Ht_met_Et.pdf}
        \caption{}
        \label{fig:Ht_met_Et}
    \end{subfigure}
    }
    \makebox[0.95\linewidth][c]{%
    \begin{subfigure}{.405\textwidth}
        \includegraphics[width=\textwidth]{Figures/FeaturesHistograms/njet_SG.pdf}
        \caption{}
        \label{fig:njet_SG}
    \end{subfigure}
    \hfill
    \begin{subfigure}{.525\textwidth}
        \includegraphics[width=\textwidth]{Figures/FeaturesHistograms/M_jj.pdf}
        \caption{}
        \label{fig:M_jj}
    \end{subfigure}
    }
    \makebox[0.95\linewidth][c]{%
    \centering
    \begin{subfigure}{.405\textwidth}
        \includegraphics[width=\textwidth]{Figures/FeaturesHistograms/nbjet77.pdf}
        \caption{}
        \label{fig:nbjet77}
    \end{subfigure}
    \hfill
    \begin{subfigure}{.525\textwidth}
        \includegraphics[width=\textwidth]{Figures/FeaturesHistograms/nbjet85.pdf}
        \caption{}
        \label{fig:nbjet85}
    \end{subfigure}
    }
    \caption[\acs{MC} simulated and measured data comparison showing the sum of $p_T$
    for the SS pair and the sum over all three leptons added with $E_t^{miss}$. The distribution of number of 
    signal jets and the mass of the leading dijet pair. Finally, the number of B-jets with $77\%$ and $85\%$ 
    certainty.]{\acs{MC} simulated and measured data comparison showing the sum of $p_T$
    for the \acs{SS} pair \ref{fig:Ht_SS} and the sum over all three leptons added with $E_t^{miss}$
    \ref{fig:Ht_met_Et}. The distribution of number of signal jets \ref{fig:njet_SG} and the mass 
    of the leading dijet pair \ref{fig:M_jj}. Finally, the number of B-jets with $77\%$ \ref{fig:nbjet77} and $85\%$ 
    \ref{fig:nbjet85} certainty.}
\end{figure}
\newpage
\newpage
\subsection{The Selection of Features}\label{subsec:FeatSelec}

\begin{table}[H]
    \centering
    $
    \begin{array}{cc}
        \hline \text {\textbf{Feature Name} }  & \text {\textbf{Description}} \\
        \hline \hline \text {$P_T$}  & \text {Transverse momentum} \\
        \text {$\eta$}  & \text {Pseudo rapidity} \\
        \text {$\phi$}  & \text {Azimuthal angle} \\
        \text {$M_T$}  & \text {Transverse mass} \\
        \text {$Charge$}  & \text {\ac{EM} charge} \\
        \text {$Flavour$}  & \text {Particle type} \\
        \text {$E_T^{miss}$}  & \text {Missing transverse energy} \\
        \text{$\phi(miss)$} & \text {Azimuthal angle of the missing transverse energy} \\
        \text{$M_{lll}$} &  \text {Invariant mass of the trilepton}\\
        \text{$M_{ll}(OSSF)$} & \text {Mass of the \ac{OSSF} pair} \\
        \text{Sig $E_T^{miss}$} & \text {Significance of $E_T^{miss}$} \\
        \text{$H_T(lll)$} &  \text {Sum of $p_T$ for all three leptons }\\
        \text{$H_T(SS)$} &  \text {Sum of $p_T$ for the \ac{SS} pair}\\
        \text{$H_T(lll)+E_T^{miss}$} & \text {-} \\
        \text{$\Delta R$} &  \text {Distance defined in the $\eta\phi$-space}\\
        \text{Flavor combo} &  \text {Combination of flavors for all three leptons}\\
        \text{Nr of signal Jets} &  \text {Nr of jets passing the signal criteria} \\
        \text{$M_{jj}$} & \text {Mass of the leading jet pair} \\
        \text{Nr of B-jets(77)} & \text {Number of B-jets with $77\%$ efficiency} \\
        \text{Nr of B-jets(85)} & \text {Number of B-jets with $85\%$ efficiency} \\
        \hline
    \end{array}
    $
    \caption{A summary and description of all features used in this analysis.}
    \label{table:Features}
\end{table}
\newpage

\appendix
\renewcommand{\thechapter}{C}
\chapter{Appendix C}
\renewcommand{\theequation}{\thechapter.\arabic{equation}}
\section{The implementation of Channel-Out, \ac{SCO} and Maxout}\label{sec:TFImp}
\subsubsection*{Channel-out}
\lstset{style=Python}
\begin{lstlisting}[caption={Python implementation for the custom activation function used to define the channel-out layer.},captionpos=b, label={lst:channel_out}]
def call(self, inputs: tf.Tensor,mask: tf.Tensor = None) -> tf.Tensor:
    # Pass input through weight kernel and adding bias terms.
    inputs = gen_math_ops.MatMul(a=inputs, b=self.kernel)
    inputs = nn_ops.bias_add(inputs, self.bias)

    num_inputs = inputs.shape[0]
    if num_inputs is None:
    num_inputs = -1

    # Reshaping inputs such that they are grouped correctly
    num_competitors = self.units // self.num_groups
    new_shape = [num_inputs, self.num_groups, num_competitors]
    inputs = tf.reshape(inputs, new_shape)

    # Finding maximum activations and setting losers to 0.
    outputs = tf.math.reduce_max(inputs, axis=-1, keepdims=True)
    outputs = tf.where(tf.equal(inputs, outputs), outputs, 0.)
    # Reshaping outputs to original input shape
    outputs = tf.reshape(outputs, [num_inputs, self.units])

    #Count the activate nodes. This variable is used when plotting the activations.
    self.counter = outputs

    return outputs 
\end{lstlisting}

\lstset{style=Python}
\begin{lstlisting}[caption={Python implementation for the custom activation function used to define the \ac{SCO} layer.},captionpos=b, label={lst:channel_out}]
    def call(self, inputs: tf.Tensor, mask: tf.Tensor = None) -> tf.Tensor:
    inputs = gen_math_ops.MatMul(a=inputs, b=self.kernel)
    inputs = nn_ops.bias_add(inputs, self.bias)
    #tf.print(inputs)
    num_inputs = inputs.shape[0]

    if num_inputs is None:
    num_inputs = -1

    shuffle_index = tf.random.shuffle(self.index)
    unshuffle_index = tf.tensor_scatter_nd_update(tensor = self.zeros , 
                                                indices = tf.reshape(shuffle_index, 
                                                                    [inputs.shape[1],1]), 
                                                updates = self.index)
    inputs_s = tf.gather(inputs, shuffle_index, axis = 1)


    # Reshaping inputs such that they are grouped correctly
    num_competitors = self.units // self.num_groups
    new_shape = [num_inputs, self.num_groups, num_competitors]
    inputs_s = tf.reshape(inputs_s, new_shape)

    # Finding maximum activations and setting losers to 0.
    outputs = tf.math.reduce_max(inputs_s, axis=-1, keepdims=True)

    outputs = tf.where(tf.equal(inputs_s, outputs), 1.0, 0.)
    # Reshaping outputs to original input shape
    outputs = tf.reshape(outputs, [num_inputs, self.units])

    outputs = tf.gather(outputs, unshuffle_index, axis = 1) 
    outputs = tf.multiply(inputs, outputs)
    #Count the activate nodes. This variable is used when plotting the activations.
    self.counter = outputs
    return outputs 
\end{lstlisting}
\newpage
\lstset{style=Python}
\begin{lstlisting}[caption={Python implementation for the custom activation function used to define the maxout layer.},captionpos=b, label={lst:max_out}]
def call(self, inputs: tf.Tensor) -> tf.Tensor:
    # Passing input through weight kernel and adding bias terms
    inputs = gen_math_ops.MatMul(a=inputs, b=self.kernel)
    inputs = nn_ops.bias_add(inputs, self.bias)

    num_inputs = inputs.shape[0]
    if num_inputs is None:
        num_inputs = -1
    num_competitors = self.units // self.num_groups
    new_shape = [num_inputs, self.num_groups, num_competitors]

    # Reshaping outputs such that they are grouped correctly
    inputs = tf.reshape(inputs, new_shape)
    # Finding maximum activation in each group
    outputs = tf.math.reduce_max(inputs, axis=-1,keepdims=True)

    counter = tf.where(tf.equal(inputs, outputs), outputs, 0.)
    
    # Reshaping outputs to original input shape
    outputs = tf.reshape(outputs,[num_inputs, self.num_groups])   

    #Count the activate nodes. This variable is used when plotting the activations.
    self.counter = tf.reshape(counter, [num_inputs, self.units])

    return outputs
\end{lstlisting}
\newpage

\appendix
\renewcommand{\thechapter}{D}
\chapter{Appendix D}
\renewcommand{\theequation}{\thechapter.\arabic{equation}}
\section{The Mellin Transform}\label{sec:Appendix Mellin Transform}
For a function $f(x)$ defined on the positive real axis, the Mellin transformation $\mathcal{M}$ is the operation mapping $f$ into the function $\Me{f}$ defined on the complex plane. It has the following definition
\begin{align}
    \mathcal{M}\big[f(x):N\big]=\tilde{f}(N)=\int_{0}^{\infty}dx\,x^{N-1}\,f(x)\,,
\end{align}
where $\Me{f}(N)$ is the Mellin transform of $f(x)$, and $N$ is the Mellin moment conjugate to $x$. In general, the integral does not exist for all functions, i.e. all functions does not have a well defined Mellin transform. But even if the transform exist, it is not guaranteed to converge. The domain of $N$ for which the integral converge for a given function is known as the fundamental strip. For a real function, the fundamental strip is denoted as $<a,b>$, given by all points on the domain $a<s<b$ such that $N=s+it$, for any t. The values of $a$ and $b$ is found by the asymptotic behaviour
\begin{align}
    a:\hspace{0.5cm}\lim_{x\rightarrow 0^{+}}f(x)&=\mathcal{O}(x^{-a})\,,
    \\
    b:\hspace{0.5cm}\lim_{x\rightarrow\infty}f(x)&=\mathcal{O}(x^{-b})\,,
\end{align}
which implies that a function defined on the domain $0<x<1$, have a fundamental strip $<a,\infty>$.

The Mellin transform is closely related to the two-sided Laplace transform, with the difference of a variable change $x\rightarrow -\ln x$. Hence, the inverse Mellin is given by
\begin{align}\label{eq:Appendix Inverse Mellin}
    \mathcal{M}^{-1}\big[\Me{f}(N):x\big]=\frac{1}{2\pi i}\int_{c-i\infty}^{c+i\infty}dN\,x^{-N}\Me{f}(N)\,,
\end{align}
where the integration contour is along a vertical line through $Re(N)=c$, as long as $c$ lies in the fundamental strip of the function. If the function is holomorphic in the strip and vanishes sufficiently fast when $Im(N)\rightarrow\pm\infty$, it follows from Cauchy's theorem that the contour may be deformed as long as no poles are crossed. 

One very important property of the Mellin transform is its effect on convolutions
\begin{align}\label{eq:Appendix Mellin convolution}
    \big(f\star g\big)(x)=\int_{0}^{1}dx_1\int_{0}^{1}dx_2\,f(x_1)g(x_2)\delta(x-x_1x_2)=\int_{x}^{1}\frac{dx_1}{x_1}\,f(x_1)g\big(\frac{x_2}{x_1}\big)\,,
\end{align}
where $x\in(0,1)$. To disentangle this convolution, one performs the transform
\begin{align}\label{eq:Appendix Mellin convolution transform}
    \mathcal{M}\big[\big(f\star g\big)(x):N\big]&=\int_{0}^{1}dx\,x^{N-1}\int_{0}^{1}dx_1\int_{0}^{1}dx_2\,f(x_1)g(x_2)\delta(x-x_1x_2)\nonumber
    \\
    &=\int_{0}^{1}dx_1\,x_1^{N-1}f(x_1)\int_{0}^{1}dx_2\,x_2^{N-1}g(x_2)\nonumber
    \\
    &=\Me{f}(N)\Me{g}(N)\,,
\end{align}
where the convolution in $x$-space has transformed into simple products in Mellin space. 

\subsection{Mellin Transforms of Functions}
This section is intended to demonstrate some of the Mellin transforms that are encountered in this thesis. Since the main focus is in the domain of large $N$, this will not be the general treatment of these transforms.

The simplest case is the Mellin transform of a constant $c$,
\begin{align}
    \int_{0}^{1}dx\,x^{N-1}c=\frac{c}{N}\,,
\end{align}
and a monomial
\begin{align}
    \int_{0}^{1}dx\,x^{N-1}x^{a}=\frac{1}{N+a}\,,
\end{align}
which combined with a logarithm
\begin{align}
    \int_{0}^{1}dx\,x^{N-1}(-x^{a}\ln x)&=-\frac{x^{N+a}}{N+a}\ln x\big|_{0}^{1}+\int_{0}^{1}dx\,\frac{x^{N+a-1}}{N+a}\nonumber
    \\
    &=\frac{1}{(N+a)^{2}}\,,
\end{align}
which can be generalized for any polynomial $P(x)$
\begin{align}
    \int_{0}^{1}dx\,x^{N-1}P(x)=\mathcal{O}(1/N)\,.
\end{align}
Furthermore, by using that $x^{N-1}=e^{(N-1)\ln x}$, repeated derivatives with respect to $N$ will give
\begin{align}
    \int_{0}^{1}dx\,x^{N-1} f(x)\ln^{k}x=\frac{d^{k}}{dN^{k}}\Me{f}(N),\hspace{1cm}\forall\,k>0\,.
\end{align}
Another useful property involves the derivative of a function
\begin{align}\label{eq:Appendix derivative of function}
    \int_{0}^{1}dx\,x^{N-1}\Big(-x\frac{d}{dx}f(x)\Big)&=-x^{N}f(x)\big|_{0}^{1}+N\int_{0}^{1}dx\,x^{N-1}f(x)\nonumber
    \\
    &=-f(1)+N\Me{f}(N)\,,
\end{align}
which is especially useful for functions that vanish for $x=1$.

The Mellin transform of plus distributions are more complicated. By using \cref{eq:plus distribution convolution}, we can write
\begin{align}
    \int_{0}^{1}dx\,x^{N-1}\Big[\frac{1}{1-x}\Big]_{+}=\int_{0}^{1}dx\Big(\frac{x^{N-1}}{1-x}-\frac{1}{1-x}\Big)\,.
\end{align}
The terms on the $rhs$ diverge when considered separately, so we can not calculate them independently. By introducing a regulator $\epsilon$, this can be rewritten by using beta integrals
\begin{align}
    \int_{0}^{1}dx \frac{x^{N-1}}{(1-x)^{1-\epsilon}}-\int_{0}^{1}dx\frac{1}{(1-x)^{1-\epsilon}}=\frac{\Gamma(N)\Gamma(\epsilon)}{\Gamma(N+\epsilon)}-\frac{\Gamma(1)\Gamma(\epsilon)}{\Gamma(1+\epsilon)}\,,
\end{align}
which by the recursion relation $x\Gamma(x)=\Gamma(x+1)$, can be shown to give\footnote{After the regulator has been removed.} 
\begin{align}
    \int_{0}^{1}dx\,x^{N-1}\Big[\frac{1}{1-x}\Big]_{+}&=\sum_{k=1}^{N-1}\frac{1}{k}\nonumber
    \\
    &=-\Big(\int_{1}^{N}dk\frac{1}{k}+\lim_{N\to\infty}\Big(\sum_{k=1}^{N-1}\frac{1}{k}-\int_{1}^{N}dk\frac{1}{k}\Big)+\mathcal{O}(1/N)\Big)\nonumber
    \\
    &=-\ln\Bar{N}+\mathcal{O}(1/N)\,,
\end{align}
where $\Bar{N}=Ne^{e^{\gamma_{E}}}$. 

Particularly useful moments are those of plus distributions with logarithms, see \cite{CATANI1989} 
\begin{align}
    \int_{0}^{1}dx\,x^{N-1}\Big[\frac{\ln(1-x)}{1-x}\Big]_{+}&=\frac{1}{2}\ln^{2}\Bar{N}+\frac{1}{2}\zeta(2)+\mathcal{O}(1/N)\,,\label{eq:App mellin of ln plus dist}
    \\
    \int_{0}^{1}dx\,x^{N-1}\Big[\frac{\ln^{2}(1-x)}{1-x}\Big]_{+}&=-\frac{1}{3}\ln^{3}\Bar{N}-\zeta{2}\ln\Bar{N}-\frac{2}{3}\zeta{3}+\mathcal{O}(1/N)\,,
\end{align}
where $\zeta(y)$ is the Riemann zeta function. In the large $N$ limit, the following behaviour follows
\begin{align}
    \int_{0}^{1}dx x^{N-1}\Big[\frac{\ln^{n}(1-x)}{(1-x)}\Big]_{+}=\frac{(-1)^{n+1}}{n+1}\ln^{n+1}(\bar{N})+\mathcal{O}(\ln^{n-1}(\bar{N}))\,.
\end{align}


\end{appendices}




\bibliography{bibliography.bib}

\end{document}
